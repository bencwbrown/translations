\documentclass{article}

\title{Duality theorem for coherent algebraic sheaves}
\author{A. Grothendieck}
\date{May 1957}

\usepackage{amssymb,amsmath}

\usepackage{hyperref}
\usepackage[nameinlink]{cleveref}
\usepackage{enumerate}
\usepackage{tikz-cd}
\usepackage{graphicx}

\usepackage{mathrsfs}
%% Fancy fonts --- feel free to remove! %%
\usepackage{Baskervaldx}
\usepackage{mathpazo}


\usepackage{fancyhdr}
\usepackage{lastpage}
\usepackage{xstring}
\makeatletter
\ifx\pdfmdfivesum\undefined
  \let\pdfmdfivesum\mdfivesum
\fi
\edef\filesum{\pdfmdfivesum file {\jobname}}
\pagestyle{fancy}
\makeatletter
\let\runauthor\@author
\let\runtitle\@title
\makeatother
\fancyhf{}
\lhead{\footnotesize\runtitle}
\rhead{\footnotesize Version: \texttt{\StrMid{\filesum}{1}{8}}}
\cfoot{\small\thepage\ of \pageref*{LastPage}}


\crefname{section}{Section}{Sections}
\crefname{equation}{}{}


%% Theorem environments %%

\usepackage{amsthm}

  \theoremstyle{plain}

  \newtheorem{innercustomproposition}{Proposition}
  \crefname{innercustomproposition}{Proposition}{Propositions}
  \newenvironment{proposition}[1]
    {\renewcommand\theinnercustomproposition{#1}\innercustomproposition}
    {\endinnercustomproposition}

  \newtheorem{innercustomcorollary}{Corollary}
  \crefname{innercustomcorollary}{Corollary}{Corollary}
  \newenvironment{corollary}[1]
    {\renewcommand\theinnercustomcorollary{#1}\innercustomcorollary}
    {\endinnercustomcorollary}


%% Shortcuts %%

\newcommand{\sh}{\mathscr}
\newcommand{\cat}{\mathcal}

\renewcommand{\geq}{\geqslant}
\renewcommand{\leq}{\leqslant}

\DeclareMathOperator{\Ext}{Ext}
\DeclareMathOperator{\Hom}{Hom}
\DeclareMathOperator{\shExt}{\underline{Ext}}
\DeclareMathOperator{\shHom}{\underline{Hom}}
\DeclareMathOperator{\RR}{R}
\DeclareMathOperator{\HH}{H}

\newcommand{\todo}{\textbf{ !TODO! }}
\newcommand{\oldpage}[1]{\marginpar{\footnotesize$\Big\vert$ \textit{p.~#1}}}


%% Document %%

\usepackage{embedall}
\begin{document}

\maketitle
\thispagestyle{fancy}

\renewcommand{\abstractname}{Translator's note.}

\begin{abstract}
  \renewcommand*{\thefootnote}{\fnsymbol{footnote}}
  \emph{This text is one of a series\footnote{\url{https://github.com/thosgood/translations}} of translations of various papers into English.}
  \emph{The translator takes full responsibility for any errors introduced in the passage from one language to another, and claims no rights to any of the mathematical content herein.}
  
  \emph{What follows is a translation of the French paper:}

  \medskip\noindent
  \textsc{Grothendieck, A.}
  Th\'{e}or\`{e}mes de dualit\'{e} pour les faisceaux alg\'{e}briques coh\'{e}rents.
  \emph{S\'{e}minaire Bourbaki}, Volume~\textbf{9} (1956--57), Talk no.~149.
\end{abstract}

\setcounter{footnote}{0}

\tableofcontents
\bigskip


%% Content %%

\oldpage{149-01}
The results that follow, inspired by Serre's ``theorem of algebraic duality'', were discovered in the winter of 1955 and the winter of 1956.
They can be established very simply, thanks to reasonably elementary results on the cohomology of projective spaces \cite{3} and an intensive use of Cartan--Eilenberg's homological algebra, in the form given in \cite{2}.


\section{$\Ext$ of sheaves of modules}
\label{section1}

(\cite[chap.~3 and 4]{2}).

Let $X$ be a topological space endowed with a sheaf $\sh{O}$ of unital (but not necessarily commutative) rings.
We consider the abelian category $\cat{C}^\sh{O}$ of sheaves of $\sh{O}$-modules, which also referred to as $\sh{O}$-modules.
We know that every object of this category admits an injective resolution, which allows us to define the $\Ext$ functors that have the well-known formal properties.
More precisely, to avoid confusion, we denote by $\Hom_\sh{O}(X;\sh{A},\sh{B})$, or simply $\Hom(X;\sh{A},\sh{B})$, the abelian \emph{groups} of $\sh{O}$-homomorphisms from $\sh{A}$ to $\sh{B}$, whereas $\shHom_\sh{O}(\sh{A},\sh{B})$ denotes the \emph{sheaf} of germs of homomorphisms from $\sh{A}$ to $\sh{B}$ (where $\sh{A},\sh{B}\in \cat{C}^\sh{O}$).
We define, for fixed $\sh{A}\in \cat{C}^\sh{O}$, functors $h_\sh{A}$ and $\underline{h}_\sh{A}$, with values in the category $\cat{C}$ of abelian groups and the category $\cat{C}^Z$ of abelian sheaves on $X$ (respectively), by the formulas:
\[
\label{equation1.1}
  \begin{aligned}
    h_\sh{A}(\sh{B}) &= \Hom_\sh{O}(X;\sh{A},\sh{B})
  \\\underline{h}_\sh{A}(\sh{B}) &= \shHom_\sh{O}(\sh{A},\sh{B}).
  \end{aligned}
\tag{1.1}
\]

The functors $h_\sh{A}$ and $\underline{h}_\sh{A}$ are left exact and covariant, and so we consider their right-derived functors, denoted by $\Ext_\sh{O}^p(X;\sh{A},\sh{B})$ and $\shExt_\sh{O}^p(\sh{A},\sh{B})$ (respectively).
We then have, by definition,
\[
\label{equation1.2}
  \begin{gathered}
    \Ext_\sh{O}^p(X;\sh{A},\sh{B}) = (\RR^p h_\sh{A})(\sh{B}) = \HH^p(\Hom_\sh{O}(X;\sh{A},C(\sh{B})))
  \\\shExt_\sh{O}^p(\sh{A},\sh{B}) = (\RR^p \underline{h}_\sh{A})(\sh{B}) = \HH^p(\shHom_\sh{O}(\sh{A},C(\sh{B})))
  \end{gathered}
\tag{1.2}
\]
where $\RR^p$ denotes the passage to right-derived functors, and where $C(\sh(B)$ denotes in arbitrary injective resolution of $\sh{B}$ in $\cat{C}^\sh{O}$.
We denote by $\Gamma\colon\cat{C}^Z\to\cat{C}$ the ``sections'' functor.
Recall that its right-derived functors are denoted by $B\mapsto\HH^p(X,\sh{B})$:
\oldpage{149-02}
\[
\label{equation1.3}
  \HH^p(X,\sh{B}) = (\RR^p\Gamma)(\sh{B}) = \HH^p(\Gamma(C(\sh{B}))).
\tag{1.3}
\]
We evidently have $h_\sh{A}=\Gamma\underline{h}_\sh{A}$;
we can also show that $\underline{h}_\sh{A}$ sends injective objects to $\Gamma$-acyclic objects.
From this, it is a well-known result that:

\begin{proposition}{1}
\label{proposition1}
  There exists, for every $\sh{O}$-module $\sh{A}$, a cohomological spectral functor on $\cat{C}^\sh{O}$, abutting to the graded functor $(\Ext_\sh{O}^\bullet(X;\sh{A},\sh{B}))$, and whose initial page is
  \[
  \label{equation1.4}
    E_2^{p,q}(\sh{A},\sh{B}) = \HH^p(X,\shExt_\sh{O}^q(\sh{A},\sh{B})).
  \tag{1.4}
  \]
\end{proposition}

From this, we obtain ``\emph{boundary homomorphisms}'', and a short exact sequence, which we will not write.

\begin{corollary}{1}
\label{corollary1}
  If $\sh{A}$ is locally isomorphism to $\sh{O}^n$, then we have canonical isomorphisms
  \[
  \label{equation1.5}
    \Ext_\sh{O}^p(X;\sh{A},\sh{B}) \xleftarrow{\sim} \HH^p(x,\shHom_\sh{O}(\sh{A},\sh{B}))
  \tag{1.5}
  \]
  (given by the boundary homomorphisms of the spectral sequence).
  In particular, we have a canonical isomorphism
  \[
  \label{equation1.6}
    \Ext_\sh{O}^p(X;\sh{O},\sh{B}) = \HH^p(X,\sh{B}).
  \tag{1.6}
  \]
\end{corollary}

To use these results, we need to know how to explicitly describe the $\Ext_\sh{O}^p(\sh{A},\sh{B})$.
They are functors that we calculate locally, i.e. if $V$ is an open subset of $X$, then
\[
  \shExt_\sh{O}^p(\sh{A},\sh{B})|U = \shExt_{\sh{O}|U}^p(\sh{A}|U,\sh{B}|U),
\]
as follows from the fact that the restriction to $U$ of an injective $\sh{O}$-module is an injective $(\sh{O}|U)$-module.
Furthermore, for fixed $x\in X$, we have functorial homomorphisms,
\[
\label{equation1.7}
  \shHom_\sh{O}(\sh{A},\sh{B})_x \to \Hom_{\sh{O}_x}(\sh{A}_x,\sh{B}_x)
\tag{1.7}
\]
that uniquely extend to a homomorphism of cohomological functors (in $\sh{B}$):
\[
\label{equation1.8}
  \shExt_\sh{O}^p(\sh{A},\sh{B})_x \to \Ext_{\sh{O}_x}^p(\sh{A}_x,\sh{B}_x).
\tag{1.8}
\]

\begin{proposition}{2}
\label{proposition2}
  If, in a neighbourhood of $x$, $\sh{A}$ is isomorphic to the cokernel of a homomorphism $\sh{O}^m\to\sh{O}^n$, then \cref{equation1.7} is an isomorphism for all $p$.
  This is the case, in particular, if $\sh{A}$ is a coherent $\sh{O}$-module \cite{3}.
\end{proposition}

\oldpage{149-03}
\begin{proposition}{3}
\label{proposition3}
  Let $\sh{L}_\bullet=(\sh{L}_i)$ be a left resolution of the $\sh{O}$-module $\sh{A}$ by $\sh{O}$-modules that are all locally isomorphic to some $\sh{O}^n$.
  Then $\shExt_\sh{O}(\sh{A},\sh{B})$ can be identified with $\HH^\bullet(\shHom_\sh{O}(\sh{L}_\bullet,\sh{B}))$, and $\Ext_\sh{O}(X;\sh{A},\sh{B})$ can be identified with the hypercohomology of $X$ with respect to the complex $\shHom_\sh{O}(\sh{L}_\bullet,\sh{B})$.
\end{proposition}

\begin{proof}
  The proof is standard: we consider the bicomplex $\shHom_\sh{O}(\sh{L}_\bullet,C(\sh{B}))$, where $C(\sh{B})$ is an injective resolution of $\sh{B}$, as well as the natural homomorphisms into this bicomplex from $\shHom_\sh{O}(\sh{L}_\bullet,\sh{B}))$ and $\shHom_\sh{O}(\sh{A},C,\sh{B}))$.
\end{proof}

To finish, we note that the two $\Ext$ functors introduced in \cref{equation1.2} are not only cohomological functors in $\sh{B}$, but in fact \emph{cohomological bifunctors}, covariant in $\sh{B}$ and contravariant in $\sh{A}$.


\section{The composition law in $\Ext$}
\label{section2}


%% Bibliography %%

\nocite{*}
\bibliographystyle{acm}

\end{document}

\documentclass{article}

\title{Duality theorem for coherent algebraic sheaves}
\author{A. Grothendieck}
\date{May 1957}

\usepackage{amssymb,amsmath}

\usepackage{hyperref}
\usepackage[nameinlink]{cleveref}
\usepackage{enumerate}
\usepackage{tikz-cd}
\usepackage{graphicx}

\usepackage{mathrsfs}
%% Fancy fonts --- feel free to remove! %%
\usepackage{Baskervaldx}
\usepackage{mathpazo}


\usepackage{fancyhdr}
\usepackage{lastpage}
\usepackage{xstring}
\makeatletter
\ifx\pdfmdfivesum\undefined
  \let\pdfmdfivesum\mdfivesum
\fi
\edef\filesum{\pdfmdfivesum file {\jobname}}
\pagestyle{fancy}
\makeatletter
\let\runauthor\@author
\let\runtitle\@title
\makeatother
\fancyhf{}
\lhead{\footnotesize\runtitle}
\rhead{\footnotesize Version: \texttt{\StrMid{\filesum}{1}{8}}}
\cfoot{\small\thepage\ of \pageref*{LastPage}}


\crefname{section}{Section}{Sections}
\crefname{equation}{}{}


%% Theorem environments %%

\usepackage{amsthm}

  \theoremstyle{plain}

  \newtheorem{innercustomproposition}{Proposition}
  \crefname{innercustomproposition}{Proposition}{Propositions}
  \newenvironment{proposition}[1]
    {\renewcommand\theinnercustomproposition{#1}\innercustomproposition}
    {\endinnercustomproposition}

  \newtheorem{innercustomcorollary}{Corollary}
  \crefname{innercustomcorollary}{Corollary}{Corollaries}
  \newenvironment{corollary}[1]
    {\renewcommand\theinnercustomcorollary{#1}\innercustomcorollary}
    {\endinnercustomcorollary}

  \newtheorem{innercustomlemma}{Lemma}
  \crefname{innercustomlemma}{Lemma}{Lemmas}
  \newenvironment{lemma}[1]
    {\renewcommand\theinnercustomlemma{#1}\innercustomlemma}
    {\endinnercustomlemma}

  \newtheorem*{corollary*}{Corollary}


%% Shortcuts %%

\newcommand{\sh}{\mathscr}
\newcommand{\cat}{\mathcal}
\newcommand{\from}{\leftarrow}

\renewcommand{\geq}{\geqslant}
\renewcommand{\leq}{\leqslant}

\DeclareMathOperator{\Ext}{Ext}
\DeclareMathOperator{\Hom}{Hom}
\DeclareMathOperator{\Tor}{Tor}
\DeclareMathOperator{\shExt}{\underline{Ext}}
\DeclareMathOperator{\shHom}{\underline{Hom}}
\DeclareMathOperator{\RR}{R}
\DeclareMathOperator{\HH}{H}

\newcommand{\todo}{\textbf{ !TODO! }}
\newcommand{\oldpage}[1]{\marginpar{\footnotesize$\Big\vert$ \textit{p.~#1}}}


%% Document %%

\usepackage{embedall}
\begin{document}

\maketitle
\thispagestyle{fancy}

\renewcommand{\abstractname}{Translator's note.}

\begin{abstract}
  \renewcommand*{\thefootnote}{\fnsymbol{footnote}}
  \emph{This text is one of a series\footnote{\url{https://github.com/thosgood/translations}} of translations of various papers into English.}
  \emph{The translator takes full responsibility for any errors introduced in the passage from one language to another, and claims no rights to any of the mathematical content herein.}
  
  \emph{What follows is a translation of the French paper:}

  \medskip\noindent
  \textsc{Grothendieck, A.}
  Th\'{e}or\`{e}mes de dualit\'{e} pour les faisceaux alg\'{e}briques coh\'{e}rents.
  \emph{S\'{e}minaire Bourbaki}, Volume~\textbf{9} (1956--57), Talk no.~149.
\end{abstract}

\setcounter{footnote}{0}

\tableofcontents
\bigskip


%% Content %%

\oldpage{149-01}
The results that follow, inspired by Serre's ``theorem of algebraic duality'', were discovered in the winter of 1955 and the winter of 1956.
They can be established very simply, thanks to reasonably elementary results on the cohomology of projective spaces \cite{3} and an intensive use of Cartan--Eilenberg's homological algebra, in the form given in \cite{2}.


\section{$\Ext$ of sheaves of modules}
\label{section1}

(\cite[chap.~3 and 4]{2}).

Let $X$ be a topological space endowed with a sheaf $\sh{O}$ of unital (but not necessarily commutative) rings.
We consider the abelian category $\cat{C}^\sh{O}$ of sheaves of $\sh{O}$-modules, which also referred to as $\sh{O}$-modules.
We know that every object of this category admits an injective resolution, which allows us to define the $\Ext$ functors that have the well-known formal properties.
More precisely, to avoid confusion, we denote by $\Hom_\sh{O}(X;\sh{A},\sh{B})$, or simply $\Hom(X;\sh{A},\sh{B})$, the abelian \emph{groups} of $\sh{O}$-homomorphisms from $\sh{A}$ to $\sh{B}$, whereas $\shHom_\sh{O}(\sh{A},\sh{B})$ denotes the \emph{sheaf} of germs of homomorphisms from $\sh{A}$ to $\sh{B}$ (where $\sh{A},\sh{B}\in \cat{C}^\sh{O}$).
We define, for fixed $\sh{A}\in \cat{C}^\sh{O}$, functors $h_\sh{A}$ and $\underline{h}_\sh{A}$, with values in the category $\cat{C}$ of abelian groups and the category $\cat{C}^Z$ of abelian sheaves on $X$ (respectively), by the formulas:
\[
\label{1.1}
  \begin{aligned}
    h_\sh{A}(\sh{B}) &= \Hom_\sh{O}(X;\sh{A},\sh{B})
  \\\underline{h}_\sh{A}(\sh{B}) &= \shHom_\sh{O}(\sh{A},\sh{B}).
  \end{aligned}
\tag{1.1}
\]

The functors $h_\sh{A}$ and $\underline{h}_\sh{A}$ are left exact and covariant, and so we consider their right-derived functors, denoted by $\Ext_\sh{O}^p(X;\sh{A},\sh{B})$ and $\shExt_\sh{O}^p(\sh{A},\sh{B})$ (respectively).
We then have, by definition,
\[
\label{1.2}
  \begin{gathered}
    \Ext_\sh{O}^p(X;\sh{A},\sh{B}) = (\RR^p h_\sh{A})(\sh{B}) = \HH^p(\Hom_\sh{O}(X;\sh{A},C(\sh{B})))
  \\\shExt_\sh{O}^p(\sh{A},\sh{B}) = (\RR^p \underline{h}_\sh{A})(\sh{B}) = \HH^p(\shHom_\sh{O}(\sh{A},C(\sh{B})))
  \end{gathered}
\tag{1.2}
\]
where $\RR^p$ denotes the passage to right-derived functors, and where $C(\sh(B)$ denotes in arbitrary injective resolution of $\sh{B}$ in $\cat{C}^\sh{O}$.
We denote by $\Gamma\colon\cat{C}^Z\to\cat{C}$ the ``sections'' functor.
Recall that its right-derived functors are denoted by $B\mapsto\HH^p(X,\sh{B})$:
\oldpage{149-02}
\[
\label{1.3}
  \HH^p(X,\sh{B}) = (\RR^p\Gamma)(\sh{B}) = \HH^p(\Gamma(C(\sh{B}))).
\tag{1.3}
\]
We evidently have $h_\sh{A}=\Gamma\underline{h}_\sh{A}$;
we can also show that $\underline{h}_\sh{A}$ sends injective objects to $\Gamma$-acyclic objects.
From this, it is a well-known result that:

\begin{proposition}{1}
\label{proposition1}
  There exists, for every $\sh{O}$-module $\sh{A}$, a cohomological spectral functor on $\cat{C}^\sh{O}$, abutting to the graded functor $(\Ext_\sh{O}^\bullet(X;\sh{A},\sh{B}))$, and whose initial page is
  \[
  \label{1.4}
    E_2^{p,q}(\sh{A},\sh{B}) = \HH^p(X,\shExt_\sh{O}^q(\sh{A},\sh{B})).
  \tag{1.4}
  \]
\end{proposition}

From this, we obtain ``\emph{boundary homomorphisms}'', and a short exact sequence, which we will not write.

\begin{corollary}{1}
\label{proposition1corollary1}
  If $\sh{A}$ is locally isomorphism to $\sh{O}^n$, then we have canonical isomorphisms
  \[
  \label{1.5}
    \Ext_\sh{O}^p(X;\sh{A},\sh{B}) \xleftarrow{\sim} \HH^p(x,\shHom_\sh{O}(\sh{A},\sh{B}))
  \tag{1.5}
  \]
  (given by the boundary homomorphisms of the spectral sequence).
  In particular, we have a canonical isomorphism
  \[
  \label{1.6}
    \Ext_\sh{O}^p(X;\sh{O},\sh{B}) = \HH^p(X,\sh{B}).
  \tag{1.6}
  \]
\end{corollary}

To use these results, we need to know how to explicitly describe the $\Ext_\sh{O}^p(\sh{A},\sh{B})$.
They are functors that we calculate locally, i.e. if $V$ is an open subset of $X$, then
\[
  \shExt_\sh{O}^p(\sh{A},\sh{B})|U = \shExt_{\sh{O}|U}^p(\sh{A}|U,\sh{B}|U),
\]
as follows from the fact that the restriction to $U$ of an injective $\sh{O}$-module is an injective $(\sh{O}|U)$-module.
Furthermore, for fixed $x\in X$, we have functorial homomorphisms,
\[
\label{1.7}
  \shHom_\sh{O}(\sh{A},\sh{B})_x \to \Hom_{\sh{O}_x}(\sh{A}_x,\sh{B}_x)
\tag{1.7}
\]
that uniquely extend to a homomorphism of cohomological functors (in $\sh{B}$):
\[
\label{1.8}
  \shExt_\sh{O}^p(\sh{A},\sh{B})_x \to \Ext_{\sh{O}_x}^p(\sh{A}_x,\sh{B}_x).
\tag{1.8}
\]

\begin{proposition}{2}
\label{proposition2}
  If, in a neighbourhood of $x$, $\sh{A}$ is isomorphic to the cokernel of a homomorphism $\sh{O}^m\to\sh{O}^n$, then \cref{1.7} is an isomorphism for all $p$.
  This is the case, in particular, if $\sh{A}$ is a coherent $\sh{O}$-module \cite{3}.
\end{proposition}

\oldpage{149-03}
\begin{proposition}{3}
\label{proposition3}
  Let $\sh{L}_\bullet=(\sh{L}_i)$ be a left resolution of the $\sh{O}$-module $\sh{A}$ by $\sh{O}$-modules that are all locally isomorphic to some $\sh{O}^n$.
  Then $\shExt_\sh{O}(\sh{A},\sh{B})$ can be identified with $\HH^\bullet(\shHom_\sh{O}(\sh{L}_\bullet,\sh{B}))$, and $\Ext_\sh{O}(X;\sh{A},\sh{B})$ can be identified with the hypercohomology of $X$ with respect to the complex $\shHom_\sh{O}(\sh{L}_\bullet,\sh{B})$.
\end{proposition}

\begin{proof}
  The proof is standard: we consider the bicomplex $\shHom_\sh{O}(\sh{L}_\bullet,C(\sh{B}))$, where $C(\sh{B})$ is an injective resolution of $\sh{B}$, as well as the natural homomorphisms into this bicomplex from $\shHom_\sh{O}(\sh{L}_\bullet,\sh{B}))$ and $\shHom_\sh{O}(\sh{A},C,\sh{B}))$.
\end{proof}

To finish, we note that the two $\Ext$ functors introduced in \cref{1.2} are not only cohomological functors in $\sh{B}$, but in fact \emph{cohomological bifunctors}, covariant in $\sh{B}$ and contravariant in $\sh{A}$.


\section{The composition law in $\Ext$}
\label{section2}

The results of this section are due, independently, to Cartier and Yoneda;
see an expos\'{e} by Cartier \cite{1} for more details.
Let $\cat{C}$ be an abelian category.
Let $K$ and $L$ be two graded objects of $\cat{C}$.
We denote by $\Hom(K,L)$ the graded abelian group whose degree-$n$ component consists of homogeneous homomorphisms of degree~$n$ from $K$ to $L$ (i.e. systems $(u_i)$ of homomorphisms $K^i\to L^{i+n}$).
If $K$ and $L$ are complexes (with differentials of degree~$+1$, to fix conventions), then we endow $\Hom(K,L)$ with the differential operator given by
\[
\label{2.1}
  \delta(u) = \mathrm{d}u + (-1)^{n+1}u\mathrm{d}
  \quad\text{where }n=\deg(u)
\tag{2.1}
\]
which makes it a complex with a differential of degree~$+1$.
The cycles of degree~$n$ are the maps of degree~$n$ that anticommute with $u$ (as homogeneous maps).
We can then consider $\HH^\bullet(\Hom(K,L))$, which is an invariant of the homotopy types of $K$ and $L$< and which we may denote by $\HH^\bullet(K,L)$.
If we have a third complex $M$, then the composition of homomorphisms defines a pairing $\Hom(K,L)\times\Hom(L,M)\to\Hom(K,M)$ that is compatible with the differential maps, whence, by passing to the cohomology of pairings,
\[
\label{2.2}
  \HH^\bullet(K,L)\times\HH^\bullet(L,M) \to \HH^\bullet(K,M)
\tag{2.2}
\]
which we write as $(u,v)\mapsto vu$.
These pairings satisfy an evident associativity property.
In particular, $\HH^\bullet(K,K)$ is an associative graded unital ring, and $\HH^\bullet(K,L)$ (resp. $\HH^\bullet(L,K)$) is a graded right (resp. left) module over this ring, etc.
In dimension~$0$, \cref{2.2} reduces to the composition of permissible homomorphisms of complexes.
Finally, an exact sequence of complexes
\oldpage{149-04}
$0\to K'\to K\to K''\to0$, such that, for all $i$, $K'^i$ can be identified with a direct factor of $K^i$, gives rise to an exact sequence of complexes of groups $\Hom(K'',L)$, etc., whence a coboundary map $\HH^i(K',L)\to\HH^{i+1}(K'',L)$.
We similarly define the boundary maps relative to an exact sequence in $L$.
The pairings in \cref{2.2} are compatible, in the usual sense, with these coboundary maps.

Now suppose that $\cat{C}$ is a category such that every element $A$ of $C$ admits an injective resolution $C(A)$.
We then note that, using one of the many variants of the theorem of bicomplexes,
\[
  \HH^\bullet(C(A),C(B)) = \HH^\bullet(\Hom(C(A),C(B)))
\]
is canonically isomorphic to
\[
  \HH^\bullet(\Hom(A,C(B))) = \Ext^\bullet(A,B).
\]
The coboundary maps described above give coboundary maps of the $\Ext$.
Furthermore, the pairings in \cref{2.2} give associative pairings here:
\[
\label{2.3}
  \Ext^\bullet(A,B)\times\Ext^\bullet(B,C) \to \Ext^\bullet(A,C)
\tag{2.3}
\]
and these are compatible with the coboundary maps.
In particular, $\Ext^\bullet(A,A)$ is an associative graded unital ring, etc.
(We show in an analogous manner that the $\Ext$ functors \todo;
we do not make use of this fact here).

In the case where the category in question is the category $\cat{C}^\sh{O}$ of $\sh{O}$-modules on $X$, we then obtain pairings
\[
\label{2.4}
  \Ext_\sh{O}^p(X;\sh{A},\sh{B})\times\Ext_\sh{O}^q(X;\sh{B},\sh{C}) \to \Ext_\sh{O}^{p+q}(X;\sh{A},\sh{C})
\tag{2.4}
\]
that can be calculated as has already been said.
The same method, but replacing the category of abelian groups with the category of abelian sheaves on $X$, and the $\Hom$ functors by the $\shHom$ functors, again defines pairings, having the same formal properties, and of a ``local nature'' this time:
\[
\label{2.5}
  \shExt_\sh{O}^p(\sh{A},\sh{B})\times\shExt_\sh{O}^q(\sh{B},\sh{C}) \to \shExt_\sh{O}^{p+q}(\sh{A},\sh{C}).
\tag{2.5}
\]
These can be understood by noting that the homomorphisms in \cref{1.8} are compatible with the pairings between the $\Ext$.

\oldpage{149-05}
Finally, recall that we also have a multiplicative structure between functors $\HH^p(X,A)$ (the cup product).
We note then that the spectral sequences of \cref{proposition1} are compatible with the multiplicative structures;
more precisely, we have a pairing from the spectral sequence $E(A,B)$ with the spectral sequence $E(B,C)$ to the spectral sequence $E(A,C)$ that abuts to the pairing between the global $\Ext$, and whose \todo initial page comes from the cup product and the local $\Ext$ pairings in the right-hand side of \cref{1.4}.
It then follows, in particular, that the ``boundary homomorphisms''
\[
\label{2.6}
  \Ext_\sh{O}^n(X;\sh{A},\sh{B}) \to \HH^0(X;\shExt_\sh{O}^n(\sh{A},\sh{B}))
\tag{2.6}
\]
\[
\label{2.7}
  \HH^n(X,\shHom_\sh{O}(\sh{A},\sh{B})) \to \Ext_\sh{O}^n(X;\sh{A},\sh{B})
\tag{2.7}
\]
are compatible with the multiplicative structures.
So if we restrict to sheaves that are locally isomorphic to some $\sh{O}^m$, then this completely explains the composition of the global $\Ext$ by means of the cup product, taking into account the isomorphism of \cref{1.5}.


\section{Results on local cohomology}
\label{section3}

Let $A$ be a unital commutative ring endowed with an ideal $\mathfrak{J}$.
We will define, for any $A$-module $M$, functorial homomorphisms
\[
\label{3.1}
  \begin{aligned}
    \Ext_A^p(A/\mathfrak{J},M) &\to \Hom_A(\wedge^p\mathfrak{J}/\mathfrak{J}^2,M\otimes A/\mathfrak{J})
  \\\Tor_p^A(A/\mathfrak{J},M) &\from (\wedge^p\mathfrak{J}/\mathfrak{J}^2)\otimes\Hom_A(A/\mathfrak{J},M)
  \end{aligned}
\tag{3.1}
\]
where the tensor and exterior products are taken over the ring $A$;
note also that $\mathfrak{J}/\mathfrak{J}^2$ is in fact an $A/\mathfrak{J}$-module, and that its exterior powers as an $A$-module agree with its exterior powers as an $A/\mathfrak{J}$-module.
The definition of the homomorphisms in \cref{3.1} come from the definition, for every system $x=(x_1,\ldots,x_p)$ of points of $\mathfrak{J}$, of homomorphisms $\varphi_x$
\[
\label{3.2}
  \begin{aligned}
    \varphi_x\colon \Ext_A^p(A/\mathfrak{J},M) &\to M\otimes A/\mathfrak{J}
  \\\varphi_x\colon \Hom_A(A/\mathfrak{J},M) &\to \Tor_p^A(A/\mathfrak{J},M)
  \end{aligned}
\tag{3.2}
\]
such that the following conditions are satisfied:
\oldpage{149-06}
\begin{enumerate}[i.]
  \item $\varphi_{x_1,\ldots,x_p}$ depends on the system of the $x_i\in\mathfrak{J}$ in an alternating $A$-multilinear way;
  \item $\varphi_{x_1,\ldots,x_p}$ is zero when any of the $x_i$ is in $\mathfrak{J}^2$.
\end{enumerate}

In fact, ii. follows from i., since $a\varphi_x=0$ for $a\in\mathfrak{J}$, as we see by noting that all the modules in \cref{3.2} are annihilated by $\mathfrak{J}$.

To define the $\varphi_x$, we consider the complex $K_x$ whose underlying $A$-modules are the $\wedge A^p$, and whose differential is the interior product $i_x$ by $x$, considered as a linear form on $A^p$ with components $x_1,\ldots,x_p$.
The differential is of degree~$-1$, the degrees of the complex are positive, and the cohomology of this complex in dimension~$0$ is $A/(x_1A+\ldots+x_pA)$.
Since the $x_i$ are in $\mathfrak{J}$, we obtain an augmentation $K_{x,0}\to A/\mathfrak{J}$.
Thus $K_x$ is a \emph{free} augmented complex, with augmentation module $A/\mathfrak{J}$.
We thus obtain known homomorphisms
\[
  \begin{aligned}
    \Ext_A^\bullet(\HH_0(K_x),M) &\to \HH^\bullet(\Hom_A(K_x,M))
  \\\Tor_\bullet^A(\HH_0(K_x),M) &\from \HH_\bullet(K_x\otimes M)
  \end{aligned}
\]
whence, by composing with the homomorphisms to the $\Ext$ and the $\Tor$ induced by the augmentation homomorphism $\HH_0(K_x)\to A/\mathfrak{J}$, we obtain homomorphisms
\[
\label{3.3}
  \begin{aligned}
    \psi_x\colon \Ext_A^\bullet(A/\mathfrak{J},M) &\to \HH^\bullet(\Hom_A(K_x,M))
  \\\psi_x\colon \Tor_\bullet^A(A/\mathfrak{J},M) &\from \HH_\bullet(K_x\otimes M).
  \end{aligned}
\tag{3.3}
\]
But we immediately note that, in maximal dimension $p$, the cohomology of the right-hand side is $M(x_1M+\ldots+x_pM)$ (resp. the set o felements of $M$ that are annihilated by each of the $x_i$).
Since the $x_i$ are in $\mathfrak{J}$, we thus obtain homomorphisms
\[
\label{3.4}
  \begin{aligned}
    \HH^p(\Hom_A(K_x,M)) &\to M\otimes A/\mathfrak{J}
  \\\HH_p(K_x\otimes M) &\from \Hom_A(A/\mathfrak{J},M).
  \end{aligned}
\tag{3.4}
\]
By composing the homomorphisms in \cref{3.3} and \cref{3.4} we obtain the homomorphisms in \cref{3.2} that we wanted to define.
The verification of i. is tedious, but does not present any difficulties.

\oldpage{149-07}
\begin{proposition}{4}
\label{proposition4}
  Let $A$ be a commutative unital ring, and let $(x_1,\ldots,x_p)$  be a sequence of elements of $A$ such that, for $1\leq i\leq p$, the image of $x_i$ in the quotient of $A$ by the ideal generated by $(x_1,\ldots,x_{i-1})$ is not a zero divisor.
  Let $\mathfrak{J}$ be the ideal generated by the $x_i$.
  Then $\mathfrak{J}/\mathfrak{J}^2$ is a free $(A/\mathfrak{J})$-module, with basis given by the canonical images of the $x_i$;
  the complex $K_x$ is a free resolution of $A/\mathfrak{J}$;
  and, for every $A$-modules $M$, the homomorphisms in \cref{3.1} in dimension~$p$ are bijective.
  The same is true for the analogous homomorphisms defined for arbitrary degree~$i$ as long as $\mathfrak{J}\cdot M=0$.
\end{proposition}

(The essential point, from which all others follow, is the acyclicity of $K_x$, which is a well-known fact under the conditions given).

\begin{corollary}{1}
\label{proposition4corollary1}
  With $A$ and $\mathfrak{J}$ as above, suppose further that $A$ is a regular affine algebra of $\dim n$ over a perfect field $k$, and that $A/\mathfrak{J}$ is a regular affine algebra.
  Denote by $\Omega^i(A)$ and $\Omega^i(A/\mathfrak{J})$ the modules of K\"{a}hler differentials.
  Then there is a canonical isomorphism
  \[
  \label{3.5}
    \Ext_A^p(\Omega^{n-p}(A/\mathfrak{J}),\Omega^n(A)) = A/\mathfrak{J}.
  \tag{3.5}
  \]
  that is compatible with localisation.
\end{corollary}

\begin{proof}
  Indeed, $\Omega^{n-p}(A/\mathfrak{J})$ is a free $(A/\mathfrak{J})$-module of rank~$1$, and similarly $\Omega^n(A)$ is a free $A$-module of rank~$n$, and so the left-hand side is equal to
  \[
    \Ext_A^p(A/\mathfrak{J},A) \otimes \Omega^{n-p}(A/\mathfrak{J})' \otimes \Omega^n(A)
  \]
  (where the "'" notation denotes the dual $(A/\mathfrak{J})$-module).
  The tensor product of these last two factors can be identified with $\wedge^p(\mathfrak{J}/\mathfrak{J}^2)$, and so the whole thing can be identified with $\Ext_A^p(A/\mathfrak{J},\wedge^p(\mathfrak{J}/\mathfrak{J}^2))$, and thus, by the proposition, with
  \[
    \Hom_A(\wedge^p \mathfrak{J}/\mathfrak{J}^2,\wedge^p \mathfrak{J}/\mathfrak{J}^2)
  \]
  i.e. to $A/\mathfrak{J}$.
\end{proof}

In particular, there is a distinguished element in $\Ext_A^p(\Omega^{n-p}(A/\mathfrak{J}),\Omega^n(A))$, corresponding to the unit of $A/\mathfrak{J}$, called the \emph{fundamental class} of the ideal $\mathfrak{J}$ in $A$.
(It can in fact be defined under rather more general conditions).
We can write \cref{proposition4corollary1} in a more geometric and more global form:

\begin{corollary}{2}
\label{proposition4corollary2}
  Let $X$ be a non-singular variety over an algebraically-closed field $k$, $Y$ a closed non-singular subvariety of $X$, $\sh{O}_X$ the structure sheaf of $X$, and $\sh{O}_Y$ the structure sheaf of $Y$, considered as a quotient sheaf of $\sh{O}_X$.
  Let $n$ be the dimension of $X$, and $n-p$ the dimension of $Y$.
\oldpage{149-08}
  Let $\Omega_X$ (resp. $\Omega_Y$) be the sheaf of germs of regular differential forms on $X$ (resp. $Y$).
  Then there are canonical isomorphisms
  \[
  \label{3.6}
    \shExt_{\sh{O}_X}^p(\Omega_Y^{n-p},\Omega_X^n) = \sh{O}_Y
  \tag{3.6}
  \]
  as well as
  \[
  \label{3.6bis}
    \Ext_{\sh{O}_X}^p(\sh{O}_Y,\Omega_X^n) = \Omega_Y^{n-p}.
  \tag{3.6 \emph{bis}}
  \]
\end{corollary}

Equation~\cref{3.6bis} can serve as the \emph{definition} of $\Omega_Y^{n-p}$ when $Y$ is a singular variety.
More precisely:

\begin{proposition}{5}
\label{proposition5}
  Let $Y$ be an algebraic subset of dimension $q=n-p$ of a non-singular algebraic variety $X$ of dimension~$n$.
  Let $\sh{F}$ be a coherent algebraic sheaf on $X$ with support contained in $Y$, and let $\sh{L}$ be a locally-free algebraic sheaf on $X$.
  Then the sheaves $\shExt_{\sh{O}_X}^i(\sh{F},\sh{L})$ are zero for $i<p$, and when $i=p$ there is a canonical isomorphism
  \[
  \label{3.7}
    \shExt_{\sh{O}_X}^p(\sh{F},\sh{L}) = \shHom_{\sh{O}_X}(\sh{F},\shExt^p(\sh{O}_X/\mathfrak{J},\sh{L}))
  \tag{3.7}
  \]
  where $\mathfrak{J}$ denotes an arbitrary sheaf of ideals on $X$ that annihilates $\sh{F}$ and has $Y$ as its set of zeros.
  In particular, if $\sh{F}$ is a coherent algebraic sheaf on $Y$, then
  \[
  \label{3.7bis}
    \shExt_{\sh{O}_X}^p(\sh{F},\sh{L}) = \shHom_{\sh{O}_Y}(\sh{F},\shExt^p(\sh{O}_Y,\sh{L})).
  \tag{3.7 \emph{bis}}
  \]
  Finally, with $\sh{F}$ still a coherent algebraic sheaf on $Y$, the sheaves $\sh{E}^i=\shExt_{\sh{O}_X}^{p+i}(\sh{F},\Omega_X^n)$ do not depend on the choice of immersion of the algebraic space $Y$ into the non-singular algebraic variety $X$.
\end{proposition}

\begin{proof}
  Since the question is local, we can assume that $X$ is affine and $\sh{L}=\sh{O}_X$.
  This then reduces to a problem of commutative algebra, and even of local algebra:
  if $A$ is a regular \todo, and $M$ an $A$-module whose support is of dimension $\leq q=n-p$, then we have to prove that $\Ext_A^i(M,A)=0$ for $i<p$ and $\Ext_A^p(M,A)=\Hom_A(M,\Ext^p(A/\mathfrak{J},A))$, where $\mathfrak{J}$ is an arbitrary ideal of ``dimension'' $\leq q$ that annihilates $M$.
  For the first claim, we proceed by induction on $q$:
  an immediate \emph{d\'{e}vissage} leads to the case where $M$ is of the form $A/\mathfrak{J}$, and thus, by replacing $\mathfrak{J}$ by a smaller ideal and using the induction hypothesis, as well as the exact sequence of the $\Ext$, to the case where $\mathfrak{J}$ is generated by a ``system of parameters'', as in \cref{proposition4}, where the result is immediate.
  The previous result implies that, if $\mathfrak{J}$ is a
\oldpage{149-09}
  fixed ideal of ``dimension'' $\leq q$, then the contravariant functor $E(M)=\Ext_A^p(M,A)$ to the category of $(A/\mathfrak{J})$-modules is left exact;
  furthermore, it sends direct sums to direct products, from which it easily follows that $E(M)=\Hom_A(M,E(A))$.
  Finally, the last claim of \cref{proposition5} is more subtle, and follows from an intrinsic characterisation of the $E^i(F)$ via a local duality theorem that cannot be stated here.
\end{proof}

\begin{corollary*}
\label{proposition5corollary}
  Denote by $\omega_Y^q$ the sheaf $\Ext_{\sh{O}_X}^p(\sh{O}_Y,\Omega_X^n)$.
  Then there is a functorial isomorphism for coherent algebraic sheaves $\sh{F}$ on $Y$:
  \[
  \label{3.8}
    \shExt_{\sh{O}_X}^p(\sh{F},\Omega_X^n) = \shHom_{\sh{O}_X}(\sh{F},\omega_Y^q).
  \tag{3.8}
  \]
\end{corollary*}


\section{Cohomology class associated to a subvariety}
\label{section4}

In all that follows, $X$ denotes an algebraic set of dimension~$n$, defined over a field $k$ that we assume, for simplicity, to be algebraically closed.
Except in \cref{section6}, $X$ is assumed to be non-singular.
We denote by $\sh{O}_X$ the structure sheaf of $X$, and by $\Omega_X^\bullet=\bigcup_p\Omega_X^p$ the sheaf of germs of differential forms on $X$.
If $Y$ is a closed subset of $X$, then we identify coherent algebraic sheaves on $Y$ with coherent algebraic sheaves on $X$ that are zero outside of $Y$;
we do this, in particular, with $\sh{O}_Y$ and $\Omega_Y$.

\begin{lemma}{1}
\label{lemma1}
  Let $\sh{F}$ be a coherent algebraic sheaf on $X$ whose support is of dimension~$\leq n-p$, and let $\sh{L}$ be a coherent algebraic sheaf on $X$ that is locally free.
  Then $\shExt_{\sh{O}_X}^i(X;\sh{F},\sh{L})$ is zero for $i<p$, and there is a canonical isomorphism
  \[
  \label{4.1}
    \Ext_\sh{O}^p(X;\sh{F},\sh{L}) = \HH^0(X,\shExt_\sh{O}^p(\sh{F},\sh{L})).
  \tag{4.1}
  \]
  If $\sh{F}$ is a coherent algebraic sheaf on a closed subset $W$ of $X$ of dimension~$\leq n-p$, then there is a canonical isomorphism
  \[
  \label{4.1bis}
    \shExt_{\sh{O}_X}^p(\sh{F},\sh{L}) = \shHom_{\sh{O}_X}(\sh{F}\otimes\sh{L}'\otimes\Omega_X^n,\omega_Y^{n-p})
  \tag{4.1 \emph{bis}}
  \]
  where $\omega_Y^{n-p}$ is the sheaf on $Y$ defined in \hyperref[proposition5corollary]{the corollary} to \cref{proposition5} (which can be identified with $\Omega_Y^{n-p}$ if $Y$ is non-singular).
\end{lemma}

\begin{proof}
  The formula in \cref{4.1} is an immediate consequence of the spectral sequence from \cref{proposition1}, as well as \cref{proposition5};
  by the formula in \cref{3.8}, we can write
\oldpage{149-10}
  \[
    \begin{gathered}
      \shExt_{\sh{O}_X}^p(\sh{F},\sh{L})
      = \sh{L}\otimes(\Omega_X^n)'\otimes\shExt_{\sh{O}_X}(\sh{F},\Omega_X^n)
    \\= \sh{L}\otimes(\Omega_X^n)'\otimes\shHom_{\sh{O}_X}(\sh{F},\omega_Y^q)
    \\= \shHom_{\sh{O}_X}(\sh{F}\otimes\sh{L}'\otimes\Omega_X^n,\omega_Y^q)
    \end{gathered}
  \]
  where $q=n-p$, whence the formula in \cref{4.1bis}.
\end{proof}

Setting, in particular, $\sh{F}=\sh{O}_Y$ and $\sh{L}=\Omega_X^p$, we find, taking into account the fact that $\Omega_X^n\otimes(\Omega_X^p)'=\Omega_X^{n-p}$, a canonical isomorphism
\[
\label{4.2}
  \Ext_{\sh{O}_X}^p(X;\sh{O}_Y,\Omega_X^p) = \Hom_{\sh{O}_X}(X;\Omega_X^{n-p},\omega_Y^{n-p}).
\tag{4.2}
\]
Now suppose, for simplicity, that $Y$ is \emph{non-singular}, so that $\omega_Y^{n-p}=\Omega_Y^{n-p}$.
There is a natural homomorphism from $\Omega_X^{n-p}$ to $\Omega_Y^{n-p}$, whence a canonical section $s_Y$ of the sheaf $\shExt_{\sh{O}_X}^p(\sh{O}_Y,\Omega_X^p)$, that we call, if all the components of $Y$ are of dimension~$n-p$, the \emph{fundamental section} of the sheaf $\shExt_{\sh{O}_X}^p(\sh{O}_Y,\Omega_X^p)$.
By \cref{4.1}, this section defines an element of $\Ext_{\sh{O}_X}^p(X;\sh{O}_Y,\Omega_X^p)$.
But the natural homomorphism $\sh{O}_X\to\sh{O}_Y$ defines a homomorphism
\[
  \Ext_{\sh{O}_X}^p(X;\sh{O}_Y,\Omega_X^p) \to \Ext_{\sh{O}_X}^p(X;\sh{O}_X,\Omega_X^p) = \HH^p(X,\Omega_X^p).
\]
We thus obtain an element of $\HH^p(X,\Omega_X^p)$, denoted by $P_X(Y)$, that we call the \emph{cohomology class of $Y$ in $X$}.



%% Bibliography %%

\nocite{*}
\bibliographystyle{acm}
\begin{thebibliography}{4}

  \bibitem{1}
  {\sc Cartier, P.}
  \newblock Des groups $\Ext^s(A,B)$.
  \newblock {\em S\'{e}minaire A.~Grothendieck: Alg\'{e}bre homologique}, Volume~1 (1957), Talk no.~3.

  \bibitem{2}
  {\sc Grothendieck, A.}
  \newblock Sur quelques points d'alg\'{e}bre homologique.
  \newblock {\em Tohoku math. J.} {\bf 9} (1957), pp.~119--183.

  \bibitem{3}
  {\sc Serre, J.-P.}
  \newblock Faisceux alg\'{e}briques coh\'{e}rents.
  \newblock {\em Annals of Math.} {\bf 61} (1955), pp.197--278.

  \bibitem{4}
  {\sc Serre, J.-P.}
  \newblock Sur la dimension homologique des anneaux et des modules noeth\'{e}riens.
  \newblock {\em Proc. Intern. Symp. on alg. number Theory [1955, Tokyo et Nikko]}.
  \newblock Tokyo, Science Council of Japan (1956), pp.~175--189.

\end{thebibliography}

\end{document}

\documentclass{article}

\title{Action of a torus in a projective variety}
\author{Michel BRION and Claudio PROCESI}
\date{}

\usepackage{amssymb,amsmath}

\usepackage{hyperref}
\usepackage[nameinlink]{cleveref}
\usepackage{enumerate}

\usepackage{mathrsfs}
%% Fancy fonts --- feel free to remove! %%
\usepackage{Baskervaldx}
\usepackage{mathpazo}


\usepackage{fancyhdr}
\usepackage{lastpage}
\usepackage{xstring}
\makeatletter
\ifx\pdfmdfivesum\undefined
  \let\pdfmdfivesum\mdfivesum
\fi
\edef\filesum{\pdfmdfivesum file {\jobname}}
\pagestyle{fancy}
\makeatletter
\let\runauthor\@author
\let\runtitle\@title
\makeatother
\fancyhf{}
\lhead{\footnotesize\runtitle}
\rhead{\footnotesize Version: \texttt{\StrMid{\filesum}{1}{8}}}
\cfoot{\small\thepage\ of \pageref*{LastPage}}


\crefname{section}{Section}{Sections}
\crefname{equation}{}{}


%% Theorem environments %%

\usepackage{amsthm}


%% Shortcuts %%

\newcommand{\sh}{\mathscr}
\newcommand{\PP}{\mathbf{P}}
\newcommand{\QQ}{\mathbf{Q}}
\DeclareMathOperator{\Pic}{Pic}
\DeclareMathOperator{\Proj}{Proj}

\renewcommand{\geq}{\geqslant}
\renewcommand{\leq}{\leqslant}

\newcommand{\todo}{\textbf{ !TODO! }}
\newcommand{\oldpage}[1]{\marginpar{\footnotesize$\Big\vert$ \textit{p.~#1}}}


%% Document %%

\usepackage{embedall}
\begin{document}

\maketitle
\thispagestyle{fancy}

\renewcommand{\abstractname}{Translator's note.}

\begin{abstract}
  \renewcommand*{\thefootnote}{\fnsymbol{footnote}}
  \emph{This text is one of a series\footnote{\url{https://thosgood.com/translations/}} of translations of various papers into English.}
  \emph{The translator takes full responsibility for any errors introduced in the passage from one language to another, and claims no rights to any of the mathematical content herein.}
  
  \emph{What follows is a translation of the French book chapter:}

  \medskip\noindent
  \textsc{Brion, M. and Procesi, C.}
  ``Action d'un tore dans une vari\'{e}t\'{e} projective''
  in \emph{Operator algebras, unitary representations, enveloping algebras, and invariant theory (Paris, 1989)}, Birkhäuser Boston, Progress in mathematics \textbf{92} (1990), pp.~509--539.
\end{abstract}

\setcounter{footnote}{0}

\tableofcontents
\bigskip


%% Content %%

\section*{Introduction}
\label{section:introduction}

If a reductive algebraic group $G$ acts on a projective algebraic variety $X$, all over an algebraically closed field, the data of an $G$-linearised ample line bundle $L$ on $X$ allows us to define the open subset of stable points of $X$.
The quotient (in the usual sense of orbit spaces) of this open subset $X^s=X^s(L)$ by $G$ exists;
it is a quasi-projective variety, denoted by $X^s/G$.
Furthermore, $X^s$ is contained in the open subset $X^{ss}=X^{ss}(L)$ of semi-stable points, and we can again define a ``quotient'' $Y$ of $X^{ss}$ by $G$ (it is the space of closed orbits of $G$ in $X^{ss}$).
The variety $Y$ is projective, and contains $X^s/G$ as an open sub set.
To define it, we introduce the algebra $A:=\oplus_{n=0}^\infty \Gamma(X,L^n)$ and its sub-algebra $A^G$ consisting of invariants of $G$;
then $Y$ is the $\Proj$ of the graded algebra $A^G$.
Thus $Y$ is endowed with a sheaf $\sh{O}(n)$ for every integer $n$, and one of them is invertible.
So we can define the class $1/n[\sh{O}(n)]$ in the Picard group of $Y$, tensored with $\mathbb{Q}$.
This class is ample, and depends only on $X$ and $L$.

Our goal is to study these objects, introduced by Mumford in \cite{MF}, when $G$ is a torus (denoted by $T$).
The idea is to simultaneously consider all the quotients associated to the $T$-linearised bundles $L^n\otimes\sh{O}(\chi)$, where $n$ is a positive integer, and $\sh{O}(\chi)$ is the trivial line bundle on $X$, with the torus acting on each fibre of $\sh{O}(\chi)$ by multiplication by the character $\chi$.
The notion of a stable or semi-stable point for $L^n\otimes\sh{O}(\chi)$ depends only, in fact, on $\chi/n$.
\oldpage{510}
To every quotient $p$ of a character of $T$ by an integer, we can thus associated $X^s(p)$, $X^{ss}(p)$, $Y(p)$, and an ample class $L_p$ in $\Pic_\mathbb{Q}(Y(p))$;
when $p=0$, we recover the above notions.

For simplicity, we assume $L$ to be very ample, that is, we consider the case where $X$ is a subvariety of a projective space $\PP(V)$, with the torus $T$ acting linearly on $V$, and where $L$ is the restriction of $\sh{O}(1)$ to $X$.
We will show that $X^s(p)$, $X^{ss}(p)$, and $Y(p)$ depend only on the position of $p$ relative to a certain finite set $\Pi$ of characters of $T$ (the set of weights of $T$ in $V$).
More precisely, we will define \cref{1.1,1.2} a partition of the convex hull $\mathcal{C}$ of $\Pi$ into ``faces''.
Each face is the interior of a convex polyhedron, and $X^s(p)$, $X^{ss}(p)$, and $Y(p)$ depend only on the face of $p$;
for every face $F$, we can define $X^s(F)$, $X^{ss}(F)$, and $Y(F)$.
We will further show \cref{1.3} that the map $p\in F\mapsto L_p\in\Pic_\QQ(Y(F))$ is affine.

Let $F$ be an open face of $\mathcal{C}$.
We will show that every point that is semistable for $F$ is stable;
thus, if $X$ is smooth, then $Y(F)$ only has singularities from quotients by finite abelian groups.
The quotients associated to the open faces are thus relatively simply.
For an arbitrary face $F$, let $F'$ be an open face whose closure contains $F$.


%% Bibliography %%

\nocite{*}
\bibliographystyle{acm}

\end{document}

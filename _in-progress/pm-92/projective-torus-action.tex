\documentclass{article}

\title{Action of a torus in a projective variety}
\author{Michel BRION and Claudio PROCESI}
\date{}

\usepackage{amssymb,amsmath}

\usepackage{hyperref}
\usepackage[nameinlink]{cleveref}
\usepackage{enumerate}

\usepackage{mathrsfs}
%% Fancy fonts --- feel free to remove! %%
\usepackage{Baskervaldx}
\usepackage{mathpazo}


\usepackage{fancyhdr}
\usepackage{lastpage}
\usepackage{xstring}
\makeatletter
\ifx\pdfmdfivesum\undefined
  \let\pdfmdfivesum\mdfivesum
\fi
\edef\filesum{\pdfmdfivesum file {\jobname}}
\pagestyle{fancy}
\makeatletter
\let\runauthor\@author
\let\runtitle\@title
\makeatother
\fancyhf{}
\lhead{\footnotesize\runtitle}
\rhead{\footnotesize Version: \texttt{\StrMid{\filesum}{1}{8}}}
\cfoot{\small\thepage\ of \pageref*{LastPage}}


\crefname{section}{Section}{Sections}
\crefname{equation}{}{}


%% Theorem environments %%

\usepackage{amsthm}


%% Shortcuts %%

\newcommand{\sh}{\mathscr}
\newcommand{\PP}{\mathbf{P}}
\newcommand{\QQ}{\mathbf{Q}}
\newcommand{\NN}{\mathbf{N}}
\newcommand{\ZZ}{\mathbf{Z}}
\newcommand{\RR}{\mathbf{R}}
\DeclareMathOperator{\Pic}{Pic}
\DeclareMathOperator{\Proj}{Proj}

\renewcommand{\geq}{\geqslant}
\renewcommand{\leq}{\leqslant}

\newcommand{\todo}{\textbf{ !TODO! }}
\newcommand{\unsure}[1]{\underline{\textbf{TODO: #1}}}

\newcommand{\oldpage}[1]{\marginpar{\footnotesize$\Big\vert$ \textit{p.~#1}}}


%% Document %%

\usepackage{embedall}
\begin{document}

\maketitle
\thispagestyle{fancy}

\renewcommand{\abstractname}{Translator's note.}

\begin{abstract}
  \renewcommand*{\thefootnote}{\fnsymbol{footnote}}
  \emph{This text is one of a series\footnote{\url{https://thosgood.com/translations/}} of translations of various papers into English.}
  \emph{The translator takes full responsibility for any errors introduced in the passage from one language to another, and claims no rights to any of the mathematical content herein.}
  
  \emph{What follows is a translation of the French book chapter:}

  \medskip\noindent
  \textsc{Brion, M. and Procesi, C.}
  ``Action d'un tore dans une vari\'{e}t\'{e} projective''
  in \emph{Operator algebras, unitary representations, enveloping algebras, and invariant theory (Paris, 1989)}, Birkh\"{a}user Boston, Progress in mathematics \textbf{92} (1990), pp.~509--539.
\end{abstract}

\setcounter{footnote}{0}

\tableofcontents
\bigskip


%% Content %%

\section*{Introduction}
\label{introduction}

If a reductive algebraic group $G$ acts on a projective algebraic variety $X$, all over an algebraically closed field, the data of an $G$-linearised ample line bundle $L$ on $X$ allows us to define the open subset of stable points of $X$.
The quotient (in the usual sense of orbit spaces) of this open subset $X^s=X^s(L)$ by $G$ exists;
it is a quasi-projective variety, denoted by $X^s/G$.
Furthermore, $X^s$ is contained in the open subset $X^{ss}=X^{ss}(L)$ of semi-stable points, and we can again define a ``quotient'' $Y$ of $X^{ss}$ by $G$ (it is the space of closed orbits of $G$ in $X^{ss}$).
The variety $Y$ is projective, and contains $X^s/G$ as an open sub set.
To define it, we introduce the algebra $A:=\oplus_{n=0}^\infty \Gamma(X,L^n)$ and its sub-algebra $A^G$ consisting of invariants of $G$;
then $Y$ is the $\Proj$ of the graded algebra $A^G$.
Thus $Y$ is endowed with a sheaf $\sh{O}(n)$ for every integer $n$, and one of them is invertible.
So we can define the class $1/n[\sh{O}(n)]$ in the Picard group of $Y$, tensored with $\mathbb{Q}$.
This class is ample, and depends only on $X$ and $L$.

Our goal is to study these objects, introduced by Mumford in \cite{MF}, when $G$ is a torus (denoted by $T$).
The idea is to simultaneously consider all the quotients associated to the $T$-linearised bundles $L^n\otimes\sh{O}(\chi)$, where $n$ is a positive integer, and $\sh{O}(\chi)$ is the trivial line bundle on $X$, with the torus acting on each fibre of $\sh{O}(\chi)$ by multiplication by the character $\chi$.
The notion of a stable or semi-stable point for $L^n\otimes\sh{O}(\chi)$ depends only, in fact, on $\chi/n$.
\oldpage{510}
To every quotient $p$ of a character of $T$ by an integer, we can thus associated $X^s(p)$, $X^{ss}(p)$, $Y(p)$, and an ample class $L_p$ in $\Pic_\mathbb{Q}(Y(p))$;
when $p=0$, we recover the above notions.

For simplicity, we assume $L$ to be very ample, i.e. we consider the case where $X$ is a subvariety of a projective space $\PP(V)$, with the torus $T$ acting linearly on $V$, and where $L$ is the restriction of $\sh{O}(1)$ to $X$.
We show that $X^s(p)$, $X^{ss}(p)$, and $Y(p)$ depend only on the position of $p$ relative to a certain finite set $\Pi$ of characters of $T$ (the set of weights of $T$ in $V$).
More precisely, we define \cref{1.1,1.2} a partition of the convex hull $\mathcal{C}$ of $\Pi$ into ``faces''.
Each face is the interior of a convex polyhedron, and $X^s(p)$, $X^{ss}(p)$, and $Y(p)$ depend only on the face of $p$;
for every face $F$, we can define $X^s(F)$, $X^{ss}(F)$, and $Y(F)$.
We further show \cref{1.3} that the map $p\in F\mapsto L_p\in\Pic_\QQ(Y(F))$ is affine.

Let $F$ be an open face of $\mathcal{C}$.
We show that every point that is semistable for $F$ is stable;
thus, if $X$ is smooth, then $Y(F)$ only has singularities from quotients by finite abelian groups.
The quotients associated to the open faces are thus relatively simply.
For an arbitrary face $F$, let $F'$ be an open face whose closure contains $F$.
We define, in \cref{1.4}, a morphism $\pi_{F,F'}\colon Y(F')\to Y(F)$, that is almost always birational.
In \cref{1.5}, we study the case where $F$ is of codimension one, and contained in the closure of the two open faces $F_-$ and $F_+$.
The morphisms $\pi_{F,F_-}$ and $\pi_{F,F_+}$ are, in general, quite complicated;
however, if $\hat{Y}$ denotes the product of $Y(F_-)$ and $Y(F_+)$ over $Y(F)$, then the morphisms from $\hat{Y}$ to $Y(F_-)$, $Y(F)$, and $Y(F_+)$ are blow-ups of subvarieties, with the same exceptional divisor (see \cref{2.3} for a precise statement).

We can extend the affine map $p\mapsto L_p$ from $F_+$ to $\Pic_\QQ(Y_+)$ to $\mathcal{C}$, and then lift each $L_p$ to $\hat{Y}$, giving a class $L_p^+$;
we define $L_p^-$ similarly.
We show in \cref{2.3} that the difference $L_p^+ - L_p^-$ is a multiple of the exceptional divisor $\hat{Y}\to Y(F)$, and is zero for all $p\in F$.
Then, when we consider the product $\bar{Y}$ of the quotients associated to all the faces, over all the morphisms $\pi_{F,F'}$, the maps $p\mapsto L_p$ glue to give a map, piecewise affine \unsure{???} and continuous, from $\mathcal{C}$ to $\Pic_\QQ(\overline{Y})$.

In the third section of this paper, we apply these results to the asymptotic study of multigraded modules;
we consider various extensions of the notions of multiplicity and of the Hilbert--Samuel function for a graded module to these modules.
More precisely, we consider a polynomial algebra $A$, graded by $\NN\times\ZZ^l$ (with the $\NN$-grading being defined by the degree of the polynomials).
We denote by $T$ the torus having $\ZZ^l$ as its group of character.
Every $(\NN\times\ZZ^l)$-graded $A$-module defines
\oldpage{511}
a $T$-linearised sheaf $\sh{M}$ on the projective space $\PP(V)$ associated to $A$.
Let $\Pi$ be the set of weights of $T$ in $V$, i.e. \unsure{opposites? duals? of $\ZZ^l$-degrees?} of generators of $A$.
For every face $F$ associated to $\Pi$, let $\sh{M}^{(F)}$ be the sheaf given by invariants of $T$ in $(\pi_F)_*(\sh{M})$, where $\pi_F\colon X^{ss}(F)\to Y(F)$ is the ``quotient''.
We show that, if $M$ is an $A$-module of finite type, then $\sh{M}^{(F)}$ is a coherent sheaf on $Y(F)$.
For every $p\in F$, we denote by $\mu_M(p)$ the degree of $\sh{M}^{(F)}$ relative to the ample class $L_p$ (see \cite[I.3]{Kle}).
We show that the function $\mu_M$ is polynomial on each face.
Furthermore, it appears as the density in various integral expressions of asymptotic invariants of $M$: the ``generalised Hilbert--Samuel polynomials'' \cref{3.5}, and the ``Joseph polynomials'' (\cite{Jos}; see \cref{3.6}).
If $M$ is the homogeneous coordinate ring of a smooth projective subvariety $X$ of $\PP(V)$, then the function $\mu_M$ is continuous, and its Fourier transform depends only on fixed points of $T$ in $X$, and on their normal bundle \cref{3.4}.

Finally, in an appendix, we briefly explain the links between our results and some theorems due to Atiyah, Duistermaat--Heckman, Guillemin--Sternberg in symplectic geometry (see \cite{Ati}, \cite{DH12}, and \cite{GS12}).
Indeed, there are close links, studied in \cite{Kir}, between the algebraic action of a complex torus on a subvariety $X$ of a projective space and the Hamiltonian action of its compact maximal sub-torus $T_c$ on $X$ considered as a symplectic variety (the symplectic form being the imaginary part of a K\"{a}hler form on $\PP(V)$ that is invariant under $T_c$).
The objects of our study can be understood in terms of the moment map $J$:
its image can be identified with the convex polyhedron $\mathcal{C}$;
the open faces consist of regular values of $J$;
the quotient $Y(p)$ is ``the \unsure{reduced phase space}'' $J^{-1}(p)/T_c$;
the class of $L_p$ in $H^2(Y(p),\QQ)$ comes from the cohomology class of the symplectic form.
Furthermore, if $M$ is the homogeneous coordinate ring of $X$, then the function $\mu_M$ is the density of the measure given by the image of the moment map.

After having obtained the results of this paper, we became aware (in May 1989) of the article \cite{GS3} of V.~Guillemin and S.~Sternberg, which proves results that are related to ours found in \cref{2}, concerning Hamiltonian actions of compact toruses on symplectic varieties.


\section{Quotients by a torus}
\label{1}

\subsection{Notation}
\label{1.1}

Let $T$ be a torus acting linearly on a vector space $V$ (with the base field $k$ being algebraically closed).
Let $X$ be a closed irreducible subvariety of the projective space $\PP(V)$, stable under the action of $T$.
For simplicity, we suppose
\oldpage{512}
that $T$ acts faithfully on $V$, and that it does not contain the homotheties, and further that $X$ is not contained in any linear subspace of $\PP(V)$.

We denote by $\mathfrak{X}(T)$ the group of characters of $T$, and by $E$ (resp. $E_\QQ$) the vector space $\mathfrak{X}(T)\otimes_\ZZ\RR$ (resp. $\mathfrak{X}(T)\otimes_\ZZ\QQ$).
Let $V=\oplus_{\chi\in\mathfrak{X}(T)}V_\chi$ be the decomposition of $V$ into proper subspaces of $T$.
We denote by $\Pi$ the set of $\chi$ such that $V_\chi\neq\{0\}$, i.e. the set of weights of $T$ in $V$.
For all $x\in\PP(V)$, we denote by $\bar{x}$ a representative of $x$ in $V$, and by $\bar{x}=\sum v_\chi$ its decomposition into eigenvectors of $T$.
We denote by $\Pi(x)$ the set of $\chi\in\Pi$ such that $v_\chi\neq=0$;
we call it the set of weights of $x$.

For all $p\in E$, let $\overline{F}(p)$ be the intersection of all the simplices that both contain $p$ and have all their vertices in $\Pi$.
This is a convex polyhedron in $E$, with vertices in $E_\QQ$.
We denote by $F(p)$ its relative interior, i.e. the interior of $\overline{F}(p)$ in the affine space that it generates;
we call $F(p)$ the face of $p$.
By the Carath\'{e}odory theorem \cite[Theorem~1.21]{Val}, every point of the convex hull $\mathcal{C}$ of $\Pi$ belongs to a simplex that has all its vertices in $\Pi$.
Thus the $F(p)_{p\in E}$ form a partition of $\mathcal{C}$.
Furthermore, every face is contained in the closure of an open face.
From this it follows that every face of codimension one is contained in the closure of at most two open faces.
Every affine hyperplane generated by a face of codimension one is also generated by $l$ points of $\Pi$, where $l$ is the dimension of $E$.


\subsection{Faces and quotients}
\label{1.2}

For all $p\in E$, we denote by $X^{ss}(p)$ (resp. $X^s(p)$) the set of $x\in X$ such that the convex hull of $\Pi(x)$ contains $p$ (resp. contains $p$ in its interior).
It is clear that these two sets depend only on the face of $p$;
for every face $F$, we can thus define $X^{ss}(F)$ and $X^s(F)$.
The following claim is evident.


%% Bibliography %%

\nocite{*}
\bibliographystyle{acm}

\end{document}

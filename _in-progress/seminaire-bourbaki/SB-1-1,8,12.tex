\documentclass{article}

\title{The work of Koszul}
\author{Henri Cartan}
\date{December 1948, March 1949, and May 1949}

\usepackage{amssymb,amsmath}

\usepackage{hyperref}
\usepackage{xcolor}
\hypersetup{colorlinks,linkcolor={red!50!black},citecolor={blue!50!black},urlcolor={blue!80!black}}
\usepackage[nameinlink]{cleveref}
\usepackage{enumerate}

\usepackage{mathrsfs}
%% Fancy fonts --- feel free to remove! %%
\usepackage{Baskervaldx}
\usepackage{mathpazo}


\usepackage{fancyhdr}
\usepackage{lastpage}
\usepackage{xstring}
\makeatletter
\ifx\pdfmdfivesum\undefined
  \let\pdfmdfivesum\mdfivesum
\fi
\edef\filesum{\pdfmdfivesum file {\jobname}}
\pagestyle{fancy}
\makeatletter
\let\runauthor\@author
\let\runtitle\@title
\makeatother
\fancyhf{}
\lhead{\footnotesize\runtitle}
\rhead{\footnotesize Version: \texttt{\StrMid{\filesum}{1}{8}}}
\cfoot{\small\thepage\ of \pageref*{LastPage}}


\crefname{section}{\S\!}{\S\S\!}
\crefname{equation}{}{}


%% Theorem environments %%

\usepackage{amsthm}


%% Shortcuts %%

\newcommand{\llp}{\mathbin{\llcorner}}
\newcommand{\lrp}{\mathbin{\lrcorner}}
\newcommand{\dd}{\mathrm{d}}
\newcommand{\RR}{\mathbb{R}}

\renewcommand{\geq}{\geqslant}
\renewcommand{\leq}{\leqslant}

\newcommand{\todo}{\textbf{ !TODO! }}
\newcommand{\oldpage}[1]{\marginpar{\footnotesize$\Big\vert$ \textit{p.~#1}}}


%% Document %%

\usepackage{embedall}
\begin{document}

\maketitle
\thispagestyle{fancy}

\renewcommand{\abstractname}{Translator's note.}

\begin{abstract}
  \renewcommand*{\thefootnote}{\fnsymbol{footnote}}
  \emph{This text is one of a series\footnote{\url{https://thosgood.com/translations/}} of translations of various papers into English.}
  \emph{The translator takes full responsibility for any errors introduced in the passage from one language to another, and claims no rights to any of the mathematical content herein.}
  
  \emph{What follows is a translation of the French seminar talks:}

  \medskip\noindent
  \textsc{Cartan, H.}
  ``Les travaux de Koszul, I''.
  \emph{S\'{e}minaire Bourbaki}, Volume~\textbf{1} (1952), Talk no.~1, 7--12.
  {\url{http://www.numdam.org/book-part/SB_1948-1951__1__7_0/}}

  \medskip\noindent
  \textsc{Cartan, H.}
  ``Les travaux de Koszul, II''.
  \emph{S\'{e}minaire Bourbaki}, Volume~\textbf{1} (1952), Talk no.~8, 45--52.
  {\url{http://www.numdam.org/book-part/SB_1948-1951__1__7_0/}}

  \medskip\noindent
  \textsc{Cartan, H.}
  ``Les travaux de Koszul, III''.
  \emph{S\'{e}minaire Bourbaki}, Volume~\textbf{1} (1952), Talk no.~12, 71--74.
  {\url{http://www.numdam.org/book-part/SB_1948-1951__1__7_0/}}
\end{abstract}

\setcounter{footnote}{0}

\tableofcontents
\bigskip


%% Content %%

\part{}

\oldpage{7}
The homology and cohomology of a \emph{compact} Lie group can be directly studied via the Lie algebra of the group (cf. {\sc Chevalley and Eilenberg}, Cohomology theory of Lie groups and Lie algebras, \emph{Trans. Amer. Math. Soc}~\textbf{63} (1948), 85--124).
Proceeding like so, we can study Lie algebras over an arbitrary field (most often of characteristic zero);
the compactness hypotheses are then replaced with \emph{semi-simplicity} hypotheses.
The current conference aims to explain certain algebraic tools that are useful for this study, and to prove the first results obtained (notably the theorem on the third Betti number).
In a later conference, we will introduce notions concerning a sub-algebra of a Lie algebra, and its corresponding ``homogeneous space''.


\section{General notions}
\label{I.1}

Denoting an operator on a set $A$ (that is, a transformation from $A$ to $A$) by an arbitrary letter, such as $T$, we write $T\cdot x$ to mean the transformation of $x\in A$ under $T$, and $TU$ to mean the composition of operators $T$ and $U$, so that $TU\cdot x = T\cdot (U\cdot x)$.


\section{Notions concerning the exterior algebra}
\label{I.2}

{(cf. {\sc Bourbaki}, \emph{Alg\`{e}bre}, Chap.~III).}

Let $E$ be a vector space over a commutative field $K$ (or, equivalently, a unital module over a commutative ring).
Let $\Lambda(E)$ be the exterior algebra of $E$, given by the direct sum of the $\Lambda^p(E)$ (with $\Lambda^0(E)=K$ and $\Lambda^1(E)=E$).
We use lowercase \emph{Latin} letters to denote elements of $E$, and \emph{Greek} for those of $\Lambda(E)$.
We write $a\wedge b$ for the product;
in particular, for a product of elements of degree~$1$, we write $x_1\wedge x_2\wedge\ldots\wedge x_p$.

The bilinear form that defines the duality between $E$ and its \emph{dual} $E'$ is denoted by $\langle x,x'\rangle$.
The (\todo) product of some $\alpha\in\Lambda^p(E)$ with some $x'\in E$ is an element of $\Lambda^{p-1}(E)$, denoted by $a\llp x'$, and defined by
\[
  (x_1\wedge x_2\wedge\ldots\wedge x_p) \llp x'
  = \sum_{i=1}^p (-1)^{i+1} \langle x_i,x' \rangle x_1\wedge\ldots\wedge\widehat{x_i}\wedge\ldots\wedge x_p
\]
(where the hat $\;\widehat{\,}\;$ over $x_i$ indicates that the $x_i$ term should be omitted).
The operator on $\Lambda(E)$ given by $\alpha\mapsto\alpha\llp x'$ is denoted by $i(x')$;
we have $i(x')i(x')=0$.
We define $i(\alpha')$ for $\alpha'\in\Lambda(E)'$ by
\[
  i(x'_1\wedge\ldots\wedge x'_p)
  = i(x'_p)\ldots i(x'_1).
\]
We write
\oldpage{8}
$\alpha\llp\alpha'$ to mean $i(\alpha')\cdot\alpha$.
We similarly define $i(\alpha)$, which acts on $\Lambda(E')$;
we denote $i(\alpha)\cdot\alpha'$ by $\alpha\lrp\alpha'$.
We have $i(\alpha\wedge\beta)=i(\beta)i(\alpha)$.
The scalar components of $\alpha\llp\alpha'$ and of $\alpha\lrp\alpha'$ are equal;
we denote this scalar component by $\langle\alpha,\alpha'\rangle$: this ``scalar product'' extends $\langle x,x'\rangle$ and defines the duality between $\Lambda(E)$ and $\Lambda(E')$.
We have $\langle\alpha,\alpha'\rangle=0$ if $\alpha$ and $\alpha'$ are homogeneous of different degrees, and
\[
  \langle x_1\wedge\ldots\wedge x_p,x'_1\wedge\ldots\wedge x'_p\rangle = \det(\langle x_i,x'_j\rangle).
\]

Let $e(\beta)$ be the exterior multiplication $\alpha\mapsto\beta\wedge\alpha$, and $e(\beta')$ the exterior multiplication $\alpha'\mapsto\beta'\wedge\alpha'$.
The equation
\[
  \langle \beta\wedge\alpha, \alpha' \rangle
  = \langle \alpha, \beta\lrp\alpha' \rangle
\]
tells us that $e(\beta)$ and $i(\beta)$ are \emph{transpose} to one another (the former acts on $\Lambda(E)$, and the latter on $\Lambda(E')$).
Similarly, $e(\beta')$ and $i(\beta')$ are transpose.


\section{Notions concerning endomorphisms of algebras in general}
\label{I.3}

We now study a \emph{graded} algebra $\Lambda$.
We denote by $\alpha\mapsto\overline{\alpha}$ the automorphism that sends a homogeneous element $\alpha$ of degree~$n$ to the element $(-1)^n\alpha$.
An endomorphism $\theta$ (of the vector structure) is said to be a \emph{derivation} if
\[
  \begin{aligned}
    \theta\cdot\overline{\alpha}
    &= \overline{\theta\cdot\alpha}
  \\\theta\cdot(\alpha\beta)
    &= (\theta\cdot\alpha)\beta + \alpha(\theta\cdot\beta),
  \end{aligned}
\]
and an \emph{antiderivation} if
\[
  \begin{aligned}
    \theta\cdot\overline{\alpha}
    &= -\overline{\theta\cdot\alpha}
  \\\theta\cdot(\alpha\beta)
    &= (\theta\cdot\alpha)\beta + \overline{\alpha}(\theta\cdot\beta).
  \end{aligned}
\]

If $\theta$ an antiderivation, then $\theta\theta$ is a derivation;
if $\theta_1$ and $\theta_2$ are antiderivations, then  $\theta_1\theta_2+\theta_2\theta_1$ is a derivation.

The ``bracket'' $[\theta_1,\theta_2]$ of two operators is, by definition, $\theta_1\theta_2-\theta_2\theta_1$.
The bracket of two derivations is again a derivation;
the bracket of a derivation and an antiderivation is an antiderivation.

If $\Lambda$ is generated by its degree~$0$ and degree~$1$ elements, then every derivation (resp. antiderivation) that is zero on the degree~$0$ and degree~$1$ elements is identically zero.

If $\Lambda$ is the exterior algebra $\Lambda(E)$ of a vector space $E$, then $i(x')$ (for $x'\in E'$) is an antiderivation.
If $\theta$ is a derivation of $\Lambda(E)$, then $e(x)\theta$ is an antiderivation.


\section{Notions concerning differentiable manifolds}
\label{I.4}

For simplicity, we will restrict our study to that of infinitely differentiable manifolds;
all the ``functions'' that we consider will be infinitely differentiable.

\oldpage{9}
At each point $M$ of the manifold $V$, we have a duality between the space $E(m)$ of \emph{tangent vectors} (at $M$) and the space $E'(M)$ of differentials of real-valued functions (to $\RR$) at the point $M$.
They are both $n$-dimensional vector spaces over the field $\RR$ of real numbers (where $n$ is the dimension of $V$).
A \emph{vector field} $X$ is a function that, to each point $M$ of $V$, associates a tangent vector at $M$;
vector fields form a module $E$ (over the ring of real-valued functions) whose dual $E'$ is the module of degree~$1$ differential forms.
We denote by $\langle X,\omega\rangle$ the bilinear form defining this duality.
The differential $\dd f$ of a real-valued function is a differential form (an element of $E'$).
The algebra of ``exterior differential forms'' can be identified with the exterior algebra $\Lambda(E')$ (where $E'$ is considered as a module over the ring of real-valued functions);
the operator $\dd$ (\emph{exterior differentiation}) is characterised by the following three properties:
\begin{enumerate}[1)]
  \item for a function $f$ (an element of $\Lambda^0(E')$), $\dd f$ is the differential of $f$ ;
  \item $\dd\dd=0$ ; and
  \item $\dd$ is an \emph{antiderivation}.
\end{enumerate}

Every vector field $X$ defines an \emph{infinitesimal} transformation, which we denote by $\theta(X)$, and which acts on $\Lambda(E)$ and $\Lambda(E')$.
We first define $\theta(X)$ on $\Lambda^0(E)=\Lambda^0(E')$ by setting $\theta(X)\cdot f=\langle X,\dd f\rangle$.



%% Bibliography %%

\nocite{*}

\end{document}

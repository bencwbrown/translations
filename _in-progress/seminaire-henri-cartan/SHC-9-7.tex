\documentclass{article}

\title{Singularities of differentiable maps}
\author{A. Haefliger}
\date{13\textsuperscript{th} and 24\textsuperscript{th} of May, 1957}

\usepackage{amssymb,amsmath}

\usepackage{hyperref}
\usepackage{xcolor}
\hypersetup{colorlinks,linkcolor={red!50!black},citecolor={blue!50!black},urlcolor={blue!80!black}}
\usepackage[nameinlink]{cleveref}
\usepackage{enumerate}
\usepackage{tikz-cd}

\usepackage{mathrsfs}
%% Fancy fonts --- feel free to remove! %%
\usepackage{fouriernc}


\usepackage{fancyhdr}
\usepackage{lastpage}
\usepackage{xstring}
\makeatletter
\ifx\pdfmdfivesum\undefined
  \let\pdfmdfivesum\mdfivesum
\fi
\edef\filesum{\pdfmdfivesum file {\jobname}}
\pagestyle{fancy}
\makeatletter
\let\runauthor\@author
\let\runtitle\@title
\makeatother
\fancyhf{}
\lhead{\footnotesize\runtitle}
\lfoot{\footnotesize Version: \texttt{\StrMid{\filesum}{1}{8}}}
\cfoot{\small\thepage\ of \pageref*{LastPage}}


\crefname{section}{\S\!}{\S\S\!}
\crefname{equation}{}{}


%% Theorem environments %%

\usepackage{amsthm}

\theoremstyle{plain}

  \newtheorem{innercustomtheorem}{Theorem}
  \crefname{innercustomtheorem}{Theorem}{Theorems}
  \newenvironment{theorem}[1]
    {\renewcommand\theinnercustomtheorem{#1}\innercustomtheorem}
    {\endinnercustomtheorem}

  \newtheorem{innercustomlemma}{Lemma}
  \crefname{innercustomlemma}{Lemma}{Lemmas}
  \newenvironment{lemma}[1]
    {\renewcommand\theinnercustomlemma{#1}\innercustomlemma}
    {\endinnercustomlemma}

  \newtheorem{innercustomcorollary}{Corollary}
  \crefname{innercustomcorollary}{Corollary}{Corollaries}
  \newenvironment{corollary}[1]
    {\renewcommand\theinnercustomcorollary{#1}\innercustomcorollary}
    {\endinnercustomcorollary}


\theoremstyle{definition}

  \newtheorem*{remark}{Remark}

  \newtheorem{innercustomdefinition}{Definition}
  \crefname{innercustomdefinition}{Definition}{Definitions}
  \newenvironment{definition}[1]
    {\renewcommand\theinnercustomdefinition{#1}\innercustomdefinition}
    {\endinnercustomdefinition}


%% Shortcuts %%

\newcommand{\RR}{\mathbb{R}}

\renewcommand{\geq}{\geqslant}
\renewcommand{\leq}{\leqslant}

\newcommand{\oldpage}[1]{\marginpar{\footnotesize$\Big\vert$ \textit{p.~#1}}}


%% Document %%

% \usepackage{embedall}
\begin{document}

\maketitle
\thispagestyle{fancy}

\renewcommand{\abstractname}{Translator's note.}

\begin{abstract}
  \renewcommand*{\thefootnote}{\fnsymbol{footnote}}
  \emph{This text is one of a series\footnote{\url{https://thosgood.com/translations/}} of translations of various papers into English.}
  \emph{The translator takes full responsibility for any errors introduced in the passage from one language to another, and claims no rights to any of the mathematical content herein.}

  \medskip
  
  \emph{What follows is a translation of the French seminar talk:}

  \medskip\noindent
  \textsc{Haefliger, A.}
  ``Les singularit\'{e}s des applications diff\'{e}rentiables''.
  \emph{S\'{e}minaire Henri Cartan}, Volume~\textbf{9} (1956--1957), Talk no.~7.
  {\url{http://www.numdam.org/item/SHC_1956-1957__9__A7_0/}}
\end{abstract}

\setcounter{footnote}{0}

\tableofcontents
\bigskip


%% Content %%

\oldpage{7-01}
The aim of this expos\'{e} is to reformulate, in the terminology of ``jets'' of C.~Ehresmann, the basis of the theory of singularities of differentiable maps developed by H.~Whitney and R.~Thom.
This language seems to us to be well adapted to make precise the definitions and the problems, as well as to bring out their invariant (under a change of coordinates) characteristics.
This first talk contains almost only the definitions needed to be able to state the Thom lemma and the fundamental conjecture (Whitney--Thom) of the theory.
The following talk will discuss the question of homological invariance of critical cycles.

We refer to \cite{2} and \cite{7} for examples.


\section{Recalling some definitions}
\label{section1}

We briefly recall here some notation and definitions of the theory of jets of C.~Ehresmann.
For more details, see \cite{1}.

We write ``$r$-differentiable map'' to mean a differentiable map of class~$r$.
We denote by $\RR^n$ the real space of dimension~$n$.

In the set of pointed local $r$-differentiable maps from $\RR^n$ to $\RR^p$ (that is, the set of pairs $(f,x)$ consisting of an $r$-differentiable map $f$ from an open subset $U$ of $\RR^n$ to $\RR^p$, and of a point $x\in U$), we say that the pair $(f,x)$ is equivalent to the pair $(f',x')$ if and only if $x=x'$ and all the partial derivatives of order $\leq r$ of $f$ and $f'$ at the point $x$ are equal.
The equivalence class of the pair $(f,x)$ is called the \emph{jet of order~$r$ of $f$ at the point $x$}, and is denoted by $f^r(x)$ (instead of $j_x^rf$).
The point $x$ is said to be the \emph{source} of the jet $f^r(x)$, and $f(x)$ is its \emph{target}.
If $V^n$ and $V^p$ are $r$-differentiable manifolds, then we similarly have the notion of a jet of order~$r$ at a point $x$ of $V^n$ of a local $r$-differentiable map from $V^n$ to $V^p$.

The set $J^r(\RR^n,\RR^p$ of jets of order $r$ (of $r$-differentiable maps) from $\RR^n$ to $\RR^p$ is endowed with a topological structure which makes it a manifold homeomorphic to some real space:
the coordinates of a jet $f^r(x)$ are the values of the coordinates of its source $x$ and of its target $f(x)$ and of the partial derivatives up to order~$r$ of $f$ at the point $x$.
The space of jets of order~$r$ from
\oldpage{7-02}
$\RR^n$ to $\RR^p$ with source and target being the origin of $\RR^n$ and $\RR^p$ (respectively) is denoted by $L_{n,p}^r$.
The space $J^r(\RR^n,\RR^p)$ is canonically isomorphic to the product $\RR^n\times\RR^p\times L_{n,p}^r$.

The partial composition law defined between pointed local maps from $\RR^n$ to $\RR^p$ and pointed local maps from $\RR^p$ to $\RR^m$ defines, by passing to the quotient, a partially-defined \emph{composition law} between $J^r(\RR^n,\RR^p)$ and $J^r(\RR^p,\RR^m)$ that is continuous.
The composition $YX$ of $X\in J^r(\RR^n,\RR^p)$ and $Y\in J^r(\RR^p,\RR^m)$ is defined if the source of $Y$ is the target of $X$.

The set of \emph{invertible jets of order~$r$ from $\RR^n$ to $\RR^n$} with source and target equal to the origin forms an algebraic \emph{group} denoted by $L_n^r$.
\emph{The product group $L_n^r\times L_p^r$ acts as a transformation group on $L_{n,p}^r$}: the transformation of an element $X\in L_{n,p}^r$ under the pair $(Z,Z')\in L_n^r\times L_p^r$ is the element $Z'XZ^{-1}$.

All of the above notions immediately generalise if we replace $\RR^n$ and $\RR^p$ by $r$-differentiable manifolds $V^n$ and $V^p$ (of dimension~$n$ and $p$, respectively).
The space $J^r(V^n,V^p)$ of jets of order~$r$ from $V^n$ to $V^p$ is endowed with the structure of a manifold;
the map that sends any element of $J^r(V^n,V^p)$ to the pair consisting of its source and target is the projection of a fibration with base $V^n\times V^p$, with fibre $L_{n,p}^r$, and with structure group $L_n^r\times L_p^r$.

Every $r$-differentiable map $f$ from $V^n$ to $V^p$ \emph{extends} to a map $f^r$ from $V^n$ to $J^r(V^n,V^p)$ that sends any point $x$ of $V^n$ to $f^r(x)$ (i.e. the jet of order~$r$ of $f$ at the point $x$).

A jet $X\in J^r(V^n,V^p)$ is said to be \emph{equivalent} to a jet $X'\in J^r(V'^n,V'^p)$ (written $X\sim X'$) if there exist invertible jets $Z_1\in J^r(V^n,V'^n)$ and $Z_2\in J^r(V^p,V'^p)$ such that $X'=Z_2XZ_1^{-1}$.
In particular, two jets $X$ and $X'$ belonging to $L_{n,p}^r$ are equivalent if and only if there exists an elements of the group $L_n^r\times L_p^r$ that sends $X$ to $X'$.

Finally, note that the space $J^{r+k}(V^n,V^p)$ can be identified with a sub-manifold of the space of jets of order~$k$ from $V^n$ to $J^r(V^n,V^p)$.
This is a direct consequence of the fact that the partial derivatives of order $\leq r+k$ of an $(r+k)$-differentiable map from $\RR^n$ to $\RR^p$ are the partial derivatives of order $\leq k$ of the partial derivatives of order $\leq r$ of $f$.


\section{The Thom lemma}
\label{section2}

\subsection*{The notion of a map transversal to a sub-manifold}

\oldpage{7-03}


%% Bibliography %%

\nocite{*}

\begin{thebibliography}{1}

  \bibitem{1}
  {\sc Ehresmann, C.}
  \newblock ``Introduction \`{a} la th\'{e}orie des structures infinit\'{e}simales et des pseudo-groups de Lie'', in
  \newblock {\em Colloques Internationaux du Centre National de la Recherche Scientifique \textbf{52}: G\'{e}om\'{e}trie diff\'{e}rentielle, Strasbourg 1953}.
  \newblock Paris, Centre National de la Recherche Scientifique (1953), 97--110.

  \bibitem{2}
  {\sc Thom, R.}
  \newblock Les singularit\'{e}s des applications diff\'{e}rentiables.
  \newblock {\em Ann. Inst. Fourier} \textbf{6} (1955--56), 43--87.

  \bibitem{3}
  {\sc Thom, R.}
  \newblock Un lemma sur les applications diff\'{e}rentiables.
  \newblock {\em Bol. Soc. Mat. Mexico} \textbf{2} (1956), 59--71.

  \bibitem{4}
  {\sc Whitney, H.}
  \newblock Differentiable manifolds.
  \newblock {\em Annals of Math.} \textbf{37} (1936), 645--680.

  \bibitem{5}
  {\sc Whitney, H.}
  \newblock The general type of singularity of a set of $2n-1$ smooth functions of $n$ variables.
  \newblock {\em Duke Math. J.} \textbf{10} (1943), 161--172.

  \bibitem{6}
  {\sc Whitney, H.}
  \newblock On singularities of mappings of Euclidean spaces. I. Mappings of the plane into the plane.
  \newblock {\em Annals of Math.} \textbf{62} (1955), 374--410.

  \bibitem{7}
  {\sc Whitney, H.}
  \newblock Singularities of mappings of Euclidean spaces.
  \newblock {\em To appear.}

  \bibitem{8}
  {\sc Whitney, H.}
  \newblock Elementary structure of real algebraic varieties.
  \newblock {\em To appear (in Annals of Math.)}.

\end{thebibliography}


\end{document}

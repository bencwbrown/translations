\documentclass{article}

\title{Singularities of differentiable maps}
\author{A. Haefliger}
\date{13\textsuperscript{th} and 24\textsuperscript{th} of May, 1957}

\usepackage{amssymb,amsmath}

\usepackage{hyperref}
\usepackage{xcolor}
\hypersetup{colorlinks,linkcolor={red!50!black},citecolor={blue!50!black},urlcolor={blue!80!black}}
\usepackage[nameinlink]{cleveref}
\usepackage{enumerate}
\usepackage{tikz-cd}

\usepackage{mathrsfs}
%% Fancy fonts --- feel free to remove! %%
\usepackage{fouriernc}


\usepackage{fancyhdr}
\usepackage{lastpage}
\usepackage{xstring}
\makeatletter
\ifx\pdfmdfivesum\undefined
  \let\pdfmdfivesum\mdfivesum
\fi
\edef\filesum{\pdfmdfivesum file {\jobname}}
\pagestyle{fancy}
\makeatletter
\let\runauthor\@author
\let\runtitle\@title
\makeatother
\fancyhf{}
\lhead{\footnotesize\runtitle}
\lfoot{\footnotesize Version: \texttt{\StrMid{\filesum}{1}{8}}}
\cfoot{\small\thepage\ of \pageref*{LastPage}}


\crefname{section}{\S\!}{\S\S\!}
\crefname{equation}{}{}


%% Theorem environments %%

\usepackage{amsthm}

\theoremstyle{plain}

  \newtheorem{innercustomtheorem}{Theorem}
  \crefname{innercustomtheorem}{Theorem}{Theorems}
  \newenvironment{theorem}[1]
    {\renewcommand\theinnercustomtheorem{#1}\innercustomtheorem}
    {\endinnercustomtheorem}

  \newtheorem{innercustomlemma}{Lemma}
  \crefname{innercustomlemma}{Lemma}{Lemmas}
  \newenvironment{lemma}[1]
    {\renewcommand\theinnercustomlemma{#1}\innercustomlemma}
    {\endinnercustomlemma}

  \newtheorem{innercustomcorollary}{Corollary}
  \crefname{innercustomcorollary}{Corollary}{Corollaries}
  \newenvironment{corollary}[1]
    {\renewcommand\theinnercustomcorollary{#1}\innercustomcorollary}
    {\endinnercustomcorollary}


\theoremstyle{definition}

  \newtheorem*{remark}{Remark}

  \newtheorem{innercustomdefinition}{Definition}
  \crefname{innercustomdefinition}{Definition}{Definitions}
  \newenvironment{definition}[1]
    {\renewcommand\theinnercustomdefinition{#1}\innercustomdefinition}
    {\endinnercustomdefinition}


%% Shortcuts %%

\newcommand{\sh}[1]{{\mathscr{#1}}}
\newcommand{\cat}[1]{{\mathcal{#1}}}
\usepackage{aurical}
\newcommand{\shHom}{\sh{H}\textup{\kern-2.2pt{\Fontauri\slshape om}}}
\newcommand{\shExt}{\sh{E}\textup{\kern-2.2pt{\Fontauri\slshape xt}}}
% \DeclareMathOperator{\shHom}{\underline{Hom}}
% \DeclareMathOperator{\shExt}{\underline{Ext}}
\newcommand{\HH}{\mathrm{H}}
\newcommand{\EE}{\mathrm{E}}
\newcommand{\RR}{\mathrm{R}}
\newcommand{\supp}{\operatorname{supp}}
\newcommand{\Ext}{\operatorname{Ext}}

\renewcommand{\geq}{\geqslant}
\renewcommand{\leq}{\leqslant}

\newcommand{\oldpage}[1]{\marginpar{\footnotesize$\Big\vert$ \textit{p.~#1}}}


%% Document %%

% \usepackage{embedall}
\begin{document}

\maketitle
\thispagestyle{fancy}

\renewcommand{\abstractname}{Translator's note.}

\begin{abstract}
  \renewcommand*{\thefootnote}{\fnsymbol{footnote}}
  \emph{This text is one of a series\footnote{\url{https://thosgood.com/translations/}} of translations of various papers into English.}
  \emph{The translator takes full responsibility for any errors introduced in the passage from one language to another, and claims no rights to any of the mathematical content herein.}

  \medskip
  
  \emph{What follows is a translation of the French seminar talk:}

  \medskip\noindent
  \textsc{Haefliger, A.}
  ``Les singularit\'{e}s des applications diff\'{e}rentiables''.
  \emph{S\'{e}minaire Henri Cartan}, Volume~\textbf{9} (1956--1957), Talk no.~7.
  {\url{http://www.numdam.org/item/SHC_1956-1957__9__A7_0/}}
\end{abstract}

\setcounter{footnote}{0}

\tableofcontents
\bigskip


%% Content %%

The aim of this expos\'{e} is


\section{Recalling some definitions}
\label{section1}


%% Bibliography %%

\nocite{*}

\begin{thebibliography}{1}

  \bibitem{1}
  {\sc Ehresmann, C.}
  \newblock ``Introduction \`{a} la th\'{e}orie des structures infinit\'{e}simales et des pseudo-groups de Lie'', in
  \newblock {\em Colloques Internationaux du Centre National de la Recherche Scientifique \textbf{52}: G\'{e}om\'{e}trie diff\'{e}rentielle, Strasbourg 1953}.
  \newblock Paris, Centre National de la Recherche Scientifique (1953), 97--110.

  \bibitem{2}
  {\sc Thom, R.}
  \newblock Les singularit\'{e}s des applications diff\'{e}rentiables.
  \newblock {\em Ann. Inst. Fourier} \textbf{6} (1955--56), 43--87.

  \bibitem{3}
  {\sc Thom, R.}
  \newblock Un lemma sur les applications diff\'{e}rentiables.
  \newblock {\em Bol. Soc. Mat. Mexico} \textbf{2} (1956), 59--71.

  \bibitem{4}
  {\sc Whitney, H.}
  \newblock Differentiable manifolds.
  \newblock {\em Annals of Math.} \textbf{37} (1936), 645--680.

  \bibitem{5}
  {\sc Whitney, H.}
  \newblock The general type of singularity of a set of $2n-1$ smooth functions of $n$ variables.
  \newblock {\em Duke Math. J.} \textbf{10} (1943), 161--172.

  \bibitem{6}
  {\sc Whitney, H.}
  \newblock On singularities of mappings of Euclidean spaces. I. Mappings of the plane into the plane.
  \newblock {\em Annals of Math.} \textbf{62} (1955), 374--410.

  \bibitem{7}
  {\sc Whitney, H.}
  \newblock Singularities of mappings of Euclidean spaces.
  \newblock {\em To appear.}

  \bibitem{8}
  {\sc Whitney, H.}
  \newblock Elementary structure of real algebraic varieties.
  \newblock {\em To appear (in Annals of Math.)}.

\end{thebibliography}


\end{document}

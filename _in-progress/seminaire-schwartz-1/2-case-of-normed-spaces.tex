\documentclass{article}

\title{The case of normed spaces. Tensor product of linear maps}
\author{L. Schwartz}
\date{25\textsuperscript{th} of November, 1953}

\usepackage{amssymb,amsmath}

\usepackage{hyperref}
\usepackage[nameinlink]{cleveref}
\usepackage{enumerate}
\usepackage{tikz-cd}

\usepackage{mathrsfs}
%% Fancy fonts --- feel free to remove! %%
\usepackage{Baskervaldx}
\usepackage{mathpazo}


\usepackage{fancyhdr}
\usepackage{lastpage}
\usepackage{xstring}
\makeatletter
\ifx\pdfmdfivesum\undefined
  \let\pdfmdfivesum\mdfivesum
\fi
\edef\filesum{\pdfmdfivesum file {\jobname}}
\pagestyle{fancy}
\makeatletter
\let\runauthor\@author
\let\runtitle\@title
\makeatother
\fancyhf{}
\lhead{\footnotesize\runtitle}
\rhead{\footnotesize Version: \texttt{\StrMid{\filesum}{1}{8}}}
\cfoot{\small\thepage\ of \pageref*{LastPage}}


\crefname{section}{Section}{Sections}
\crefname{equation}{}{}


%% Theorem environments %%

\usepackage{amsthm}

  \theoremstyle{plain}

  \newtheorem{innercustomproposition}{Proposition}
  \crefname{innercustomproposition}{Proposition}{Propositions}
  \newenvironment{proposition}[1]
    {\renewcommand\theinnercustomproposition{#1}\innercustomproposition}
    {\endinnercustomproposition}

  \newtheorem*{theorem}{Theorem}

  \theoremstyle{definition}

  \newtheorem*{remark}{Remark}
  \newtheorem*{definition}{Definition}


%% Shortcuts %%

\newcommand{\sh}{\mathscr}
\newcommand{\cat}{\mathcal}
\newcommand{\BB}{\mathcal{B}}
\newcommand{\LL}{\mathcal{L}}
\newcommand{\cotimes}{\widehat{\otimes}}

\renewcommand{\geq}{\geqslant}
\renewcommand{\leq}{\leqslant}

\newcommand{\todo}{\textbf{ !TODO! }}
\newcommand{\oldpage}[1]{\marginpar{\footnotesize$\Big\vert$ \textit{p.~#1}}}


%% Document %%

\usepackage{embedall}
\begin{document}

\maketitle
\thispagestyle{fancy}

\renewcommand{\abstractname}{Translator's note.}

\begin{abstract}
  \renewcommand*{\thefootnote}{\fnsymbol{footnote}}
  \emph{This text is one of a series\footnote{\url{https://thosgood.com/translations/}} of translations of various papers into English.}
  \emph{The translator takes full responsibility for any errors introduced in the passage from one language to another, and claims no rights to any of the mathematical content herein.}
  
  \emph{What follows is a translation of the French paper:}

  \medskip\noindent
  \textsc{Schwartz, L.}
  ``Cas des espaces norm\'{e}s. Produit tensoriel d'applications lin\'{e}aires''.
  \emph{S\'{e}minaire Schwartz}, Volume~\textbf{1} (1953-1954), Talk no.~2, pp.~3--7.
  {\footnotesize\url{http://www.numdam.org/item/SLS_1953-1954__1__A3_0/}}
\end{abstract}

\setcounter{footnote}{0}


%% Content %%

\oldpage{3}
\begin{proposition}{1}
  If $U$ (resp. $V$) is a closed balanced convex neighbourhood of $0$ in $E$ (resp. in $F$), of gauge $p$ (resp. $q$), then the gauge of $\Gamma(U\otimes V)$ is given by
  \[
  \label{equation1}
    p\otimes q(u) = \inf\sum_\nu p(\xi_\nu)q(\eta_\nu)
    \quad\text{for}\quad
    u=\sum_\nu\xi_\nu\otimes\eta_\nu.
  \tag{1}
  \]
  If $p$ (resp. $q$) runs over a fundamental system of continuous seminorms of $E$ (resp. $F$), then the $p\otimes q$ form a fundamental system of continuous seminorms on $E\otimes F$.
  Then
  \[
  \label{equation2}
    p\otimes q(\xi\otimes\eta) = p(\xi)q(\eta).
  \tag{2}
  \]
\end{proposition}

\begin{proof}
\oldpage{4}
  We will show that $\Gamma(U\otimes V)$ has exactly $r=p\otimes q$ as its gauge.
  This gauge is defined by
  \[
  \label{equation3}
    r(u) = \inf_{\substack{u\in\lambda\Gamma(U\otimes V)\\\lambda>0}}(\lambda).
  \tag{3}
  \]
  But $u\in\lambda\Gamma(U\otimes V)$ if and only if $u=\sum_{\nu=1}^N t_\nu x_\nu \otimes y_\nu$, $p(x_\nu)\leq1$, $q(y_\nu)\leq1$, and $\sum|t_\nu|\leq\lambda$.
  This implies that $u=\sum_{\nu=1}^N\xi_\nu\otimes\eta_\nu$ and $\sum_{\nu=1}^N p(\xi_\nu)q(\eta_\nu)\leq\lambda$, by taking $\xi_\nu=t_\nu x_\nu$ and $\eta_\nu=y_\nu$.

  Conversely, let $u=\sum_{\nu=1}^N \xi_\nu\otimes\eta_\nu$, with $\sum p(\xi_\nu)q(\eta_\nu)\leq\lambda$.
  Let $x_\nu$, $y_\nu$, and $t_\nu$ be given, depending on the value of $\nu$, by
  \[
    (x_\nu,y_\nu,t_\nu) =
    \begin{cases}
      \left(
        \frac{\xi_\nu}{p(\xi_\nu)}, \frac{\eta_\nu}{p(\eta_\nu)}, p(\xi_\nu)q(\eta_\nu)
      \right) &\mbox{if }
      p(\xi_\nu)q(\eta_\nu)\neq0;
    \\\left(
        N\frac{\xi_\nu q(\eta_\nu)}{\varepsilon}, \frac{\eta_\nu}{q(\eta_\nu)}, \frac{\varepsilon}{N}
      \right) &\mbox{if }
      \begin{cases}
        p(\xi_\nu)=0,
      \\q(\eta_\nu)\neq0;
      \end{cases}
    \\\left(
        \frac{\xi_\nu}{p(\xi_\nu)}, N\frac{\eta_\nu p(\xi_\nu)}{\varepsilon}, \frac{\varepsilon}{N}
      \right) &\mbox{if }
      \begin{cases}
        p(\xi_\nu)\neq0,
      \\q(\eta_\nu)=0;
      \end{cases}
    \\\left(
        N\frac{\xi_\nu}{\varepsilon}, \eta_\nu, \frac{\varepsilon}{N}
      \right) &\mbox{if }
      p(\xi_\nu) = q(\eta_\nu) = 0;
    \end{cases}
  \]
  we then obtain a decomposition $u=\sum_{\nu=1}^N t_\nu x_\nu\otimes y_\nu$, with $p(x_\nu)\leq1$, $q(y_\nu)\leq1$, and $\sum_{\nu=1}^N|t_\nu|\leq\lambda+\varepsilon$.

  Since $\varepsilon$ can be as small as we like, $r(u)$ defined by \cref{equation3} is indeed equal to $p\otimes q(u)$ defined by \cref{equation1}.

  We now prove \cref{equation2}.
\end{proof}


\end{document}

\documentclass{article}

\title{Formal geometry and algebraic geometry\\(FGA~2)}
\author{A. Grothendieck}
\date{May 1959}

\usepackage{amssymb,amsmath}

\usepackage{hyperref}
\usepackage[nameinlink]{cleveref}
\usepackage{enumerate}
\usepackage{tikz-cd}
\usepackage{graphicx}

\usepackage{mathrsfs}
%% Fancy fonts --- feel free to remove! %%
\usepackage{Baskervaldx}
\usepackage{mathpazo}


\usepackage{fancyhdr}
\usepackage{lastpage}
\usepackage{xstring}
\makeatletter
\ifx\pdfmdfivesum\undefined
  \let\pdfmdfivesum\mdfivesum
\fi
\edef\filesum{\pdfmdfivesum file {\jobname}}
\pagestyle{fancy}
\makeatletter
\let\runauthor\@author
\let\runtitle\@title
\makeatother
\fancyhf{}
\lhead{\footnotesize\runtitle}
\rhead{\footnotesize Version: \texttt{\StrMid{\filesum}{1}{8}}}
\cfoot{\small\thepage\ of \pageref*{LastPage}}


\crefname{section}{\S\!}{\S\S\!}
\crefname{equation}{}{}


%% Theorem environments %%

\usepackage{amsthm}

\theoremstyle{plain}

\newtheorem{innercustomtheorem}{Theorem}
\crefname{innercustomtheorem}{Theorem}{Theorems}
\newenvironment{theorem}[1]
  {\renewcommand\theinnercustomtheorem{#1}\innercustomtheorem}
  {\endinnercustomtheorem}

\newtheorem{innercustomproposition}{Proposition}
\crefname{innercustomproposition}{Proposition}{Propositions}
\newenvironment{proposition}[1]
  {\renewcommand\theinnercustomproposition{#1}\innercustomproposition}
  {\endinnercustomproposition}

\newtheorem{innercustomlemma}{Lemma}
\crefname{innercustomlemma}{Lemma}{Lemmas}
\newenvironment{lemma}[1]
  {\renewcommand\theinnercustomlemma{#1}\innercustomlemma}
  {\endinnercustomlemma}

\newtheorem{innercustomcorollary}{Corollary}
\crefname{innercustomcorollary}{Corollary}{Corollaries}
\newenvironment{corollary}[1]
  {\renewcommand\theinnercustomcorollary{#1}\innercustomcorollary}
  {\endinnercustomcorollary}


\theoremstyle{definition}

\newtheorem*{remark}{Remark}
\newtheorem*{remarks}{Remarks}

%% Shortcuts %%

\newcommand{\sh}{\mathscr}
\newcommand{\cat}{\mathcal}
\newcommand{\fk}{\mathfrak}
\newcommand{\PP}{\mathbb{P}}
\newcommand{\ZZ}{\mathbb{Z}}
\newcommand{\kres}{\kappa}
\newcommand{\red}{\mathrm{red}}

\renewcommand{\geq}{\geqslant}
\renewcommand{\leq}{\leqslant}

\DeclareMathOperator{\Spec}{Spec}
\DeclareMathOperator{\supp}{supp}
\DeclareMathOperator{\RR}{R}
\DeclareMathOperator{\Hom}{Hom}
\DeclareMathOperator{\Aut}{Aut}
\DeclareMathOperator{\shHom}{\underline{\Hom}}
\DeclareMathOperator{\shAut}{\underline{\Aut}}
\DeclareMathOperator{\gr}{gr}

\newcommand{\todo}{\textbf{ !TODO! }}
\newcommand{\oldpage}[1]{\marginpar{\footnotesize$\Big\vert$ \textit{p.~#1}}}


%% Document %%

\usepackage{embedall}
\begin{document}

\maketitle
\thispagestyle{fancy}

\renewcommand{\abstractname}{Translator's note.}

\begin{abstract}
  \renewcommand*{\thefootnote}{\fnsymbol{footnote}}
  \emph{This text is one of a series\footnote{\url{https://thosgood.com/translations/}} of translations of various papers into English.}
  \emph{The translator takes full responsibility for any errors introduced in the passage from one language to another, and claims no rights to any of the mathematical content herein.}

  \medskip
  
  \emph{What follows is a translation of the French paper:}

  \medskip\noindent
  \textsc{Grothendieck, A.}
  G\'{e}om\'{e}trie formelle et g\'{e}om\'{e}trie alg\'{e}brique.
  \emph{S\'{e}minaire Bourbaki}, Volume~\textbf{11} (1958--59), Talk no.~182.
\end{abstract}

\setcounter{footnote}{0}

\tableofcontents


%% Content %%

\section*{}

\emph{[Trans.] The following ``introduction'' comes from the May 1962 ``Commentaires'' (\emph{S\'{e}minaire Bourbaki} \textbf{14}, 1961--62, Compl\'{e}ment). We have also made changes throughout the text following the errata (loc. cit.), and preface them with ``[Comp.]''.}

\oldpage{C-03}
The substance of \cref{section1} to \cref{section5} is contained in the published part of EGA~III; that of \cref{section6} and \cref{section7} is contained in SGA~III.
For the study of the fundamental group, see SGA~V, IX, X, and XI, as well as SGA~1962 (expos\'{e}s X, XII, and XIII) for the Lefschetz-type theorems and numerous open questions.
Only the theory of moderately ramified coverings (cf. \cref{theorem14}) has not yet been the object of a dedicated expos\'{e}.
The \hyperref[theorem14corollary1]{corollary} of \cref{theorem14}, which completely determines Galois coverings of order prime to the characteristic of an algebraic curve over an algebraically closed field, has been used in an essential manner on three separate occasions:
\begin{enumerate}[1.]
  \item in the proof by Igusa of the Picard inequality for non-singular projective surfaces in arbitrary characteristic;
  \item in the study (developed independently by Ogg and \v{S}afarevi\v{c}) of the group of homogeneous principal bundles over an abelian variety defined over a function field in one variable, in arbitrary characteristic;
  \item in the recent proof, by Artin, of certain key theorems concerning the ``Weil cohomology'' of algebraic varieties.
\end{enumerate}


\section{Schemes}
\label{section1}

\oldpage{182-01}
We know that an affine algebraic space defined over a field $k$ is essentially determined by its affine algebra $A$ (the ring of regular functions defined over $k$), and the morphisms $X\to Y$ of algebraic spaces correspond bijectively to homomorphisms $A(Y)\to A(X)$ of $k$-algebras.
The affine algebra corresponding to an algebraic space is a $k$-algebra of finite type, and, from the ``classical'' point of view, it has no nilpotent elements;
conversely, every such algebra is obtained as the affine algebra of an algebraic space defined over $k$.
There is thus a known dictionary that allows us to interpret situations concerning affine algebraic spaces in terms of commutative algebra.
We have long since noted that we thus obtain more general statements, since it was not generally necessary to suppose that the rings in play were of the form just described, with the Noetherian hypothesis being sufficient the most of the time.
In particular, whether or not a base field were given, it was not necessary to exclude the case where these rings contained nilpotent elements.
Up until now, geometers had refused to take into account this information, and were obstinate in restricting to the consideration of affine algebra without nilpotent elements, i.e. algebraic spaces in whose structure sheaves there are no nilpotent elements (and even, most of the time, ``absolutely irreducible'' algebraic spaces).
The speaker thinks that this state of mind has been a serious obstacle to the development of truly natural methods in algebraic geometry.

Let $A$ be a commutative ring.
It is well known that the set $X=\Spec(A)$ of prime ideals of $A$ is endowed with a natural topology: the ``\emph{Zariski topology}'', or the spectral topology.
Also, there is a sheaf of commutative rings $\sh{O}_X$ on $X$, whose fibre at $\fk{p}\in X$ is the localised ring $A_\fk{p}$, and whose ring of sections can be identified with $A$.
Thus $X$ becomes a \emph{ringed space}, and is called the \emph{prime spectrum} of $A$.
A ring homomorphism $f\colon A\to B$ defines a morphism $f'\colon\Spec(B)\to\Spec(A)$ of ringed spaces, with the underlying map of sets being exactly $\fk{p}\mapsto f^{-1}(\fk{p})$.
The homomorphisms
\oldpage{182-02}
$\Spec(B)\to\Spec(A)$ of ringed spaces obtained in this manner are exactly those for which the homomorphisms $\sh{O}_x\to\sh{O}_y$ (where $x=f'(y)$) are local (i.e. the inverse image of a maximal ideal is a maximal ideal).

We define an \emph{affine scheme} to be a ringed space that is isomorphic to some $\Spec(A)$, and a \emph{prescheme} to be a locally-affine ringed space, i.e. such that every point has an open neighbourhood that is an affine scheme for the induced structure.
We define, in an evident way, \emph{morphisms} of preschemes;
locally, they correspond to ring homomorphisms.

When we fix a prescheme $S$, and we look at morphisms $X\to S$ of preschemes, then $S$ plays the role of a base field or base ring (or, even better, of a base space in a fibration).
We then say that $X$ is an \emph{$S$-prescheme};
if $S=\Spec(A)$, then this also implies that $\sh{O}_X$ is a sheaf of \emph{$A$-algebras}.
So every prescheme can be regarded in a unique way as a $\mathbb{Z}$-prescheme.
Of course, $S$-preschemes form a category, and we can further show that, in this category, the product of two objects $X$ and $Y$ always exists;
it is denoted by $X\times_S Y$.
This notion of product allows us to define the \emph{change of base} of an $S$-prescheme, corresponding to a morphism $S'\to S$, since $X\times_S S'$ can be considered as an $S'$-prescheme.

We say that $X$ is \emph{separated} over $S$ if the diagonal of $X\times_S X$ is closed.
We define a \emph{scheme} to be a prescheme that is separated over $\mathbb{Z}$;
it is then separated over anything.
For simplicity, we will now only speak of schemes, which we will further suppose to be \emph{Noetherian}, i.e. finite unions of affine opens that are spectra of Noetherian rings.
We say that $X$ is \emph{of finite type} over $S$ if, for every affine open subset $U$ of $S$, its inverse image in $X$ is a finite union of affine opens whose rings are algebras of finite type over the ring of $U$.
It is such $S$-schemes that lend themselves to a properly geometry study.
In particular, for every $s\in S$, the fibre $f^{-1}(s)$ of $X$ over $s$ is an algebraic scheme over the residue field $k(s)$ of the local ring $\sh{O}_s$ of $s$ in $S$.
Thus $X$ can be, to a certain extent, considered as a family of ``algebraic spaces'' $f^{-1}(s)$, with the parameter $s$ running over $S$ (i.e., from the local point of view, the set of prime ideals of a given ring).
Of course, the $k(s)$ can have different characteristics.
If $S=\Spec(k)$, where $k$ is a field, then we essentially recover the usual notion of ``algebraic space'', with the only difference being that now the structure sheaf can have nilpotent elements.

\oldpage{182-03}
Inspired by well-known ideas, we can define the notion of a \emph{projective morphism}, and, more generally, of a \emph{proper morphism}.
Such a morphism is of finite type, and further sends closed subsets to closed subsets, and retains this property under an arbitrary change of base.

With $X$ being a (Noetherian, as always) scheme, the sheaf $\sh{O}_X$ is a \emph{coherent sheaf of rings} in the sense of \cite{2}.
The coherent sheaves of modules on $X$ are thus also the sheaves which are locally isomorphic to a cokernel of some morphism $\sh{O}_X^m\to\sh{O}_X^n$.


\section{Formal schemes}
\label{section2}

Let $X$ be a scheme, and $X'$ a closed subset of $X$.
Then there exists a coherent subsheaf $\sh{J}$ of $\sh{O}_X$ such that $X'=\supp\sh{O}_X/\sh{J}$ (and there even exists a larger one \todo?).
Endowing $X'$ with the sheaf $\sh{O}_X/\sh{J}$ makes $X'$ a scheme, denoted $X_0$;
such a scheme is called a \emph{closed subscheme of $X$}.
We can also, for any $n$, consider $X'$ endowed with $\sh{O}_X/\sh{J}^{n+1}$, denoted $X_n$, which is a closed subprescheme of $X$ whose underlying set is again $X'$, but with a different structure sheaf, namely $\sh{O}_{X_n}=\sh{O}_X/\sh{J}^{n+1}$.
Clearly the $\sh{O}_{X_n}$ form a projective system of sheaves of rings on $X$, whose projective limit $\overline{\sh{O}_X}$ is called the \emph{formal completion of $\sh{O}_X$ along $X'$}.
Endowed with this sheaf of rings, $X'$ is called the \emph{formal completion of $X$ along $X'$}, and is thus a ringed space, but not a scheme in general.
For every coherent sheaf $\sh{F}$ on $X$, we can similarly consider the formal completion $\overline{\sh{F}}=\varprojlim_n\sh{F}_n$ of $\sh{F}$ along $X'$ (where $\sh{F}_n=\sh{F}\otimes_{\sh{O}_X}\sh{O}_X/\sh{J}^{n+1}$), which is a sheaf of modules on $\overline{X}$.
Its sections are called \emph{formal sections of $\sh{F}$ along $X$}, and can be identified with elements of $\varprojlim_n\Gamma(X',\sh{F}_n)$.
For $\sh{F}=\sh{O}_X$, we recover the ``holomorphic functions'' of $X$ along $X'$, in the sense of Zariski (whose terminology we will not follow, due to its interferences with classical terminology).

We define a \emph{formal scheme} (implicitly assumed to be Noetherian) to be a topological space $\fk{X}$ endowed with a sheaf of topological rings $\sh{O}_{\fk{X}}$ satisfying the following condition:
there is an isomorphism of sheaves of topological rings $\sh{O}_{\fk{X}}=\varprojlim_n\sh{O}_n$, where the $\sh{O}_n$ form a projective system of sheaves of rings on $\fk{X}$, with each one making $\fk{X}$ into a scheme $\fk{X}_n$, and such that, for $m\geq n$, the homomorphism $\sh{O}_m\to\sh{O}_n$ is surjective and has $\sh{J}_m^{n+1}$ as its kernel, where $\sh{J}_m$ is
\oldpage{182-04}
the kernel of $\sh{O}_m\to\sh{O}_0$.
We will show that $\sh{O}_{\fk{X}}$ is a \emph{coherent} sheaf of \emph{local Noetherian} rings.

By the definitions, a formal completion $\overline{X}$ as above is a formal scheme, and, conversely, every formal scheme is \emph{locally} of this type.
In fact, the data of a formal \emph{affine} scheme (i.e. such that $\fk{X}_0$ is affine, which implies that all the $\fk{X}_n$ are affine) is equivalent to the data of a separated complete $\sh{J}$-adic Noetherian topological ring.

The usual definitions (morphism, morphism of finite type, proper morphism, etc.) for ordinary schemes generalise without problem to formal schemes.


\section{The three fundamental theorems}
\label{section3}

Let $f\colon X\to Y$ be a proper morphism of schemes (Noetherian, as always), and let $Y'$ be a closed subset of $Y'$, with $X'$ its inverse image in $X$, and consider the corresponding formal completions $\overline{Y}$ and $\overline{X}$.
Then $f$ induces a morphism $\overline{f}\colon\overline{X}\to\overline{Y}$ of formal schemes, which is also proper.
Let $\sh{F}$ be a coherent sheaf on $X$, then $\overline{\sh{F}}$ is a coherent sheaf on $\overline{X}$.
In \cref{theorem1}, we forget $X$, $Y$, and $\sh{F}$, and consider only the proper morphism $\overline{f}$ of formal schemes, along with the coherent sheaf $\overline{\sh{F}}$ on $\overline{X}$.
(However, the speaker has only written the complete proof in the case where we start with some $X$, $Y$, $f$, and $F$).

\begin{theorem}{1}
\label{theorem1}
  \emph{(Finiteness theorem).}
  \begin{enumerate}[i.]
    \item The $\RR^q\overline{f}_*(\overline{\sh{F}})$ are coherent sheaves on $\overline{Y}$.
    \item The natural homomorphisms
      \[
        \RR^q\overline{f}_*(\overline{\sh{F}}) \to \varprojlim_n\RR^q (f_n)_*(\sh{F}_n)
      \]
      are isomorphisms.
  \end{enumerate}
\end{theorem}

In this statement, we suppose that we already have some coherent subsheaf $\sh{J}$ of $\sh{O}_Y$ that defines $Y'$, whence, by taking the inverse image, a coherent subsheaf of $\sh{O}_X$ that defines $X'$, whence, by definition, $\sh{F}_n$, $X_n$, $Y_n$, and $f_n\colon X_n\to Y_n$ as in \cref{section2}.
The minor changes that need to be made to the notation in the explanation if we started with an arbitrary proper morphism between two formal schemes are evident.

\Cref{theorem1} deals only with ``formal cohomology''.
The following theorem
\oldpage{182-05}
relates it with ``algebraic cohomology'', and resembles a well-known theorem of Serre \cite{4} on the comparison between algebraic cohomology and analytic cohomology.

\begin{theorem}{2}
\label{theorem2}
  \emph{(First comparison theorem).}
  The $\RR^q f_*(\sh{F})$ are coherent sheaves on $Y$ (which is a particular case of \cref{theorem1}), and the natural homomorphisms
  \[
    \overline{\RR^q f_*(\sh{F})} \to \varprojlim_n \RR^q (f_n)_*(\sh{F}_n)
  \]
  are isomorphisms.
\end{theorem}

\begin{corollary}{1}
\label{theorem2corollary1}
  There are canonical isomorphisms $\overline{\RR^q f_*(\sh{F})} = \RR^q\overline{f}_*(\overline{\sh{F}})$.
\end{corollary}

This corollary is, for $q=0$, a generalisation of Zariski's ``fundamental theorem of holomorphic functions'', from which we will deduce a generalisation of Zariski's ``connection theorem''.
We also note that, while \cref{theorem1}~(ii) is trivial for $q=$, this is not at all the case for \cref{theorem2} nor for its equivalent formulation (\hyperref[theorem2corollary1]{Corollary~1}).
In fact, the proof proceeds by decreasing induction on $q$ (being trivial for large $q$, since then both sides of the equation are zero), and the case $q=0$ thus appears as the last induction step, and so could be called the ``most difficult'' case.

\begin{corollary}{2}
\label{theorem2corollary2}
  Let $Y=\Spec(A)$, and let $Y'$ be defined by an ideal $\sh{J}$ of $A$.
  Then, for every coherent sheaf $\sh{F}$ on $X$, the $H^q(X,\sh{F})$ are $A$-modules of finite type, whose $\sh{J}$-adic completions are the $H^q(\overline{X},\overline{\sh{F}})$.
\end{corollary}

Finally, applying this corollary to $H=\shHom_{\sh{O}_X}(\sh{F},\sh{G})$, we obtain:

\begin{corollary}{3}
\label{theorem2corollary3}
  Let $Y=\Spec(A)$, and let $Y'$ be defined by an ideal $\sh{J}$ of $A$.
  Let $\sh{F}$ and $\sh{G}$ be coherent sheaves on $X$.
  Then $\Hom(\sh{F},\sh{G})$ is an $A$-module of finite type, whose $\sh{J}$-adic completion can be identified with $\Hom(\overline{\sh{F}},\overline{\sh{G}})$.
\end{corollary}

Of course, the natural map $\Hom(\sh{F},\sh{G})\to\Hom(\overline{\sh{F}},\overline{\sh{G}})$ is that which sends a homomorphism $u\colon\sh{F}\to\sh{G}$ to its extension ``by continuity'' $\overline{u}\colon\overline{\sh{F}}\to\overline{\sh{G}}$ (so that $\overline{\sh{F}}$ becomes a functor in $\sh{F}$).

Now suppose that $A$ is separated and complete for its $\sh{J}$-adic topology.
Then the above corollaries \hyperref[theorem2corollary2]{2} and \hyperref[theorem2corollary3]{3} give:
\[
  \begin{aligned}
    H^q(X,\sh{F}) &= H^q(\overline{X},\overline{\sh{F}}),
  \\\Hom(\sh{F},\sh{G}) &= \Hom(\overline{\sh{F}},\overline{\sh{G}}).
  \end{aligned}
\]
\oldpage{182-06}
This latter identity shows that the category of coherent sheaves on $X$ can be identified with a \emph{subcategory} (with morphisms being the induced morphisms) of the category of coherent sheaves on $\overline{X}$.
In fact, we even have:

\begin{theorem}{3}
\label{theorem3}
  For a sheaf of modules on $\overline{X}$ to be coherent, it is necessary and sufficient for it to be isomorphic to a sheaf of the form $\overline{\sh{F}}$, where $\sh{F}$ is a coherent sheaf on $X$ (determined up to canonical isomorphism, by \hyperref[theorem2corollary3]{Corollary~3} of \cref{theorem2}).
  [We recall that now $Y=\Spec(A)$, with $A$ being a complete and separated $\sh{J}$-adic Noetherian topological ring].
\end{theorem}

\begin{corollary}{1}
\label{theorem3corollary1}
  The closed subschemes of $X$ are in bijective correspondence with the closed formal subschemes of $\overline{X}$.
\end{corollary}

Indeed, they correspond to coherent subsheaves of $\sh{O}_X$ (resp. of $\sh{O}_{\overline{X}}$).
Considering the graphs of morphisms as closed subschemes, \hyperref[theorem3corollary1]{Corollary~1} implies:

\begin{corollary}{2}
\label{theorem3corollary2}
  Let $X$ and $Z$ be proper schemes over $A$ (which is a separated complete $\sh{J}$-adic Noetherian ring).
  Then the map $g\mapsto\overline{g}$ defines a bijective correspondence between $Y$-morphism from $X$ to $Z$ and $\overline{Y}$-morphisms from $\overline{X}$ to $\overline{Z}$.
\end{corollary}

In other words, proper algebraic schemes over $A$ give a subcategory (with the morphisms being the induced morphisms) of the category of proper formal schemes over $\overline{Y}$.
We note, however, that \emph{there exist proper formal schemes over $\overline{Y}$ that are not ``algebraisable''}, i.e. isomorphic to some $\overline{X}$, where $X$ is proper over $A$ (just as there exist compact complex-analytic varieties that do not come from algebraic varieties defined over the field of complex numbers).
Such formal schemes naturally appear in ``module theory''.
We note, however, an interesting case where a formal scheme is algebraisable:

\begin{theorem}{4}
\label{theorem4}
  Let $A$ be a complete local Noetherian ring, with residue field $k$, and let $\fk{X}$ be a proper formal scheme over $A$ (endowed with its \todo $\Gamma(A)$-adic topology).
  We suppose that
  \begin{enumerate}[i.]
    \item the local rings of $\sh{O}_{\fk{X}}$ are \emph{flat} $A$-modules, or, equivalently, that, if we endow $\sh{O}_{\fk{X}}$ and $A$ with the filtration given by powers of the maximal ideal of $A$, then the associated graded algebras satisfy
      \[
        \gr(\sh{O}_{\fk{X}}) \simeq \gr^0(\sh{O}_{\fk{X}})\otimes_k\gr(A);
      \]
    \item $H^2(\fk{X}_0,\sh{O}_{\fk{X}_0})=0$, where we consider $\fk{X}_0=\fk{X}\otimes_Ak$ as an algebraic scheme over $k$;
\oldpage{182-07}
    \item $\fk{X}_0$ is projective.
  \end{enumerate}
  Then, under these conditions, $\fk{X}$ is algebraisable, and, more precisely, is isomorphic to $\overline{X}$, where $X$ is some projective $A$-scheme.
\end{theorem}

Conditions~(ii) and (iii) will be satisfied if, in particular, $\fk{X}_0$ is a \emph{simple curve} over $k$, and \cref{theorem4} can be applied, in particular, in the ``module theory'' of curves of a given genus\ldots.
We give a hint on how to prove \cref{theorem4}:
we show (cf. \cref{proposition3} below) that (i) and (ii) imply that every coherent sheaf on $\fk{X}_0$ that is locally isomorphic to a fundamental sheaf can be obtained by reduction starting from a sheaf of the same nature on $\fk{X}$.
So, starting with an ``ample'' sheaf on $\fk{X}_0$ (which, by (iii), exists), we lift it to obtain an invertible sheaf on $\fk{X}$, and, using \cref{theorem1}, we prove that a multiple of this invertible sheaf defined an immersion of $X$ into the formal completion of a scheme $\PP_A^r$ (``projective type'' of dimension~$r$ over $A$).

For the proof of \cref{theorem1,theorem2,theorem3}, we refer the reader to \cite{1}.


\section{Application to Zariski's connection theorem and ``main theorem''}
\label{section4}

Let $f\colon X\to Y$ be a proper morphism of schemes.
Then, by the \hyperref[theorem1]{finiteness theorem}, $f_*(\sh{O}_X)=\underline{A}$ is a coherent sheaf on $Y$, and is also a sheaf of commutative algebras, and thus corresponds to a $Y$-scheme $g\colon Y'\to Y$ that is finite over $Y$ (defined by the condition of being affine over $Y$, i.e. the inverse image of an affine open is affine, and $g_*(\sh{O}_{Y'})=\underline{A}$).
It is immediate that $f$ then canonically factors as $f=gf'$, where $f'\colon X\to Y'$ is a morphism from $X$ to $Y$ that is now such that $f'_*(\sh{O}_X)=\sh{O}_{Y'}$.
This factorisation of $f$ is called \emph{the Stein factorisation} of $f$.
Applying the \hyperref[theorem2]{first comparison theorem} and \hyperref[theorem2corollary1]{Corollary~1} to $f'$ and the subset $Y'$ consisting of a single point $y'$, we see that $(f')^{-1}(y')=X'$ is connected (or, said differently, the formal sections of $X$ along $X'$ do not form a local ring, but the completion $f'_*(\sh{O}_X)_{y'}=\sh{O}_{y'}$ is local!)
We have proven:

\begin{theorem}{5}
\label{theorem5}
  \emph{(Zariski's ``connection theorem'').}
  Let $f\colon X\to Y$ be a proper morphism.
  Then $f$ factors uniquely (up to isomorphism) as $f=gf'$, where $g\colon Y'\to Y$ is finite, and $f'\colon X\to Y'$ is such that $f'_*(\sh{O}_X)=\sh{O}_{Y'}$ (whence $g_*(\sh{O}_{Y'})=f_*(\sh{O}_X)$).
  The fibres of $f'$ are connected,
\oldpage{182-06}
  i.e. the set of connected components of a fibre $f^{-1}(y)$ of $f$ is in bijective correspondence with the set of points of $Y'$ over $y$, i.e. the set of maximal ideals in $f_*(\sh{O}_X)_y$.
\end{theorem}

From this, we immediately deduce the usual variants of the connection theorem.
We state here only the following:

\begin{corollary}{1}
\label{theorem5corollary1}
  For a point $x$ of $X$ to be isolated in its fibre $f^{-1}(y)$, it is necessary and sufficient for the fibre $(f')^{-1}(y')$ (where $y'=f'(x)$) to consist of a single point $x$, or for $f'$ to induce an isomorphism from a neighbourhood of $x$ to a neighbourhood of $y'$.
  The set of these points is an open subset $U$, and $f'$ induces an isomorphism from $U$ to an open subset of $Y'$.
\end{corollary}

To show that $f'$ is a local isomorphism at $x$, we note that $f'$ induces an \emph{isomorphism} $\sh{O}_{y'}\to\sh{O}_x$, as we see thanks to $f'(\sh{O}_X)=\sh{O}_{Y'}$;
we also note that the $(f')^{-1}(V)$ give a fundamental system of neighbourhoods of $x$ when $V$ runs over a fundamental system of neighbourhoods of $y'$ (since $f'$ is a closed map whose fibre at $y'$ consists of the single point $x$).
We thus immediately deduce the following result, due to Chevalley in the ``geometric'' case:

\begin{corollary}{2}
\label{theorem5corollary2}
  For $f$ to be a finite morphism, it is necessary and sufficient for it to be proper with finite fibres.
\end{corollary}

If this is so, then $f'$ is effectively an isomorphism, by the above.

Let $f\colon X\to Y$ be a morphism that is not necessarily proper, but suppose that $X$ is contained in some proper $Y$-scheme $\overline{f}\colon\overline{X}\to Y$ as an open subset (which is the case if, in particular, $\overline{f}$ is quasi-projective).
Applying \hyperref[theorem5corollary1]{Corollary~1}, we see that $\overline{f'}$ (\todo \textbf{or $\overline{f}'$?}) induces an isomorphism from the set $U$ of points of $X$ that are isolated in their fibre to an open subset of $Y'$ (and that $U$ is indeed an open subset).
We thus deduce the following global version of Zariski's ``main theorem'':

\begin{theorem}{6}
\label{theorem6}
  Let $f\colon X\to Y$ be a morphism of finite type.
  Then the set $U$ of points of $X$ that are isolated in their fibre is open, and, if $f$ is quasi-projective\footnote{\emph{[Comp.]} This hypothesis can be replaced by the weaker hypothesis ``if $f$ is separated'', by means of the following result (see SGA~VIII, 6.2): every morphism $f\colon X\to Y$ which is quasi-finite and separated is also projective.}, then $U$ is $Y$-isomorphic to an open subset of some scheme $Y'$ that is finite over $Y$.
\end{theorem}

Since a morphism of finite type is locally affine, and \emph{a fortiori} locally quasi-projective, we immediately deduce from \cref{theorem6} the usual local variants of the main theorem.


\section{Application to the cohomological study of proper and flat morphisms}
\label{section5}

Let $f\colon X\to Y$ be a proper morphism, and $\sh{F}$ a coherent sheaf on $X$,
\oldpage{182-09}
with $\sh{F}$ assumed to be $Y$-flat, i.e. the $\sh{F}_x$ are flat modules over the rings $\sh{O}_y$ (where $y=f(x)$).
This also implies that, for every $y\in Y$, if we filter $\sh{F}$ along the fibre $f^{-1}(y)$ by the $\fk{m}_y^n\sh{F}$ (where $\fk{m}_y$ is the maximal ideal of $\sh{O}_y$), then the associated graded algebra is isomorphic to $(\sh{F}/\fk{m}_y\sh{F})\otimes_{\kres(y)}\gr(\sh{O}_y)$;
in other words, we have that
\[
  \fk{m}_y^n\sh{F}/\fk{m}_y^{n+1}
  = \sh{F}_y\otimes_{\kres(y)}(\fk{m}_y^n/\fk{m}_y^{n+1})
\]
for every integer $n$, where $\sh{F}_y$ denotes the sheaf $\sh{F}/\fk{m}_y\sh{F}$ induced by $\sh{F}$ on $X_y$ (with $X_y$ denoting the fibre $f^{-1}(y)$ considered as a proper scheme over the residue field $\kres(y)$ of $y$).
Taking this isomorphism, as well as \cref{theorem2}, into account, we obtain augmentations, and sometimes computations, of the $\RR^q f_*(\sh{F})$ in a neighbourhood of $y$, knowing the cohomology of $X_y$ with coefficients in $\sh{F}_y$.
Here \cref{theorem2} takes the form
\[
  \overline{\RR^q f_*(\sh{F})} = \varprojlim_n H^q(\sh{F}_y,\sh{F}/\fk{m}_y^n\sh{F}).
\]
We will mention here only the following consequence:

\begin{proposition}{1}
\label{proposition1}
  Let $f\colon X\to Y$ be a proper morphism, and $\sh{F}$ a coherent $Y$-flat sheaf on $X$.
  Let $y\in Y$, let $q$ be an integer, and suppose that $H^q(X_y,\sh{F}_y)=0$.
  Then $\RR^q f_*(\sh{F})$ is zero on a a neighbourhood of $y$, and, for every $n$, the natural homomorphism
  \[
    \RR^{q-1}f_*(\sh{F})_y \to H^{q-1}(X_y,\sh{F}_y/\fk{m}_y^n\sh{F}_y)
  \]
  is surjective.
\end{proposition}

In particular, if $f$ is a flat morphism (i.e. if $\sh{O}_X$ is $Y$-flat), then every locally free coherent sheaf $\sh{F}$ on $X$ is $Y$-flat.
Let $\sh{F}$ and $\sh{G}$ be two such sheaves, and apply \cref{proposition1} to $\shHom_{\sh{O}_X}(\sh{F},\sh{G})$ and $q=1$ to obtain:

\begin{theorem}{7}
\label{theorem7}
  Let $f$ be a flat proper morphism, $\sh{F}$ and $\sh{G}$ locally free coherent sheaves on $X$, and $y\in Y$;
  suppose that $H^1(X_y,\shHom_{\sh{O}_X}(\sh{F}_y,\sh{G}))=0$.
  Then every homomorphism $u_0\colon\sh{F}_y\to\sh{G}_y$ is induced by a homomorphism $u\colon\sh{F}|V\to\sh{G}|V$, where $V=f^{-1}(U)$ is the inverse image of a neighbourhood $U$ of $y$.
\end{theorem}

\begin{corollary}{1}
\label{theorem7corollary1}
  If $u_0$ is an isomorphism (resp. a monomorphism, resp. an epimorphism), then so too is $u$, for small enough $U$.
\end{corollary}

In particular:

\oldpage{182-10}
\begin{corollary}{2}
\label{theorem7corollary2}
  Let $E_0$ be a locally free coherent sheaf on $X_y$ such that $H^1(X_y;\shHom_{\sh{O}_X}(E_0,E_0))=0$.
  Then any two locally free sheaves whose restrictions to $X_y$ are isomorphic to $E_0$ are themselves isomorphic to one another in a neighbourhood of $X_y$.
\end{corollary}

Thus:

\begin{corollary}{3}
\label{theorem7corollary3}
  Suppose that $H^1(X_y,\sh{O}_{X_y})=0$.
  Then any two invertible sheaves on $X$ (i.e. locally isomorphic to $\sh{O}_X$) whose restrictions to $X_y$ are isomorphic are themselves isomorphic to one another.
\end{corollary}

It thus follows that:

\begin{proposition}{2}
\label{proposition2}
  Let $Y$ be a connected scheme, and $E$ a locally free coherent sheaf on $Y$.
  Consider the bundle of projective spaces $X=\PP(E)$ associated to $E$, endowed with its well-known invertible sheaf $\sh{O}_X(1)$.
  Then every invertible sheaf $\sh{L}$ on $X$ is isomorphic to a sheaf of the form $f^*(\sh{L}')\otimes\sh{O}_X(n)$, where $\sh{L}'$ is an invertible sheaf on $Y$, and $n$ is an integer.
  Further, $n$ is uniquely determined, and $\sh{L}'$ is determined up to isomorphism.
\end{proposition}

The above \hyperref[theorem7corollary3]{Corollary~3} proves that $\sh{L}$ is isomorphic to an $\sh{O}_X(n)$-module on a neighbourhood of each fibre.
The rest is more or less formal.

\Cref{proposition2} allows us to determine the $Y$-morphisms from $X=\PP(E)$ to another projective bundle.
We see, in particular:

\begin{corollary}{1}
\label{proposition2corollary1}
  Let $u$ be an automorphism of $X=\PP(E)$.
  Then there exists an invertible sheaf $\sh{L}'$ on $Y$, and an isomorphism $v$ from $E$ to $E\otimes\sh{L}'$ such that $u$ is the isomorphism corresponding to $\PP(E)\xrightarrow{\sim}\PP(E\otimes\sh{L}')=\PP(E)$;
  the pair $(v,\sh{L}')$ is determined up to isomorphism.
\end{corollary}

Let $\Gamma$ be the set of classes of invertible bundles $\sh{L}'$ on $Y$ such that $E\otimes\sh{L}'$ is isomorphic to $E$.
Its elements are torsion, since, if $n$ is the rank of $E$, then (by taking $n$-th exterior powers) we must have that $(\sh{L}')^{\otimes n}\xrightarrow{\sim}\sh{O}_Y$.
The above corollary can then be expressed by saying that we have an exact sequence of groups:
\[
  e \to \Aut(E)/\Gamma(Y,\sh{O}_Y^*) \to \Aut_Y(X) \to \Gamma \to e
\]
(which can also be deduced from the exact sequence in cohomology induced by the exact sequence of \emph{sheaves} of groups
\oldpage{182-11}
\[
  e \to \sh{O}_X^* \to \shAut \to \shAut_Y(X) \to e,
\]
where $\sh{O}_X^*$ is the sheaf of ``units'' of $\sh{O}_X$, identified with the centre of $\shAut(E)$.)


\section{Application to existence and uniqueness theorems for sheaves and schemes over a complete $\mathscr{J}$-adic ring}
\label{section6}

\Cref{theorem7} gave a uniqueness result for locally free coherent sheaves, by using \cref{theorem1} and \cref{theorem2}.
Using \cref{theorem3}, we now obtain \emph{existence} theorems for sheaves, for morphisms of schemes, or for schemes.
In the following, $A$ denotes a local Noetherian ring, assumed to be separated and \emph{complete}.
The general method still consists of making \emph{formal} construction, which consists essentially of doing \emph{algebraic geometry over an Artinian ring}, and deducing conclusions from this that are ``algebraic'' in nature, by using the three fundamental theorems.

\begin{proposition}{3}
\label{proposition3}
  Let $\fk{X}$ be a proper formal scheme that is flat over $A$, and let $\sh{F}_0$ be a locally free sheaf on $X_0$ such that $H^2(X_0,\shHom_{\sh{O}_{X_0}}(\sh{F}_0,\sh{F}_0))=0$.
  Then there exists a locally free sheaf $\sh{F}$ on $\fk{X}$ that induces, on $X_0$, a sheaf isomorphic to $\sh{F}_0$.
  (This $\sh{F}$ is also unique up to isomorphism if $H^1(X_0,\shHom_{\sh{O}_{X_0}}(\sh{F}_0,\sh{F}_0))=0$).
\end{proposition}

We construct, step by step, locally free sheaves $\sh{F}_n$ on the $X_n$, that induce one another.
The construction of $\sh{F}_0$ is met with an obstruction in $H^2(X_0,\shHom_{\sh{O}_{X_0}}(\sh{F}_0,\sh{F}_0))\otimes_{A/\sh{J}}(\sh{J}^n/\sh{J}^{n+1})$, but this is zero, by hypothesis.
Now using \cref{theorem3}, we obtain:

\begin{corollary}{1}
\label{proposition3corollary1}
  Let $X$ be a proper scheme that is flat over $A$, and let $\sh{F}_0$ be as above.
  Then there exists a locally free sheaf $\sh{F}$ on $X$ that induces, on $X_0$, a sheaf that is isomorphic to $\sh{F}_0$.
  This $\sh{F}$ is also unique up to isomorphism if $H^1(X_0,\shHom_{\sh{O}_{X_0}}(\sh{F}_0,\sh{F}_0))=0$.
\end{corollary}

Let $X_0$ be a scheme of finite type over the field $k$, and suppose that $X_0$ is \emph{simple} (by which we mean \emph{absolutely} simple) over $k$, but not necessarily proper over $k$.
Let $A$ be a local Artinian ring with residue field $k$.
We are interested in finding schemes $X$ that are flat over $A$, and such that $X\otimes_A k=X_0$ (this is the starting point of the ``\emph{theory of modules}'', or of ``structure variations''
\oldpage{182-12}
of $X_0$).
It is equivalent to give either such an $X$ or, on the topological space $X_0$, a sheaf $\sh{O}_X$ endowed with the following structures:
\begin{enumerate}[i.]
  \item $\sh{O}_X$ is a sheaf of $A$-algebras;
  \item $\sh{O}_X$ is endowed with an augmentation homomorphism $\sh{O}_X\to\sh{O}_{X_0}$ (that is compatible with the $A$-algebra structures);
\end{enumerate}
and with the above data being subject to the following conditions: the augmentation induces an isomorphism $\sh{O}_X\otimes_A k\xrightarrow{\sim}\sh{O}_{X_0}$;
$\sh{O}_X$ is flat over $A$, i.e. the graded algebra associated to $\sh{O}_X$ filtered by the powers of the maximal ideal $\fk{m}$ of $A$ is isomorphic to $\gr^0(\sh{O}_X)\otimes_k\gr(A)$, i.e. we have isomorphisms $\fk{m}^n\sh{O}_X/\fk{m}^{n+1}\sh{O}_X = \sh{O}_{X_0}\otimes_k(\fk{m}^n/\fk{m}^{n+1})$.
The fundamental fact is the following:

\begin{theorem}{8}
\label{theorem8}
  Let $X_0$ be a simple scheme of finite type over the field $k$, and assume $X_0$ to be affine.
  Let $A$ be a local Artinian ring of residue field $k$.
  Then there exists an $A$-scheme $X$ that is flat over $A$ and such that $X\otimes_A k=X_0$.
  Further, any two such schemes are necessarily isomorphic.
\end{theorem}

Note that the isomorphic in question is not canonical, since $X$ will have, in general, non-trivial $A$-automorphisms that induce the identity on $X_0$.
Furthermore, there is not, in general, a ``canonical'' choice of $X$ satisfying the given conditions, except in the case where $A$ is a $k$-algebra (the case of \emph{equal characteristics}), where we can take $X=X_0\otimes_k A$, i.e. $\sh{O}_X=\sh{O}_{X_0}\otimes_k A$ (whether or not $X_0$ is affine, in fact).
In the case of unequal characteristics, I do not know in general, when $X_0$ is not affine, if we can ``lift'' $X_0$ to an $X$ defined over $A$.
However, for any integer $n>0$, let $A_{n-1}=A/\fk{m}^n$, and suppose that we have lifted $X_0$ to a flat $A_{n-1}$-scheme $X_{n-1}$;
we intend to lift $X_{n-1}$ to a flat $A_n$-scheme $X_n$.
We already know, by \cref{theorem8}, that this is possible locally;
we can also easily verify that, if $U_n$ lifts an open subset $U_{n-1}$ of $X_{n-1}$, then the sheaf of groups of automorphisms of $U_n$ (that induce the identity on $U_{n-1}$) is canonically isomorphic to $\fk{G}_{X_0/k}\otimes_k\fk{m}^n/\fk{m}^{n+1}$ restricted to $U_{n-1}$, and thus, in particular, commutative (where $\fk{G}_{X_0/k}$ denotes the sheaf of germs of $k$-derivations on $X_0$).
It follows easily that we have an \emph{obstruction of constructing $X_n$ lifting $X_{n-1}$}, which lives in $H^2(X_0,\fk{G}_{X_0/k})\otimes\fk{m}^n/\fk{m}^{n+1}$.
Then:

\begin{corollary}{1}
\label{theorem8corollary1}
  Let $X_0$ be a simple scheme of finite type over $k$, and suppose that
\oldpage{182-13}
  \[H^2(X_0,\fk{G}_{X_0/k})=0.\]
  Then, for every local Artinian ring $A$ with residue field $k$, there exists a flat $A$-scheme $X$ such that $X\otimes_A k=X_0$.
\end{corollary}

Also, if we can find \emph{one} $X$ that is flat over $A$ and that lifts $X_0$, then, by \cref{theorem8}, the set of classes (up to isomorphism) of flat $A$-schemes that lift $X_0$ can be identified with $H^1(X_0,\shAut(X))$, where $\shAut(X)$ denotes the sheaf of germs of automorphisms of the sheaf $\sh{O_X}$ of $A$-algebras \emph{that are compatible with the augmentation}.
The filtration of $\sh{O}_X$ defines a filtration of $\shAut(X)$, with the quotient of this sheaf by the $n$-th subgroup of the filtration being $\shAut(X_n)$;
the graded algebra associated to this filtration is commutative, and can be identified with $\fk{G}_{X_0/k}\otimes_k\gr(A)$.
In particular, if $\fk{m}^{n+1}$ is the first power of $\fk{m}$ that is not zero, then $F^n(\shAut(X))$ (the last stage of the filtration) is in the centre of $\shAut(X)$, and is isomorphic to $\fk{G}_{X_0/k}\otimes_k\fk{m}^n$;
it is also the sheaf of germs of automorphisms of $X$ that induce the identity on $X_{n-1}=X\otimes_A A/\fk{m}^n$.
Using these results, we immediately obtain the following statements:

\begin{corollary}{2}
\label{theorem8corollary2}
  Let $X_0$ be a simple scheme of finite type over $k$, and let $A$ be a local Artinian ring with residue field $k$ and maximal ideal $\fk{m}$.
  Suppose that $\fk{m}^{n+1}=0$.
  Let $A_{n-1}=A/\fk{m}^n$, and let $X_{n-1}$ be a flat $A_{n-1}$-scheme such that $X_{n-1}\otimes_Ak=X_0$.
  Then the set of classes (up to isomorphism, inducing the identity on $X_{n-1}$ \todo?) of flat $A$-schemes $X_n$ such that $X\otimes_AA_{n-1}=X_{n-1}$ is either empty, or a homogeneous principal space under $H^1(X_0,\fk{G}_{X_0/k})\otimes_k\fk{m}^n$.
\end{corollary}

(Note that, in general, there is no privileged choice of origin in the latter homogeneous principal space, since there is no privileged way of lifting $X_{n-1}$ to $X_n$).

\begin{corollary}{3}
\label{theorem8corollary3}
  Let $X_0$ be a simple scheme of finite type over $k$, and suppose that $H^1(X_0,\fk{G}_{X_0/k})=0$.
  Then, for every local Artinian ring $A$ with residue field $k$, \emph{there exists at most one flat $A$-scheme $X$} (up to isomorphism) such that $X\otimes_Ak=X_0$.
\end{corollary}

\Cref{theorem8corollary1} and \cref{theorem8corollary3} immediately imply the claims, which seem more general, obtained by supposing only that $A$ is a \emph{complete local Noetherian ring with residue field $k$}, provided that we introduce $X$ as a formal scheme over $A$:

\begin{theorem}{9}
\label{theorem9}
  Let $k$ be a field, and $X_0$ a \emph{simple} scheme of finite type over $k$.
  For every complete locally Noetherian ring $A$ with residue field $k$, let $F(A)$ be the set of classes (up to isomorphism, inducing the identity on $X_0$ \todo)
\oldpage{182-14}
  of formal schemes $\fk{X}$ over $A$, of finite type, and flat over $A$, such that $X\otimes_Ak=X_0$.
  With this notation:
  \begin{enumerate}[i.]
    \item if $H^1(X_0,\fk{G}_{X_0/k})=0$, then, for all $A$, $F(A)$ has at most one element;
    \item if $H^2(X_0,\fk{G}_{X_0/k})=0$, then, for all $A$, $F(A)$ has at least one element.
  \end{enumerate}
\end{theorem}

\begin{corollary}{1}
\label{theorem9corollary1}
  Suppose that $X_0$ is proper over $k$.
  Under condition~(i) of \cref{theorem9}, for all $A$, there exists at most (up to isomorphism, inducing the identity on $X_0$ \todo) one scheme $X$ that is proper, flat over $A$, and such that $X\otimes_Ak=X_0$.
\end{corollary}

We can use \hyperref[theorem3corollary2]{Corollary~2} of \cref{theorem3}.
For example:
\begin{corollary}{2}
\label{theorem9corollary2}
  If $X$ is a proper flat $A$-scheme such that $X\otimes_Ak$ is isomorphic to the projective-type scheme $\fk{P}_k^r$ of dimension~$r$ over $k$, then $X$ is isomorphic to $\fk{P}_k^r$.
\end{corollary}

(We can also deduce this result from \hyperref[proposition3corollary1]{Corollary~1} of \cref{proposition3}).

\begin{corollary}{3}
\label{theorem9corollary3}
  Let $X_0$ be a simple projective scheme over $k$, and suppose that
  \[
    H^2(X_0,\sh{O}_{X_0}) = H^2(X_0,\fk{G}_{X_0/k}) = 0.
  \]
  Then, for all $A$, there exists a flat projective $A$-scheme such that $X\otimes_Ak=X_0$.
\end{corollary}

We can combine \cref{theorem9}~(ii) with \cref{theorem4}.
In particular:

\begin{corollary}{4}
\label{theorem9corollary4}
  Let $X_0$ be the scheme of a complete simple algebraic curve over $k$.
  Then, for every complete local Noetherian ring $A$ with residue field $k$, there exists a ``simple curve scheme'' $X$ over $A$, such that $X\otimes_Ak=X_0$.
\end{corollary}

\begin{remarks}
  \begin{enumerate}[1)]
    \item \hyperref[theorem9corollary3]{Corollary~3} and \hyperref[theorem9corollary4]{Corollary~4} are above all interesting if $k$ is of characteristic~$p\neq0$, taking $A$ to be a discrete valuation ring \emph{of characteristic~$0$}, with residue field $k$;
      for example, the ``smallest possible $A$'', i.e. that for which $p$ generates the maximal ideal.
      (In fact, by theorems of Cohen, it suffices to have \hyperref[theorem9corollary3]{Corollary~3} and \hyperref[theorem9corollary4]{Corollary~4} for such a ring $A$).
      We note that, concerning this point, according to the specialists, we do not keow if there exist schemes over a field $k$ that are not reductions $\mod p$ of a flat scheme defined over such a ring $A$.
      At the least, the results of this section give a way of systematically investigating this question.
      We must start
\oldpage{182-15}
      by seeing if the first obstruction that we have, in $H^2(X_0,\fk{G}_{X_0/k})$, is necessarily zero.
    \item We note that \cref{theorem3}, and the corresponding technique, only works for a \emph{complete} (local, for simplicity) base ring.

      In order to go from known results concerning the completion of a local ring to the corresponding results for the local ring itself, we would need a fourth ``fundamental theorem'', whose precise statement still needs to be found.
    \item We will compare the results from this section (mainly the above \hyperref[theorem9corollary1]{Corollary~1} and \hyperref[theorem9corollary2]{Corollary~2}), as well as those from the following, with the results of Kodaira--Spencer on the variation of complex structures.
      Using the conjectural theorem to which we have just alluded, we should be able to conclude, under the conditions of \hyperref[theorem9corollary1]{Corollary~1}, but where $A$ is no longer assumed to be complete, that there exists a ring $A'$ that contains $A$, and that is finite and unramified over $A$, such that $X\otimes_AA'$ and $X'\otimes_AA'$ are $A'$-isomorphic (where $X$ and $X'$ are given, and are proper flat $A$-schemes such that $X\otimes_Ak=X'\otimes_Ak=X_0$).
      This is what we can prove, at least, when $X_0=\fk{P}_k^r$, by using \hyperref[proposition2corollary1]{Corollary~1} of \cref{proposition2}.
      In any case, if $H^1(X_0,\fk{G}_{X_0/k})=0$, then we can prove that the fibres of $X$ and $X'$ over any point $y$ of $Y=\Spec(A)$ are isomorphic, or at least when we pass to the algebraic closure of the residue field $\kres(y)$.
      (We have a local result, seemingly stronger, when we don't suppose that $A$ is necessarily local).
      As for ``structure variations'' of the projective space, we again point out the following question, suggested by a corresponding problem of Kodaira--Spencer.
      Let $X$ be a proper flat scheme, over a local integral ring $A$ with field of fractions $K$ and residue field $k$, and suppose that $X\otimes_AK$ is isomorphic to $\fk{P}_K^r$.
      Is it then true that $X\otimes_Ak=X_0$ is isomorphic to $\fk{P}_k^R$ (or at least, over the algebraic closure of $k$)?
      In this question, we can assume that $A$ is a complete discrete valuation ring.
      There is an analogous question when $X_0$ is an abelian variety.
  \end{enumerate}
\end{remarks}

\begin{remark}
  \emph{[Comp.] Concerning Remark~1) above.}
  We note that J.-P. Serre has constructed a non-singular projective variety, of dimension~$3$, over an algebraically closed field $k$, of characteristic~$p>0$, which does not come from reduction of a proper scheme over a local integral ring with residue field $k$, and having a field of fractions of characteristic~$0$.\footnote{\textsc{Serre, J.-P.} ``Exemples des vari\'{e}t\'{e}s projectives en caract\'{e}ristique $p$ non relevables en caract\'{e}ristique z\'{e}ro''. \emph{Proc. Nat. Acad. Sc. U.S.A.} \textbf{47} (1961), 108--109.}
  Mumford would have found an analogous result, with a non-singular projective \emph{surface}.
\end{remark}

\begin{remark}
  \emph{[Comp.] Concerning Remarks~2) and 3) above.}
  I am now less optimistic concerning the results conjectured here.
  However, the question concerning structure variations for projective space, mentioned at the end of Remark~3) above, has been positively resolved by Hironaka, and the analogous question for abelian varieties has been resolved by Koizumi.
\end{remark}


\section{Application to the ``theory of modules''}
\label{section7}

Since the speaker has only recently encountered this theory himself, we will be obliged to limit ourselves to just cursory remarks.
For simplicity, we work over a \emph{field} $k$, i.e. we work in \emph{equal characteristic}, even though \cref{theorem8} allows us to also discuss the more general case, without any fundamental changes, so it seems.
We have not yet gotten past the ``formal'' stage, but the speaker still hopes to be able to construct true schemes of modules
\oldpage{182-16}
in certain cases from this, and, in particular, construct, for every integer $g$, a scheme over the integers that plays the role of universal scheme of modules for the simple curves of genus~$g$.

\begin{remark}
  \emph{[Comp.]}
  We note that Mumford has recently constructed schemes of modules for the curves of genus~$g$ (cf. \emph{Mumford--Tate seminar}, Harvard University, 1961--62).
  \Cref{theorem10} also proves that the ``Jacobi schemes of level~$n$\todo'' from the theory of modules are non-singular (and even simple over $\underline{\ZZ}$).
\end{remark}

We continue to use the setting and notation of \cref{theorem9}, and now suppose that $A$ is a local algebra of finite rank over $k$, which is assumed to be algebraically closed, for simplicity.
Then $F(A)$ can be thought of as a covariant functor in $A$, with values in the category of sets, with a homomorphism $A\to B$ of $k$-algebras defining a map $F(A)\to F(B)$, since every flat $A$-scheme $X$ with $X\otimes_Ak=X_0$ gives rise to a $B$-scheme $X\otimes_AB$ with the same properties.
\emph{Suppose} that we can find a complete local Noetherian $k$-algebra $\cat{O}$, as well as a functorial isomorphism
\[
  \label{isomorphism*}
  \Hom(\cat{O},A) \xrightarrow{\sim} F(A)
  \tag{$*$}
\]
(where the left-hand side denotes homomorphisms of $k$-algebras).
We can easily see that such an $\cat{O}$ is determined up to canonical isomorphism, and so we call the formal spectrum $\fk{Y}$ of $\cat{O}$ (i.e. the topological space consisting of a single point, endowed with a sheaf of topological rings consisting of just $\cat{O}$) the \emph{formal scheme of modules for $X_0$}.
(Note that it does not necessarily exist).
Let $\fk{r}$ be the maximal ideal of $\cat{O}$, and, for all $n$, let $\sh{O}_n=\cat{O}/\fk{r}^{n+1}$ (so that $\sh{O}_0=k$).
Then the canonical homomorphism $\cat{O}\to\sh{O}_n$ is an element of $\Hom(\cat{O},\sh{O}_n)$, and thus defines an element of $F(\sh{O}_n)$, i.e. a flat $\sh{O}_n$-scheme $X_n$ whose restriction $\mod\fk{r}$ is $X_0$.
These $X_n$ are induced from one another by extension of scalars (i.e. here by reductions), whence it follows that they come from a formal scheme $\fk{X}$ that is well determined by the formal scheme of modules $\fk{Y}$;
further, $\fk{X}$ is flat over $\fk{Y}$, and $\fk{X}_0=X_0$.
The \hyperref[isomorphism*]{isomorphism~($*$)} is then given, as we can immediately see, by associating to each homomorphism $\cat{O}\to A$ of $k$-algebras the class of the $A$-scheme $\fk{X}\otimes_{\cat{O}} A$ (i.e. to every morphism $\fk{Y}'=\Spec(A)\to\fk{Y}$ of $k$-schemes, we associate the $\fk{Y}'$-scheme $\fk{X}\otimes_{\fk{Y}}\fk{Y}'$ given by base change).
Furthermore, we see that the \hyperref[isomorphism*]{isomorphism~($*$)} and its above description still hold even if we only suppose that $A$ is a complete local Noetherian $k$-algebra (not necessarily Artinian).
Of course, as always, $\cat{O}$ can indeed \emph{a priori} have nilpotent elements, and it seems likely that there should exist cases where $\cat{O}$ is itself Artinian, without being identical to $k$.
This tells us at which point the point of view of Kodaira--Spencer (restricting to considering the $A$ that are \emph{regular} rings) is \emph{a priori} inadequate in the general case.

\oldpage{182-17}
It remains to give sufficient conditions for there to exist a formal scheme of modules for $X_0$, assumed to be proper over $k$.
Generally, it is easy to give simple necessary and sufficient conditions on a functor $A\to F(A)$ (from local $k$-algebras of finite rank to sets) in order for it to be of the form $\Hom(\cat{O},A)$ for some suitable $\cat{O}$.
We do not give the details here.
We point out only that, in the case which we are studying, these conditions impose non-trivial conditions of a cohomological nature on $X_0$, and it seems unlikely that they will always be satisfied, even though the speaker has not constructed any counterexamples.
It seems plausible, however, that the condition $H^0(X,\fk{G}_{X_0/k})=0$ is \emph{sufficient} (even if not at all necessary) in order to guarantee the existence of a formal scheme of modules.
We restrict ourselves to stating here a theorem that deals with a particularly simple case (whose analogue in the theory of analytic spaces is well known, cf. Kodaira--Spencer), which can easily be proven using the results from the previous section:

\begin{theorem}{10}
\label{theorem10}
  Let $X_0$ be a simple proper scheme over the field $k$ such that
  \[
    H^0(X_0,\fk{G}_{X_0/k}) = H^2(X_0,\fk{G}_{X_0/k}) = 0.
  \]
  Then there exists a formal scheme of modules for $X_0$, corresponding to a local regular ring $\cat{O}$ (i.e. an algebra of formal series over $k$).
\end{theorem}

As we have already pointed out, it is not true in general that the formal scheme $\fk{X}$ over $\cat{O}$ is algebraisable;
but we know that this is true, however, when $X_0$ is projective and $H^2(X_0,\sh{O}_{X_0})=0$ (\cref{theorem4}), such as when $X_0$ is of dimension~$1$.
This is what gives some hope of constructing a scheme of modules over the integers for curves of a given genus\ldots{}.

Note also that methods such as those described in this section can be applied in the construction and study of Picard varieties, as well as in many other constructions.
We will return to this soon.


\section{Application to the fundamental group}
\label{section8}

The techniques described allows us to tackle the system study of the fundamental group, using the example of topological theory.
The first two theorems stated in this section are generalisations of results in a recent work by Lang--Serre.

Let $X$ be a scheme.
Then an $X$-scheme $X'$ is said to be an \emph{unramified covering of $X$}
\oldpage{182-18}
if
\begin{enumerate}[i.]
  \item $X'$ is finite over $X$, i.e. it is defined by a coherent sheaf of algebras $\sh{A}=\sh{A}(X')$ on $X$;
  \item $\sh{A}$ is a locally free sheaf on $X$;
  \item for all $x\in X$, the quotient $\sh{A}_x/\fk{m}_x\sh{A}_x = \sh{A}_x\otimes_{\sh{O}_X}\pi(x)$ is a separable algebra over $\kres(x)$.
\end{enumerate}

This notion of unramified covering (due to Serre and the speaker) posses all the elementary properties for which we can reasonably hope, and which we will not list.
We restrict ourselves to saying that it gives rise to a \emph{Galois theory} modelled on classical Galois theory (and containing it; the proofs being overall simpler than the proofs generally seen for the latter) and the Galois theory of topological coverings.
More precisely, we define a \emph{geometric point} of a scheme $X$ to be a morphism $a$ from the spectrum $\xi$ of an algebraically closed field $\Omega$ to $X$, i.e. the data of an algebraically closed extension of the residue field $\kres(x)$ of a point $x=|a|$ of $X$ (called the \emph{locality} of the geometric point $a$).
If $X'$ is an unramified covering of $X$, then we can associate to it the set $E_a(X')$ of ``geometric points of $X'$ over $a$'', i.e. the set of pairs consisting of an $x'\in X'$ over $x$ and a $\kres(x)$-homomorphism to $\Omega$.
We thus obtain (for fixed $(X,a)$) a functor $F(X,a)$ from the category $R(X)$ of unramified coverings $X'$ of $X$ to the category of finite sets.
If $X$ is connected, then the pair given by $R(X)$ and $F(X,a)$ has all the formal properties necessary in order to be isomorphic to the analogous pair defined by a suitable totally disconnected compact topological group $\pi$ (i.e. a projective limit of finite groups): we take the category $\cat{C}(\pi)$ of finite sets on which $\pi$ acts \todo continuously, and the identity functor $F(\pi)(E)=E$ from this category to the category of finite sets.
The group $\pi$ is also determined up to canonical isomorphic by the condition that $(\cat{C}(\pi),F(\pi))$ is isomorphic to a given pair.
To be precise, $\pi$ is called the \emph{fundamental group of the connected scheme $X$ at the geometric point $a$}, and we denote it by $\pi_1(X,a)$.
If $X$ is not connected, then we can replace it by the connected component containing $x=|a|$.
If, however, $X$ is connected, then the groups $\pi_1(X,a')$ and $\pi_1(X,a')$ are isomorphic for any two geometric points $a'$ and $a''$ of $X$ (with the isomorphism being determined up to inner automorphism), and thus we can, as per usual, choose the most suitable $a$ for our purposes, such as the generic point of
\oldpage{182-19}
$X$ that is assumed to be irreducible.
Of course, $\pi_1(X,a)$ is a \emph{covariant functor in the pointed scheme $(X,a)$}.
Every statement concerning the classification of inseparable coverings can then be translated into the language of group theory, following the well-known dictionary (except that we must take into account the fact that here we have \emph{topological groups}).

Our goal is to develop an analogue of the homotopy exact sequence of fibre bundles, relative to a proper morphism $f\colon X\to Y$.
Clearly, since we don't know what the higher homotopy groups are, we will only have necessarily incomplete results.
In order to be able to apply the fundamental theorems from \cref{section3}, we must first explain certain elementary lemmas concerning schemes over Artinian rings or fields (following the general procedure!).

\begin{lemma}{1}
\label{lemma1}
  Let $(X',a')$ be a pointed unramified covering associated to a pointed representation of $\pi_1(X,a)$ in a finite set $E$ (endowed with a marked point $e$),
  Then the canonical morphism $\pi_1(X',a')\to\pi_1(X,a)$ identifies the domain with the stabiliser of $e$ in $\pi_1(X,a)$ (and is thus injective).
\end{lemma}

\begin{lemma}{2}
\label{lemma2}
  Let $X$ be an algebraic scheme over the field $k$, and let $k'$ be a radicial extension of $k$.
  Then every unramified covering of $X\otimes_kk'$ is given by the inverse image (i.e. extension of scalars) of an unramified covering of $X$, determined up to isomorphism.
\end{lemma}

It follows, in particular, from these two lemmas that, for \emph{every} algebraic extension $K$ of $k$, and every geometric point $a'$ of $X'=X\otimes_kK$ that projects to the geometric point $a$ of $X$, that the functorial homomorphism $\pi_1(X',a')\to\pi_1(X,a)$ is injective.

\begin{lemma}{3}
\label{lemma3}
  Let $X$ be a complete scheme over a local Artinian ring $A$, such that $H^0(X,\sh{O}_X)=A$.
  Let $X'$ be an unramified covering of $X$, and let $A'=H^0(X',\sh{O}_{X'})$, which is thus a ring that is finite over $A$ (and which may a priori be ramified over $A$).
  Let $X_0$ and $X'_0$ be the reduced subschemes associated to $X$ and $X'$, respectively (obtained by \todo by the sheaves of nilpotent elements in $\sh{O}_X$ and $\sh{O}_{X'}$, respectively).
  Let $k$ be a subfield of $A/\fk{r}(A)$ over which $A/\fk{r}(A)$ is finite (so $X_0$ is a complete algebraic scheme over $k$, and $X'_0$ is an unramified covering).
  Finally, let $\Omega$ be an algebraically closed extension of $k$, and consider the unramified covering $X'_0\otimes_k\Omega$ of $X_0\otimes_k\Omega$.
  Then the following two conditions are equivalent:
\oldpage{182-20}
  \begin{enumerate}[i.]
    \item $X'_0\otimes_k\Omega$ is complete \todo over $X_0\otimes_k\Omega$;
    \item the natural morphism $X'\to X\otimes_AA'$ is an isomorphism.
  \end{enumerate}

  Under these conditions, $A'$ is an \emph{unramified} extension of $A$.
  Finally, if $X'$ is connected, then condition~(i) is equivalent to the following, seemingly weaker, condition:
  \begin{enumerate}[i bis.]
    \item $X'_0\otimes_k\Omega$ admits a regular section over $X_0\otimes_k\Omega$.
  \end{enumerate}
\end{lemma}

When condition~(ii) is satisfied, we say that the unramified covering $X'$ of $X$ is \emph{geometrically trivial}.

\begin{lemma}{4}
\label{lemma4}
  Let $f\colon X\to Y$ be a proper morphism such that $f_*(\sh{O}_X)=\sh{O}_Y$.
  Let $a$ be a geometric point of $X$, and $b$ its projection over $Y$.
  Then $\pi_1(X,a)\to\pi_1(Y,b)$ is \emph{surjective}.
\end{lemma}

What we need to show is effectively the following: if an unramified covering $Y'$ of $Y$ (corresponding to a locally free sheaf of algebras $\sh{A}$) is such that $X\otimes_YY'$ is disconnected, then $Y'$ is also disconnected.
In fact, $\sh{A}$ (\todo \emph{\textbf{right?}}) is then the direct sum of two non-zero sheaves of rings, and thus so too is its direct image, which is exactly $\sh{A}\otimes f_*(\sh{O}_X)=\sh{A}$.

\begin{lemma}{5}
\label{lemma5}
  Let $X$ be a complete scheme over a field $k$, and suppose that $H^0(X,\sh{O}_X)$ is a local ring $A$, and that $A/\fk{r}(A)$ is radicial over $k$.
  Let $\Omega$ be an algebraic closure of $k$, and let $\overline{X}=X\otimes_k\Omega$ (which is connected).
  Pick a geometric point $\overline{a}$ of $\overline{X}$ that projects to the geometric point $a$ of $X$.
  Then we have an exact sequence
  \[
    e
    \to \pi_1(\overline{X},\overline{e})
    \to \pi_1(X,a)
    \to \pi_1(k,b)
    \to e
  \]
  (where $\pi_1(k,b)$ is the Galois group of $\Omega$ over $k$).
\end{lemma}

The fact that the first homomorphism is injective has already been shown with \cref{lemma1} and \cref{lemma2};
the exactness in the middle follows from \cref{lemma3};
finally, the surjectivity of the last homomorphism (which is the only thing to rely on the fact that $A/\fk{r}(A)$ is radicial) follows from \cref{lemma4}.

\begin{proposition}{4}
\label{proposition4}
  Let $f\colon X\to Y$ be a proper flat morphism such that, for all $y\in Y$, the algebra $H^0(f^{-1}(y),\sh{O}_{f^{-1}(y)})$ is separable over the residue field $\kres(y)$ (which is the case, for example, if $f^{-1}(y)$ is a \emph{separable scheme} over $\kres(y)$, i.e. reduced and such that the fields corresponding to its irreducible components are separable extensions of $\kres(y)$).
  Then the covering
\oldpage{182-21}
  $Y'$ of $Y$ associated to $f_*(\sh{O}_X)$ is unramified.
\end{proposition}

The proof is easy, thanks to \cref{theorem2}.

This proposition, combined with \cref{lemma1}, practically reduces the homotopical study of proper and flat morphisms (with separable fibres) to the case where $f_*(\sh{O}_X)=\sh{O}_Y$ (since, using Stein factorisation, we can replace $Y$ by $Y'$).

\begin{remark}
  A flat morphism of finite type whose fibres are separable (resp. simple) schemes is said to be \emph{separable} (resp. \emph{simple}).
  We show that, if $f$ is flat and if $f^{-1}(y)$ is separable (resp. simple) then there exists a neighbourhood of $f^{-1}(y)$ on which $f$ is separable (resp. simple).
  The same result holds true for ``absolutely normal'' (this is \emph{Bertini's theorem}).
\end{remark}

Let $f\colon X\to Y$ be a proper morphism such that
\[
\label{equationi}
  f_*(\sh{O}_X) = \sh{O}_Y
\tag{i}
\]
and let $X'$ be a finite scheme over $X$.
Let $Y'$ be the covering of $Y$ corresponding to the Stein factorisation of $X'\to Y$ (cf. \cref{theorem5}).
Let $y\in Y$, so that the set of connected components of the fibre $F'$ of $X'$ over $y$ can be identified with the set of points $y'\in Y'$ over $y$ (\cref{theorem5}).
Consider the evident morphism
\[
\label{equation*}
  X'\to X\times_Y Y'
\tag{$*$}
\]
induced by the natural morphisms $X'\to X$ and $X\to Y'$;
this will be an isomorphism whenever $X'$ is of the form $X\times_Y Y''$, where $Y''$ is an unramified covering of $Y$, and then $Y'$ will be exactly $Y''$, and \hyperref[equation*]{equation~($*$)} will be the identity.
We wish to precisely give the conditions for which $X'$ is of the form that we have just indicated, i.e. such that $Y'$ is unramified and \hyperref[equation*]{equation~($*$)} is an isomorphism.
For this, we introduce the fibre $F$ of $X$ at $y$, which is a proper scheme over $\kres(y)$, for which $F'$ is a cover (an unramified one if $X$ is).
Let $F'_1$ be a connected component of $F'$ corresponding to a point $y'_1$ of $Y'$ over $y$.
Suppose further that
\begin{enumerate}[(i)]
\setcounter{enumi}{1}
  \item $X'$ is unramified over $X$ at the points of $F'_1$ (and thus $F'_1$ is an unramified cover of $F$), and
  \item $F'_1$ is a geometrically trivial covering of $F$ (cf. \cref{lemma3}).
\end{enumerate}

\begin{theorem}{11}
\label{theorem11}
  Under these conditions, there exists an open neighbourhood $U'$ of $y'_1$
\oldpage{182-22}
  in $Y'$ such that \hyperref[equation*]{equation~($*$)} is an isomorphism over $U'$.
  Furthermore, $Y'$ is unramified at $y'_1$ over $Y$ (but can be ramified at other points $y'$ of $Y'$ over $y$).
\end{theorem}

Of course, conditions~(ii) and (iii) are also necessary for the conclusion of the theorem.
The proof of the theorem is easy, thanks to \cref{lemma3} and \cref{theorem2}.

\begin{corollary}{1}
\label{theorem11corollary1}
  Suppose that (i) is still satisfied.
  For an unramified covering over $X$ to be isomorphic to the inverse image of an unramified covering $Y'$ of $Y$, it is necessary and sufficient for $X'$ to induce, on each fibre $f^{-1}(y)$, a geometrically trivial covering.
\end{corollary}

By \cref{theorem11}, the set of points of $Y$ for which this condition is satisfied is open, and so it suffices to verify it at the points $y$ which are closed\ldots.
Note that the following statement is equivalent to \cref{theorem11corollary1}:
\begin{quote}
  \itshape
  The kernel of the homomorphism $\pi_1(X)\to\pi_1(Y)$ (which is surjective, by \cref{lemma4}) is the closed invariant subgroup generated by the images in $\pi_1(X)$ of $\pi_1(f^{-1}(y))$, where $f^{-1}(y)$ denotes the scheme $f^{-1}(y)\otimes_{k(y)}\overline{\kres(y)}$ (where $\overline{\kres(y)}$ denotes an algebraic closure of $\kres(y)$).
\end{quote}
We note that, since we cannot \emph{choose} the same base point for all the fibres, the homomorphisms $\pi_1(f^{-1}(y))\to\pi_1(X)$ are determined (after having picked a base point for $X$, and then for $Y$) only up to composition with an inner automorphism of $\pi_1(X)$.

\begin{corollary}{2}
\label{theorem11corollary2}
  Under the general conditions of \cref{theorem11}, suppose further that $Y$, $X$, and $X'$ are integral, and let $K$, $L$, and $L'$ be their fields (respectively).
  Then there exists a separable sub-extension $K'$ of $K$ in $L'$, linearly disjoint from $L$, such that $L'=LK'$ (whence $L'=L\otimes_KK'$).
\end{corollary}

(We apply the last part of \cref{lemma3} to the generic fibre of $X$).
The most interesting case in which we can apply \cref{theorem11} is when $f$ is a \emph{separable} morphism.
Then $X'$ is also separable over $Y$, and so, by \cref{proposition4}, $Y'$ is unramified over $Y$, and so the right-hand side $X\times_YY'$ in \hyperref[equation*]{equation~($*$)} is unramified over $X$.
From this, we easily conclude:

\begin{corollary}{3}
\label{theorem11corollary3}
  Suppose, in addition to (i), that $f$ is separable.
  Let $X'$ be a connected unramified covering of $X$.
  For $X$ to be the inverse image of an unramified covering $Y'$ of $Y$, it is necessary and sufficient for the induced covering $\overline{F}'$ of a geometric fibre $\overline{F}=\overline{f^{-1}(y)}$ to admit a regular section.
\end{corollary}

Note that it was not necessary to suppose that $\overline{F}'$ be \emph{geometrically}
\oldpage{182-23}
trivial over $\overline{F}$ (which will be true \emph{a posteriori}, even though \emph{a priori} this condition is a lot stronger).
In, \hyperref{theorem11corollary3}[Corollary~3] is equivalent to the following statement:

\begin{corollary}{4}
\label{theorem11corollary4}
  Let $f\colon X\to Y$ be a proper and separable morphism such that $f_*(\sh{O}_X)=\sh{O}_Y$.
  Let $\overline{F}$ be the geometric fibre of a point $y\in Y$, and pick a geometric point in $\overline{F}$, which, by the morphisms $\overline{F}\to X\to Y$, gives geometric points in $X$ and $Y$; we take these three points as base points for the fundamental groups of $\overline{F}$, $X$, and $Y$, respectively.
  Under these conditions, we have the exact sequence
  \[
    \boxed{\pi_1(\overline{F}) \to \pi_1(X) \to \pi_1(Y) \to 0.}
  \]
\end{corollary}

From this, we easily deduce the two following statements of Serre--Lang, with all normality hypotheses removed:

\begin{corollary}{5}
\label{theorem11corollary5}
  Let $X$ and $Y$ be connected schemes that are proper over a field $k$, and suppose that the reduced scheme $X_\red$ is separable over $k$ (which is automatically true if $k$ is perfect) and complete.
  Pick a geometric point $a$ (resp. $b$) in $X$ (resp. $Y$); this gives a geometric point $c=(a,b)$ in $X\times_kY$, and a natural morphism
  \[
    \pi_1(X\times_kY,c) \to \pi_1(X,a)\times\pi_1(Y,b)
  \]
  (induced by the functorial morphisms from $\pi_1(X\times Y,c)$ to $\pi_1(X,a)$ and $\pi_1(Y,b)$).
  This morphism is injective, and further bijective if $k$ is algebraically closed.
\end{corollary}

(The surjectivity in the above claim is almost trivial).
We thus deduce, with Serre--Lang:

\begin{corollary}{6}
\label{theorem11corollary6}
  Let $X$ be a connected algebraic scheme over an algebraically closed field $k$, and let $K$ be an algebraically closed extension of $k$.
  Then the fundamental groups of $X$ and $X\times_kK$ are the same, i.e. every unramified covering of the latter scheme is given by extension of scalars of an unramified covering (which is unique up to isomorphism) of $X$.
\end{corollary}

\begin{remarks}
  \begin{enumerate}[1)]
    \item Using \cref{proposition4}, we see that the hypothesis that $f_*(\sh{O}_X)=\sh{O}_Y$ in \hyperref[theorem11corollary4]{Corollary~4} is not essential.
      In the general case, instead of putting the trivial group $e$ after $\pi_1(Y)$, one must continue by $\pi_0(\overline{F})\to\pi_0(X)\to\pi_0(Y)\to e$,
\oldpage{182-24}
      as in algebraic topology.
    \item In general, we cannot say anything at the moment about the kernel of $\pi_1(\overline{F})\to\pi_1(X)$, although it should involve a $\pi_2(Y)$.
      It seems, however, that we should be able to prove that $\pi_1(\overline{F})\to\pi_1(X)$ is \emph{injective} if $Y$ is the spectrum of a local ring $A$, by appealing to \cref{theorem12} below (which shows that this is the case if $A$ is \emph{complete}).
  \end{enumerate}
\end{remarks}

\Cref{theorem11} used only \cref{theorem1} and \cref{theorem2}.
We will now use \cref{theorem3}, along with the following elementary lemma:

\begin{lemma}{6}
\label{lemma6}
  Let $X$ be a scheme, and $X_0$ the corresponding reduced scheme (i.e. where we have killed all the nilpotent elements).
  Then every unramified covering $X'_0$ of $X_0$ is induced by an unramified covering $X'$ of $X$, determined up to isomorphism.
\end{lemma}

This lemma, which is of a purely local nature, plays a role analogous to that of \cref{theorem8} here, in the theory of modules.
Combining it with the existence theorem (\cref{theorem3}), we obtain:

\begin{theorem}{12}
\label{theorem12}
  Let $A$ be a complete local Noetherian ring with residue field $k$.
  Let $X$ be a proper scheme over $A$.
  Then every unramified covering $X'_0$ of $X_0=X\otimes_Ak$ is induced by an unramified covering $X'$ of $X$, unique up to isomorphism.
\end{theorem}

In other words:

\begin{corollary}{1}
\label{theorem12corollary1}
  Pick a geometric point in $X_0$ as the base point for the fundamental groups of $X_0$ and $X$.
  Then the canonical homomorphism $\pi_1(X_0)\to\pi_1(X)$ is an \emph{isomorphism}.
\end{corollary}

Applying \cref{lemma5} to $X_0$ (supposing that $H^0(X_0,\sh{O}_{X_0})=k$, for simplicity), and noting that, since $A$ is complete, the unramified extensions of $A$ correspond to unramified extensions of its residue field, i.e. $\pi_1(Y)=\pi_1(k)$ (where $Y=\Spec(A)$).
We obtain the exact sequence:
\[
  e \to \pi_1(\overline{X_0}) \to \pi_1(X) \to \pi_1(Y) \to e.
\]

\begin{corollary}{2}
\label{theorem12corollary2}
  Let $f\colon X\to Y$ be a proper flat morphism, and let $y_1$ be a point of $Y$, and $y_0$ a specialisation of $y_1$.
  Consider the corresponding ``geometric'' fibres $\overline{X_1}$ and $\overline{X_0}$, and suppose that $\overline{X_0}$ is separable and connected (which implies that $\overline{X_1}$ satisfies the same conditions).
  Then we can find a group homomorphism $\pi_1(X_1)\to\pi_1(X_0)$, defined up to inner automorphism.
  Further, this homomorphism is \emph{surjective}.
\end{corollary}

\oldpage{182-25}
We might hope that this homomorphism is always \emph{bijective}.
Unfortunately, this is not the case in general if $\kres(y_0)$ is of characteristic~$>0$.
We will, however, obtain below a \todo of the kernel of this homomorphism (at least in the case where $\overline{X_0}$ is simple), implying that, if $\kres(y_0)$ is of characteristic~$0$, then the above homomorphism is bijective (which is a result that we can also prove by transcendentality).
At the very least, we already have, in any case, a \todo of $\pi_1$ by \todo.
Using, for example, the fact that an algebraic curve in characteristic~$p$ lifts to a curve in characteristic~$0$ (\hyperref[theorem9corollary4]{Corollary~4} of \cref{theorem9}), we obtain, by transcendentality:

\begin{corollary}{3}
\label{theorem12corollary3}
  Let $X_0$ be the scheme of complete simple curve over an algebraically closed field of arbitrary characteristic, and let $g$ be the genus of $X_0$.
  Then $\pi_1(X_0)$ admits $2g$ topological generators, related by the well-known relation.
\end{corollary}

We thus deduce, by a well-known technique using hyperplane sections:

\begin{corollary}{4}
\label{theorem12corollary4}
  Let $X$ be a simple projective scheme over an algebraically closed field of arbitrary characteristic.
  Then $\pi_1(X)$ admits a finite number of topological generators.
\end{corollary}


%% Bibliography %%

\nocite{*}
\begin{thebibliography}{4}

  \bibitem{1}
  {\sc Dieudonn\'{e}, J. and Grothendieck, A.}
  \newblock El\'{e}ments de g\'{e}om\'{e}trie alg\'{e}brique.
  \newblock {\em Publications math\'{e}matiques de l'Institut des Hautes Etudes Scientifiques} (to appear).

  \bibitem{2}
  {\sc Grothendieck, A.}
  \newblock The cohomology theory of abstract algebraic varieties.
  \newblock {\em International Congress of Mathematicians, 1958, Edinburgh} (to appear).

  \bibitem{3}
  {\sc Serre, J.-P.}
  \newblock Faisceux alg\'{e}briques coh\'{e}rents.
  \newblock {\em Annals of Math.} {\bf 61} (1955), 197--278.

  \bibitem{4}
  {\sc Serre, J.-P.}
  \newblock G\'{e}om\'{e}trie alg\'{e}brique et g\'{e}om\'{e}trie analytique.
  \newblock {\em Ann. Institut Fourier Grenoble} \textbf{6} (1955--56), 1--42.

\end{thebibliography}

\end{document}

\documentclass{article}

\title{Techniques of descent and existence theorems in algebraic geometry, I\\\emph{Generalities, and descent by faithfully flat morphisms}}
\author{A. Grothendieck}
\date{December 1959}

\usepackage{amssymb,amsmath}

\usepackage{hyperref}
\usepackage{xcolor}
\hypersetup{colorlinks,linkcolor={red!50!black},citecolor={blue!50!black},urlcolor={blue!80!black}}
\usepackage[nameinlink]{cleveref}
\usepackage{enumerate}
\usepackage{tikz-cd}
\usepackage{graphicx}

\usepackage{mathrsfs}
%% Fancy fonts --- feel free to remove! %%
\usepackage{Baskervaldx}
\usepackage{mathpazo}


\usepackage{fancyhdr}
\usepackage{lastpage}
\usepackage{xstring}
\makeatletter
\ifx\pdfmdfivesum\undefined
  \let\pdfmdfivesum\mdfivesum
\fi
\edef\filesum{\pdfmdfivesum file {\jobname}}
\pagestyle{fancy}
\makeatletter
\let\runauthor\@author
\let\runtitle\@title
\makeatother
\fancyhf{}
\lhead{\footnotesize\runtitle}
\rhead{\footnotesize Version: \texttt{\StrMid{\filesum}{1}{8}}}
\cfoot{\small\thepage\ of \pageref*{LastPage}}


\crefname{section}{\S\!}{\S\S\!}
\crefname{equation}{}{}
\renewcommand{\thesection}{\Alph{section}}
\renewcommand{\thesubsection}{\arabic{subsection}}
\renewcommand{\thesubsubsection}{(\alph{subsubsection})}


%% Theorem environments %%

\usepackage{amsthm}

\theoremstyle{plain}

\newtheorem{innercustomtheorem}{Theorem}
\crefname{innercustomtheorem}{Theorem}{Theorems}
\newenvironment{theorem}[1]
  {\renewcommand\theinnercustomtheorem{#1}\innercustomtheorem}
  {\endinnercustomtheorem}

\newtheorem{innercustomproposition}{Proposition}
\crefname{innercustomproposition}{Proposition}{Propositions}
\newenvironment{proposition}[1]
  {\renewcommand\theinnercustomproposition{#1}\innercustomproposition}
  {\endinnercustomproposition}

\newtheorem{innercustomlemma}{Lemma}
\crefname{innercustomlemma}{Lemma}{Lemmas}
\newenvironment{lemma}[1]
  {\renewcommand\theinnercustomlemma{#1}\innercustomlemma}
  {\endinnercustomlemma}

\newtheorem{innercustomcorollary}{Corollary}
\crefname{innercustomcorollary}{Corollary}{Corollaries}
\newenvironment{corollary}[1]
  {\renewcommand\theinnercustomcorollary{#1}\innercustomcorollary}
  {\endinnercustomcorollary}


\theoremstyle{definition}

\newtheorem{innercustomdefinition}{Definition}
\crefname{innercustomdefinition}{Definition}{Definitions}
\newenvironment{definition}[1]
  {\renewcommand\theinnercustomdefinition{#1}\innercustomdefinition}
  {\endinnercustomdefinition}

\newtheorem{innercustomexample}{Example}
\crefname{innercustomexample}{Example}{Examples}
\newenvironment{example}[1]
  {\renewcommand\theinnercustomexample{#1}\innercustomexample}
  {\endinnercustomexample}

\newtheorem*{remark}{Remark}
\newtheorem*{remarks}{Remarks}

%% Shortcuts %%

\newcommand{\sh}[1]{{\mathscr{#1}}}
\newcommand{\cat}[1]{{\mathcal{#1}}}

\renewcommand{\geq}{\geqslant}
\renewcommand{\leq}{\leqslant}

\DeclareMathOperator{\id}{id}
\DeclareMathOperator{\Hom}{Hom}

\newcommand{\todo}{\textbf{ !TODO! }}
\newcommand{\oldpage}[1]{\marginpar{\footnotesize$\Big\vert$ \textit{p.~#1}}}


%% Document %%

\usepackage{embedall}
\begin{document}

\maketitle
\thispagestyle{fancy}

\renewcommand{\abstractname}{Translator's note.}

\begin{abstract}
  \renewcommand*{\thefootnote}{\fnsymbol{footnote}}
  \emph{This text is one of a series\footnote{\url{https://thosgood.com/translations/}} of translations of various papers into English.}
  \emph{The translator takes full responsibility for any errors introduced in the passage from one language to another, and claims no rights to any of the mathematical content herein.}

  \medskip
  
  \emph{What follows is a translation of the French seminar talk:}

  \medskip\noindent
  \textsc{Grothendieck, A.}
  Technique de descente et th\'{e}or\`{e}mes d'existence en g\'{e}om\'{e}trie alg\'{e}brique. I. G\'{e}n\'{e}ralit\'{e}s. Descente par morphismes fid\`{e}lement plats.
  \emph{S\'{e}minaire Bourbaki}, Volume~\textbf{12} (1959--60), Talk no.~190.
\end{abstract}

\setcounter{footnote}{0}

\setcounter{tocdepth}{3}
\tableofcontents


%% Content %%

\subsubsection*{}

\emph{[Trans.] We have made changes throughout the text following the errata (\emph{S\'{e}minaire Bourbaki} \textbf{14}, 1961--62, Compl\'{e}ment), and preface them with ``[Comp.]''.}
\medskip

\textbf{\todo(errata)}

\oldpage{190-01}
From a technical point of view, the current talk, and those that will follow, can be considered as variations on the celebrated ``Theorem 90'' of Hilbert.
The introduction of the method of descent in algebraic geometry seems to be due to A.~Weil, under the name of ``descent of the base field''.
Weil considered only the case of separable finite field extensions.
The case of radicial extensions of height~$1$ was then studied by P.~Cartier.
Lacking the language of schemes, and, more particularly, lacking nilpotent elements in the rings that were under consideration, the essential identity between these two cases could not have been formulated by Cartier.

Currently, it seems that the general technique of descent that will be explained (combined with, when necessary, the fundamental theorems of ``formal geometry'', cf. \cite{3}) is at the base of the majority of existence theorems in algebraic geometry.
It is worth noting as well that this aforementioned technique of descent can certainly be transported to ``analytic geometry'', and we can hope that, in the not-too-distant future, the specialists will know how to prove the ``analytic'' analogues of the existence theorems in formal geometry that will be given in talk~II.
In any case, the recent work of Kodaira--Spencer, whose methods seem unfit for defining and studying ``varieties of modules'' in the neighbourhood of their singular points, show reasonably clearly the necessity of methods that are closer to the theory of schemes (which should naturally complement transcendental techniques).

In the current talk, namely talk~I, we will discuss the most elementary case of descent (the one indicated in the title).
The applications of \cref{theorem-B.1,theorem-B.2,theorem-B.3} below are, however, already vast in number.
We will restrict ourselves to giving only some of them as examples, without aiming for the maximum generality possible.

We will freely use the language of schemes, for which we refer to the already cited talk, as well as \cite{2}.
We make clear to point out, however, that the preschemes considered in this current talk are not necessarily Noetherian, and that the
\oldpage{190-02}
morphisms are not necessarily of finite type.
So, if $A$ is a local Noetherian ring, with completion $\overline{A}$, then we will need to consider the non-Noetherian ring $\overline{\overline{A}}\otimes_A\overline{A}$, as well as the morphisms of affine schemes that correspond to the inclusions between the rings in question.


\section{Preliminaries on categories}
\label{A}


\subsection{Fibred categories, descent data, \texorpdfstring{$\sh{F}$}{F}-descent morphisms}
\label{A.1}

\subsubsection{}
\label{A.1.a}

\begin{definition}{1.1}
  A \emph{fibred category $\sh{F}$ with base $\cat{C}$} consists of
  \begin{itemize}
    \item a category $\cat{C}$
    \item for every $X\in\cat{C}$, a category $\sh{F}_X$
    \item for every $\cat{C}$-morphism $f\colon X\to Y$, a functor $f^*\colon\sh{F}_Y\to\sh{F}_X$, which we also write as
      \[
        f^*(\xi) = \xi \times_Y X
      \]
      for $\xi\in\sh{F}_Y$ (with $X$ being thought of as an ``object of $\cat{C}$ over $Y$'', i.e. as being endowed with the morphism $f$)
    \item for any two composible morphisms $X\xrightarrow{f}Y\xrightarrow{g}Z$, an isomorphism of functors
      \[
        c_{f,g}\colon (gf)^* \to f^*g^*
      \]
  \end{itemize}
  with the above data being subject to the conditions that
  \begin{enumerate}[(i)]
    \item $\id^*=\id$
    \item $c_{f,g}$ is the identity isomorphism if $f$ or $g$ is an identity isomorphism
    \item for any three composible morphisms $X\xrightarrow{f}Y\xrightarrow{g}Z\xrightarrow{h}T$, the following diagram, given by using the isomorphisms of the form $c_{u,v}$, commutes:
      \[
        \begin{tikzcd}[column sep=0]
          (h(gf))^* \dar &=& ((hg)f)^* \dar
        \\(gf)^*h^* \dar && f^*(hg)^* \dar
        \\(f^*g^*)h^* &=& f^*(g^*h^*).
        \end{tikzcd}
      \]
  \end{enumerate}
\end{definition}

\begin{example}{1}
\label{example:A.1.1}
  Let $\cat{C}$ be a category where all fibre products exist.
  We then define a fibred category $\sh{F}$ with base $\cat{C}$ by setting $\sh{F}_X$ to be the category of objects of $\cat{C}$ over $X$, and the functor $f^*\colon\sh{F}_Y\to\sh{F}_X$ corresponding to a morphism $f\colon X\to Y$ being defined by the \emph{fibre product} $Z\rightsquigarrow Z\times_Y X$.
\end{example}

\begin{example}{2}
  Let $\cat{C}$ be the category of preschemes, and, for $X\in\cat{C}$, let $\sh{F}_X$ be the category of quasi-coherent sheaves of modules on $X$.
  If $f\colon X\to Y$ is a morphism of preschemes, then $f^*\colon\sh{F}_Y\to\sh{F}_X$ is the
\oldpage{190-03}
  \emph{inverse image of sheaves of modules} functor.
  We thus obtain a category fibred over $\cat{C}$.
\end{example}


\subsubsection{}
\label{A.1.b}

\begin{definition}{1.2}
\label{definition:A.1.2}
  A diagram
  \[
    \begin{tikzcd}
      E \rar["u"]
      & E' \rar[shift left=1,"v_1"] \rar[shift right=1,swap,"v_2"]
      & E''
    \end{tikzcd}
  \]
  of maps of sets is said to be \emph{exact} if $u$ is a bijection from $E$ to the subset of $E'$ consisting of the $x'\in E'$ such that $v_1(x')=v_2(x')$.
\end{definition}

\begin{definition}{1.3}
\label{definition:A.1.3}
  Let $\sh{F}$ be a fibred category with base $\cat{C}$, and consider a diagram
  \[
    \begin{tikzcd}
      S
      & S' \lar[swap,"\alpha"]
      & S'' \lar[shift right=1,swap,"\beta_1"] \lar[shift left=1,"\beta_2"]
    \end{tikzcd}
  \]
  of morphisms in $\cat{C}$ such that $\alpha\beta_1=\alpha\beta_2$;
  this diagram is said to be \emph{$\sh{F}$-exact} if, for every pair $(\xi,\eta)$ of elements of $\sh{F}_S$, the diagram
  \[
  \label{equation-definition:A.1.3}
    \begin{tikzcd}
      \Hom(\xi,\eta) \rar["\alpha^*"]
      & \Hom(\alpha^*(\xi),\alpha^*(\eta)) \rar[shift left=1,"\beta_1^*"] \rar[shift right=1,swap,"\beta_2^*"]
      & \Hom(\gamma^*(\xi),\gamma^*(\eta))
    \end{tikzcd}
  \tag{+}
  \]
  (where $\gamma=\alpha\beta_1=\alpha\beta_2$) of sets is exact.

  In this latter diagram, for simplicity, we have identified $\beta_i^*\alpha^*$ with $(\alpha\beta_i)^*=\gamma^*$, using $c_{\beta_i,\alpha}$.
\end{definition}

\begin{definition}{1.4}
  Let $\sh{F}$ be a fibred category with base $\cat{C}$, and consider morphisms $\beta_1,\beta_2\colon S''\to S'$ in $\cat{C}$.
  Let $\xi'\in\sh{F}_{S'}$.
  We define a \emph{gluing data} on $\xi'$ (with respect to the pair $(\beta_1,\beta_2)$) to be an isomorphism from $\beta_1^*(\xi')$ to $\beta_2^*(\xi')$.
  If $\xi',\eta'\in\sh{F}_{S'}$ are both endowed with gluing data, then a morphism $u\colon\xi'\to\eta'$ in $\sh{F}_{S'}$ is said to be \emph{compatible with the gluing data} if the following diagram commutes:
  \[
    \begin{tikzcd}
      \beta_1^*(\xi') \rar \dar
      &\beta_2^*(\xi') \dar
    \\\beta_1^*(\eta') \rar
      &\beta_2^*(\eta').
    \end{tikzcd}
  \]
\end{definition}

With this definition, the objects of $\sh{F}_{S'}$ that are endowed with gluing data (with respect to $\beta_1$ and $\beta_2$) then form a \emph{category}.
If $\alpha\colon S'\to S$ is a morphism such that $\alpha\beta_1=\alpha\beta_2$, then, for every $\xi\in\sh{F}_{S'}$, the object $\xi'=\alpha^*(\xi)$
\oldpage{190-04}
of $\sh{F}_{S'}$ is endowed with a canonical gluing data, since
\[
  \beta_i^*\alpha^*(\xi)
  \simeq (\alpha\beta_i)^*(\xi)
  = \gamma^*(\xi),
\]
where we again set $\gamma=\alpha\beta_1=\alpha\beta_2$;
furthermore, if $u\colon\xi\to\eta$ is a morphism in $\sh{F}_s$, then
\[
  \alpha^*(u)\colon \alpha^*(\xi) \to \alpha^*(\eta)
\]
is a morphism in $\sh{F}_{S'}$ that is compatible with the canonical gluing data.
We thus obtain a \emph{canonical functor} from the category $\sh{F}_S$ to the category of objects of $\sh{F}_{S'}$ endowed with gluing data with respect to the pair $(\beta_1,\beta_2)$.
With this, we can also rephrase \cref{definition:A.1.3} by saying that the diagram~\cref{equation-definition:A.1.3} is \emph{$\sh{F}$-exact} if the above functor is \emph{fully faithful}, i.e. if the above functor defines an equivalence between the category $\sh{F}_S$ and a subcategory of the category of objects of $\sh{F}_S$ endowed with gluing data with respect to $(\beta_1,\beta_2)$.

\begin{definition}{1.5}
\label{definition:A.1.5}
  We say that a gluing data on $\xi'\in\sh{F}_{S'}$ is \emph{effective} (with respect to $\alpha$) if $\xi'$, endowed with this data, is isomorphic to $\alpha^*(\xi)$ for some $\xi\in\sh{F}_S$.
\end{definition}

In the case where the diagram~\cref{equation-definition:A.1.3} is $\sh{F}$-exact, the object $\xi$ in \cref{definition:A.1.5} is then determined up to unique isomorphism, and \emph{the category $\sh{F}_S$ is equivalent to the category of objects of $\sh{F}_{S'}$ endowed with effective gluing data}.


\subsubsection{}
\label{A.1.c}
The most important case is that where
\[
  S'' = S' \times_S S',
\]
with the $\beta_i$ being the two projections $p_1$ and $p_2$ from $S'\times_S S'$ to its two factors (where we now suppose that $\cat{C}$ has all fibre products).
We then have two immediate necessary conditions for a gluing data $\varphi\colon p_1^*(\xi')\to p_2^*(\xi')$ on some $\xi'\in\sh{F}_S$ to be effective:
\begin{enumerate}[(i)]
  \item $\Delta^*(\varphi) = \id_\xi$, where $\Delta\colon S'\to S'\times_S S'$ denotes the diagonal morphism, and where we identify $\Delta^* p_i^*(\xi')$ with $(p_i\Delta)^*(\xi')=\xi'$.
  \item $p_{31}^*(\varphi) = p_{32}^*(\varphi)p_{21}^*(\varphi)$,
\oldpage{190-05}
  where $p_{ij}$ denotes the canonical projection from $S'\times_S S'\times_S S'$ to the partial product of its $i$th and $j$th factors.
\end{enumerate}

\begin{definition}{1.6}
\label{definition:A.1.6}
  We define \emph{descent data} on $\xi'\in\sh{F}_{S'}$, with respect to the morphism $\alpha\colon S'\to S$, to be a gluing data on $\xi'$ with respect to the pair $(p_1,p_2)$ of canonical projections $S'\times_S S'\to S'$ that satisfies conditions~(i) and (ii) above.
\end{definition}

\begin{definition}{1.7}
\label{definition:A.1.7}
  A morphism $\alpha\colon S'\to S$ is said to be an \emph{$\sh{F}$-descent morphism} if the diagram
  \[
    \begin{tikzcd}
      S
      & S' \lar[swap,"\alpha"]
      & S'\times_S S' \lar[shift left=1,"p_2"] \lar[shift right=1,swap,"p_1"]
    \end{tikzcd}
  \]
  of morphisms is $\sh{F}$-exact (\cref{definition:A.1.3});
  we say that $\alpha$ is a \emph{strict $\sh{F}$-descent morphism} if, further, every descent data (\cref{definition:A.1.6}) on any object of $\sh{F}_{S'}$ is effective.

  This latter condition (of strictness) can also be stated in a more evocative way:
  ``giving an object of $\sh{F}_S$ is equivalent to giving an object of $\sh{F}_{S'}$ endowed with a descent data''.
\end{definition}

Note that, if an $\sh{F}$-descent morphism $\alpha\colon S'\to S$ admits a \emph{section} $s\colon S\to S'$ (i.e. a morphism $s$ such that $\alpha s=\id_S$), then it is a strict $\sh{F}$-descent morphism:
if $\xi'\in\sh{F}_{S'}$ is endowed with descent data with respect to $\alpha$, then ``it comes from'' $\xi=s^*(\xi')$.


\subsubsection{}
\label{A.1.d}
We can present the above notions in a more intuitive manner, by introducing, for an object $T$ of $\cat{C}$ over $S$, the set
\[
  \Hom_S(T,S') = S'(T),
\]
whore elements will be denoted by $t$, $t'$, etc.
Given an object $\xi'\in\sh{F}_{S'}$, there then corresponds, to every $t\in S'(T)$, an object $t^*(\xi')$ of $\sh{F}_T$, which will also be denoted by $\xi'_t$.
A gluing data on $\xi'$ (with respect to $(p_1,p_2)$) is then defined by the data, for every $T$ over $S$, and every pair of points $t,t'\in S'(T)$, of an isomorphism
\[
  \varphi_{t',t}\colon \xi'_t \to \xi'_{t'}
\]
(satisfying the evident conditions of functoriality in $T$).
Conditions~(i) and (ii) of \cref{A.1.c} can then be written as
\begin{enumerate}[(i {bis})]
  \item $\varphi_{t,t}=\id_{\xi'_t}$, for all $T$ and all $t\in S'(T)$.
\oldpage{190-06}
  \item $\varphi_{t,t''}=\varphi_{t,t'}\varphi_{t',t''}$, for all $T$ and all $t,t',t''\in S'(T)$.
\end{enumerate}

We can show that (ii~bis) implies that $\varphi_{t,t}^2=\varphi_{t,t}$, by taking $t=t'=t''$, and thus, since $\varphi_{t,t}$ is an isomorphism by hypothesis, implies (i~bis), which is thus a consequence of (ii~bis) (and so (i) is also a consequence of (ii)).
But if we no longer suppose a priori that the $\varphi_{t,t}$ are isomorphisms (i.e. that $\varphi\colon p_1^*(\xi')\to p_2^*(\xi')$ is an isomorphism), then (ii~bis) no longer necessarily implies (i~bis);
the combination of (ii~bis) and (i~bis), however, does imply that the $\varphi_{t,t'}$ are isomorphisms (since we then have $\varphi_{t,t'}\varphi_{t',t}=\varphi_{t,t}=\id_{\xi'_t}$).


\subsection{Exact diagrams and strict epimorphisms, descent morphisms, and examples.}
\label{A.2}

\subsubsection{}
\label{A.2.a}

\begin{definition}{2.1}
  Let $\cat{C}$ be a category.
  A diagram
  \[
    \begin{tikzcd}
      T \rar["\alpha"]
      & T' \rar[shift left=1,"\beta_1"] \rar[shift right=1,swap,"\beta_2"]
      & T''
    \end{tikzcd}
  \]
  of morphisms is said to be \emph{exact} if, for all $Z\in\cat{C}$, the corresponding diagram
  \[
    \begin{tikzcd}
      \Hom(Z,T) \rar
      & \Hom(Z,T') \rar[shift left=1] \rar[shift right=1]
      & \Hom(Z,T'')
    \end{tikzcd}
  \]
  of sets is exact (\cref{definition:A.1.2}).
  We then say that $(T,\alpha)$ (or, by an abuse of language, $T$) is a \emph{kernel} of the pair $(\beta_1,\beta_2)$ of morphisms.
\end{definition}

This kernel is evidently determined up to unique isomorphism.
If $\cat{C}$ is the category of sets, then the above definition is compatible with \cref{definition:A.1.2}.
Dually, we define the exactness of a diagram
\[
  \begin{tikzcd}
    S
    & S' \lar["\alpha"]
    & S''\lar[shift left=1,swap,"\beta_1"] \lar[shift right=1,"\beta_2"]
  \end{tikzcd}
\]
of morphisms in $\cat{C}$;
we then say that $(S,\alpha)$ is a \emph{cokernel} of the pair $(\beta_1,\beta_2)$ morphisms.

\begin{definition}{2.2}
\label{definition:A.2.2}
  A morphism $\alpha\colon S'\to S$ is said to be a \emph{strict epimorphism} if it is an epimorphism and, for every morphism $u\colon S'\to Z$, the following necessary condition is also sufficient for $u$ to factor as $S'\to S\to Z$:
  for every $S''\in\cat{C}$ and every pair $\beta_1,\beta_2\colon S''\to S$ of morphisms such that $\alpha\beta_1=\alpha\beta_2$, we also have that $u\beta_1=u\beta_2$.
\end{definition}

If the fibre product $S'\times_S S'$ exists, then it is equivalent to say that the diagram
\[
  \begin{tikzcd}
    S
    & S' \lar["\alpha"]
    & S'\times_S S'\lar[shift left=1,swap,"p_1"] \lar[shift right=1,"p_2"]
  \end{tikzcd}
\]
\oldpage{190-07}
is exact, i.e. that $S$ is a cokernel of the pair $(p_1,p_2)$.
In any case, a cokernel morphism is a strict epimorphism.
Note also that, if a strict epimorphism is also a monomorphism, then it is an isomorphism.
We leave to the reader the task of developing the dual notion of a \emph{strict monomorphism}.

To make the relation between the ideas of $\sh{F}$-descent morphisms and strict epimorphisms more precise, we introduce the following definitions:

\begin{definition}{2.3}
  A morphism $\alpha\colon S'\to S$ is said to be a \emph{universal epimorphism} (resp. a \emph{strict universal epimorphism}) if, for every $T$ over $S$, the fibre product $T'=S'\times_S T$ exists, and the projection $T'\to T$ is an epimorphism (resp. a strict epimorphism).
\end{definition}

In very nice categories (such as the category of sets, the category of sets over a topological space, abelian categories, etc.), the four notions of ``epijectivity'' introduced above all coincide;
they are, however, all distinct in a category such as the category of preschemes, or the category of preschemes over a given non-empty prescheme $S$, even if we restrict to $S$-schemes that are finite over $S$.

\begin{definition}{2.4}
  A morphism $\alpha\colon S'\to S$ is said to be a \emph{descent morphism} (resp. a \emph{strict descent morphism}) if it is an $\sh{F}$-descent morphism (resp. a strict $\sh{F}$-descent morphism) (cf. \cref{definition:A.1.7}), where $\sh{F}$ denotes the fibred category over $\cat{C}$ of objects of $\cat{C}$ over objects of $\cat{C}$ (\cref{example:A.1.1}).
\end{definition}

\begin{proposition}{2.1}
  If $\cat{C}$ has all finite products and (finite) fibre products, then there is an identity between descent morphisms in $\cat{C}$ and strict universal epimorphisms in $\cat{C}$.
\end{proposition}


\subsubsection{}
\label{A.2.b}


%% Bibliography %%

\nocite{*}
\begin{thebibliography}{6}

  \bibitem{1}
  {\sc Dieudonn\'{e}, J. and Grothendieck, A.}
  \newblock El\'{e}ments de g\'{e}om\'{e}trie alg\'{e}brique.
  \newblock {\em Publications math\'{e}matiques de l'Institut des Hautes Etudes Scientifiques} (to appear).

  \bibitem{2}
  {\sc Grauert, H. and Remmert, R.}
  \newblock Komplexe R\"{a}ume.
  \newblock {\em Math. Annalen} \textbf{136} (1958), 245--318.

  \bibitem{3}
  {\sc Grothendieck, A.}
  \newblock G\'{e}om\'{e}trie formelle et g\'{e}om\'{e}trie alg\'{e}brique.
  \newblock {\em S\'{e}minaire Bourbaki} \textbf{11} (1958--59), Talk no.~182.

  \bibitem{4}
  {\sc Murre, J.P.}
  \newblock On a connectedness theorem for a birational transformation at a simple point.
  \newblock {\em Amer. J. Math.} \textbf{80} (1958), 3--15

  \bibitem{5}
  {\sc Serre, J.-P.}
  \newblock G\'{e}om\'{e}trie alg\'{e}brique et g\'{e}om\'{e}trie analytique.
  \newblock {\em Ann. Institut Fourier Grenoble} \textbf{6} (1955--56), 1--42.

  \bibitem{6}
  {\sc Serre, J.-P.}
  \newblock Espaces fibr\'{e}s alg\'{e}briques.
  \newblock {\em S\'{e}minaire Chevalley} \textbf{3} (1958), Talk no.~1.

\end{thebibliography}

\end{document}

\documentclass{article}

\title{Techniques of descent and existence theorems in algebraic geometry\\II. \emph{The existence theorem and the formal theory of modules}}
\author{A. Grothendieck}
\date{February 1960}

\usepackage{amssymb,amsmath}

\usepackage{hyperref}
\usepackage{xcolor}
\hypersetup{colorlinks,linkcolor={red!50!black},citecolor={blue!50!black},urlcolor={blue!80!black}}
\usepackage[nameinlink]{cleveref}
\usepackage{enumerate}
\usepackage{tikz-cd}
\usepackage{graphicx}

\usepackage{mathrsfs}
%% Fancy fonts --- feel free to remove! %%
\usepackage{fouriernc}


\usepackage{fancyhdr}
\usepackage{lastpage}
\usepackage{xstring}
\makeatletter
\ifx\pdfmdfivesum\undefined
  \let\pdfmdfivesum\mdfivesum
\fi
\edef\filesum{\pdfmdfivesum file {\jobname}}
\pagestyle{fancy}
\makeatletter
\let\runauthor\@author
\let\runtitle\@title
\makeatother
\fancyhf{}
\lhead{\footnotesize\runtitle}
\lfoot{\footnotesize Version: \texttt{\StrMid{\filesum}{1}{8}}}
\cfoot{\small\thepage\ of \pageref*{LastPage}}


\crefname{section}{}{}
\crefname{equation}{}{}
\renewcommand{\thepart}{\Alph{part}}
\renewcommand{\thesection}{\arabic{section}}
\renewcommand{\thesubsection}{(\alph{subsection})}


%% Theorem environments %%

\usepackage{amsthm}

\theoremstyle{plain}

\newtheorem{innercustomtheorem}{Theorem}
\crefname{innercustomtheorem}{Theorem}{Theorems}
\newenvironment{theorem}[1]
  {\renewcommand\theinnercustomtheorem{#1}\innercustomtheorem}
  {\endinnercustomtheorem}

\newtheorem{innercustomproposition}{Proposition}
\crefname{innercustomproposition}{Proposition}{Propositions}
\newenvironment{proposition}[1]
  {\renewcommand\theinnercustomproposition{#1}\innercustomproposition}
  {\endinnercustomproposition}

\newtheorem{innercustomlemma}{Lemma}
\crefname{innercustomlemma}{Lemma}{Lemmas}
\newenvironment{lemma}[1]
  {\renewcommand\theinnercustomlemma{#1}\innercustomlemma}
  {\endinnercustomlemma}

\newtheorem{innercustomcorollary}{Corollary}
\crefname{innercustomcorollary}{Corollary}{Corollaries}
\newenvironment{corollary}[1]
  {\renewcommand\theinnercustomcorollary{#1}\innercustomcorollary}
  {\endinnercustomcorollary}

\newtheorem*{corollary*}{Corollary}
\newtheorem*{lemma*}{Lemma}


\theoremstyle{definition}

\newtheorem{innercustomdefinition}{Definition}
\crefname{innercustomdefinition}{Definition}{Definitions}
\newenvironment{definition}[1]
  {\renewcommand\theinnercustomdefinition{#1}\innercustomdefinition}
  {\endinnercustomdefinition}

\newtheorem{innercustomexample}{Example}
\crefname{innercustomexample}{Example}{Examples}
\newenvironment{example}[1]
  {\renewcommand\theinnercustomexample{#1}\innercustomexample}
  {\endinnercustomexample}

\newtheorem*{remark*}{Remark}
\newtheorem*{remarks*}{Remarks}
\newtheorem*{example*}{Example}

%% Shortcuts %%

\newcommand{\sh}[1]{{\mathscr{#1}}}
\newcommand{\cat}[1]{{\mathcal{#1}}}
\newcommand{\fk}[1]{{\mathfrak{#1}}}
\newcommand{\kres}{k}
\newcommand{\simto}{\xrightarrow{\raisebox{-0.7ex}[0ex][0ex]{$\sim$}}}

\newcommand{\Set}{\mathsf{Set}}

\renewcommand{\geq}{\geqslant}
\renewcommand{\leq}{\leqslant}

\DeclareMathOperator{\id}{id}
\DeclareMathOperator{\Hom}{Hom}
\DeclareMathOperator{\shHom}{\underline{\Hom}}
\DeclareMathOperator{\Aut}{Aut}
\DeclareMathOperator{\HH}{H}
\DeclareMathOperator{\RR}{R}
\DeclareMathOperator{\GL}{GL}
\DeclareMathOperator{\Ga}{G_a}
\DeclareMathOperator{\Gm}{G_m}
\DeclareMathOperator{\SL}{SL}
\DeclareMathOperator{\Sp}{Sp}
\DeclareMathOperator{\Spec}{Spec}
\DeclareMathOperator{\Pro}{Pro}

\newcommand{\todo}{\textbf{ !TODO! }}
\newcommand{\oldpage}[1]{\marginpar{\footnotesize$\Big\vert$ \textit{p.~#1}}}


%% Document %%

\usepackage{embedall}
\begin{document}

\maketitle
\thispagestyle{fancy}

\renewcommand{\abstractname}{Translator's note.}

\begin{abstract}
  \renewcommand*{\thefootnote}{\fnsymbol{footnote}}
  \emph{This text is one of a series\footnote{\url{https://thosgood.com/translations/}} of translations of various papers into English.}
  \emph{The translator takes full responsibility for any errors introduced in the passage from one language to another, and claims no rights to any of the mathematical content herein.}

  \medskip
  
  \emph{What follows is a translation of the French seminar talk:}

  \medskip\noindent
  \textsc{Grothendieck, A.}
  Technique de descente et th\'{e}or\`{e}mes d'existence en g\'{e}om\'{e}trie alg\'{e}brique. I. Le th\'{e}or\`{e}me d'existence et th\'{e}orie formelle des modules.
  \emph{S\'{e}minaire Bourbaki}, Volume~\textbf{12} (1959--60), Talk no.~195.
\end{abstract}

\setcounter{footnote}{0}

\setcounter{tocdepth}{1}
\tableofcontents


%% Content %%

\subsubsection*{}

\emph{[Trans.] We have made changes throughout the text following the errata (\emph{S\'{e}minaire Bourbaki} \textbf{14}, 1961--62, Compl\'{e}ment); we preface them with ``[Comp.]''.}
\medskip

\todo{errata}

\oldpage{195-01}
\part{Representable and pro-representable functors}
\label{A}

\section{Representable functors}
\label{A.1}

Let $\cat{C}$ be a category.
For all $X\in\cat{C}$, let $h_X$ be the contravariant functor from $\cat{C}$ to the category $\Set$ of sets given by
\[
  \begin{aligned}
    h_X\colon \cat{C} &\to \Set
  \\Y&\mapsto \Hom(Y,X).
  \end{aligned}
\]
If we have a morphism $X\to X'$ in $\cat{C}$, then this evidently induces a morphism $h_X\to h_{X'}$ of functors;
$h_X$ is a covariant functor in $X$, i.e. we have defined a \emph{canonical covariant functor}
\[
  h\colon \cat{C} \to \Hom(\cat{C}^o,\Set)
\]
from $\cat{C}$ to the category of covariant functors from the dual $\cat{C}^o$ of $\cat{C}$ to the category of sets.
We then recall:

\begin{proposition}{1}
\label{A.1-proposition1}
  This functor $h$ is \emph{faithfully flat};
  in other words, for every pair $X,X'$ of objects of $\cat{C}$, the natural map
  \[
    \Hom(X,X') \to \Hom(h_X,h_{X'})
  \]
  is \emph{bijective}.
\end{proposition}

In particular, if a functor $F\in\Hom(\cat{C}^o,\Set)$ is isomorphic to a functor of the form $h_X$, then \emph{$X$ is determined up to unique isomorphism}.
We then say that the functor $F$ is \emph{representable}.
The above proposition then implies that the canonical functor $h$ defines an \emph{equivalence} between the category $\cat{C}$ and the full subcategory of $\Hom(\cat{C}^o,\Set)$ consisting of representable functors.
This fact is the basis of \emph{the idea of a ``solution of a universal problem''}, with such a problem always consisting of examining if a given (contravariant, as here, or covariant, in the dual case) functor from $\cat{C}$ to $\Set$ is representable.
\oldpage{195-02}
Note further that, even by the definition of products in a category \cite{1}, the functor $h\colon X\mapsto h_X$ commutes with products whenever they exist (and, more generally, with finite or infinite projective limits, and, in particular, with fibred products, taking ``kernels'' \cite{2}, etc., whenever such things exist): we have an isomorphism of functors
\[
  h_{X\times X'} \simto h_X\times h_{X'}
\]
whenever $X\times X'$ exists, i.e. we have functorial (in $Y$) bijections
\[
  h_{X\times X'} \simto h_X(Y)\times h_{X'}(Y).
\]
In particular, the data of a morphism
\[
  X\times X' \to X''
\]
in $\cat{C}$ (i.e. of a ``\emph{composition law}'' in $\cat{C}$ between $X$, $X'$, and $X''$) is equivalent to the data of a morphism $h_{X\times X'}=h_X\times h_{X'}\to h_{X''}$, i.e. to the data, for all $Y\in\cat{C}$, of a composition law of \emph{sets}
\[
  h_X(Y)\times h_{X'}(Y) \to h_{X''}(Y)
\]
such that, for every morphism $Y\to Y'$ in $\cat{C}$, the system of set maps
\[
  h_{X^{(i)}}(Y) \to h_{X^{(i)}}(Y')
  \qquad\mbox{(for $i=0,1,2$)}
\]
is a morphism for the two composition laws, with respect to $Y$ and $Y'$.
In this way, we see that the idea of a ``$\cat{C}-group$'' structure, or a ``$\cat{C}$-ring'' structure, etc. on an object $X$ of $\cat{C}$ can be expressed in the most manageable way (in theory as much as in practice) by saying that, for every $Y\in\cat{C}$, we have a group law (resp. ring law, etc.) in the usual sense on the set $h_X(Y)$, with the maps $h_X(Y)\to h_X(Y')$ corresponding to morphisms $Y\to Y'$ that should be group homomorphisms (resp. ring homomorphisms, etc.).
This is the most intuitive and manageable way of defining, for example, the various classical groups $\Ga$, $\Gm$, $\GL(n)$, etc. on a prescheme $S$ over an arbitrary base, and of writing the classical relations between these groups, or of placing a ``vector bundle'' structure on the affine scheme $V(\sh{F})$ over $S$ defined by a quasi-coherent sheaf $\sh{F}$, and of defining and studying the many associated flag varieties (Grassmannians, projective bundles), etc.;
\emph{the general yoga is purely and simply identifying, using the canonical functor $h$, the objects of $\cat{C}$ with particular contravariant functors (namely, representable functors)}
\oldpage{195-03}
\emph{from $\cat{C}$ to the category of sets.}

The usual procedure of reversing the arrows that is necessary, for example, in the case of affine schemes in order to pass from the geometric language to the language of commutative algebra, leads us to dualise the above considerations, and, in particular, to also introduce \emph{covariant representable functors $\cat{C}\to\Set$}, i.e. those of the form $Y\mapsto\Hom(X,Y)=h'_X(Y)$.


\section{Pro-representable functors, pro-objects}
\label{A.2}

Let $\cat{X}=(X_i)_{i\in I}$ be a projective system of objects of $\cat{C}$;
there is a corresponding covariant functor
\[
  h'_{\cat{X}} = \varinjlim_i h'_{X_i}
\]
which can be written more explicitly as
\[
  h'_{\cat{X}}(Y) = \varinjlim_i h'_{X_i}(Y) = \varinjlim_i\Hom(X_i,Y)
\]
which is a functor from $\cat{C}$ to $\Set$.
A functor from $\cat{C}$ to $\Set$ that is isomorphic to a functor of this type \emph{with $I$ filtered} is said to be \emph{pro-representable}.
By \hyperref[A.1]{the previous section}, these are exactly the functors that are isomorphic to \emph{filtered inductive limits of representable functors}.
Let $\cat{X}'=(X_j)_{j\in J}$ be another filtered projective system in $\cat{C}$ (indexed by another filtered pre-ordered set of indices $J$).
Then we can easily show that we have a canonical bijection
\[
  \Hom(h_{\cat{X}'},h_{\cat{X}}) = \varprojlim_j\varinjlim_i\Hom(X_i,X'_j)
\]
(generalising \hyperref[A.1-proposition1]{Proposition~1, (A.1)}).
This leads to introducing the \emph{category $\Pro(\cat{C})$ of pro-objects of $\cat{C}$}, whose objects are projective systems of objects of $\cat{C}$ (indexed by arbitrary filtered pre-ordered sets of indices), and whose morphisms between objects $\cat{X}=(X_i)_{i\in I}$ and $\cat{X}'=(X_j)_{j\in J}$ are given by
\[
  \Pro\Hom(\cat{X},\cat{X}') = \varprojlim_j\varinjlim_i\Hom(X_i,X'_j),
\]
with the composition of pro-homomorphisms being evident.
By the very construction itself, $\cat{X}\mapsto h'_{\cat{X}}$ can be considered as a contravariant functor in $\cat{X}$, establishing an \emph{equivalence between the dual category of the category $\Pro(\cat{C})$ of pro-objects of $\cat{C}$ and the category of pro-representable covariant functors from $\cat{C}$ to $\Set$}.
Of course, an object $X$ of $\cat{C}$ canonically defines a pro-object, denoted
\oldpage{195-04}
again by $X$, so that \emph{$\cat{C}$ is equivalent to a full subcategory of $\Pro(\cat{C})$}.
Then, if $\cat{X}=(X_i)_{i\in I}$ is an arbitrary pro-object of $\cat{C}$, then (with the above identification) we have that
\[
  \cat{X} = \varprojlim_i X_i
\]
with the projective limit being \emph{taken in $\Pro(\cat{C})$} (since $h_{\cat{X}}=\varinjlim_i h_{X_i}$).

We draw attention to the fact that, even if the projective limit of the $X_i$ \emph{exists in $\cat{C}$}, it will generally \emph{not} be isomorphic to the projective limit $\cat{X}$ in $\Pro(\cat{C})$, as is already evident in the case where $\cat{C}$ is the category of sets.
We note that, by the definition itself, $\varprojlim{}_{\cat{C}}X_i=L$ is defined by the condition that the functor
\[
  \varprojlim_i\Hom_{\cat{C}}(Y,X_i)=\Hom_{\Pro(\cat{C})}(Y,\cat{X})
\]
in $Y\in\cat{C}$ and with values in $\Set$ be representable via $\cat{L}$, i.e. that it be isomorphic to $\Hom_{\cat{C}}(Y,\cat{L})$;
then \emph{$\lim{}_{\cat{C}}X_i$ is already defined in terms of the \emph{pro-object} $\cat{X}$}, and, in a precise way, depends functorially on the pro-object $\cat{X}$ whenever it is defined;
there is therefore no problem with denoting it by $\lim{}_{\cat{C}}(\cat{X})$.
If projective limits in $\cat{C}$ always exist, then $\lim{}_{\cat{C}}(\cat{X})$ is a functor from $\Pro(\cat{C})$ to $\cat{C}$, and there is a canonical homomorphism of functors $\lim_\cat{C}(\cat{X})\to\cat{X}$.
Since every (covariant, say, for simplicity) functor
\[
  F\colon \cat{C} \to \cat{C}'
\]
can be extended in an obvious way to a functor
\[
  \Pro(F)\colon \Pro(\cat{C}) \to \Pro(\cat{C}'),
\]
it follows that, if projective limits always exist in $\cat{C}'$, then $F$ also canonically defines a composite functor
\[
  \overline{F} = \varprojlim{}_{\cat{C}'}\colon \Pro(\cat{C}) \to \cat{C}'
\]
sending $\cat{X}=(X_i)_{i\in I}$ to $\varprojlim{}_{\cat{C}'}F(X_i)$.

A pro-object $\cat{X}$ is said to be a \emph{strict pro-object} if it is isomorphic to a pro-object $(X_i)_{i\in I}$, where the transition morphisms $X_i\to X_j$ are \emph{epimorphisms};
a functor defined by such an object is said to be \emph{strictly pro-representable}.
We can thus further demand that $I$ be a filtered \emph{ordered} set, and that every epimorphism
\oldpage{195-05}
$X_i\to X'$ be equivalent to an epimorphism $X_i\to X_j$ for some suitable $j\in I$ (uniquely determined by this condition).
Under these conditions, the projective system $(X_i)_{i\in I}$ is determined \emph{up to unique isomorphism} (in the usual sense of isomorphisms of projective systems).
It thus follows that \emph{the projective limit of a projective system $\cat{X}^{(\alpha)}$ of strict pro-objects always exists in $\Pro(\cat{C})$}, and that, with the above notation of $F$ and $\overline{F}$, we have that
\[
  \overline{F}\varprojlim_\alpha\cat{X}^{(\alpha)} = \varprojlim_\alpha{}_{\cat{C}'}F(X^{(\alpha)}).
\]
In particular, if every pro-object of $\cat{C}$ is strict (cf. the \hyperref[A.1]{previous section}), then the extended functor $\overline{F}$ commutes with projective limits.


\section{Characterisation of pro-representable functors}
\label{A.3}


%% Bibliography %%

\nocite{*}
\begin{thebibliography}{4}

  \bibitem{1}
  {\sc Grothendieck, A.}
  \newblock Sur quelques points d'alg\'{e}bre homologique.
  \newblock {\em Tohoku math. J.} {\bf 9} (1957), 119--221.

  \bibitem{2}
  {\sc Grothendieck, A.}
  \newblock G\'{e}om\'{e}trie formelle et g\'{e}om\'{e}trie alg\'{e}brique.
  \newblock {\em S\'{e}minaire Bourbaki} \textbf{11} (1958--59), Talk no.~182.

  \bibitem{3}
  {\sc Grothendieck, A.}
  \newblock Technique de descente et th\'{e}or\`{e}mes d'existence en g\'{e}om\'{e}trie alg\'{e}brique, I: G\'{e}n\'{e}ralit\'{e}s, Descente pas morphismes fid\`{e}lement plats.
  \newblock {\em S\'{e}minaire Bourbaki} \textbf{12} (1959--60), Talk no.~190.

  \bibitem{4}
  {\sc Serre, J.-P.}
  \newblock Corps locaux et isog\'{e}nies.
  \newblock {\em S\'{e}minaire Bourbaki} \textbf{11} (1958--59), Talk no.~185.

\end{thebibliography}

\end{document}

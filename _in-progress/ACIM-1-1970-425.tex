\documentclass{article}

\title{Hodge Theory I}
\author{Pierre Deligne}
\date{}

\usepackage{amssymb,amsmath}

\usepackage{hyperref}
\usepackage{xcolor}
\hypersetup{colorlinks,linkcolor={red!50!black},citecolor={blue!50!black},urlcolor={blue!80!black}}
\usepackage[nameinlink]{cleveref}
\usepackage{enumerate}
\usepackage{booktabs}

\usepackage{mathrsfs}
%% Fancy fonts --- feel free to remove! %%
\usepackage{fouriernc}


\usepackage{fancyhdr}
\usepackage{lastpage}
\usepackage{xstring}
\makeatletter
\ifx\pdfmdfivesum\undefined
  \let\pdfmdfivesum\mdfivesum
\fi
\edef\filesum{\pdfmdfivesum file {\jobname}}
\pagestyle{fancy}
\makeatletter
\let\runauthor\@author
\let\runtitle\@title
\makeatother
\fancyhf{}
\lhead{\footnotesize\runtitle}
\lfoot{\footnotesize Version: \texttt{\StrMid{\filesum}{1}{8}}}
\cfoot{\small\thepage\ of \pageref*{LastPage}}


\crefname{section}{\S\!}{\S\S\!}
\crefname{equation}{}{}


%% Theorem environments %%

\usepackage{amsthm}

\theoremstyle{plain}

\theoremstyle{definition}

  \newtheorem{innercustomdefinition}{Definition}
  \crefname{innercustomdefinition}{Definition}{Definitions}
  \newenvironment{definition}[1]
    {\renewcommand\theinnercustomdefinition{#1}\innercustomdefinition}
    {\endinnercustomdefinition}

  \newtheorem{innercustomprinciple}{Principle}
  \crefname{innercustomprinciple}{Principle}{Principles}
  \newenvironment{principle}[1]
    {\renewcommand\theinnercustomprinciple{#1}\innercustomprinciple}
    {\endinnercustomprinciple}

  \newtheorem*{translation*}{Translation}


%% Shortcuts %%

\newcommand{\ZZ}{\mathbb{Z}}
\newcommand{\QQ}{\mathbb{Q}}
\newcommand{\CC}{\mathbb{C}}

\DeclareMathOperator{\Gr}{Gr}
\DeclareMathOperator{\Ker}{Ker}
\DeclareMathOperator{\Gal}{Gal}
\DeclareMathOperator{\Spec}{Spec}
\DeclareMathOperator{\Hom}{Hom}

\renewcommand{\geq}{\geqslant}
\renewcommand{\leq}{\leqslant}

\newcommand{\todo}{\textbf{ !TODO! }}
\newcommand{\oldpage}[1]{\marginpar{\footnotesize$\Big\vert$ \textit{p.~#1}}}


%% Document %%

\usepackage{embedall}
\begin{document}

\maketitle
\thispagestyle{fancy}

\renewcommand{\abstractname}{Translator's note.}

\begin{abstract}
  \renewcommand*{\thefootnote}{\fnsymbol{footnote}}
  \emph{This text is one of a series\footnote{\url{https://thosgood.com/translations/}} of translations of various papers into English.}
  \emph{The translator takes full responsibility for any errors introduced in the passage from one language to another, and claims no rights to any of the mathematical content herein.}
  
  \emph{What follows is a translation of the French paper:}

  \medskip\noindent
  \textsc{Deligne, P.}
  ``Th\'{e}orie de Hodge I''.
  \emph{Actes du Congr\`{e}s intern. math.}, Volume~\textbf{1} (1970), 425--430.
  \url{https://publications.ias.edu/deligne/paper/359}
\end{abstract}

\setcounter{footnote}{0}

% \tableofcontents
\bigskip


%% Content %%

\oldpage{425}
We intend to give a heuristic dictionary between statements in $l$-adic cohomology and statements in Hodge theory.
This dictionary has, as its most notable sources sources, \cite{3} and the conjectural theory of Grothendieck motives \cite{2}.
Up until now, it has mainly served to formulate conjectures in Hodge theory, and it has sometimes even suggested a proof.


\section{}
\label{1}

\begin{definition}{1.1}
  A \emph{mixed Hodge structure $H$} consists of
  \begin{enumerate}[(a)]
    \item a $\ZZ$-module $H_\ZZ$ of finite type (the ``\emph{integer lattice}'');
    \item a finite increasing filtration $W$ of $H_\QQ = H_\ZZ\otimes\QQ$ (the ``\emph{weight filtration}'');
    \item a finite decreasing filtration $F$ of $H_\CC = H_\ZZ\otimes\CC$ (the ``\emph{Hodge filtration}'').
  \end{enumerate}
  This data is subject to the following condition:
  there exists a (unique) bigradation of $\Gr_W(H_\CC)$ by subspaces $H^{p,q}$ such that
  \begin{enumerate}[(i)]
    \item $\Gr_W^n(H_\CC) = \bigoplus_{p+q=n}H^{p,q}$
    \item the filtration $F$ induces on $\Gr_W(H_\CC)$ the filtration
      \[
        \Gr_W(F)^p = \bigoplus_{p'\geq p} H^{p',q'}
      \]
    \item $\overline{H^{pq}}=H^{qp}$.
  \end{enumerate}
\end{definition}

A \emph{morphism} $f\colon H\to H'$ is a homomorphism $f_\ZZ\colon H_\ZZ\to H'_\ZZ$ such that $f_\QQ\colon H_\QQ\to H'_\QQ$ and $f_\CC\colon H_\CC\to H'_\CC$ are compatible with the filtrations $W$ and $F$, respectively.

The \emph{Hodge numbers} of $H$ are the integers
\[
  h^{pq} = \dim H^{pq} = h^{qp}.
\tag{1.2}
\]

We say that $H$ is pure \emph{of weight~$n$} if $h^{pq}=0$ for $p+q\neq n$ (i.e. if $\Gr_W^i(H)=0$ for $i\neq n$).
We also say that $H$ is a \emph{Hodge structure of weight~$n$}.

The \emph{Tate Hodge structure} $\ZZ(1)$ is the Hodge structure of weight~$-2$, purely of type~$(-1,-1)$, for which $\ZZ(1)_\CC=\CC$ and $\ZZ(1)_\ZZ = 2\pi i\ZZ = \Ker(\exp\colon\CC\to\CC^*)\subset\CC$.
We set $\ZZ(n)=\ZZ(1)^{\otimes n}$.

We can show that mixed Hodge structures form an abelian category.
If $f\colon H\to H'$ is a morphism, then $f_\QQ$ and $f_\CC$ are strictly compatible with the filtrations $W$ and $F$ (cf. \cite[2.3.5]{1}).


\section{}
\label{2}

Let $A$ be a normal integral ring of finite type over $\ZZ$, with field of fractions $K$,
\oldpage{426}
and $\overline{K}$ an algebraic closure of $K$.
Let $K_{nr}$ be the largest sub-extension of $\overline{K}$ that is unramified at each prime ideal of $A$.
We know that, or we set,
\[
  \pi_1(\Spec(A),\overline{K}) = \Gal(K_{nr}/K).
\]

For every closed point $x$ of $\Spec(A)$, defined by some maximal ideal $m_x$ of $A$, the residue field $k_x=A/m_x$ is finite;
the point $x$ defines a conjugation class of ``Frobenius substitutions'' $\varphi_x\in\pi_1(\Spec(A),\overline{K})$.
We set $q_x=\#k_x$ and $F_x=\varphi_x^{-1}$.

Let $K$ be a field of finite type over the prime field of characteristic~$p$, let $\overline{K}$ be an algebraic closure of $K$, let $l$ be a prime number $\neq p$, and let $H$ be a $\ZZ_l$- (or a $\QQ_l$-) module of finite type endowed with a continuous action $\rho$ of $\Gal(\overline{K}/K)$.
We will still suppose in what follows that there exists an $A$ as above, with $l$ invertible in $A$, and such that $\rho$ factors through $\pi_1(\Spec(A),\overline{K}) = \Gal(K_{nr}/K)$.
We say that $H$ is \emph{pure of weight~$n$} if, for every closed point $x$ of an non-empty open subset of $\Spec(A)$, the eigenvalues $\alpha$ of $F_x$ acting on $H$ are algebraic integers whose complex conjugates are all of absolute value $|\alpha|=q_x^{n/2}$.

\begin{principle}{2.1}
\label{principle-2.1}
  If the Galois module $H$ ``comes from algebraic geometry'', then there exists a (unique) increasing filtration $W$ on $H_{\QQ_l}=H\otimes_{\ZZ_l}\QQ_l$ (the ``\emph{weight filtration}'') that is Galois invariant and such that $\Gr_n^W(H)$ is pure of weight~$n$.
\end{principle}

We can also further suppose that $\Gr_n^W(H)$ is semi-simple.

When we have a resolution of singularities, we can often give a conjectural definition of $W$, whose validity follows from the Weil conjectures \cite{5} (cf. \cref{6}).

Let $\mu$ be the subgroup of $\overline{K}^*$ given by the roots of unity.
The \emph{Tate module $\ZZ_l(1)$}, defined by
\[
  \ZZ_l(1) = \Hom(\QQ_l/\ZZ_l,\mu)
\]
is pure of weight~$-2$.
We set $\ZZ_l(n)=\ZZ_l(1)^\otimes n$.

It is trivial that every morphism $f\colon H\to H'$ is strictly compatible with the weight filtration.

\Cref{principle-2.1} agrees with the fact that every extension of $\mathbb{G}_m$ (``weight~$-2$'') by an abelian variety (``weight~$-1>-2$'') is trivial.


\section{}
\label{3}

\begin{translation*}
  The Galois modules that appear in $l$-adic cohomology have, as analogues, over $\CC$, mixed Hodge structures.
  We further have the dictionary
  
  \bigskip
  \begin{tabular}{p{0.5\linewidth}|p{0.5\linewidth}}
    pure module of weight $n$
    & Hodge structure of weight $n$
  \\weight filtration
    & weight filtration
  \\Galois-compatible homomorphism
    & morphism
  \\Tate module $\ZZ_l(1)$
    & Tate Hodge structure $\ZZ(1)$
  \end{tabular}
\end{translation*}


\section{}
\label{4}

Let $X$ be a complex algebraic variety (i.e. a scheme of finite type over $\CC$ that we assume to be separated).
Then there exists a subfield $K$ of $\CC$, of finite type over $\QQ$, such that $X$ can be defined over $K$ (i.e. it comes from an extension of scalars of $K$ to $\CC$ applied to a $K$-scheme $X'$).
Let $\overline{K}$ be the algebraic closure of $K$ in $\CC$.
The Galois group $\Gal(\overline{K}/K)$ then acts on the $l$-adic cohomology groups $H^\bullet(X,\ZZ_l)$;
we have
\[
  H^\bullet(X(\CC),\ZZ)\otimes\ZZ_l
  = H^\bullet(X,\ZZ_l)
  = H^\bullet(X'_{\overline{K}},\ZZ_l).
\]

\oldpage{427}
By \cref{3}, we



%% Bibliography %%

\nocite{*}

\end{document}

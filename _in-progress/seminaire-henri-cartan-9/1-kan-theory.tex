\documentclass{article}

\title{On Kan theory}
\author{H. Cartan}
\date{10\textsuperscript{th} and 17\textsuperscript{th} of December, 1956}

\usepackage{amssymb,amsmath}

\usepackage{hyperref}
\usepackage{xcolor}
\hypersetup{colorlinks,linkcolor={red!50!black},citecolor={blue!50!black},urlcolor={blue!80!black}}
\usepackage[nameinlink]{cleveref}
\usepackage{enumerate}
\usepackage{tikz-cd}

\usepackage{mathrsfs}
%% Fancy fonts --- feel free to remove! %%
\usepackage{Baskervaldx}
\usepackage{mathpazo}


\usepackage{fancyhdr}
\usepackage{lastpage}
\usepackage{xstring}
\makeatletter
\ifx\pdfmdfivesum\undefined
  \let\pdfmdfivesum\mdfivesum
\fi
\edef\filesum{\pdfmdfivesum file {\jobname}}
\pagestyle{fancy}
\makeatletter
\let\runauthor\@author
\let\runtitle\@title
\makeatother
\fancyhf{}
\lhead{\footnotesize\runtitle}
\rhead{\footnotesize Version: \texttt{\StrMid{\filesum}{1}{8}}}
\cfoot{\small\thepage\ of \pageref*{LastPage}}


\crefname{section}{\S\!}{\S\S\!}
\crefname{equation}{}{}


%% Theorem environments %%

\usepackage{amsthm}

\theoremstyle{plain}

  \newtheorem{innercustomtheorem}{Theorem}
  \crefname{innercustomtheorem}{Theorem}{Theorems}
  \newenvironment{theorem}[1]
    {\renewcommand\theinnercustomtheorem{#1}\innercustomtheorem}
    {\endinnercustomtheorem}

  \newtheorem*{corollary}{Corollary}
  \newtheorem*{lemma}{Lemma}

\theoremstyle{definition}

  \newtheorem*{remark}{Remark}


%% Shortcuts %%

\newcommand{\cat}{\mathcal}
\newcommand{\id}{\mathrm{id}}
\renewcommand{\SS}{\mathrm{S}}

\renewcommand{\geq}{\geqslant}
\renewcommand{\leq}{\leqslant}

\newcommand{\oldpage}[1]{\marginpar{\footnotesize$\Big\vert$ \textit{p.~#1}}}


%% Document %%

% \usepackage{embedall}
\begin{document}

\maketitle
\thispagestyle{fancy}

\renewcommand{\abstractname}{Translator's note.}

\begin{abstract}
  \renewcommand*{\thefootnote}{\fnsymbol{footnote}}
  \emph{This text is one of a series\footnote{\url{https://thosgood.com/translations/}} of translations of various papers into English.}
  \emph{The translator takes full responsibility for any errors introduced in the passage from one language to another, and claims no rights to any of the mathematical content herein.}

  \medskip
  
  \emph{What follows is a translation of the French paper:}

  \medskip\noindent
  \textsc{Cartan, H.}
  ``Sur la th\'{e}orie de Kan''.
  \emph{S\'{e}minaire Henri Cartan}, Volume~\textbf{9} (1956-1957), Talk no.~1, 1–19.
  {\url{http://www.numdam.org/item/SHC_1956-1957__9__A1_0/}}
\end{abstract}

\setcounter{footnote}{0}

\tableofcontents
\bigskip


%% Content %%

\oldpage{1-01}
\subsection*{Bibliography}

\begin{itemize}
  \item \textsc{Daniel M. Kan}, Notes to \emph{Proc. Nat. Acad. Sci. U.S.A} \textbf{42} (1956), 419--421 and 542--546.
  \item Various ``secret papers'' by Kan.
  \item The notes from the lectures of J.C.~Moore, Princeton 1955-56.
\end{itemize}


\section{Categories with simplicial structure}
\label{section1}

Let $\cat{C}$ be a category.
We associate to $\cat{C}$ a new category $\cat{C}^\SS$ as follows: an object of $\cat{C}^\SS$ is defined by the data of a sequence $(K_n)$ (for integer $n\geq0$) of objects of $\cat{C}$, along with, for each $n$, two sequences of maps
\[
  \begin{array}{ll}
    d_i\colon K_n \to K_{n-1}
    & \mbox{($0\leq i\leq n$, $n\geq1$)}
  \\s_i\colon K_n \to K_{n+1}
    & \mbox{($0\leq i\leq n$, $n\geq0$)}
  \end{array}
\]
that are ``morphisms'' of the category $\cat{C}$.
The $d_i$ (``face operators'') and the $s_i$ (``degeneracy operators'') are also subject to the following relations:
\[
\label{equation1}
  \begin{cases}
    d_i d_j = d_{j-1} d_i &\mbox{for $i<j$}
  \\s_i s_j = s_{j+1} s_i &\mbox{for $i\leq j$}
  \\d_i s_j = s_{j-1} d_i &\mbox{for $i<j$}
  \\d_i s_j = \id &\mbox{for $i=j$ or $i=j+1$}
  \\d_i s_j = s_j d_{i-1} &\mbox{for $i>j+1$.}
  \end{cases}
\tag{1}
\]


\end{document}

\documentclass{article}

\title{A comment about the integral representation of the Riemann $\xi$-function}
\author{George P\'{o}lya}
\date{}

\usepackage{amssymb,amsmath}

\usepackage{hyperref}
\usepackage[nameinlink]{cleveref}
\usepackage{enumerate}

\usepackage{mathrsfs}
%% Fancy fonts --- feel free to remove! %%
\usepackage{Baskervaldx}
\usepackage{mathpazo}


\usepackage{fancyhdr}
\usepackage{lastpage}
\usepackage{xstring}
\makeatletter
\ifx\pdfmdfivesum\undefined
  \let\pdfmdfivesum\mdfivesum
\fi
\edef\filesum{\pdfmdfivesum file {\jobname}}
\pagestyle{fancy}
\makeatletter
\let\runauthor\@author
\let\runtitle\@title
\makeatother
\fancyhf{}
\lhead{\footnotesize\runtitle}
\rhead{\footnotesize Version: \texttt{\StrMid{\filesum}{1}{8}}}
\cfoot{\small\thepage\ of \pageref*{LastPage}}


\crefname{section}{\S}{\SS}
\crefname{equation}{}{}


%% Theorem environments %%

\usepackage{amsthm}


%% Shortcuts %%

\newcommand{\sh}{\mathscr}
\newcommand{\cat}{\mathcal}
\newcommand{\dd}{\operatorname{d}\!}
\newcommand{\GG}{\mathfrak{G}}
\newcommand{\UU}{\mathscr{U}}
\newcommand{\TT}{\mathscr{T}}
\renewcommand{\SS}{\mathscr{S}}

\renewcommand{\geq}{\geqslant}
\renewcommand{\leq}{\leqslant}

\newcommand{\todo}{\textbf{ !TODO! }}
\newcommand{\unsure}[1]{\underline{\emph{#1 (?)}}}
\newcommand{\oldpage}[1]{\marginpar{\,\\\footnotesize$\Big\vert$ \textit{p.~#1}}}


%% Document %%

\usepackage{embedall}
\begin{document}

\maketitle
\thispagestyle{fancy}

\renewcommand{\abstractname}{Translator's note.}

\begin{abstract}
  \renewcommand*{\thefootnote}{\fnsymbol{footnote}}
  \emph{This text is one of a series\footnote{\url{https://thosgood.com/translations/}} of translations of various papers into English.}
  \emph{The translator takes full responsibility for any errors introduced in the passage from one language to another, and claims no rights to any of the mathematical content herein.}
  
  \emph{What follows is a translation of the German paper:}

  \medskip\noindent
  \textsc{P\'{o}lya, G.}
  ``Bemerkung \"{U}ber die Integraldarstellung der Riemannschen $\xi$-Funktion.''
  \emph{Acta Math.}, Volume~\textbf{48} (1926), pp.~305--317.
  \textsc{DOI:} \href{https://doi.org/10.1007/BF02565336}{10.1007/BF02565336}.
  % \emph{Journal}, Volume~\textbf{X} (Date), pp.~Y--Z.
  % {\footnotesize\url{URL}}
\end{abstract}

\setcounter{footnote}{0}

\bigskip


%% Content %%

\emph{[Translator.] The numbering of the footnotes in the original has not been replicated in this translation, since this would have resulted in multiple footnotes with the same number on one page.}

\bigskip

The Riemann $\xi$-function, defined by the formula
\oldpage{305}
\[
\label{1}
  \xi(iz)
  =
  \frac12 \left(
    z^2 - \frac14
  \right) \pi^{-\frac{z}{2}-\frac14} \Gamma \left(
    \frac{z}{2} + \frac14
  \right) \zeta \left(
    \frac12 + z
  \right),
\tag{1}
\]
was represented by Riemann himself by an infinite trigonometric integral, namely\footnote{\textsc{B. Riemann}, Werke (1876), S.~138.}
\[
\label{2}
  \xi(z) = 2\int_0^\infty \Phi(u)\cos(zu)\dd u
\tag{2}
\]
\[
\label{3}
  \Phi(u) = 2\pi e^{\frac{5u}{2}} \sum_{n=1}^\infty (2\pi e^{2u}n^2 - 3) n^2 e^{-n^2\pi e^{2u}}.
\tag{3}
\]
It is evident that
\[
\label{4}
  \Phi(u) \sim 4\pi^2 e^{\frac{9u}{2}-\pi e^{2u}}
  \quad\mbox{as $u\to+\infty$.}
\tag{4}
\]
Furthermore (see \cref{section4}), $\Phi(u)$ is an even function.
Therefore
\[
\label{5}
  \Phi(u) \sim 4\pi^2 \left(
    e^{\frac{9u}{2}} + e^{-\frac{9u}{2}}
  \right) e^{-\pi(e^{2u}+e^{-2u})}
  \quad\mbox{as $u\to\pm\infty$.}
\tag{5}
\]

\oldpage{306}
With regards to the Riemann hypothesis, one could ask the following question\footnote{This was casually mentioned by Prof.~Landau in a conversation in 1913.}: does the function given by replacing $\Phi(u)$ in the right-hand side of \cref{2} with the right-hand side of \cref{4} have only real zeros?

The answer is no (see \cref{section4}): the resulting function has infinitely many imaginary zeros.
If, however, the right-hand side of \cref{5} is used, instead of the right-hand side of \cref{4}, then we obtain the function
\[
\label{6}
  \xi^*(z) = 8\pi^2 \int_0^\infty \left(
    e^{\frac{9u}{2}} + e^{-\frac{9u}{2}}
  \right) e^{-\pi(e^{2u}+e^{-2u})} \cos(zu) \dd u,
\tag{6}
\]
which one could call the ``modified $\xi$-function'', and $\xi^*(z)$ in fact \emph{has only real zeros}.
Incidentally,
\[
  \xi(z) \sim \xi^*(z)
\]
if $z$ tends towards $\infty$ in some closed ray based at the point $0$ and not containing the real axis.
If we denote by $N(r)$ the number of zeros of $\xi(z)$ in the circular region $|z|\leq r$, and by $N^*(r)$ the number of zeros of $\xi^*(z)$ in the same region, then
\[
  N(r) \sim N^*(r)
\]
and even
\[
  N(r) - N^*(r) = O(\log r).
\]

In what follows, I provide a proof of the fact that all the zeros of $\xi^*(z)$ are real.
The proof focuses on another entire function, namely the function
\[
\label{7}
  \GG(z) = \GG(z;a) = \int_{-\infty}^{+\infty} e^{-a(e^u+e^{-u})+zu} \dd u,
\tag{7}
\]
which has some nice, simple properties.
The parameter $a$ is always a positive value.
We can express $\xi^*(z)$ in terms of $\GG(z)$.
Indeed,
\[
\label{8}
  \xi^*(z) = 2\pi^2 \left\{
    \GG\left(
      \frac{iz}{2} - \frac94; \pi
    \right) +
    \GG\left(
      \frac{iz}{2} + \frac94; \pi
    \right)
  \right\}
\tag{8}
\]

\oldpage{307}
as one can easily derive from \cref{6} and \cref{7}.
I will show that $\GG(iz)$ \emph{has only real zeros}.
From this, the same property then easily follows for $\xi^*(z)$.


\section{}
\label{section1}

The most important property of the entire function $\GG(z)$ is that is satisfies a simple difference equation;
namely
\[
\label{9}
  z\GG(z) = a(\GG(z+1) - \GG(z-1))
\tag{9}
\]
as can easily be proven by partial integration.
Note also that $\GG(z)$ is even:
\[
\label{10}
  \GG(-z) = \GG(z).
\tag{10}
\]
Furthermore, if $z$ is not an integer, then
\[
\label{11}
  \begin{aligned}
    \GG(z)
    &= a^{-z}\Gamma(z) \left(
      1 + \sum_{n=1}^\infty \frac{a^{2n}}{n!(1-z)(2-z)\ldots(n-z)}
    \right)
  \\&+ a^z\Gamma(-z) \left(
      1 + \sum_{n=1}^\infty \frac{a^{2n}}{n!(1+z)(2+z)\ldots(n+z)}
    \right).
  \end{aligned}
\tag{11}
\]

The proof of \cref{11} goes as follows: write
\[
\label{12}
  \Gamma(z)
  = \int_0^a e^{-v} v^{z-1} \dd v + \int_a^\infty e^{-v} v^{z-1} \dd v
  = P(z) + Q(z),
\tag{12}
\]
\[
\label{13}
  P(z) = P(z;a) = \sum_{n=0}^\infty \frac{(-1)^n a^{z+n}}{n!(z+n)},
\tag{13}
\]
\[
\label{14}
  Q(z) = Q(z;a) = a^z \int_0^\infty e^{-ae^u+uz} \dd u.
\tag{14}
\]

Now,
\[
  \begin{aligned}
    \GG(z)
    &= \int_0^\infty e^{-a(e^u+e^{-u})} (e^{uz}+e^{-uz}) \dd u
  \\&= \int_0^\infty \sum_{n=0}^\infty e^{-ae^u} \frac{(-a)^n}{n!} (e^{(-n+z)u}+e^{(-n-z)u}) \dd u.
  \end{aligned}
\]
From this it follows, by \cref{14}, after exchanging the order of summation and integration (which can easily be justified by absolute convergence), that
\oldpage{308}
\[
\label{15}
  \GG(z)
  = a^{-z}\sum_{n=0}^\infty \frac{(-1)^n a^{2n}}{n!} Q(-n+z)
  + a^z\sum_{n=0}^\infty \frac{(-1)^n a^{2n}}{n!} Q(-n-z).
\tag{15}
\]
On the other hand, evidently
\[
  \begin{aligned}
    0
    &= \sum_{k=0}^\infty \sum_{l=0}^\infty \frac{(-1)^{k+l}a^{k+l}}{k!l!} \left(
      \frac{1}{-k+z+l} + \frac{1}{k-z-l}
    \right)
  \\&= a^{-z} \sum_{k=0}^\infty \frac{(-1)^ka^{2k}}{k!} \sum_{l=0}^\infty \frac{(-1)^la^{-k+z+l}}{l!(-k+z+l)}
  \\&+ a^z \sum_{l=0}^\infty \frac{(-1)^la^{2l}}{l!} \sum_{k=0}^\infty \frac{(-1)^ka^{-l-z+k}}{k!(-l-z+k)}
  \end{aligned}
\]
and so, by \cref{13},
\[
\label{16}
  \GG(z)
  = a^{-z} \sum_{k=0}^\infty \frac{(-1)^ka^{2k}}{k!} P(-k+z)
  + a^z \sum_{l=0}^\infty \frac{(-1)^la^{2l}}{l!} P(-l-z).
\tag{16}
\]
It follows from \cref{12}, \cref{15}, and \cref{16} that
\[
  \GG(z)
  = a^{-z} \sum_{n=0}^\infty \frac{(-1)^na^{2n}}{n!} \Gamma(-n+z)
  + a^z \sum_{n=0}^\infty \frac{(-1)^na^{2n}}{n!} \Gamma(-n-z),
\]
which is equivalent to \cref{11}.

I will also state here, without proof, the two following representations, which will not be used in what follows:
\[
  \begin{aligned}
    \GG(z)
    &= \frac{1}{2\pi i} \int_{a-i\infty}^{a+i\infty} \Gamma\left(
      s - \frac{z}{2}
    \right) \Gamma \left(
      s + \frac{z}{2}
    \right) a^{-2s} \dd s,
  \\\GG(z)
    &= \frac{\pi}{\sin(\pi z)} e^{\frac{i\pi z}{2}} J_{-z}(2ia) - \frac{\pi}{\sin(\pi z)} e^{-\frac{i\pi z}{2}} J_z(2ia).
  \end{aligned}
\]
In the first one, $2a>|\Re z|$; in the second, both terms on the right-hand side satisfy the same difference equation, which is also satisfied by $\GG(z)$, by \cref{9}.


\section{}
\label{section2}

The asymptotic representation and the estimates of $\GG(z)$, with which we now concern ourselves, can be obtained by expanding \cref{11}.

\oldpage{309}
Different areas of the $z$-plane have to be examined on a case-by-case basis.
Fix $z = x+iy + re^{i\varphi}$, where $x$, $y$, $r$, and $\varphi$ are real; $r\geq0$; if $y>0$, then we assume that $0<\varphi<\pi$; $x-iy = \overline{z}$.

\begin{enumerate}[I.]
  \item \emph{For $|y|\geq1$ and $a\leq A$,}
    \[
    \label{17}
      \GG(z)
      = a^{-z} \Gamma(z) \left(
        1 + \frac{\chi(z)}{z}
      \right) + a^z \Gamma(z) \left(
        1 - \frac{\chi(-z)}{z}
      \right)
    \tag{17}
    \]
    \emph{where the absolute value of the function $\chi(z)=\chi(z;a)$ is bounded above by a value depending only on $A$.}
    \label{I}

    We know, by \cref{11}, that
    \[
      \chi(z)
      = \frac{za^2}{1-z} \left(
        1 + \sum_{n=2}^\infty \frac{a^{2n-2}}{n!(2-z)(3-z)\ldots(n-z)}
      \right)
    \]
    whence, without further ado, the claim.
  \item \emph{Let $\varepsilon$ and $A$ be fixed positive numbers.}
    \emph{Set}
    \[
    \label{18}
      \frac{a^z\GG(z)}{\Gamma(z)} - 1
      = \psi(z) = \psi(z;a).
    \tag{18}
    \]
    \emph{Then, in the half-plane $x\geq\varepsilon$,}
    \[
    \label{19}
      \lim_{|z|\to\infty} \psi(z) = 0,
    \tag{19}
    \]
    \emph{and, in fact, if $a\leq A$, then this convergence is uniform in $z$ and $a$.}
    \label{II}

    It follows, from Stirling's formula, that there exists some constant $C$ such that, for $y\geq1$, $x\geq\varepsilon$, and $r\geq ae$,
    \[
    \label{20}
      \left\vert
        a^{2z} \frac{\Gamma(-z)}{\Gamma(z)}
      \right\vert \leq a^{2x}Cr^{-2x}e^{-(\pi-2\varphi)y+2x}
      \leq C\left(
        \frac{ae}{r}
      \right)^{2\varepsilon}.
    \tag{20}
    \]
    In the part of the half-plane where $x\geq\varepsilon$, where $y\geq1$, claim~II follows directly from \cref{17}, \cref{18}, and \cref{20};
    it also follows, by symmetry, in the part where $y\leq-1$.
    Then the only remaining part is the half-strip
    \[
      x\geq\varepsilon,
      \quad -1\leq y\leq 1.
    \]
    By the already-proven part of claim~II, function~\cref{18} tends to zero as $z\to\infty$.
    Since function~\cref{18} is entire and
\oldpage{310}
    of finite order, by well-known general theorems\footnote{See \textsc{G.~P\'{o}lya and G.~Szeg\"{o}}, Aufgaben und Lehrs\"{a}tze aus der Analysis (Berlin 1925), Bd.~1, Aufgaben~III~333, III~339.}, it must tend uniformly to zero in \todo half-strips as $z\to\infty$;
    this completes the proof of claim~II.
    Of course, instead of using general theorems, function~\cref{18} could also be used on suitable curves intersecting the half-strip, e.g. to estimate on the straight lines $x=k+\frac12$, for $k=0,1,2,3,\ldots$.
    In the case where $k=0$, we investigate such an estimation for a different purpose, c.f. \cref{22}.
  \item \emph{If $a\leq\frac14$, then there are no zeros of $\GG(z;a)$ outside of the strip $-\frac12<x<\frac12$.}
    \label{III}

    Since $\GG(z)$ is even, it suffices to consider the half-plane $x\geq\frac12$.
    On the straight line $x=\frac12$,
    \[
      z+\overline{z} = 1,
      \quad \overline{z} = 1-z
    \]
    and so, by the symmetry of \unsure{?} with respect to the real axis,
    \[
      |\Gamma(1-z)| = |\Gamma(z)|,
    \]
    whence
    \[
      \left\vert
        \frac{\Gamma(-z)}{\Gamma(z)}
      \right\vert = \left\vert
        \frac{\Gamma(1-z)}{-z\Gamma(z)}
      \right\vert = \frac{1}{|z|} \leq 2
      \qquad\left(
        x=\frac12
      \right).
    \]
    Using \cref{21}, it follows from \cref{11} that, on the straight line $x=\frac12$,
    \[
    \label{22}
      \begin{aligned}
        |\psi(z)|
        &= \left\vert
          \frac{a^z\GG(z)}{\Gamma(z)} - 1
        \right\vert
      \\&= \left\vert
          \sum_{n=1}^\infty \frac{a^{2n}}{n!(1-z)\ldots(n-z)}
          + \frac{a^{2z}\Gamma(-z)}{\Gamma(z)} \left(
            1 + \sum_{n=1}^\infty \frac{a^{2n}}{n!(1+z)\ldots(n+z)}
          \right)
        \right\vert
      \\&\leq \sum_{n=1}^\infty \frac{a^{2n}}{n!\frac12\cdot\frac32\cdot\ldots\cdot\frac{2n-1}{2}} + a\cdot2 \left(
        1 + \sum_{n=1}^\infty \frac{a^{2n}}{n!\frac32\cdot\frac52\cdot\ldots\cdot\frac{2n+1}{2}}
      \right)
      \\&= \sum_{n=1}^\infty \frac{(2a)^{2n}}{(2n)!} + \sum_{n=0}^\infty \frac{(2a)^{2n+1}}{(2n+1)!}
      \\&= e^{2a}-1.
      \end{aligned}
    \tag{22}
    \]

  \oldpage{311}
    It then follows from \cref{22} that
    \[
    \label{23}
      |\psi(z)| = |\psi(z;a)| < 1
      \qquad\mbox{for $x=\frac12$ and $a\leq\frac14$.}
    \tag{23}
    \]
    It follows from claim~II that there exists some number $R$ such that
    \[
    \label{24}
      |\psi(z)| = |\psi(z;a)| <1
      \qquad\mbox{for $x\geq\frac12$, $r\geq R$, and $a\leq\frac14$.}
    \tag{24}
    \]
    Furthermore, $|\psi(z)|<1$ is also satisfied in the circle segment $\{x\geq\frac12,r\leq R\}$, since the inequality in question, by \cref{23} and \cref{24}, is satisfied on the boundary of the circle segment.
    Thus $|\psi(z)|<1$ is satisfied in the whole half-plane $x\geq\frac12$;
    thus $\GG(z)$, cf.~\cref{18}, does not disappear in this half-plane.

    \todo (PICTURE)

    For what follows, it is useful to have in mind a certain partition of the plane (cf. the above figure): the strip $-1\leq x\leq 1$ is denoted by $\SS$; the part of the plane where both $r\leq R$ and $|x|>1$ is denoted by $\TT$; the remaining part, which extends to infinity, is denoted by $\UU$.
    We pick the number $R=R(A)$ depending on a certain positive number $A$ such that, inside, and on the boundary of, $\UU$, the inequality $|\psi(z)|<1$ is satisfied for $a\leq A$.
    From \cref{18} it follows that $\GG(z)$ does not vanish inside, or on the boundary of, $\UU$.
\end{enumerate}


\section{}
\label{section3}

We will now study consequences of the difference equation in \cref{9}, and link them to those that we have just obtained from the representation in \cref{11}.

\begin{enumerate}[I.]
\setcounter{enumi}{3}
  \item \emph{$\GG(z)$ has no zeros on the straight lines $x=1$ or $x=-1$, i.e. on the boundary of $\SS$.}
    \label{IV}

    There are two things that we have to consider:
\oldpage{312}
    firstly, the difference equation and the symmetry of $\GG(z)$; secondly, the asymptotic properties of $\GG(z)$.

    Firstly:
    the function $\GG(z)$ is even, and takes real values for all $z$.
    From this it follows that, if $y$ is real, then $\GG(iy)$ is real, and the two values $\GG(1+iy)$ and $\GG(-1+iy)$ are complex conjugate.
    So \cref{9} implies, for $z=iy$, that
    \[
    \label{25}
      y\GG(iy) = 2a\Im\GG(1+iy).
    \tag{25}
    \]

    Secondly:
    it is not possible for $\GG(z)$ to vanish at two points simultaneously if the line connecting them is parallel to the real axis and is of length~$1$.
    This is since, if $\GG(c)=0$ and $\GG(c-1)=0$, then \cref{9} would imply that $\GG(c+1)=0$ as well, and thus that $\GG(c+2)=0$, $\GG(c+3)=0$, ..., and so $\GG(z)$ would have zeros in the region $\UU$ (see the figure); but this would be a contradiction.

    For real $z$, $\GG(z)$ is positive (cf. \cref{7}).
    On one hand, $\GG(1+iy)=0$ implies that $y\neq0$, and, on the other hand, $\GG(1+iy)=0$ implies that $\Im\GG(1+iy)=0$, and thus \cref{25} implies that $\GG(iy)=0$.
    However, the simultaneous vanishing of $\GG(1+iy)$ and $\GG(iy)$ is impossible, and thus $\GG(1+iy)\neq0$.
  \item \emph{There are no zeros of $\GG(z)$ outside of the strip $\SS$.}
    \label{V}

    This has already been proven for $a\leq\frac14$, cf. \hyperref[III]{III}.
    Let
    \[
    \label{26}
      \frac14 \leq a \leq A.
    \tag{26}
    \]
    As we have already established, there are no zeros of $\GG(z)$ in the region $\UU$ nor on its boundary.
    If we take \hyperref[IV]{IV} into account as well, we see that there are no zeros of $\GG(z)$ on the boundary of $\TT$ either.
    If $z$ varies along the boundary of $\TT$, and $a$ in the closed interval \cref{26}, then $\GG(z;a)$ is a continuous non-zero function of $z$ and $a$.
    Thus the integral
    \[
    \label{27}
      \frac{1}{2\pi i} \int \frac{\GG'(z;a)\dd z}{\GG(z;a)}
    \tag{27}
    \]
    (along the boundary of $\TT$ in the positive direction) is a continuous function of $a$.
    The integral \cref{27} calculates an integer, namely the number of zeros of $\GG(z;a)$ that lie inside $\TT$.
    A continuous function that takes only integer values is constant.
    Thus the integral is always equal to $0$, since it is equal to zero for $a=\frac14$, cf. \hyperref[III]{III}.
\oldpage{313}
  \item \emph{All the zeros of $\GG(z)$ lying inside the strip $\SS$ are simple and purely imaginary.}
    \label{VI}

    It follows from \cref{17} and from Stirling's formula that, in the strip $1\leq x\leq 1$, for $y\to+\infty$, \todo
    \[
    \label{28}
      \GG(x+iy)
      =
      \frac{1}{\sqrt{2\pi y}} e^{-\frac\pi2 y+ i\frac\pi2 x}
      \left\{
        \left(\frac{y}{a}\right)^x e^{i\Phi} + 
        \left(\frac{y}{a}\right)^{-x} e^{-i\Phi}
      \right\} + O\left(
        e^{-\frac\pi2 y} y^{|z| - \todo}
      \right)
    \tag{28}
    \]
    where, for abbreviation, we set
    \[
    \label{29}
      \Phi = y\log\frac{y}{a} - y - \frac\pi4.
    \tag{29}
    \]
\end{enumerate}



%% Bibliography %%

\nocite{*}
\bibliographystyle{acm}

\end{document}

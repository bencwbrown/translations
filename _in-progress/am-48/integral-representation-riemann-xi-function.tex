\documentclass{article}

\title{A comment about the integral representation of the Riemann $\xi$-function}
\author{George P\'{o}lya}
\date{}

\usepackage{amssymb,amsmath}

\usepackage{hyperref}
\usepackage[nameinlink]{cleveref}
\usepackage{enumerate}

\usepackage{mathrsfs}
%% Fancy fonts --- feel free to remove! %%
\usepackage{Baskervaldx}
\usepackage{mathpazo}


\usepackage{fancyhdr}
\usepackage{lastpage}
\usepackage{xstring}
\makeatletter
\ifx\pdfmdfivesum\undefined
  \let\pdfmdfivesum\mdfivesum
\fi
\edef\filesum{\pdfmdfivesum file {\jobname}}
\pagestyle{fancy}
\makeatletter
\let\runauthor\@author
\let\runtitle\@title
\makeatother
\fancyhf{}
\lhead{\footnotesize\runtitle}
\rhead{\footnotesize Version: \texttt{\StrMid{\filesum}{1}{8}}}
\cfoot{\small\thepage\ of \pageref*{LastPage}}


\crefname{section}{\S}{\SS}
\crefname{equation}{}{}


%% Theorem environments %%

\usepackage{amsthm}


%% Shortcuts %%

\newcommand{\sh}{\mathscr}
\newcommand{\cat}{\mathcal}
\newcommand{\dd}{\operatorname{d}\!}
\newcommand{\GG}{\mathfrak{G}}

\renewcommand{\geq}{\geqslant}
\renewcommand{\leq}{\leqslant}

\newcommand{\todo}{\textbf{ !TODO! }}
\newcommand{\oldpage}[1]{\marginpar{\footnotesize$\Big\vert$ \textit{p.~#1}}}


%% Document %%

\usepackage{embedall}
\begin{document}

\maketitle
\thispagestyle{fancy}

\renewcommand{\abstractname}{Translator's note.}

\begin{abstract}
  \renewcommand*{\thefootnote}{\fnsymbol{footnote}}
  \emph{This text is one of a series\footnote{\url{https://thosgood.com/translations/}} of translations of various papers into English.}
  \emph{The translator takes full responsibility for any errors introduced in the passage from one language to another, and claims no rights to any of the mathematical content herein.}
  
  \emph{What follows is a translation of the German paper:}

  \medskip\noindent
  \textsc{P\'{o}lya, G.}
  ``Bemerkung \"{U}ber die Integraldarstellung der Riemannschen $\xi$-Funktion.''
  \emph{Acta Math.}, Volume~\textbf{48} (1926), pp.~305--317.
  \textsc{DOI:} \href{https://doi.org/10.1007/BF02565336}{10.1007/BF02565336}.
  % \emph{Journal}, Volume~\textbf{X} (Date), pp.~Y--Z.
  % {\footnotesize\url{URL}}
\end{abstract}

\setcounter{footnote}{0}

\bigskip


%% Content %%

\emph{[Translator.] The numbering of the footnotes in the original has not been replicated in this translation, since this would have resulted in multiple footnotes with the same number on one page.}

\bigskip

The Riemann $\xi$-function, defined by the formula
\oldpage{305}
\[
\label{1}
  \xi(iz)
  =
  \frac12 \left(
    z^2 - \frac14
  \right) \pi^{-\frac{z}{2}-\frac14} \Gamma \left(
    \frac{z}{2} + \frac14
  \right) \zeta \left(
    \frac12 + z
  \right),
\tag{1}
\]
was represented by Riemann himself by an infinite trigonometric integral, namely\footnote{\textsc{B. Riemann}, Werke (1876), S.~138.}
\[
\label{2}
  \xi(z) = 2\int_0^\infty \Phi(u)\cos(zu)\dd u
\tag{2}
\]
\[
\label{3}
  \Phi(u) = 2\pi e^{\frac{5u}{2}} \sum_{n=1}^\infty (2\pi e^{2u}n^2 - 3) n^2 e^{-n^2\pi e^{2u}}.
\tag{3}
\]
It is evident that
\[
\label{4}
  \Phi(u) \sim 4\pi^2 e^{\frac{9u}{2}-\pi e^{2u}}
  \quad\mbox{as $u\to+\infty$.}
\tag{4}
\]
Furthermore (see \cref{section4}), $\Phi(u)$ is an even function.
Therefore
\[
\label{5}
  \Phi(u) \sim 4\pi^2 \left(
    e^{\frac{9u}{2}} + e^{-\frac{9u}{2}}
  \right) e^{-\pi(e^{2u}+e^{-2u})}
  \quad\mbox{as $u\to\pm\infty$.}
\tag{5}
\]

\oldpage{306}
With regards to the Riemann hypothesis, one could ask the following question\footnote{This was casually mentioned by Prof.~Landau in a conversation in 1913.}: does the function given by replacing $\Phi(u)$ in the right-hand side of \cref{2} with the right-hand side of \cref{4} have only real zeros?

The answer is no (see \cref{section4}): the resulting function has infinitely many imaginary zeros.
If, however, the right-hand side of \cref{5} is used, instead of the right-hand side of \cref{4}, then we obtain the function
\[
\label{6}
  \xi^*(z) = 8\pi^2 \int_0^\infty \left(
    e^{\frac{9u}{2}} + e^{-\frac{9u}{2}}
  \right) e^{-\pi(e^{2u}+e^{-2u})} \cos(zu) \dd u,
\tag{6}
\]
which one could call the ``modified $\xi$-function'', and $\xi^*(z)$ in fact \emph{has only real zeros}.
Incidentally,
\[
  \xi(z) \sim \xi^*(z)
\]
if $z$ tends towards $\infty$ in some closed ray based at the point $0$ and not containing the real axis.
If we denote by $N(r)$ the number of zeros of $\xi(z)$ in the circular region $|z|\leq r$, and by $N^*(r)$ the number of zeros of $\xi^*(z)$ in the same region, then
\[
  N(r) \sim N^*(r)
\]
and even
\[
  N(r) - N^*(r) = O(\log r).
\]

In what follows, I provide a proof of the fact that all the zeros of $\xi^*(z)$ are real.
The proof focuses on another entire function, namely the function
\[
\label{7}
  \GG(z) = \GG(z;a) = \int_{-\infty}^{+\infty} e^{-a(e^u+e^{-u})+zu} \dd u,
\tag{7}
\]
which has some nice, simple properties.
The parameter $a$ is always a positive value.
We can express $\xi^*(z)$ in terms of $\GG(z)$.
Indeed,
\[
\label{8}
  \xi^*(z) = 2\pi^2 \left\{
    \GG\left(
      \frac{iz}{2} - \frac94; \pi
    \right) +
    \GG\left(
      \frac{iz}{2} + \frac94; \pi
    \right)
  \right\}
\tag{8}
\]

\oldpage{307}
as one can easily derive from \cref{6} and \cref{7}.
I will show that $\GG(iz)$ \emph{has only real zeros}.
From this, the same property then easily follows for $\xi^*(z)$.


\section{}
\label{section1}

The most important property of the entire function $\GG(z)$ is that is satisfies a simple difference equation;
namely
\[
\label{9}
  z\GG(z) = a(\GG(z+1) - \GG(z-1))
\tag{9}
\]
as can easily be proven by partial integration.
Note also that $\GG(z)$ is even:
\[
\label{10}
  \GG(-z) = \GG(z).
\tag{10}
\]
Furthermore, if $z$ is not an integer, then
\[
\label{11}
  \begin{aligned}
    \GG(z)
    &= a^{-z}\Gamma(z) \left(
      1 + \sum_{n=1}^\infty \frac{a^{2n}}{n!(1-z)(2-z)\ldots(n-z)}
    \right)
  \\&+ a^z\Gamma(-z) \left(
      1 + \sum_{n=1}^\infty \frac{a^{2n}}{n!(1+z)(2+z)\ldots(n+z)}
    \right).
  \end{aligned}
\tag{11}
\]

The proof of \cref{11} goes as follows: write
\[
\label{12}
  \Gamma(z)
  = \int_0^a e^{-v} v^{z-1} \dd v + \int_a^\infty e^{-v} v^{z-1} \dd v
  = P(z) + Q(z),
\tag{12}
\]
\[
\label{13}
  P(z) = P(z;a) = \sum_{n=0}^\infty \frac{(-1)^n a^{z+n}}{n!(z+n)},
\tag{13}
\]
\[
\label{14}
  Q(z) = Q(z;a) = a^z \int_0^\infty e^{-ae^u+uz} \dd u.
\tag{14}
\]

Now,
\[
  \begin{aligned}
    \GG(z)
    &= \int_0^\infty e^{-a(e^u+e^{-u})} (e^{uz}+e^{-uz}) \dd u
  \\&= \int_0^\infty \sum_{n=0}^\infty e^{-ae^u} \frac{(-a)^n}{n!} (e^{(-n+z)u}+e^{(-n-z)u}) \dd u.
  \end{aligned}
\]
From this it follows, by \cref{14}, after exchanging the order of summation and integration (which can easily be justified by absolute convergence), that
\oldpage{308}
\[
\label{15}
  \GG(z)
  = a^{-z}\sum_{n=0}^\infty \frac{(-1)^n a^{2n}}{n!} Q(-n+z)
  + a^z\sum_{n=0}^\infty \frac{(-1)^n a^{2n}}{n!} Q(-n-z).
\tag{15}
\]
On the other hand, evidently
\[
  \begin{aligned}
    0
    &= \sum_{k=0}^\infty \sum_{l=0}^\infty \frac{(-1)^{k+l}a^{k+l}}{k!l!} \left(
      \frac{1}{-k+z+l} + \frac{1}{k-z-l}
    \right)
  \\&= a^{-z} \sum_{k=0}^\infty \frac{(-1)^ka^{2k}}{k!} \sum_{l=0}^\infty \frac{(-1)^la^{-k+z+l}}{l!(-k+z+l)}
  \\&+ a^z \sum_{l=0}^\infty \frac{(-1)^la^{2l}}{l!} \sum_{k=0}^\infty \frac{(-1)^ka^{-l-z+k}}{k!(-l-z+k)}
  \end{aligned}
\]
and so, by \cref{13},
\[
\label{16}
  \GG(z)
  = a^{-z} \sum_{k=0}^\infty \frac{(-1)^ka^{2k}}{k!} P(-k+z)
  + a^z \sum_{l=0}^\infty \frac{(-1)^la^{2l}}{l!} P(-l-z).
\tag{16}
\]
It follows from \cref{12}, \cref{15}, and \cref{16} that
\[
  \GG(z)
  = a^{-z} \sum_{n=0}^\infty \frac{(-1)^na^{2n}}{n!} \Gamma(-n+z)
  + a^z \sum_{n=0}^\infty \frac{(-1)^na^{2n}}{n!} \Gamma(-n-z),
\]
which is equivalent to \cref{11}.

I will also state here, without proof, the two following representations, which will not be used in what follows:
\[
  \begin{aligned}
    \GG(z)
    &= \frac{1}{2\pi i} \int_{a-i\infty}^{a+i\infty} \Gamma\left(
      s - \frac{z}{2}
    \right) \Gamma \left(
      s + \frac{z}{2}
    \right) a^{-2s} \dd s,
  \\\GG(z)
    &= \frac{\pi}{\sin(\pi z)} e^{\frac{i\pi z}{2}} J_{-z}(2ia) - \frac{\pi}{\sin(\pi z)} e^{-\frac{i\pi z}{2}} J_z(2ia).
  \end{aligned}
\]
In the first one, $2a>|\Re z|$; in the second, both terms on the right-hand side satisfy the same difference equation, which is also satisfied by $\GG(z)$, by \cref{9}.


\section{}
\label{section2}



%% Bibliography %%

\nocite{*}
\bibliographystyle{acm}

\end{document}

\documentclass{report}

\usepackage[margin=1.6in]{geometry}

\title{Regular singular differential equations}
\author{Pierre Deligne}
\date{}

\usepackage{amssymb,amsmath}

\usepackage{hyperref}
\usepackage{xcolor}
\hypersetup{colorlinks,linkcolor={red!50!black},citecolor={blue!50!black},urlcolor={blue!80!black}}
\usepackage[nameinlink]{cleveref}
\usepackage{enumerate}
\usepackage{graphicx}
\usepackage{tikz-cd}

\usepackage{mathrsfs}
%% Fancy fonts --- feel free to remove! %%
\usepackage{Baskervaldx}
\usepackage{mathpazo}


\usepackage{fancyhdr}
\usepackage{lastpage}
\usepackage{xstring}
\makeatletter
\ifx\pdfmdfivesum\undefined
  \let\pdfmdfivesum\mdfivesum
\fi
\edef\filesum{\pdfmdfivesum file {\jobname}}
\pagestyle{fancy}
\fancypagestyle{plain}{}
\fancyhf{}
\lhead{\footnotesize\nouppercase\leftmark}
\rhead{\footnotesize Version: \texttt{\StrMid{\filesum}{1}{8}}}
\cfoot{\small\thepage\ of \pageref*{LastPage}}


\crefname{section}{\S\!}{\S\S\!}
\crefname{equation}{}{}
\crefdefaultlabelformat{#2(#1)#3}


%% Theorem environments %%

\usepackage{amsthm}

\theoremstyle{plain}

  \newtheorem{innercustomproposition}{Proposition}
  \crefname{innercustomproposition}{}{}
  \newenvironment{proposition}[1]
    {\renewcommand\theinnercustomproposition{#1}\innercustomproposition}
    {\endinnercustomproposition}

  \newtheorem{innercustomlemma}{Lemma}
  \crefname{innercustomlemma}{}{}
  \newenvironment{lemma}[1]
    {\renewcommand\theinnercustomlemma{#1}\innercustomlemma}
    {\endinnercustomlemma}

  \newtheorem{innercustomtheorem}{Theorem}
  \crefname{innercustomtheorem}{}{}
  \newenvironment{theorem}[1]
    {\renewcommand\theinnercustomtheorem{#1}\innercustomtheorem}
    {\endinnercustomlemma}

  \newtheorem{innercustomcorollary}{Corollary}
  \crefname{innercustomcorollary}{}{}
  \newenvironment{corollary}[1]
    {\renewcommand\theinnercustomcorollary{#1}\innercustomcorollary}
    {\endinnercustomcorollary}


\theoremstyle{definition}

  \newtheorem{innercustomdefinition}{Definition}
  \crefname{innercustomdefinition}{}{}
  \newenvironment{definition}[1]
    {\renewcommand\theinnercustomdefinition{#1}\innercustomdefinition}
    {\endinnercustomdefinition}

  \newtheorem{innercustomreminder}{Reminder}
  \crefname{innercustomreminder}{}{}
  \newenvironment{reminder}[1]
    {\renewcommand\theinnercustomreminder{#1}\innercustomreminder}
    {\endinnercustomreminder}

  \newtheorem{innercustomenv}{}
  \crefname{innercustomenv}{}{}
  \newenvironment{env}[1]
    {\renewcommand\theinnercustomenv{#1}\innercustomenv}
    {\endinnercustomenv}


%% Shortcuts %%

\newcommand{\sh}{\mathscr}
\newcommand{\cat}{\mathcal}
\newcommand{\sbullet}{{\mathbin{\vcenter{\hbox{\scalebox{.5}{$\bullet$}}}}}}
\newcommand{\id}{\mathrm{Id}}
\newcommand{\PP}{\mathbb{P}}
\renewcommand{\AA}{\mathbb{A}}
\newcommand{\CC}{\mathbb{C}}
\newcommand{\ZZ}{\mathbb{Z}}
\newcommand{\NN}{\mathbb{N}}
\newcommand{\dd}{\mathrm{d}}
\newcommand{\DD}{\mathrm{D}}
\newcommand{\pr}{\mathrm{pr}}
\newcommand{\II}{\mathrm{II}}
\newcommand{\RR}{\mathbf{R}}
\newcommand{\GL}{\mathrm{GL}}
\newcommand{\tg}{\mathrm{tg}}
\newcommand{\detrow}[2]{\operatorname{det}(#1,#2)}

\renewcommand{\geq}{\geqslant}
\renewcommand{\leq}{\leqslant}

\DeclareMathOperator{\Spec}{Spec}
\DeclareMathOperator{\Ker}{Ker}
\DeclareMathOperator{\Hom}{Hom}
\DeclareMathOperator{\End}{End}
\DeclareMathOperator{\shHom}{\underline{Hom}}
\DeclareMathOperator{\shEnd}{\underline{End}}
\DeclareMathOperator{\HH}{H}
\DeclareMathOperator{\Gr}{Gr}
\DeclareMathOperator{\tr}{tr}

\newcommand{\todo}{\textbf{ !TODO! }}
\newcommand{\oldpage}[1]{\marginpar{\footnotesize$\Big\vert$ \textit{p.~#1}}}


%% Document %%

\usepackage{embedall}
\begin{document}

\hypersetup{pageanchor=false}
\begin{titlepage}
\maketitle
\end{titlepage}

\renewcommand{\abstractname}{Translator's note.}

\begin{abstract}
  \renewcommand*{\thefootnote}{\fnsymbol{footnote}}
  \emph{This text is one of a series\footnote{\url{https://thosgood.com/translations/}} of translations of various papers into English.}
  \emph{The translator takes full responsibility for any errors introduced in the passage from one language to another, and claims no rights to any of the mathematical content herein.}

  \medskip
  
  \emph{What follows is a translation of the French book:}

  \medskip\noindent
  \textsc{Deligne, P.}
  \emph{Equations Diff\'{e}rentielles \`{a} Points Singuliers R\'{e}guliers.}
  Springer--Verlag, Lecture Notes in Mathematics \textbf{163} (1970).
  {\url{https://publications.ias.edu/node/355}}

  \medskip
  \noindent\emph{We have also made changes following the errata, which was written in April 1971, by P. Deligne, at Warwick University.}
\end{abstract}

\hypersetup{pageanchor=true}

\setcounter{footnote}{0}

\tableofcontents


%% Content %%

\setcounter{chapter}{-1}

\chapter{Introduction}
\label{0}

(\todo \textbf{errata})

\oldpage{1}
If $X$ is a (non-singular) complex-analytic manifold, then there is an equivalence between the notions of
\begin{enumerate}[a)]
  \item local systems of complex vectors on $X$ ; and
  \item vector bundles on $X$ endowed with an integrable connection.
\end{enumerate}

The latter of these two notions can be adapted in an evident way to the case where $X$ is a non-singular algebraic variety over a field $k$ (which we will take here to be of characteristic $0$).
However, general algebraic vector bundles with integrable connections are pathological (see \cref{II.6.19});
we only obtain a reasonable theory if we impose a ``regularity'' condition at infinity.
By a theorem of Griffiths \cite{8}, this condition is automatically satisfied for ``Gauss-Manin connections'' (see \cref{II.7}).
In dimension one, this is closely linked to the idea of regular singular points of a differential equation (see \cref{I.4} and \cref{II.1}).

In Chapter~I, we explain the different forms that the notion of an integrable connection can take.
In Chapter~II, we prove the fundamental facts concerning regular connections.
In Chapter~III, we translate certain results that we have obtained into the language of Nilsson class functions, and, as an application of the regularity theorem (\cref{II.7}), we explain the proof by Brieskorn \cite{5} of the monodromy theorem.

These notes came from the non-crystalline part of a seminar given at Harvard during the autumn of 1969, under the title: ``Regular singular differential equations and crystalline cohomology''.

I thank the assistants of this seminar, who had to be subjected to often unclear expos\'{e}s, and who allowed me to bring numerous simplifications.

I also thank N.~Katz, with whom I had numerous and useful conversations, and to whom are due the principal results of section~\cref{II.1}.


\section*{Notation and terminology}

\oldpage{2}
Within a single chapter, the references follow the decimal system.
A reference to a different chapter (resp. to the current introduction) is preceded by the Roman numeral of the chapter (resp. by 0).

We will use the following definitions:
\begin{enumerate}[({0.}1)]
  \item\label{0.1}
    \emph{analytic space}:
    the analytic spaces are complex and of locally-finite dimension.
    They are assumed to be $\sigma$-compact, but not necessarily separated.
  \item\label{0.2}
    \emph{multiform function}:
    a synonym for multivalued function --- for a precise definition, see \cref{I.6.2}.
  \item\label{0.3}
    \emph{immersion}:
    following the tradition of algebraic geometers, immersion is a synonym for ``embedding''.
  \item\label{0.4}
    \emph{smooth}:
    a morphism $f\colon X\to S$ of analytic spaces is smooth if, locally on $X$, it is isomorphic to the projection from $D^n\times S$ to $S$, where $D^n$ is an open polydisc.
  \item\label{0.5}
    \emph{locally paracompact}:
    a topological space is locally paracompact if every point has a paracompact neighbourhood (and thus a fundamental system of paracompact neighbourhoods).
  \item\label{0.6}
    non-singular (or smooth) \emph{complex algebraic variety}:
    a smooth scheme of finite type over $\Spec(\mathbb{C})$.
  \item\label{0.7}
    (complex) \emph{analytic manifold}:
    a non-singular (or smooth) analytic space.
  \item\label{0.8}
    \emph{covering}:
    following the tradition of topologists, a covering is a continuous map $f\colon X\to Y$ such that every point $y\in Y$ has a neighbourhood $V$ such that $f|V$ is isomorphic to the projection from $F\times V$ to $V$, where $F$ is discrete.
\end{enumerate}


\renewcommand{\thechapter}{\Roman{chapter}}

\chapter{Dictionary}
\label{I}

\oldpage{3}
In this chapter, we explain the relations between various aspects and various uses of the notion of ``local systems of complex vectors''.
The equivalence between the points of view considered has been well known for a long time.

We do not consider the ``crystalline'' point of view;
see \cite{4,10}.


\section{Local systems and the fundamental group}
\label{I.1}

\begin{definition}{1.1}
\label{I.1.1}
  Let $X$ be a topological space.
  A \emph{complex local system} on $X$ is a sheaf of complex vectors on $X$ that, locally on $X$, is isomorphic to a constant sheaf $\mathbb{C}^n$ (n$\in\mathbb{N}$).
\end{definition}

\begin{env}{1.2}
\label{I.1.2}
  Let $X$ be a locally path-connected and locally simply path-connected topological space, along with a basepoint $x_0\in X$.
  To avoid any ambiguity, we point out that:
  \begin{enumerate}[a)]
    \item The fundamental group $\pi_1(X,x_0)$ of $X$ at $x_0$ has elements given by homotopy classes of loops based at $x_0$;
    \item If $\alpha,\beta\in\pi_1(X,x_0)$ are represented by loops $a$ and $b$, then $\alpha\beta$ is represented by the loop $ab$ obtained by juxtaposing $b$ and $a$, in that order.
  \end{enumerate}

  Let $\sh{F}$ be a locally constant sheaf on $X$.
  For every path $a\colon[0,1]\to X$, the inverse image $a^*\sh{F}$ of $\sh{F}$ on $[0,1]$ is a locally constant, and thus constant, sheaf, and there exists exactly one isomorphism between $a^*\sh{F}$ and the constant sheaf defined by the set $(a^*\sh{F})_0 = \sh{F}_{a(0)}$.
  This isomorphism defines an isomorphism $a(\sh{F})$ between $(a^*\sh{F})_0$ and $(a^*\sh{F})_1$, i.e. an isomorphism
  \[
    a(\sh{F})\colon \sh{F}_{a(0)} \to \sh{F}_{a(1)}.
  \]
  This isomorphism depends only on the homotopy class of $a$, and satisfies $ab(\sh{F}) = a(\sh{F})\cdot b(\sh{F})$.
  In particular, $\pi_1(X,x_0)$ acts (on the left) on the fibre $\sh{F}_{x_0}$ of $\sh{F}$ at $x_0$.
  It is well known that:
\end{env}

\begin{proposition}{1.3}
\label{I.1.3}
  Under the hypotheses of \cref{I.1.2}, with $X$ connected, the functor $\sh{F}\mapsto\sh{F}_{x_0}$ is an equivalence between the category of locally constant sheaves on $X$ and the category of sets endowed with an action by the group $\pi_1(X,x_0)$.
\end{proposition}

\oldpage{4}
\begin{corollary}{1.4}
\label{I.1.4}
  Under the hypotheses of \cref{I.1.2}, with $X$ connected, the functor $\sh{F}\mapsto\sh{F}_{x_0}$ is an equivalence between the category of complex local systems on $X$ and the category of complex finite-dimensional representations of $\pi_1(X,x_0)$.
\end{corollary}

\begin{env}{1.5}
\label{I.1.5}
  Under the hypotheses of \cref{I.1.2}, if $a\colon[0,1]\to X$ is a path, and $b$ a loop based at $a(0)$, then $aba^{-1}=a(b)$ is a path based at $a(1)$.
  Its homotopy class depends only on the homotopy classes of $a$ and $b$.
  This construction defines an isomorphism between $\pi_1(X,a(0))$ and $\pi_1(X,a(1))$.
\end{env}

\begin{proposition}{1.6}
\label{I.1.6}
  Under the hypotheses of \cref{I.1.5}, there exists, up to unique isomorphism, exactly one locally constant sheaf of groups $\Pi_1(X)$ on $X$ (\emph{the fundamental groupoid}), endowed, for all $x_0\in X$, with an isomorphism
  \[
  \label{I.1.6.1}
    \Pi_1(X)_{x_0} \simeq \pi_1(X,x_0)
  \tag{1.6.1}
  \]
  and such that, for every path $a\colon[0,1]\to X$, the isomorphism in \cref{I.1.5} between $\pi_1(X,a(0))$ and $\pi_1(X,a(1))$ can be identified, via \cref{I.1.6.1}, with the isomorphism in \cref{I.1.2} between $\Pi_1(X)_{a(0)}$ and $\Pi_1(X)_{a(1)}$.
\end{proposition}

If $X$ is connected, with base point $x_0$, then the sheaf $\Pi_1(X)$ corresponds, via the equivalence in \cref{I.1.3}, to the group $\pi_1(X,x_0)$ endowed with its action over itself by inner automorphisms.

\begin{proposition}{1.7}
\label{I.1.7}
  If $\sh{F}$ is a locally constant sheaf on $X$, then there exists exactly one action (said to be \emph{canonical}) of $\Pi_1(X)$ on $\sh{F}$ that, at each $x_0\in X$, induces the action from \cref{I.1.2} of $\pi_1(X,x_0)$ on $\sh{F}$.
\end{proposition}


\section{Integrable connections and local systems}
\label{I.2}

\oldpage{5}

\begin{env}{2.1}
  Let $X$ be an analytic space \hyperref[0.1]{(0.1)}.
  We define a (holomorphic) \emph{vector bundle} on $X$ to be a locally free sheaf of modules that is of finite type over the structure sheaf $\sh{O}$ of $X$.
  If $\sh{V}$ is a vector bundle on $X$, and $x$ a point of $X$, then we denote by $\sh{V}_{(x)}$ the free $\sh{O}_{(x)}$-module of finite type of germs of sections of $\sh{V}$.
  If $\mathfrak{m}_x$ is the maximal ideal of $\sh{O}_{(x)}$, then we define the \emph{fibre at $x$ of the vector bundle $\sh{V}$} to be the \todo of finite rank
  \[
  \label{I.2.1.1}
    \sh{V}_x = \sh{V}_{(x)} \otimes_{\sh{O}_{(x)}} \sh{O}_{(x)}/\mathfrak{m}_x.
  \tag{2.1.1}
  \]

  If $f\colon X\to Y$ is a morphism of analytic spaces, then the \emph{inverse image} of a vector bundle $\sh{V}$ on $Y$ is the vector bundle $f^*\sh{V}$ on $X$ given by the inverse image of $\sh{V}$ as a coherent module:
  if $f^\sbullet\sh{V}$ is the sheaf-theoretic inverse image of $\sh{V}$, then
  \[
  \label{I.2.1.2}
    f^*\sh{V} \simeq \sh{O}_X \otimes_{f^*\sh{O}_Y} f^\sbullet\sh{V}
  \tag{2.1.2}
  \]

  In particular, if $x\colon P\to X$ is the morphism from the point space $P$ to $X$ defined by a point $x$ of $X$, then
  \[
    \label{I.2.1.3}
      \sh{V}_x \simeq x^*\sh{V}.
    \tag{2.1.3}
  \]
\end{env}

\begin{env}{2.2}
\label{I.2.2}
  Let $X$ be a complex-analytic manifold \hyperref[0.7]{(0.7)} and $\sh{V}$ a vector bundle on $X$.
  The elders would have defined a (holomorphic) connection on $\sh{V}$ as the data, for every pair of points $(x,y)$ that are first order infinitesimal neighbours in $X$, of an isomorphism $\gamma_{y,x}\colon\sh{V}_x\to\sh{V}_y$ that depends holomorphically on $(x,y)$ and is such that $\gamma_{x,x}=\id$.

  Suitably interpreted, this ``definition'' coincides with the currently fashionable definition \cref{I.2.2.4} given below (which we not be use in the rest of the section).

  It suffices to understand ``point'' to mean ``point with values in any analytic space'':

\oldpage{6}
  \begin{env}{2.2.1}
  \label{I.2.2.1}
    \emph{A point in an analytic space $X$ with values in an analytic space $S$} is a morphism from $S$ to $X$.
  \end{env}

  \begin{env}{2.2.2}
  \label{I.2.2.2}
    If $Y$ is a subspace of $X$, then the \emph{$n$th infinitesimal neighbourhood} of $Y$ in $X$ is the subspace of $X$ defined locally by the $(n+1)$th power of the ideal of $\sh{O}_X$ that defines $Y$.
  \end{env}

  \begin{env}{2.2.3}
  \label{I.2.2.3}
    Two points $x,y\in X$ with values in $S$ are said to be \emph{first order infinitesimal neighbours} if the map $(x,y)\colon S\to X\times X$ that they define factors through the first order infinitesimal neighbourhood of the diagonal of $X\times X$.
  \end{env}

  \begin{env}{2.2.4}
  \label{I.2.2.4}
    If $X$ is a complex-analytic manifold and $\sh{V}$ is a vector bundle on $X$, then a (\emph{holomorphic}) \emph{connection} $\gamma$ on $\sh{V}$ consists of the following data:

    for every pair $(x,y)$ of points of $X$ with values in an arbitrary analytic space $S$, with $x$ and $y$ first order infinitesimal neighbours, an isomorphism $\gamma_{x,y}\colon x^*\sh{V}\to y^*\sh{V}$;
    this data is subject to the conditions:
    \begin{enumerate}[(i)]
      \item (functoriality) For any $f\colon T\to S$ and any first order infinitesimal neighbours $x,y\colon S\rightrightarrows X$, we have $f^*(\gamma_{y,x})=\gamma_{yf,xf}$.
      \item We have $\gamma_{x,x}=\id$.
    \end{enumerate}
  \end{env}
\end{env}

\begin{env}{2.3}
\label{I.2.3}
  Let $X_1$ be the first-order infinitesimal neighbourhood of the diagonal $X_0$ of $X\times X$, and let $p_1$ and $p_2$ be the two projections of $X_1$ to $X$.
  By definition, the vector bundle $P^1(\sh{V})$ of first-order jets of sections of $\sh{V}$ is the bundle $(p_1)_*p_2^*\sh{V}$.
  We denote by $j^1$ the first-order differential operator that sends each section of $\sh{V}$ to its first-order jet:
  \[
    j^1\colon \sh{V} \to P^1(\sh{V}) \simeq \sh{O}_{X_1}\otimes_{\sh{O}_X}\sh{V}.
  \]

  A connection (\cref{I.2.2.4}) can be understood as a homomorphism (which is automatically an isomorphism)
  \[
  \label{I.2.3.1}
    \gamma = p_1^*\sh{V} \to p_2^*\sh{V}
  \tag{2.3.1}
  \]
  which induces the identity over $X_0$.
  Since
  \[
    \Hom_{X_1}(p_1^*\sh{V},p_2^*\sh{V}) \simeq \Hom(\sh{V},(p_1)_*p_2^*\sh{V}),
  \]
\oldpage{7}
  a connection can also be understood as a ($\sh{O}$-linear) homomorphism
  \[
  \label{I.2.3.2}
    \DD\colon \sh{V} \to P^1(\sh{V})
  \tag{2.3.2}
  \]
  such that the obvious composite arrow
  \[
    \sh{V}\xrightarrow{\DD} P^1(\sh{V}) \to \sh{V}
  \]
  is the identity.
  The sections $\DD s$ and $j^1(s)$ of $P^1(v)$ thus have the same image in $\sh{V}$, and $j^1(s)-\DD(s)$ can be identified with a section $\nabla s$ of $\Omega_X^1\otimes\sh{V} \simeq \Ker(P^1(\sh{V})\to\sh{V})$:
  \[
  \label{I.2.3.3}
    \nabla\colon \sh{V} \to \Omega^1(X)
  \tag{2.3.3}
  \]
  \[
  \label{I.2.3.4}
    j^1(s) = \DD(s)+\nabla s.
  \tag{2.3.4}
  \]

  In other words, a connection (\cref{I.2.2.4}), allowing us to compare two neighbouring fibres of $\sh{V}$, also allows us to define the differential $\nabla s$ of a section of $\sh{V}$.

  Conversely, equation~\cref{I.2.3.4} allows us to define $\DD$, and thus $\gamma$, from the covariant derivative $\nabla$.
  For $\DD$ to be linear, it is necessary and sufficient for $\nabla$ to satisfy the identity
  \[
  \label{I.2.3.5}
    \nabla(fs) = \dd f\cdot s + f\cdot\nabla s
  \tag{2.3.5}
  \]

  Definition~\cref{I.2.2.4} is thus equivalent to the following definition, due to J.L.~Koszul.
\end{env}

\begin{definition}{2.4}
\label{I.2.4}
  Let $\sh{V}$ be a (holomorphic) vector bundle on a complex-analytic manifold $X$.
  A \emph{holomorphic connection} (or simply, \emph{connection}) on $\sh{V}$ is a $\mathbb{C}$-linear homomorphism
  \[
    \nabla\colon \sh{V} \to \Omega_X^1(\sh{V}) = \Omega_X^1\otimes_{\sh{O}}\sh{V}
  \]
  that satisfies the Leibniz identity (\cref{I.2.3.5}) for local sections $f$ of $\sh{O}$ and $s$ of $\sh{V}$.
  We call $\nabla$ the \emph{covariant derivative} defined by the connection.
\end{definition}

\begin{env}{2.5}
\label{I.2.5}
  If the vector bundle $\sh{V}$ is endowed with a connection $\Gamma$ with covariant derivative $\nabla$, and if $w$ is a holomorphic vector field on $X$, then we set, for every local section $v$ of $\sh{V}$ over an open subset $U$ of $X$,
  \[
    \nabla_w(v) = \langle \nabla v,w \rangle \in \sh{V}(U).
  \]
  We call $\nabla_w\colon \sh{V} \to \sh{V}$ the \emph{covariant derivative along the vector field $w$}.
\end{env}

\oldpage{8}
\begin{env}{2.6}
\label{I.2.6}
  If ${}_1\!\Gamma$ and ${}_2\!\Gamma$ are connections on $X$, with covariant derivatives ${}_1\!\nabla$ and ${}_2\!\nabla$ (respectively), then ${}_2\!\nabla-{}_1\!\nabla$ is a $\sh{O}$-linear homomorphism from $\sh{V}$ to $\Omega_X^1(\sh{V})$.
  Conversely, the sum of ${}_1\!\nabla$ and such a homomorphism defines a connection on $\sh{V}$.
  Thus connections on $\sh{V}$ form a principal homogeneous space (or torsor) on $\shHom(\sh{V},\Omega_X^1(\sh{V})) \simeq \Omega_X^1(\shEnd(\sh{V}))$.
\end{env}

\begin{env}{2.7}
\label{I.2.7}
  If vector bundles are endowed with connections, then every vector bundle obtained by a ``tensor operation'' is again endowed with a connection.
  This is evident with \cref{I.2.2.4}.
  More precisely, let $\sh{V}_1$ and $\sh{V}_2$ be vector bundles endowed with connections with covariant derivatives $\nabla_1$ and $\nabla_2$.

  \begin{env}{2.7.1}
  \label{I.2.7.1}
    We define a connection on $\sh{V}_1\oplus\sh{V}_2$ by the formula
    \[
      \nabla_w(v_1+v_2) = {}_1\!\nabla_w(v_1) + {}_2\!\nabla_w(v_2)
    \]
  \end{env}

  \begin{env}{2.7.2}
  \label{I.2.7.2}
    We define a connection on $\sh{V}_1\otimes\sh{V}_2$ by the Leibniz formula
    \[
      \nabla_w(v_1\otimes v_2) = \nabla_w v_1\cdot v_2 + v_1\cdot\nabla_w v_2.
    \]
  \end{env}

  \begin{env}{2.7.3}
  \label{I.2.7.3}
    We define a connection on $\shHom(\sh{V}_1,\sh{V}_2)$ by the formula
    \[
      (\nabla_w f)(v_1) = {}_2\!\nabla_2(f(v_1)) - f({}_1\!\nabla v_1).
    \]
  \end{env}

  The canonical connection on $\sh{O}$ is the connection for which $\nabla f=\dd f$.
  
  Let $\sh{V}$ be a vector bundle endowed with a connection.
  \begin{env}{2.7.4}
  \label{I.2.7.4}
    We define a connection on the dual $\sh{V}^\vee$ of $\sh{V}$ via \cref{I.2.7.3} and the defining isomorphism $\sh{V}^\vee = \shHom(\sh{V},\sh{O})$.
    We have
    \[
      \langle \nabla_w v',v \rangle = \partial_w\langle v',v \rangle - \langle v',\nabla_w v \rangle.
    \]
  \end{env}

  We leave it to the reader to verify that these formulas do indeed define connections.
  For \cref{I.2.7.2}, for example, one must verify that, firstly, the given formula defines a $\mathbb{C}$-bilinear map from $(\sh{V}_1\otimes\sh{V}_2)$, which means that the right-hand side $\II(v_1,v_2)$ is $\mathbb{C}$-bilinear and such that $\II(fv_1,v_2)=\II(v_1,fv_2)$;
  secondly, one must also verify identity~\cref{I.2.3.5}.
\end{env}

\begin{env}{2.8}
\label{I.2.8}
  An $\sh{O}$-homomorphism $f$ between vector bundles $\sh{V}_1$ and $\sh{V}_2$ endowed with connections
\oldpage{9}
  is said to be \emph{compatible with the connections} if
  \[
    {}_2\!\nabla\cdot f = f\cdot{}_1\!\nabla.
  \]
  By \cref{I.2.7.3}, this reduces to saying that $\nabla f=0$, if $f$ is thought of as a section of $\shHom(\sh{V}_1,\sh{V}_2)$.
  For example, by \cref{I.2.7.3}, the canonical map
  \[
    \Hom(\sh{V}_1,\sh{V}_2)\otimes\sh{V}_1 \to \sh{V}_2
  \]
  is compatible with the connections.
\end{env}

\begin{env}{2.9}
\label{I.2.9}
  A local section $v$ of $\sh{V}$ is said to be \emph{horizontal} if $\nabla v=0$.
  If $f$ is a homomorphism between bundles $\sh{V}_1$ and $\sh{V}_2$ endowed with connections, then it is equivalent to say either that $f$ is horizontal, or that $f$ is compatible with the connections \cref{I.2.8}.
\end{env}

\begin{env}{2.10}
\label{I.2.10}
  Let $\sh{V}$ be a holomorphic vector bundle on $X$.
  Define $\Omega_X^p=\wedge^p\Omega_X^1$ and $\Omega_X^p(\sh{V})=\Omega_X^p\otimes_\sh{O}\sh{V}$ (the sheaf of \emph{exterior differential $p$-forms with values in $\sh{V}$}).
  Suppose that $\sh{V}$ is endowed with a holomorphic connection.
  We then define $\CC$-linear morphisms
  \[
  \label{I.2.10.1}
    \nabla\colon \Omega_X^p(\sh{V}) \to \Omega_X^{p+1}(\sh{V})
  \tag{2.10.1}
  \]
  characterised by the following formula:
  \[
  \label{I.2.10.2}
    \nabla(\alpha,v) = \dd\alpha\cdot v + (-1)^p\alpha\wedge\nabla v,
  \tag{2.10.2}
  \]
  where $\alpha$ is any local section of $\Omega^p$, $v$ is any local section of $\sh{V}$, and $\dd$ is the exterior differential.
  To prove that the right-hand side $\II(\alpha,v)$ of \cref{I.2.10.2} defines a homomorphism \cref{I.2.10.1}, it suffices to show that $\II(\alpha,v)$ is $\CC$-bilinear and satisfies
  \[
    \II(f\alpha,v) = \II(\alpha,fv).
  \]
  But we have that
  \[
    \begin{aligned}
      \II(f\alpha,v)
      &= \dd(f\alpha)v + (-1)^pf\alpha\wedge\nabla v
    \\&= \dd\alpha\cdot fv + \dd f\wedge\alpha v + (-1)^pf\alpha\wedge\nabla v
    \\&= \dd\alpha\cdot fv + (-1)^p\alpha\wedge(f\nabla v+\dd f\cdot v)
    \\&= \II(\alpha,fv).
    \end{aligned}
  \]

  Let $\sh{V}_1$ and $\sh{V}_2$ be vector bundles endowed with connections, and let $\sh{V}$ be their tensor product \cref{I.2.7.2}.
  We denote by $\wedge$ the evident maps
  \[
    \wedge\colon \Omega^p(\sh{V}_1)\otimes\Omega^1(\sh{V}_2) \to \Omega^{p+q}(\sh{V})
  \]
\oldpage{10}
  such that, for any local section $\alpha$ (resp. $\beta$, resp. $v_1$, resp. $v_2$) of $\Omega^p$ (resp. $\Omega^q$, resp. $\sh{V}_1$, resp. $\sh{V}_2$), we have that $(\alpha\otimes v_1)\wedge(\beta\otimes v_2) = (\alpha\wedge\beta)\otimes(v_1\otimes v_2)$.
  If $\nu_1$ (resp. $\nu_2$) is any local section of $\Omega^p(\sh{V}_1)$ (resp. $\Omega^q(\sh{V}_2)$), then
  \[
  \label{I.2.10.3}
    \nabla(\nu_1\wedge\nu_2) = \nu_1\wedge\nu_2 + (-1)^p\nu_1\wedge\nu_2.
  \tag{2.10.3}
  \]
  Indeed, if $\nu_1=\alpha v_1$ and $\nu_2=\beta v_2$, then
  \[
    \begin{aligned}
      \nabla(\nu_1\wedge\nu_2)
      &= \nabla(\alpha\wedge\beta\otimes v_1\otimes v_2)
    \\&= \dd(\alpha\wedge\beta)v_1\otimes v_2 + (-1)^{p+q}\alpha\wedge\beta\wedge\nabla(v_1\otimes v_2)
    \\&= \dd\alpha\wedge\beta v_1\otimes v_2 + (-1)^p\alpha\wedge\dd\beta v_1\otimes v_2
    \\&\quad+ (-1)^{p+q}\alpha\wedge\beta\wedge\nabla v_1\otimes v_2 + (-1)^{p+q}\alpha\wedge\beta v_1\wedge\nabla v_2
    \\&= \dd\alpha v_1\wedge\nu_2 + (-1)^p\nu_1\wedge\dd\beta v_2 + (-1)^p\alpha\wedge\nabla v_1\wedge\nu_2
    \\&\quad+ (-1)^{p+q}\nu_2\wedge\beta\wedge\nabla v_2
    \\&= (\dd\alpha v_1 + (-1)^p\alpha\wedge\nabla v_1)\wedge\nu_2 + (-1)^p\nu_1\wedge(\dd\beta v_2 + (-1)^q\beta\wedge\nabla v_2)
    \\&= \nabla\nu_1\wedge\nu_2 +(-1)^p\nu_1\wedge\nabla\nu_2.
    \end{aligned}
  \]

  Let $\sh{V}$ be a vector bundle endowed with a connection.
  If we apply the above formula to $\sh{O}$ and $\sh{V}$, then, for any local section $\alpha$ (resp. $\nu$) of $\Omega^p$ (resp. $\Omega^q(\sh{V})$), we have that
  \[
  \label{I.2.10.4}
    \nabla(\alpha\wedge\nu) = \dd\alpha\wedge\nu + (-1)^p\alpha\wedge\nabla\nu.
  \tag{2.10.4}
  \]

  Iterating this formula gives
  \[
  \label{I.2.10.5}
    \begin{aligned}
      \nabla\nabla(\alpha\wedge\nu)
      &= \nabla(\dd\alpha\wedge\nu + (-1)^p\alpha\wedge\nabla\nu)
      \\&= \dd\alpha\wedge\nu + (-1)^{p+1}\dd\alpha\wedge\nabla\nu + (-1)^p\dd\alpha\wedge\nabla\nu + \alpha\wedge\nabla\nabla\nu
      \\&= \alpha\wedge\nabla\nabla\nu.
    \end{aligned}
  \tag{2.10.5}
  \]
\end{env}

\begin{definition}{2.11}
\label{I.2.11}
  Under the hypotheses of \cref{I.2.10}, the \emph{curvature} $R$ of the given connection on $\sh{V}$ is the composite homomorphism
  \[
    R\colon \sh{V} \to \Omega_X^2(\sh{V})
  \]
  considered as a section of $\Hom(\sh{V},\Omega_X^2(\sh{V})) \simeq \Omega_X^2(\End(\sh{V}))$.
\end{definition}

\begin{env}{2.12}
\label{I.2.12}
  Taking $q=0$ in \cref{I.2.10.4} gives
  \[
  \label{I.2.12.1}
    \nabla\nabla(\alpha v) = \alpha\wedge R(v),
  \tag{2.12.1}
  \]
  which we write as
\oldpage{11}
  \[
  \label{I.2.12.2}
    \nabla\nabla(\nu) = R\wedge\nu
    \qquad\mbox{(the \emph{Ricci identity}).}
  \tag{2.12.2}
  \]

  We endow $\shEnd(\sh{V})$ with the connection given in \cref{I.2.7.3}.
  The equation $\nabla(\nabla\nabla)=(\nabla\nabla)\nabla$ can be written as $\nabla(R\wedge\nu) = R\wedge\nabla\nu$.
  By \cref{I.2.7.3}, we have that $\nabla R\wedge\nu = \nabla(R\wedge\nu) - R\wedge\nabla\nu$, so that
  \[
  \label{I.2.12.3}
    \nabla R=0
    \qquad\mbox{(the \emph{Bianchi identity}).}
  \tag{2.12.3}
  \]
\end{env}

\begin{env}{2.13}
\label{I.2.13}
  If $\alpha$ is an exterior differential $p$-form, then we know that
  \[
    \begin{aligned}
      \langle \dd\alpha, X_0\wedge\ldots\wedge X_p \rangle
      &= \sum_i(-1)^i j_{X_i}\langle \alpha, X_0\wedge\ldots\wedge\widehat{X_i}\wedge\ldots\wedge X_p \rangle
    \\&\quad+ \sum_{i<j})(-1)^{i+j} \langle \alpha, [X_i,X_j]\wedge X_0\wedge\ldots\wedge\widehat{X_i}\wedge\ldots\wedge\widehat{X_j}\wedge\ldots\wedge X_p \rangle.
    \end{aligned}
  \]
  From this formula, and from \cref{I.2.10.2}, we see that, for any local section $\nu$ of $\Omega_X^p(\sh{V})$, and holomorphic vector fields $X_0,\ldots,X_p$,
  \[
  \label{I.2.13.1}
    \begin{aligned}
      \langle \nabla\nu, X_0\wedge\ldots\wedge X_p \rangle
      &= \sum_i(-1)^i \nabla_{X_i}\langle \nu, X_0\wedge\ldots\wedge\widehat{X_i}\wedge\ldots\wedge X_p \rangle
    \\&\quad+ \sum_{i<j})(-1)^{i+j} \langle \nu, [X_i,X_j]\wedge X_0\wedge\ldots\wedge\widehat{X_i}\wedge\ldots\wedge\widehat{X_j}\wedge\ldots\wedge X_p \rangle.
    \end{aligned}
  \tag{2.13.1}
  \]

  In particular, for any local section $v$ of $\sh{V}$, we have that
  \[
    \langle \nabla\nabla v, X_1\wedge X_2 \rangle
    = \nabla_{X_1}\langle \nabla v, X_2 \rangle - \nabla_{X_2}\langle v, X_1 \rangle - \langle \nabla v, [X_1,X_2] \rangle.
  \]
  That is,
  \[
  \label{I.2.13.2}
    R(X_1,X_2)(v) = \nabla_{X_1}\nabla_{X_2}v - \nabla_{X_2}\nabla_{X_1}v - \nabla_{[X_1,X_2]}v.
  \tag{2.13.2}
  \]
\end{env}

\begin{definition}{2.14}
\label{I.2.14}
  A connection is said to be \emph{integrable} if its curvature is zero, i.e. \cref{I.2.13.2} if the following holds identically:
  \[
    \nabla_{[X,Y]} = [\nabla_X,\nabla_Y].
  \]
\end{definition}
If $\dim(X)\leq1$, then every connection is integrable.

If $\Gamma$ is an integrable connection on $\sh{V}$, then the morphism $\nabla$ of \cref{I.2.10.1} satisfy $\nabla\nabla=0$, and so the $\Omega^p(\sh{V})$ give a differential complex $\Omega^\bullet(\sh{V})$.

\begin{definition}{2.15}
\label{I.2.15}
  Under the above hypotheses, the complex $\Omega^\bullet(\sh{V})$ is called the \emph{holomorphic de Rham complex} with values in $\sh{V}$.
\end{definition}

\oldpage{12}
The results \cref{I.2.16} to \cref{I.2.19} that follow will be proven in a more general setting in \cref{I.2.23}.

\begin{proposition}{2.16}
\label{I.2.16}
  Let $V$ be a local complex system on a complex-analytic variety $X$ \hyperref[0.6]{(0.6)}, and let $\sh{V}=\sh{O}\otimes_\CC V$.
  \begin{enumerate}[(i)]
    \item There exists, on the vector bundle $\sh{V}$, exactly one connection (said to be \emph{canonical}) whose horizontal sections are the local sections of the subsheaf $V$ of $\sh{V}$.
    \item The canonical connection on $\sh{V}$ is integrable.
    \item For any local section $f$ (resp. $v$) of $\sh{O}$ (resp. $V$),
      \[
      \label{I.2.16.1}
        \nabla(fv) = \dd f\cdot v.
      \tag{2.16.1}
      \]
  \end{enumerate}
\end{proposition}

\begin{proof}
  If $\nabla$ satisfies (i), then \cref{I.2.16.1} is a particular case of \cref{I.2.3.5}.
  Conversely, the right-hand side $\II(f,v)$ of \cref{I.2.16.1} is $\CC$-bilinear, and thus extends uniquely to a $\CC$-linear map $\nabla\colon\sh{V}\to\Omega^1(\sh{V})$, which we can show defines a connection.
  Claim~(ii) is local on $X$, which allows us to reduce to the case where $V=\underline{\CC}$.
  Then $\sh{V}=\sh{O}$, $\nabla=\dd$, and $\nabla_{[X,Y]}=[\nabla_X,\nabla_Y]$ by the definition of $[X,Y]$.
\end{proof}

It is well known that:
\begin{theorem}{2.17}
\label{I.2.17}
  Let $X$ be a complex-analytic variety.
  Then the following functors are quasi-inverse to one another, and thus give an equivalence between the category of complex local systems on $X$ and the category of holomorphic vector bundles with on $X$ with integrable connections (with the morphisms being the horizontal morphisms of vector bundles):
  \begin{enumerate}[a)]
    \item the complex local system $V$ is sent to $\sh{V}=\sh{O}\otimes V$ endowed with its canonical connection;
    \item the holomorphic vector bundle $\sh{V}$ endowed with its integrable connection is sent to the subsheaf $V$ of $\sh{V}$ consisting of horizontal sections (i.e. those $v$ such that $\nabla v=0$).
  \end{enumerate}
\end{theorem}

These equivalences are compatible with taking the tensor product, the internal $\Hom$, and the dual;
to the unit complex local system $\underline{\CC}$ corresponds the bundle $\sh{O}$ endowed with the connection $\nabla$ such that $\nabla f=\dd f$.

Definition~\cref{I.2.10.2} implies the following:
\begin{proposition}{2.18}
\label{I.2.18}
  If $V$ is a complex local system on $X$, and if $\sh{V}=\sh{O}\otimes_\CC V$,
\oldpage{13}
  then the system of isomorphisms
  \[
    \Omega_X^p\otimes_\CC V
    \simeq
    \Omega_X^p\otimes_\sh{O}\sh{O}\otimes_\CC V
    \simeq
    \Omega_X^p\otimes_\sh{O}\sh{V}
  \]
  is an isomorphism of complexes
  \[
    \Omega_X^\bullet\otimes_\CC V \to \Omega_X^\bullet(\sh{V}).
  \]
\end{proposition}
From this, the holomorphic Poincar\'{e} lemma gives the following:
\begin{proposition}{2.19}
\label{I.2.19}
  Under the hypotheses of \cref{I.2.16}, the complex $\Omega_X^\bullet(\sh{V})$ is a resolution of the sheaf $\sh{V}$.
\end{proposition}

\begin{env}{2.20}
\label{I.2.20}
  \textbf{Variants.}
  
  \begin{env}{2.20.1}
  \label{I.2.20.1}
    If $X$ is a differentiable manifold, and we consider $C^\infty$ connections on $C^\infty$ vector bundles, then all of the above results still hold true, mutatis mutandis.
    We will not use this fact.
  \end{env}

  \begin{env}{2.20.2}
  \label{I.2.20.2}
    Theorem~\cref{I.2.17} makes essential use of the non-singularity of $X$;
    it is thus unimportant to note that this hypothesis has not been used in an essential way before \cref{I.2.17}
  \end{env}

  \begin{env}{2.20.3}
  \label{I.2.20.3}
    The definition \cref{I.2.4} of a connection and the definition \cref{I.2.11} of an integrable connection are formal enough that we can transport them to the category of schemes, or in relative settings:
  \end{env}
\end{env}

\begin{definition}{2.21}
\label{I.2.21}
  \begin{enumerate}[(i)]
    \item Let $f\colon X\to S$ be a smooth morphism of schemes, and $\sh{V}$ a quasi-coherent sheaf on $X$.
      A \emph{relative connection} on $\sh{V}$ is an $f^*\sh{O}_S$-linear sheaf morphism
      \[
        \nabla\colon \sh{V} \to \Omega_{X/S}^1(\sh{V})
      \]
      (called the \emph{covariant derivative} defined by the connection) that identically satisfies, for any local section $f$ (resp. $v$) of $\sh{O}_X$ (resp. $\sh{V}$),
      \[
        \nabla(fv) = f\cdot\nabla v + \dd f\cdot v.
      \]
    \item Given $\sh{V}$ endowed with a relative connection, there exists exactly one system of $f^*\sh{O}_S$-homomorphisms of sheaves
      \[
        \nabla^{(p)}\colon \Omega_{X/S}^p(\sh{V}) \to \Omega_{X/S}^{p+1}(\sh{V})
      \]
      that satisfies \cref{I.2.10.4} and is such that $\nabla^{(0)}=\nabla$.
\oldpage{14}
    \item The \emph{curvature} of a connection is defined by
      \[
        R = \nabla^{(1)}\nabla^{(0)} \in \shHom(\sh{V},\Omega_{X/S}^2(\sh{V})) \cong \Omega_{X/S}^2(\shEnd(\sh{V})).
      \]
      The curvature satisfies the Ricci identity \cref{I.2.12.2} and the Bianchi identity \cref{I.2.12.3}.
    \item An \emph{integrable connection} is a connection with zero curvature.
    \item The \emph{de Rham complex} defined by an integrable connection is the complex $(\Omega_{X/S}^p(\sh{V}),\nabla)$.
  \end{enumerate}
\end{definition}

\begin{env}{2.22}
\label{I.2.22}
  Let $f\colon X\to S$ be a \emph{smooth} morphism of complex-analytic spaces;
  by hypothesis, $f$ is thus locally (in the domain) isomorphic to a projection $\pr_2\colon\CC^n\times S\to S$ (for some $n\in\NN$).
  A \emph{local relative system} on $X$ is a sheaf of $f^*\sh{O}_S$-modules that is locally isomorphic to the sheaf-theoretic inverse image of a coherent analytic sheaf on $S$.
  If $\sh{V}$ is a coherent analytic sheaf on $X$, then a \emph{relative connection} on $\sh{V}$ is an $f^*\sh{O}_S$-linear homomorphism
  \[
    \nabla\colon \sh{V} \to \Omega_{X/S}^1(\sh{V})
  \]
  that identically satisfies, for any local section $f$ (resp. $v$) of $\sh{O}$ (resp. $\sh{V}$),
  \[
    \nabla(fv) = f\cdot\nabla v + \dd f\cdot v.
  \]
  A \emph{morphism} between vector bundles endowed with relative connections is a morphism of vector bundles that commutes with $\nabla$.
  We define, as in \cref{I.2.11,I.2.21}, the \emph{curvature} $R\in\Omega_{X/S}^2(\shEnd(\sh{V}))$ of a relative connection.
  A relative connection is said to be \emph{integrable} if $R=0$, in which case we have the \emph{relative de Rham complex with values in $\sh{V}$}, denoted by $\Omega_{X/S}^\bullet(\sh{V})$, and defined as in \cref{I.2.15,I.2.21}.
\end{env}

The ``absolute'' statements \cref{I.2.17,I.2.18,I.2.19} have ``relative'' (i.e. ``with parameters'') analogues:

\begin{theorem}{2.23}
\label{I.2.23}
  Under the hypotheses of \cref{I.2.22}, we have the following.
  \begin{enumerate}[(i)]
    \item For every relative local system $V$ on $X$, there exists a coherent analytic sheaf \mbox{$\sh{V}=\sh{O}_X\otimes_{f^*\sh{O}_S}V$}, and exactly one relative connection, said to be canonical, such that a local section $v$ of $\sh{V}$ is horizontal (i.e. such that $\nabla v=0$) if and only if $v$ is a section of $V$;
      this connection is integrable.
    \item Given a relative local system $V$ on $X$, the de Rham complex defined
\oldpage{15}
      by $\sh{V}=\sh{O}_X\otimes_{f^*\sh{O}_S}V$, endowed with its canonical connection, is a resolution of the sheaf $V$.
    \item The following functors are quasi-inverse to one another, and thus give an equivalence between the category of relative local systems on $X$ and the category of coherent analytic sheaves on $X$ endowed with a relative integrable connection:
      \begin{enumerate}[a)]
        \item the relative local system $V$ is sent to $\sh{V}=\sh{O}_X\otimes_{f^*\sh{O}_S}V$ endowed with its canonical connection;
        \item the coherent analytic sheaf $\sh{V}$ on $X$ endowed with a relative integrable connection is sent to the subsheaf consisting of its horizontal sections (i.e. the sections $v$ such that $\nabla v=0$).
      \end{enumerate}
  \end{enumerate}
\end{theorem}

\begin{proof}
  We first prove (i).
  To show that $\sh{V}$ is coherent, it suffices to do so locally, for $V=f^\sbullet V_0$, in which case $\sh{V}$ is the inverse image, in the sense of coherent analytic sheaves, of $\sh{V}_0$.
  The canonical relative connection necessarily satisfies, for any local section $f$ (resp $v_0$) of $\sh{O}_X$ (resp. $V$),
  \[
  \label{I.2.23.1}
    \nabla(fv_0) = \dd f\cdot v_0.
  \tag{2.23.1}
  \]
  The right-hand side $\II(f,v_0)$ of this equation is biadditive in $f$ and $v_0$, and satisfies, for any local section $g$ of $f^*\sh{O}_S$, the identity
  \[
    \II(fg,v_0) = \II(f,gv_0)
  \]
  (using the fact that $\dd g=0$ in $\Omega_{X/S}^1$).
  We thus deduce the existence and uniqueness of a relative connection $\nabla$ that satisfies \cref{I.2.23.1}.
  Finally, we have that
  \[
    \nabla\nabla(fv_0) = \nabla(\dd f\cdot v_0) = \dd\dd f\cdot v_0 = 0,
  \]
  and so the canonical connection $\nabla$ is integrable.
  The fact that only the sections of $V$ are horizontal is a particular case of (ii), which is proven below.
\end{proof}

\begin{env}{2.23.2}
\label{I.2.23.2}
  We first of all consider the particular case of (ii) where $S=D^n$, $X=D^n\times D^m$, $f=\pr_2$, and the relative local system $V$ is the inverse image of $\sh{O}_S$.
  The complex of global sections
  \[
    0 \to \Gamma(f^\sbullet\sh{O}_S) \to \Gamma(\sh{O}_X) \to \Gamma(\Omega_{X/S}^1) \to \ldots
  \]
  is acyclic, since it admits the homotopy operator $H$ defined below.
\oldpage{16}
  \begin{enumerate}[a)]
    \item $H\colon \Gamma(\sh{O}_X) \to \Gamma(f^\sbullet\sh{O}_S) = \Gamma(S,\sh{O}_S)$ is the inverse image under the zero section of $f$;
    \item an element $\omega\in\Gamma(\Omega_{X/S}^p)$ (where $p>0$) can be represented in a unique way as a sum of convergent series:
      \[
        \omega = \sum_{\substack{I\subset[1,m]\\|I|=p}} \; \sum_{\underline{n}\in \NN^{m+n}} a_{\underline{n}}^I
        \left(
          \prod_{i\in I} x_i^{n_i}\dd x_i
        \right)
        \left(
          \prod_{i\in[1,m+n]\setminus I} x_i^{n_i}
        \right)
      \]
      and we set
      \[
        H(\omega) = \sum_{I\subset[1,m]} \; \sum_{j\in I} \; \sum_{\underline{n}\in\NN^{m+n}} a_{\underline{n}}^I
        \left(
          \prod_{\substack{j\in I\\i\neq j}} x_j^{n_j}\dd x_j \frac{x_j^{n_j+1}}{n_j+1}
        \right)
        \left(
          \prod_{i\in[1,m+n]\setminus I} x_i^{n_i}.
        \right)
      \]
  \end{enumerate}

  This remains true if we replace $D^{m+n}$ by a smaller polycylinder, and so $\Omega_{X/S}^\bullet$ is a resolution of $f^\sbullet\sh{O}_S$.
\end{env}

\begin{proof}[\normalfont\textbf{2.23.3}]
\label{I.2.23.3}
  We now prove (ii), which is of a local nature on $X$ and $S$.
  Denoting by $D$ the open unit disc, we can thus restrict to the case where $S$ is a closed analytic subset of the polycylinder $D^n$, where $X=D^m\times S$, with $f=\pr_2$, and where $V$ is the inverse image of a coherent analytic sheaf $V_0$ on $S$.
  Applying the syzygy theorem, and possibly shrinking $X$ and $S$, we can further suppose that the direct image of $V_0$ on $D^n$, which we again denote by $V_0$, admits a finite resolution $\sh{L}$ by free coherent $\sh{O}_{D^n}$-modules.
  To prove (ii), we are allowed to replace $V_0$ by its direct image on $D^n$, and to suppose that $D^n=S$, which we now do.

  If $\Sigma_0$ is a short exact sequence of coherent $\sh{O}_S$-modules
  \[
    \Sigma_0\colon 0 \to V'_0 \to V_0 \to V''_0 \to 0,
  \]
  then let $\Sigma=f^\sbullet\Sigma_0$ be the exact sequence of relative local systems given by the inverse image of $\Sigma_0$ (which is exact since $f^\sbullet$ is an exact functor), and let $\Omega_{X/S}^\bullet(\Sigma)$ be the corresponding exact sequence of relative de Rham complexes:
  \[
    \Omega_{X/S}^\bullet(\Sigma)\colon 0 \to \Omega_{X/S}^p\otimes_{f^\sbullet\sh{O}_S}f^\sbullet V'_0 \to \Omega_{X/S}^p\otimes_{f^\sbullet\sh{O}_S}f^\sbullet V_0 \to \Omega_{X/S}^p\otimes_{f^\sbullet\sh{O}_S}f^\sbullet V''_0 \to 0.
  \]
  This sequence is exact since $\Omega_{X/S}^p$ is flat over $f^\sbullet\sh{O}_S$, since it is locally free over $\sh{O}_X$ which is itself flat over $f^\sbullet\sh{O}_S$.

  The snake lemma applied to $\Omega_{X/S}^\bullet(\Sigma)$ shows that, if claim~(ii)
\oldpage{17}
  is satisfied for any two of relative local systems $f^\sbullet V_0$, $f^\sbullet V'_0$, and $f^\sbullet V''_0$, then it is again satisfied for the third.
  We thus deduce, by induction, that, if $V_0$ admits a finite resolution $M_\bullet$ by modules that satisfy (ii), then $V_0$ satisfies (ii).
  This, applied to $V_0$ and $\sh{L}^\bullet$, finishes the proof of (i) and (ii).

  It follows from (ii) that the composite $\mbox{(iii)b}\circ\mbox{(iii)a}$ of the functors from (iii) is canonically isomorphic to the identity;
  furthermore, if $V_1$ and $V_2$ are relative local systems, and if $u\colon\sh{V}_1\to\sh{V}_2$ is a homomorphism that induces $0$ on $V_1$, then $u=0$, since $V_1$ generates $\sh{V}_1$;
  it thus follows that the functor $\mbox{(iii)a}$ is fully faithful.
  It remains to show that every vector bundle $\sh{V}$ endowed with a relative connection $\nabla$ is given locally by a relative local system.

  \begin{enumerate}[\bf {Case}~1:]
    \item \emph{$S=D^n$, $X=D^{n+1}=D^n\times D$, $f=\pr_1$, and $\sh{V}$ is free.}

      Under these hypotheses, if $v$ is an arbitrary section of the inverse image of $\sh{V}$ under the zero section $s_0$ of $f$, then there exists exactly one horizontal section $\widetilde{v}$ of $\sh{V}$ that agrees with $v$ on $s_0(S)$ (as follows from the existence and uniqueness of solutions for Cauchy problems with parameters).
      If $(e_i)_{1\leq i\leq k}$ is a basis of $s_0^*\sh{V}$, then the $\widetilde{e_i}$ form a horizontal basis of $\sh{V}$, and $(\sh{V},{\nabla})$ is defined by the relative local system $f^\sbullet s_0^*\sh{V}\simeq f^\sbullet\sh{O}_S^k$.
    \item \emph{$S=D^n$, $X=D^{n+1}=D^n\times D$, and $f=\pr_1$.}

      By possibly shrinking $X$ and $S$, we can suppose that $\sh{V}$ admits a free presentation:
      \[
        \sh{V}_1 \xrightarrow{d} \sh{V}_0 \xrightarrow{\varepsilon} \sh{V} \to 0.
      \]
      By then possibly shrinking again, we can further suppose that $\sh{V}_0$ and $\sh{V}_1$ admit connections $\nabla_1$ and $\nabla_0$ (respectively) such that $\varepsilon$ and $d$ are compatible with the connections (if $(e_i)$ is a basis of $\sh{V}_0$, then $\nabla_0$ is determined by the $\nabla_0 e_i$, and it suffices to choose $\nabla_0 e_i$ such that $\varepsilon(\nabla_0 e_i)=\nabla(\varepsilon(e_i))$; we proceed similarly for $\nabla_1$).
      The connections $\nabla_0$ and $\nabla_1$ are automatically integrable, since $f$ is of relative dimension~$1$.
      There thus exist (by Case~1) relative local systems $V_0$ and $V_1$ such that $(\sh{V}_i,\nabla_i)\simeq\sh{O}_X\otimes_{f^\sbullet\sh{O}_S}V_i$.
      We then have that
      \[
        (\sh{V},\nabla) \simeq \sh{O}_X\otimes_{f^\sbullet\sh{O}_S}(V_0/dV_1).
      \]
\oldpage{18}
    \item \emph{$f$ is of relative dimension~$1$.}

      We can suppose that $S$ is a closed analytic subset of $D^n$, and that $X=S\times D$ and $f=\pr_1$.
      The relative local systems (resp. the modules with relative connections) on $X$ can then be identified with the local relative systems (resp. the modules with relative connections) on $D^n\times D$ that are annihilated by the inverse image of the ideal that defines $S$, and we conclude by using Case~2.
    \item[\bf General case.]
      We proceed by induction on the relative dimension $n$ of $f$.
      The case $n=0$ is trivial.
      If $n\neq0$, then we are led to the case where $X=S\times D^{n-1}\times D$ and $f=\pr_1$.
      The bundle with connection $(\sh{V},\nabla)$ induces a bundle $\sh{V}_0$ with connection on $X_0=S\times D^{n-1}\times\{0\}$ which is, by induction, of the form $(\sh{V}_0,\nabla_0) = \sh{O}_{X_0}\otimes_{\pr_1^\sbullet\sh{O}_S}V$.
      The projection $f$ from $X$ to $S\times D^{n-1}$ is of relative dimension~$1$, and the relative connection $\nabla$ induces a relative connection for $\sh{V}$ on $X/S\times D^{n-1}$.
      By Case~3, there exists a vector bundle $V_1$ on $S\times D^{n-1}$, as well as an isomorphism
      \[
        \sh{V} \simeq \sh{O}_X\otimes_{p^\sbullet\sh{O}_{S\times D^{n-1}}}p^\sbullet V_1.
      \]
      of bundles with relative connections (with respect to $p$).

      The vector bundle $V_1$ can be identified with the restriction of $\sh{V}$ to $X_0$, whence we obtain an isomorphism
      \[
        \alpha\colon \sh{V}\simeq\sh{O}_X\otimes_{f^\sbullet\sh{O}_S}V
      \]
      of vector bundles, such that
      \begin{enumerate}[(i)]
        \item the restriction of $\alpha$ to $X_0$ is horizontal ; and
        \item $\alpha$ is ``relatively horizontal'' with respect to $p$.
      \end{enumerate}

      If $v$ is a section of $V$, then condition~(ii) implies that
      \[
        \nabla_{x_n}v = 0.
      \]

      If $1\leq i<n$, since $R=0$, then, by an analogous statement to \cref{I.2.13.2}, we have that
      \[
        \nabla_{x_n}\nabla_{x_i}v = \nabla_{x_i}\nabla_{x_n}v = 0.
      \]

      In other words, $\nabla_{x_i}v$ is a relative horizontal section, with respect to $p$, of $\sh{V}$;
      by (i), it is zero on $X_0$, and is thus zero, and we conclude that $\nabla v=0$.
      The isomorphism $\alpha$ is thus horizontal, and this finishes the proof
\oldpage{19}
      of \cref{I.2.23}.
  \end{enumerate}
\end{proof}

Some results in general topology (\cref{I.2.24} to \cref{I.2.27}) will be necessary to deduce \cref{I.2.28} from \cref{I.2.23}.

\begin{reminder}{2.24}
\label{I.2.24}
  Let $Y$ be a closed subset of a topological space $X$, and suppose that $Y$ has a paracompact neighbourhood.
  For every sheaf $\sh{F}$ on $X$, we have that
  \[
    \varinjlim_{U\supset Y} \HH^\bullet(U,\sh{F}) \xrightarrow{\sim} \HH^\bullet(Y,\sh{F}).
  \]
\end{reminder}

\begin{proof}
  See Godement~\cite[II, 4.11.1, p.~193]{7}.
\end{proof}

\begin{corollary}{2.25}
\label{I.2.25}
  Let $f\colon X\to S$ be a proper separated morphism between topological spaces.
  Suppose that $S$ is locally paracompact \hyperref[0.5]{(0.5)}.
  Then, for every $s\in S$, and for every sheaf $\sh{F}$ on $X$, we have that
  \[
    (\RR^i f_*\sh{F})_s \simeq \HH^i(f^{-1}(s), \sh{F}|f^{-1}(s)).
  \]
\end{corollary}

\begin{proof}
  Since $f$ is closed, the $f^{-1}(U)$ form a fundamental system of neighbourhoods of $f^{-1}(s)$, where the $U$ are neighbourhoods of $s$.
  Furthermore, if $U$ is paracompact, then $f^{-1}(U)$ is paracompact, since $f$ is proper and separated.
  We conclude by \cref{I.2.24}.
\end{proof}

\begin{reminder}{2.26}
\label{I.2.26}
  Let $X$ be a contractible locally paracompact topological space, $i$ an integer, and $V$ a complex local system on $X$, such that $\dim_\CC\HH^i(X,V)<\infty$.
  Then, for every vector space $A$ over $\CC$, possibly of infinite dimension, we have that
  \[
  \label{I.2.26.1}
    A\otimes_\CC\HH^i(X,V) \xrightarrow{\sim} \HH^i(X,A\otimes_\CC V).
  \tag{2.26.1}
  \]
\end{reminder}

\begin{proof}
  We denote by $\HH_\bullet(X,V^*)$ the singular homology of $X$ with coefficients in $V^*$.
  The universal coefficient formula, which holds here, gives
  \[
  \label{I.2.26.2}
    \HH^i(X,A\otimes V) \simeq \Hom_\CC(\HH_i(X,V^*),A).
  \tag{2.26.2}
  \]
  For $A=\CC$, we thus conclude that $\dim\HH_i(X,V^*)<\infty$.
  Equation~\cref{I.2.26.1} then follows from \cref{I.2.26.2}.
\end{proof}

\begin{env}{2.27}
\label{I.2.27}
  Let $f\colon X\to S$ be a smooth morphism of complex-analytic spaces, and let $V$ be a local system on $X$.
  Then the sheaf
\oldpage{20}
  \[
  \label{I.2.27.1}
    V_\mathrm{rel} = f^\sbullet\sh{O}_S \otimes_\CC V
  \tag{2.27.1}
  \]
  is a relative local system.
  We denote by $\Omega_{X/S}^\bullet(V)$ the corresponding de Rham complex.
  By \cref{I.2.23}, $\Omega_{X/S}^\bullet$ is a resolution of $V_\mathrm{rel}$.
  We thus have that
  \[
  \label{I.2.27.2}
    \RR^i f_* V_\mathrm{rel} \xrightarrow{\sim} \RR^i f_*(\Omega_{X/S}^\bullet(V))
  \tag{2.27.2}
  \]
  where the right-hand side is the relative hypercohomology.
  From \cref{I.2.27.1}, we thus obtain an arrow
  \[
  \label{I.2.27.3}
    \sh{O}_S \otimes_\CC \RR^i f_*V \to \RR^i f_*(V_\mathrm{rel}),
  \tag{2.27.3}
  \]
  whence, by composition, an arrow
  \[
  \label{I.2.27.4}
    \sh{O}_S \otimes \RR^i f_*V \to \RR^i f_*(\Omega_{X/S}(V)).
  \tag{2.27.4}
  \]
\end{env}

\begin{proposition}{2.28}
\label{I.2.28}
  Let $f\colon X\to S$ be a smooth separated morphism of analytic spaces, $i$ an integer, and $V$ a complex local system on $X$.
  We suppose that
  \begin{enumerate}[a)]
    \item $f$ is topologically trivial locally on $S$ ; and
    \item the fibres of $f$ satisfy
      \[
        \dim\HH^i(f^{-1}(s),V) < \infty.
      \]
      Then the arrow \cref{I.2.27.4} is an isomorphism:
      \[
        \sh{O}_S \otimes_\CC \RR^i f_*V \xrightarrow{\sim} \RR^i f_*(\Omega_{X/S}^\bullet(V)).
      \]
  \end{enumerate}
\end{proposition}

\begin{proof}
  Let $s\in S$, $Y=f^{-1}(s)$, and $V_0=V|Y$.
  To show that \cref{I.2.27.4} is an isomorphism, it suffices to construct a fundamental system $T$ of neighbourhoods of $s$ such that the arrows
  \[
  \label{I.2.28.1}
    \HH^0(T,\sh{O}_S) \otimes \HH^i(T\times Y,\pr_2^\sbullet V_0) \xrightarrow{\sim} \HH^i(T\times Y,\pr_1^\sbullet\sh{O}_S \otimes \pr_2^\sbullet V_0)
  \tag{2.28.1}
  \]
  are isomorphisms.
  In fact, the fibre at $s$ of \cref{I.2.27.3}, which is the inductive limit of the arrows \cref{I.2.28.1}, will then be an isomorphism.

  We will prove \cref{I.2.28.1} for a compact Stein neighbourhood $T$ of $s$, assumed to be contractible.
  The arrow in \cref{I.2.28.1} can then be written as
\oldpage{21}
  \[
  \label{I.2.28.2}
    \HH^0(T,\sh{O}_S) \otimes \HH^i(Y,V_0) \xrightarrow{\sim} \HH^i(T\times Y,\pr_1^\sbullet\sh{O}_S \otimes \pr_2^\sbullet V_0).
  \tag{2.28.2}
  \]

  We can calculate the right-hand side of \cref{I.2.28.2} by using the Leray spectral sequence for $\pr_2\colon T\times Y\to Y$.
  By \cref{I.2.25}, since $\HH^i(T,\sh{O}_S)=0$, we have that
  \[
    \HH^i(T\times Y,\pr_1^\sbullet\sh{O}_S \otimes \pr_2^\sbullet V_0) \simeq \HH^i(Y,\HH^0(T,\sh{O}_S) \otimes V_0),
  \]
  and we conclude by \cref{I.2.26}.
\end{proof}

\begin{env}{2.29}
\label{I.2.29}
  Under the hypotheses of \cref{I.2.28}, with $S$ smooth, we define the \emph{Gauss--Manin connection} on $\RR^i f_*\Omega_{X/S}^\bullet(V)$ as being the unique integrable connection that admits the local sections of $\RR^i f_*V$ as its horizontal sections \cref{I.2.17}.
\end{env}



\section{Translation in terms of first-order partial differential equations}
\label{I.3}

\begin{env}{3.1}
\label{I.3.1}
  Let $X$ be a complex-analytic variety.
  If $\sh{V}$ is the holomorphic vector bundle defined by a $\CC$-vector space $V_0$, then we have seen that $\sh{V}$ admits a canonical connection with covariant derivative ${}_0\!\nabla$.
  If $\nabla$ is the covariant derivative defined by another connection on $\sh{V}$, then we have seen \cref{I.2.6} that $\nabla$ can be written in the form
  \[
    \nabla = {}_0\!\nabla + \Gamma,
    \quad\mbox{where $\Gamma\in\Omega(\shEnd(\sh{V}))$.}
  \]

  If we identify sections of $\sh{V}$ with holomorphic maps from $X$ to $V_0$, then we have that
  \[
  \label{I.3.1.1}
    \nabla v = \dd v + \Gamma\cdot v
  \tag{3.1.1}
  \]

  If we suppose that we have chosen a basis of $V$, i.e. an isomorphism $e\colon\CC^n\to V_0$ with coordinates (identified with basis vectors) $e_\alpha\colon\CC\to V_0$, then $\Gamma$ can be written as a matrix $\omega_\beta^\alpha$ of differential forms (the \emph{matrix of forms of the connection}), and \cref{I.3.1.1} can then be written as
  \[
  \label{I.3.1.2}
    (\nabla v)^\alpha = \dd v^\alpha + \sum_\beta \omega_\beta^\alpha v^\beta.
  \tag{3.1.2}
  \]

  Let $\sh{V}$ be an arbitrary holomorphic vector bundle on $X$.
  The choice of a basis $e\colon\CC^n\xrightarrow{\sim}\sh{V}$ of $\sh{V}$ allows us to think of $\sh{V}$ as being defined by a
\oldpage{22}
  constant vector \todo ($\CC^n$), and the above remarks apply:
  connections on $\sh{V}$ correspond, via \cref{I.3.1.2}, with $(n\times n)$-matrices of differential forms on $X$.
  If $\omega_e$ is the matrix of the connection $\nabla$ in the basis $e$, and if $f\colon\CC^n\to\xrightarrow{\sim}\sh{V}$ is another basis of $\sh{V}$, with matrix $A\in\GL_n(\sh{O})$ (where $A=e^{-1}f$), then \cref{I.3.1.2}
  \[
    \begin{aligned}
      \nabla v
      &= e\dd(e^{-1}v) + e\omega_e e^{-1}v
    \\&= fA^{-1}\dd(Af^{-1}v) + fA^{-1}\omega_e Af^{-1}v
    \\&= f\dd f^{-1}v + f(A^{-1}\dd A + A^{-1}\omega_e A)f^{-1}v.
    \end{aligned}
  \]
  Comparing this with \cref{I.3.1.2} in the basis $f$, we find that
  \[
  \label{I.3.1.3}
    \omega_f = A^{-1}\dd A + A^{-1}\omega_e A.
  \tag{3.1.3}
  \]

  If, further, $(x^i)$ is a system of local coordinates on $X$, which defines a basis $(\dd x^i)$ of $\Omega_X^1$, we set
  \[
    \omega_\beta^\alpha = \sum_i \Gamma_{\beta i}^\alpha \dd x^i
  \]
  and we call the holomorphic functions $\Gamma_{\beta i}^\alpha$ the \emph{coefficients of the connection}.
  Equation~\cref{I.3.1.2} can the be written as
  \[
  \label{I.3.1.4}
    (\nabla_i v)^\alpha = \partial_i v^\alpha + \sum_\beta \Gamma_{\beta i}^\alpha v^\beta.
  \tag{3.1.4}
  \]

  The differential equation $\nabla v=0$ of horizontal sections of $\sh{V}$ can be written as the linear homogeneous system of first-order partial differential equations
  \[
  \label{I.3.1.5}
    \partial_i v^\alpha = -\sum_\beta \Gamma_{\beta i}^\alpha v^\beta.
  \tag{3.1.5}
  \]
\end{env}

\begin{env}{3.2}
\label{I.3.2}
  With the notation of \cref{I.3.1.2}, and using Einstein summation notation, we have that
  \[
    \begin{aligned}
      \nabla\nabla v
      &= \nabla((\dd v^\alpha + \omega_\beta^\alpha v^\beta)e_\alpha)
    \\&= \dd(\dd v^\alpha + \omega_\beta^\alpha v^\beta)e_\alpha - (\dd v^\alpha + \omega_\beta^\alpha v^\beta)\wedge\omega_\alpha^\gamma\cdot e_\gamma
    \\&= \dd\omega_\beta^\alpha\cdot v^\beta\cdot e_\alpha - \omega_\beta^\alpha\wedge\dd v^\beta\cdot e_\alpha - \dd v^\alpha\wedge\omega_\alpha^\gamma\cdot e_\gamma - \omega_\beta^\alpha\wedge\omega_\alpha^\gamma\cdot v^\beta e_\gamma
    \\&= (\dd\omega_\beta^\gamma - \omega_\beta^\alpha\wedge\omega_\alpha^\gamma)v^\beta e_\gamma.
    \end{aligned}
  \]

  The curvature tensor matrix is thus
\oldpage{23}
  \[
  \label{I.3.2.1}
    R_\beta^\alpha = \dd\omega_\beta^\alpha + \sum_\gamma \omega_\gamma^\alpha\wedge\omega_\beta^\gamma,
  \tag{3.2.1}
  \]
  which we can also write as
  \[
  \label{I.3.2.2}
    R = \dd\omega + \omega\wedge\omega.
  \tag{3.2.2}
  \]
  Equation~\cref{I.3.2.1} gives, in a system $(x^i)$ of local coordinates,
  \[
  \label{I.3.2.3}
    \begin{cases}
      R_{\beta i j}^\alpha
      &= (\partial_i\Gamma_{\beta j}^\alpha - \partial_j\Gamma_{\beta i}^\alpha) + (\Gamma_{\gamma i}^\alpha\Gamma_{\beta j}^\gamma - \Gamma_{\gamma j}^\gamma\Gamma_{\beta i}^\gamma)
    \\R_\beta^\alpha
      &= \sum_{i<j} R_{\beta i j}^\alpha \dd x^i\wedge\dd x^j.
    \end{cases}
  \tag{3.2.3}
  \]

  The condition $R_{\beta i j}^\alpha = 0$ is the integrability condition of the system \cref{I.3.1.5}, in the classical sense of the word;
  it can be obtained by eliminating $v^\alpha$ from the equations given by substituting \cref{I.3.1.5} into the identity $\partial_i\partial_j v^\alpha = \partial_j\partial_i v^\alpha$.
\end{env}



\section{\texorpdfstring{$n^\mathrm{th}$}{nth}-order differential equations}
\label{I.4}

\begin{env}{4.1}
\label{I.4.1}
  The solution of a linear homogeneous $n^\mathrm{th}$-order differential equation
  \[
  \label{I.4.1.1}
    \frac{\dd^n}{\dd x^n}y = \sum_{i=1}^n a_i(x) \frac{\dd^{n-i}}{\dd x^i}y
  \tag{4.1.1}
  \]
  is equivalent to that of the system
  \[
  \label{I.4.1.2}
    \begin{cases}
      \frac{\dd}{\dd x}y_i
      = y_{i+1} &\mbox{($1\leq i<n$);}
    \\\frac{\dd}{\dd x}y_n
      = \sum_{i=1}^n a_i(x) y_{n+1-i}
    \end{cases}
  \tag{4.1.2}
  \]
  of $n$ first-order equations.

  By \cref{I.3}, this system can be described as the differential equation of horizontal sections of a rank-$n$ vector bundle endowed with a suitable connection, and this is what we aim to further explore.
\end{env}

\begin{env}{4.2}
\label{I.4.2}
  Let $X$ be a non-singular complex-analytic variety of pure dimension~$1$.
  Let $X_n$ be the $n^\mathrm{th}$ infinitesimal neighbourhood of the diagonal of $X\times X$, and $p_1$ and $p_2$ the two projections from $X_n$ to $X$.
  We denote by $\pi_{k,l}$ the injection from $X_l$ to
\oldpage{24}
  $X_k$, for $l\leq k$.

  Let $\Omega^{\otimes n}$ be the $n^\mathrm{th}$ tensor power of the invertible sheaf $\Omega_X^1$ (for $n\in\ZZ$).
  If $I$ is the ideal that defines the diagonal of $X\times X$, then $I/I^2\simeq\Omega_X^1$ canonically, and
  \[
  \label{I.4.2.1}
    I^n/I^{n+1} \simeq \Omega^{\otimes n}.
  \tag{4.2.1}
  \]

  If $\sh{L}$ is an invertible sheaf on $X$, then we denote by $P^n(\sh{L})$ the vector bundle
  \[
  \label{I.4.2.2}
    P^n(\sh{L}) = (p_1)_*p_2^*\sh{L}
  \tag{4.2.2}
  \]
  of $n^\mathrm{th}$-order jets of sections of $\sh{L}$.
  The $I$-adic filtration of $p_2^*\sh{L}$ defines a filtration of $P^n(\sh{L})$ for which
  \[
  \label{I.4.2.3}
    \begin{aligned}
      \Gr P^n(\sh{L}) &\simeq \Gr P^n(\sh{O})\otimes\sh{L}
    \\\Gr^i P^n(\sh{L}) \simeq \Omega^{\otimes i}\otimes\sh{L} \qquad\mbox{($0\leq i\leq n$).}
    \end{aligned}
  \tag{4.2.3}
  \]

  Recall that we define, by induction on $n$, a \emph{differential operator of order $\leq n$} as being a morphism $A\colon\sh{M}\to\sh{N}$ of abelian sheaves such that
  \[
    \begin{cases}
      \mbox{$A$ is $\sh{O}$-linear} & \mbox{for $n=0$ ;}
    \\\mbox{$[A,f]$ is of order $\leq m$ for every local section $f$ of $\sh{O}$} & \mbox{for $n=m+1$.}
    \end{cases}
  \]

  For every local section $s$ of $\sh{L}$, $p_2^*s$ defines a local section $\DD^n(s)$ of $P^n(\sh{L})$ \cref{I.4.2.2}.
  The $\CC$-linear sheaf morphism $\DD^n\colon\sh{L}\to P^n(\sh{L})$ is the universal differential operator of order $\leq n$ with domain $\sh{L}$.
\end{env}

\begin{definition}{4.3}
\label{I.4.3}
  \begin{enumerate}[(i)]
    \item A \emph{linear homogeneous $n^\mathrm{th}$-order differential equation} on $\sh{L}$ is an $\sh{O}_X$-homomorphism $E\colon P^n(\sh{L})\to\Omega^{\otimes n}\otimes\sh{L}$ that induces the identity on the submodule $\Omega^{\otimes n}\otimes\sh{L}$ of $P^n(\sh{L})$.
    \item A local section $s$ of $\sh{L}$ is a \emph{solution} of the differential equation $E$ if $E(\DD^n(s))=0$.
  \end{enumerate}
\end{definition}

In fact, I have cheated with this definition, in that I have only considered equations that can be put in the ``resolved'' form \cref{I.4.1.1}.

\oldpage{25}
\begin{env}{4.4}
\label{I.4.4}
  Suppose that $\sh{L}=\sh{O}$, and let $x$ be a local coordinate on $X$.
  The choice of $x$ allows us to identify $p^k(\sh{O})$ with $\sh{O}^{[0,k]}$, with the arrow $\DD^k$ becoming
  \[
    \begin{aligned}
      \DD^k\colon \sh{O} &\to P^k(\sh{O}) \simeq \sh{O}^{[0,k]}
    \\f &\mapsto (\partial_x^i f)_{0\leq i\leq k}.
    \end{aligned}
  \]
  The choice of $x$ also allows us to identify $\Omega^1$ with $\sh{O}$, so that the $n^\mathrm{th}$-order differential equation can be identified with a morphism $E\in\Hom(\sh{O}^{[0,n]},\sh{O})$, and, as such, has coordinates $(b_i)_{0\leq i\leq n}$ with $b_n=1$.
  The solutions of $E$ are then exactly the (holomorphic) functions $f$ that satisfy
  \[
  \label{I.4.4.1}
    \sum_{i=0}^n b_i(x) \partial_x^i f = 0
    \qquad\mbox{($b_n=1$).}
  \tag{4.4.1}
  \]

  The existence and uniqueness theorem for solutions of the Cauchy problem in \cref{I.4.4.1} implies the following.
\end{env}

\begin{theorem}{4.5}
\label{I.4.5}
  \emph{(Cauchy).}
  Let $X$ and $\sh{L}$ be as in \cref{I.4.2}, and let $E$ be an $n^\mathrm{th}$-order differential equation on $\sh{L}$.
  Then
  \begin{enumerate}[(i)]
    \item the subsheaf of $\sh{L}$ given by solutions of $E$ is a local system $\sh{L}^E$ of rank~$n$ on $X$ ; and
    \item the canonical arrow $\DD^{n-1}\colon\sh{L}^E \to P^{n-1}(\sh{L})$ induces an isomorphism
      \[
        \sh{O}\otimes_\CC\sh{L}^E \xrightarrow{\sim} P^{n-1}(\sh{L}).
      \]
  \end{enumerate}
\end{theorem}

In particular, it follows from \cref{I.4.5}(ii) and from \cref{I.2.17} that $E$ defines a canonical connection on $P^{n-1}(\sh{L})$, whose horizontal sections are the images under $\DD$ of solutions of $E$.

\begin{env}{4.6}
\label{I.4.6}
  To a differential equation $E$ on $\sh{L}$, we have thus associated
  \begin{enumerate}[a)]
    \item a holomorphic vector bundle $\sh{V}$ endowed with a connection (which is automatically integrable) \todo ; and
    \item a surjective homomorphism $\lambda\colon\sh{V}\to\sh{L}$ (by $i=0$ in \cref{I.4.2.3}).
  \end{enumerate}
  Furthermore, the solutions of $E$ are exactly the images under $\lambda$ of the horizontal sections of $\sh{V}$.
  This is just another way of expressing how to obtain \cref{I.4.1.2} from \cref{I.4.1.1}.
\end{env}

\oldpage{26}
\begin{env}{4.7}
\label{I.4.7}
  Let $\sh{V}$ be a rank-$n$ vector bundle on $X$, endowed with a connection with covariant derivative $\nabla$.
  Let $v$ be a local section of $\sh{V}$, and $w$ a vector field on $X$ that doesn't vanish at any point.
  We say that $v$ is \emph{cyclic} if the local sections $(\nabla_w)^i(v)$ of $\sh{V}$ (for $0\leq i<n$) form a basis of $\sh{V}$.
  This condition does not depend on the choice of $w$, and if $f$ is an invertible holomorphic function, then $v$ is cyclic if and only if $fv$ is cyclic.
  In fact, we can show, by induction on $i$, that $(\nabla_{gw})^i(fv)$ lies in the submodule of $\sh{V}$ generated by the $(\nabla_w)^j(v)$ (for $0\leq j\leq i$).

  If $\sh{L}$ is an invertible module, then we say that a section $v$ of $\sh{V}\otimes\sh{L}$ is cyclic if, for every every local isomorphism between $\sh{L}$ and $\sh{O}$, the corresponding section of $\sh{V}$ is cyclic.
  This applies, in particular, to sections $v$ of $\shHom(\sh{V},\sh{L})=\sh{V}^\vee\otimes\sh{L}$.
\end{env}

\begin{lemma}{4.8}
\label{I.4.8}
  With the hypotheses and notation of \cref{I.4.6}, $\lambda$ is a cyclic section of $\shHom(\sh{V},\sh{L})$.
\end{lemma}

\begin{proof}
  The problem is local on $X$;
  we can reduce to the case where $\sh{L}=\sh{O}$ and where there exists a local coordinate $x$.

  We use the notation of \cref{I.4.4}.
  A section $(f^i)$ of $P^{n-1}(\sh{O})\simeq\sh{O}^{[0,n-1]}$ is horizontal if and only if it satisfies
  \[
    \begin{cases}
      \partial_x f^i = f^{i+1}
      & \mbox{($0\leq i\leq n-2$)}
    \\\partial_x f^{n-1} = -\sum_{i=0}^{n-1} b_i f^i.
    \end{cases}
  \]
  This gives us the coefficients of the connection: the matrix of the connection is
  \[
  \label{I.4.8.1}
    \left(
      \def\arraystretch{1.8}
      \begin{array}{rrrrrr}
        0 & -1 & 0 & 0 & \ldots & 0
      \\0 & 0 & -1 & 0 & \ldots & 0
      \\0 & 0 & 0 & -1 & & 0
      \\\vdots & \vdots & & \ddots & \ddots & \vdots
      \\\vdots & \vdots & & & 0 & -1
      \\b^0 & b^1 & \ldots & \ldots & b^{n-2} & b^{n-1}
      \end{array}
    \right)
  \tag{4.8.1}
  \]

\oldpage{27}
  In the chosen system of coordinates, $\lambda=e^0$, and we calculate that
  \[
    \nabla_x^i\lambda = e^i
    \qquad\mbox{($0\leq i\leq n-1$)}
  \]
  which proves \cref{I.4.8}.
\end{proof}

\begin{proposition}{4.9}
\label{I.4.9}
  The construction in \cref{I.4.6} establishes an equivalence between the following categories, where we take morphisms to be isomorphisms:
  \begin{enumerate}[a)]
    \item the category of invertible sheaves on $X$ endowed with an $n^\mathrm{th}$-order differential equation \cref{I.4.3} ; and
    \item the category of triples consisting of a rank-$n$ vector bundle $\sh{V}$ endowed with a connection, an invertible sheaf $\sh{L}$, and a cyclic homomorphism $\lambda\colon\sh{V}\to\sh{L}$.
  \end{enumerate}
\end{proposition}

\begin{proof}
  We will construct a functor that is quasi-inverse to that in \cref{I.4.6}.
  Let $\sh{V}$ be a vector bundle with connection, and $\lambda$ a homomorphism from $\sh{V}$ to an invertible sheaf $\sh{L}$.
  We denote by $V$ the local system of horizontal sections of $\sh{V}$.
  For every $\sh{O}$-module $\sh{M}$, we have \cref{I.2.17}
  \[
    \Hom_\sh{O}(\sh{V},\sh{M}) \xrightarrow{\sim} \Hom_\CC(V,\sh{M}).
  \]
  In particular, we define a map $\gamma^k$ from $\sh{V}$ to $P^k(\sh{L})$ by setting, for any \emph{horizontal section} $v$ of $\sh{V}$,
  \[
  \label{I.4.9.1}
    \gamma^k(v) = \DD^k(\lambda(v)).
  \tag{4.9.1}
  \]

  \begin{lemma}{4.9.2}
  \label{I.4.9.2}
    The homomorphism $\lambda$ is cyclic if and only if
    \[
      \gamma^{n-1}\colon \sh{V} \to P^{n-1}(\sh{L})
    \]
    is an isomorphism.
  \end{lemma}

  The problem is local on $X$.
  We can restrict to the case where $\sh{L}=\sh{O}$ and where we have a local coordinate $x$.
  With the notation of \cref{I.4.4}, the morphism $\gamma^k$ then admits the morphisms $\partial_x^i\lambda = \nabla_x^i$ (for $0\leq i\leq k$) as coordinates.
  For $k=n-1$, these form a basis of $\Hom(\sh{V},\sh{O})$ if and only if $\gamma^{n-1}$ is an isomorphism.

  For $k\geq l$, the diagram
\oldpage{28}
  \[
  \label{I.4.9.3}
    \begin{tikzcd}[row sep=huge]
      & \sh{V} \dlar[swap,"\gamma^k"] \drar["v^l"] &
    \\P^k(\sh{L}) \ar[rr,"\pi_{l,k}"] && P^k(\sh{L})
    \end{tikzcd}
  \tag{4.9.3}
  \]
  commutes;
  if $\lambda$ is cyclic, then this, along with \cref{I.4.9.2}, implies that $\gamma^n(v)$ is locally a direct factor, of codimension~$1$ in $P^n(\sh{L})$, and admits $\omega^{\otimes n}\otimes\sh{L} \simeq \Ker(\pi_{n-1,n})$ as a complement.
  There thus exists exactly one $n^\mathrm{th}$-order differential equation
  \[
    E\colon P^n(\sh{L}) \to \Omega^{\otimes n}\otimes\sh{L}
  \]
  on $\sh{L}$ such that $E\circ\gamma^n=0$.

  By \cref{I.4.9.1}, if $v$ is a horizontal section of $\sh{V}$, then $E\DD^n\lambda v = E\gamma^n v = 0$, and so $\lambda v$ is a solution of $E$.
  We endow $P^{n-1}(\sh{L})$ with the connection \cref{I.4.6} defined by $E$.
  If $v$ is a horizontal section of $\sh{V}$, then $\gamma^{n-1}(v)=\DD^{n-1}\lambda v$, where $\lambda v$ is a solution of $E$, and $\gamma^{n-1}(v)$ is thus horizontal.
  We thus deduce that $\gamma^{n-1}$ is compatible with the connections.
  A particular case of \cref{I.4.9.3} shows that the diagram
  \[
    \begin{tikzcd}[row sep=huge]
      \sh{V} \ar[rr,"\gamma^{n-1}"] \drar[swap,"\lambda"]
      && P^{n-1}(\sh{L}) \dlar["\todo"]
    \\&\sh{L}&
    \end{tikzcd}
  \]
  commutes, whence we have an isomorphism between $(\sh{V},\sh{L},\lambda)$ and the triple given by \cref{I.4.6} applied to $(\sh{L},E)$.
  The functor
  \[
    (\sh{V},\sh{L},\lambda) \mapsto (\sh{L},E)
  \]
  is thus quasi-inverse to the functor in \cref{I.4.6}.
\end{proof}

\begin{env}{4.10}
\label{I.4.10}
  We now summarise the relations between two systems $(\sh{V},\sh{L},\lambda)$ and $(\sh{L},E)$ that correspond under \cref{I.4.6} and \cref{I.4.9}.

  We have homomorphisms $\gamma^k\colon\sh{V}\to P^k(\sh{L})$, such that
  \begin{enumerate}
    \item[(4.10.1)]\label{I.4.10.1}
      $\gamma^k(v) = \DD^k\lambda v$ for $v$ horizontal ;
\oldpage{29}
    \item[(4.10.2)]\label{I.4.10.2}
      $\gamma^0=\lambda$ and $\pi_{l,k}\gamma^k=\gamma^l$ ;
    \item[(4.10.3)]\label{I.4.10.3}
      $\gamma^{n-1}$ is an isomorphism ($\lambda$ is cyclic) ;
    \item[(4.10.4)]\label{I.4.10.4}
      $E\gamma^n=0$ ; and
    \item[(4.10.5)]\label{I.4.10.5}
      $\lambda$ induces an isomorphic between the local system $V$ of sections of $\sh{V}$ and the local system $\sh{L}^E$ of solutions of $E$.
  \end{enumerate}
\end{env}


\section{Second-order differential equations}
\label{I.5}

In this section, we specialise the results of \cref{I.4} to the case where $n=2$, and we express certain results given in R.C.~Gunning~\cite{11} in a more geometric form.

\begin{env}{5.1}
\label{I.5.1}
  Let $S$ be an analytic space, and let $q\colon X_2\to S$ be an analytic space over $S$ that is locally isomorphic to the finite analytic space over $S$ defined by the $\sh{O}_S$-algebra $\sh{O}_S[T]/(T^3)$.

  The fact that the group $\mathrm{PGL}_2$ acts \todo on $\PP^1$ has the following infinitesimal analogue.
\end{env}

\begin{lemma}{5.2}
\label{I.5.2}
  Under the hypotheses of \cref{I.5.1}, let $u$ and $v$ be $S$-immersions of $X_2$ into $\PP_S^1$, i.e.
  \[
    \begin{tikzcd}
      X_2 \rar[shift left=1,"u"] \rar[shift right=1,swap,"v"]
      & \PP_S^1.
    \end{tikzcd}
  \]
  Then there exists exactly one \todo ($S$-automorphism) of $\PP_S^1$ that sends $u$ to $v$.
\end{lemma}

\begin{proof}
  The problem is local on $S$, which allows us to suppose that $X_2$ is defined by the $\sh{O}_S$-algebra $\sh{O}_S[T]/(T^3)$, and that $u(X_2)$ and $v(X_2)$ are contained inside the same affine line, say, $\AA_S^1$.
  By translation, we can assume that $u(0)=v(0)=0$.
  We must then prove the existence and uniqueness of a \todo $p(x)=(ax+b)(cx+d)$ that satisfies $p(0)=0$, with first derivative $p'(0)\neq0$ and given second derivative $p''(0)$.
  We have that $b=0$, and $p$ can be written uniquely in the form
\oldpage{30}
  \[
    \begin{aligned}
      p(x)
      &= e\frac{x}{1-fx} \qquad\mbox{($e\neq0$)}
    \\&= ex + efx^2 \mod x^3.
    \end{aligned}
  \]
  The claim then follows immediately.
\end{proof}

\begin{env}{5.3}
\label{I.5.3}
  By \cref{I.5.2}, there exists exactly (up to isomorphism) one pair $(u,P)$ consisting of a projective line $P$ on $S$ (with structure group $\mathrm{PGL}_2(\sh{O}_S)$) and an $S$-immersion $u$ of $X_2$ into $P$.
  We call $P$ the \emph{osculating projective line of $X_2$}.

  Let $X$ be a smooth curve.
  Let $X_2$ the second infinitesimal neighbourhood of the diagonal of $X\times X$, and let $q_1$ and $q_2$ be the two projections from $X_2$ to $X$.

  The morphism $q_1\colon X_2\to X$ is of the type considered in \cref{I.5.1}.
\end{env}

\begin{definition}{5.4}
\label{I.5.4}
  We define the \emph{osculating projective line bundle of $X$}, denoted by $P_\tg$, to be the osculating projective line bundle of $q_1\colon X_2\to X$.
\end{definition}

By definition, we thus have a canonical commutative diagram
\[
\label{I.5.4.1}
  \begin{tikzcd}
    X_2 \rar[hook] \drar[swap,"q_1"]
    & P_\tg \dar
  \\& X
  \end{tikzcd}
\tag{5.4.1}
\]
and, in particular, $P_\tg$ is endowed with a canonical section $e$, which is the image of the diagonal section of $X_2$, and we have that
\[
\label{I.5.4.2}
  e^*\Omega_{P_\tg/X}^1 \simeq \Omega_X^1.
\tag{5.4.2}
\]

\begin{env}{5.5}
\label{I.5.5}
  If $X$ is a projective line, then $\pr_1\colon X\times X\to X$ is a projective bundle on $X$ such that $P_\tg$ can be identified with the constant projective bundle of fibre $X$ on $X$ endowed with the inclusion homomorphism of $X_2$ into $X\times X$, i.e.
  \[
    \begin{tikzcd}
      X_2 \rar[hook] \drar[swap,"q_1"]
      & X\times X \dar["\pr_1"]
    \\& X
    \end{tikzcd}
  \]
\oldpage{31}
  In this particular case, we have a canonical commutative diagram
  \[
    \begin{tikzcd}[column sep=huge]
      X_3 \drar[hook] \ar[ddr]
    \\X_2 \uar[hook] \rar[hook] \drar
      & P_\tg \dar
    \\& X
    \end{tikzcd}
  \]
\end{env}

Now let $X$ be an arbitrary smooth curve.

\begin{definition}{5.6}
\label{I.5.6}
  \emph{(Local version).}
  A \emph{projective connection} on $X$ is a sheaf on $X$ of germs of local isomorphisms from $X$ to $\PP^1$, which is a principal homogeneous sheaf (i.e. a torsor) for the constant sheaf of groups with value $\mathrm{PLG}_2(\CC)$.
\end{definition}

If $X$ is endowed with a projective connection, then every local construction on $\PP^1$ that is invariant under the projective group can be transported to $X$;
in particular, the construction in \cref{I.5.5} gives us a morphism $\gamma$ that fits into a commutative diagram:
\[
\label{I.5.6.1}
  \begin{tikzcd}[column sep=huge]
    X_3 \drar[hook,"\gamma"] \ar[ddr]
  \\X_2 \uar \rar[hook] \drar
    & P_\tg \dar
  \\& X
  \end{tikzcd}
\tag{5.6.1}
\]

It is not difficult to show that such a morphism $\gamma$ is defined by a unique projective connection (a proof of this will be given in \cref{I.5.10}), and so Definition~\cref{I.5.6} is equivalent to the following.

\begin{definition}{5.6~bis}
\label{I.5.6bis}
  \emph{(Infinitesimal version).}
  A \emph{projective connection} on $X$ is a morphism $\gamma\colon X_3\hookrightarrow P_\tg$ that makes the diagram in \cref{I.5.6.1} commute.
\end{definition}

Intuitively, giving a projective connection (the infinitesimal version) allows us to define \todo

\begin{env}{5.7}
\label{I.5.7}
  Set $\Omega^{\otimes n}=(\Omega_X^1)^{\otimes n}$ \cref{I.4.2}.
  The sheaf of ideals on $X_3$ that defines $X_2$ is
\oldpage{32}
  canonically isomorphic to $\Omega^{\otimes3}$, and is annihilated by the sheaf of ideals that defines the diagonal.
  Also, if $\Delta$ is the diagonal map, then, by \cref{I.5.4.2}, we have that
  \[
    \Delta^*\gamma^*\Omega_{P_\tg/X}^1 \simeq \Omega^1.
  \]
  We thus deduce that the set of $X$-homomorphisms from $X_3$ to $P_\tg$ that induce the canonical homomorphism from $X_2$ to $P_\tg$ is either empty, or a principal homogeneous space for
  \[
    \Hom_X(\Delta^*\gamma^*\Omega_{P_\tg/X}^1, \Omega^{\otimes3})
    = \Hom_X(\Omega^1,\Omega^{\otimes3})
    = \HH^0(X,\Omega^{\otimes2}).
  \]
  If we replace $X$ by a small enough open subset, then this set is non-empty:
\end{env}

\begin{proposition}{5.8}
\label{I.5.8}
  Projective connections of open subsets of $X$ form a principal homogeneous sheaf (i.e. a torsor) for the sheaf $\Omega^{\otimes2}$.
\end{proposition}

\begin{proof}
  If $\eta$ is a section of $\Omega^{\otimes2}$, and $\gamma_1\colon X_3\to P_\tg$ is a projective connection, then the connection $\gamma_2=\gamma_1+\eta$ is defined, for any function $f$ on $P_\tg$, by
  \[
  \label{I.5.8.1}
    \gamma_2^*f = \gamma_1^*f + \eta\cdot e^*\dd f
  \tag{5.8.1}
  \]
  (modulo the identification of $\Omega^{\otimes3}$ with an ideal of $\sh{O}_{X_3}$).
\end{proof}

\begin{env}{5.9}
\label{I.5.9}
  Let $f\colon X\to Y$ be a homomorphism between smooth curves endowed with projective connections $\gamma_X$ and $\gamma_Y$, and suppose that $f$ is a local isomorphism (i.e. $\dd f\neq0$ at all points).
  Set
  \[
    \theta f = f^*\gamma_Y - \gamma_X \in \Gamma(X,\Omega^{\otimes2}).
  \]
  For a composite map $g\circ f\colon X\xrightarrow{f}Y\xrightarrow{g}Z$, we trivially have that
  \[
  \label{I.5.9.1}
    \begin{aligned}
      \theta(g\circ f) &= \theta(f) + f^*\theta(g),
    \\\mbox{whence }\theta(f^{-1}) &= -f^*\theta(f).
    \end{aligned}
  \tag{5.9.1}
  \]

  Suppose that $X$ and $Y$ are open subsets of $\CC$, and endowed with the projective connection induced by that of $\PP^1(\CC)$.
  Denoting by $x$ the injection from $X$ into $\CC$, we then have that
\oldpage{33}
  \[
  \label{I.5.9.2}
    \theta f = \frac{f'(f'''/6) - (f''/2)^2}{(f')^2} \dd x^{\otimes2}.
  \tag{5.9.2}
  \]

  To prove this, we identify, using \cref{I.5.5}, the projective double tangent bundle to $X$ or $Y$ with the constant projective bundle.
  The morphism $\delta f\colon P_{\tg,X}\to P_{\tg,Y}$ induced by $f$ can be written as
  \[
    \delta f\colon (x,p) \mapsto
    \left(
      f(x),
      f(x) + \frac{f'(x)(p-x)}{1-\frac12(f''(x)/f'(x))(p-x)}
    \right).
  \]

  Consider the diagram
  \[
    \begin{tikzcd}
      X_3 \rar["f"] \dar[hook]
      & Y_3 \dar[hook]
    \\P_{\tg,X} \rar["\delta f"]
      & P_{\tg,Y}
    \end{tikzcd}
  \]
  Then $\theta(f)$ describes the lack of commutativity of the diagram, i.e. the difference between the jets
  \[
    (x,x+\varepsilon)
    \mapsto
    \left(
      f(x),
      f(x) + f'(x)\varepsilon + f''(x)\frac{\varepsilon^2}{2} + f'''(x)\frac{\varepsilon^3}{6}
    \right)
    \qquad\mbox{($\varepsilon^4=0$)}
  \]
  and
  \[
    \begin{aligned}
      (x,x+\varepsilon)
      &\mapsto
      \left(
        f(x),
        f(x) + \frac{f'(x)\varepsilon}{1-\frac12(f''(x)/f'(x))\varepsilon}
      \right)
    \\&= \left(
        f(x),
        f(x) + f'(x)\varepsilon + f''(x)\frac{\varepsilon^2}{2} + \frac14(f''(x)^2/f'(x))\varepsilon^3
      \right).
    \end{aligned}
  \]
  We thus have that
  \[
    \theta(f) =
    \left(
      \frac16f'''(x) - \frac14f''(x)^2/f'(x)
    \right) \dd x^{\otimes3} \dd f^{\otimes-1},
  \]
  and \cref{I.5.9.2} then follows.

  Equation~\cref{I.5.9.2} shows that $6\theta f$ is the classical \emph{Schwarz derivative} of $f$.

  \bigskip

  If a map $f$ from $X\subset\CC$ to $\PP^1(\CC)$ is described by projective coordinates $f=(g,h)$, then
  \[
  \label{I.5.9.3}
    \theta(f) =
    \frac{
      \left\vert
        \begin{array}{cc}
          g&g'\\h&h'
        \end{array}
      \right\vert
      \left(
        \left\vert
          \begin{array}{cc}
            g&g'''/6\\h&h'''/6
          \end{array}
        \right\vert
        +
        \left\vert
          \begin{array}{cc}
            g'&g'/2\\h'&h'/2
          \end{array}
        \right\vert
      \right)
      -
      \left\vert
        \begin{array}{cc}
          g&g''/2\\h&h''/2
        \end{array}
      \right\vert^2
    }{
      \left\vert
        \begin{array}{cc}
          g&g'\\h&h'
        \end{array}
      \right\vert^2
    }
  \tag{5.9.3}
  \]

  To prove \cref{I.5.9.3}, the simplest method is to note the following.
\oldpage{34}
  \begin{enumerate}[(i)]
    \item The right-hand side of \cref{I.5.9.3} is invariant under a linear substitution of constant coefficients $L$ acting on $g$ and $h$: the numerator and denominator $\det(L)^2$.
    \item The right-hand side of \cref{I.5.9.3} is invariant under the substitution
      \[
        (g,h) \mapsto (\lambda g,\lambda h).
      \]

      Denoting the determinant of $\left(\begin{array}{cc}a&b\\c&d\end{array}\right)$ by $\detrow{a}{b}$, we have that
      \[
        \begin{aligned}
          \detrow{g}{(\lambda g)'}
          \quad&=\quad \detrow{\lambda g}{\lambda g'+\lambda' g}
        \\[0.5em]
          &=\quad \lambda^2\detrow{g}{g'}
        \\[1em]
          \detrow{\lambda g}{(\lambda g)''/2}
          \quad&=\quad \lambda^2\detrow{g}{g''/2}
          + \lambda\lambda'\detrow{g}{g'}
        \\[1em]
          \detrow{\lambda g}{(\lambda g)'''/6}
          \quad&=\quad \lambda^2\detrow{g}{g'''/6}
          + \lambda\lambda'\detrow{g}{g''/2}
        \\[0.5em]
          &\qquad+ (\lambda\lambda''/2)\detrow{g}{g'}
        \\[1em]
          \detrow{(\lambda g)'}{(\lambda g)''/2}
          \quad&=\quad \lambda^2\detrow{g'}{g''/2}
          - (\lambda\lambda''/2)\detrow{g}{g'}
        \\[0.5em]
          &\qquad+ (\lambda')^2\detrow{g}{g'}
          + \lambda\lambda'\detrow{g}{g''/2}.
        \end{aligned}
      \]

      The new denominator $D_\lambda$ (resp. numerator $N_\lambda$) is thus given in terms of the old denominator $D$ (resp. numerator $N$) by
      \[
        \begin{aligned}
          D_\lambda
          &= \lambda^4 D
        \\N_\lambda
          &= \lambda^4 N + \lambda^4\detrow{g}{g'}\cdot
          \left[
            \begin{aligned}
              &\Big(
                \lambda'/\lambda\detrow{g}{g''/2}
                + (\lambda''/2\lambda)\detrow{g}{g'}
              \Big)
            \\+&\Big(
                -(\lambda''/2\lambda)\detrow{g}{g'}
                +(\lambda'/\lambda)^2\detrow{g}{g'}
                +(\lambda'/\lambda)\detrow{g}{g''}
              \Big)
            \\-&\Big(
                2(\lambda'/\lambda)\detrow{g}{g''/2}
                + (\lambda'/\lambda)^2\detrow{g}{g'}
              \Big)
            \end{aligned}
          \right]
        \end{aligned}
      \]
      and $N_\lambda/D_\lambda=N/D$.

      With these variance properties agreeing with those of the left-hand side of \cref{I.5.9.3}, it suffices to prove \cref{I.5.9.3} in the particular case where $h=1$.
      The equation then reduces to \cref{I.5.9.2}.
  \end{enumerate}

  We will only need to use the fact that $\theta(f)$ can be expressed in terms of \todo:
  we have, for $Z_i\sim Z$,
  \[
  \label{I.5.9.4}
    \frac{(f(Z_1),f(Z_2),f(Z_3),f(Z_4))}{(Z_1,Z_2,Z_3,Z_4)} - 1
    = \theta(f)(Z_1-Z_2)(Z_3-Z_4) + \mathcal{O}((Z_i-Z)^3).
  \tag{5.9.4}
  \]
\end{env}

\oldpage{35}
\begin{env}{5.10}
\label{I.5.10}
  The differential equation $\theta(f)=0$ (for $f\colon X\to\PP^1(\CC)$ with non-zero first derivative) is a third-order differential equation.
  It thus admits $\infty^3$ solutions, locally, and these solutions are permuted amongst themselves by the projective group (since the group transitively permutes the Cauchy data: \cref{I.5.2}).
  The set of solutions is thus a projective connection (the local version, \cref{I.5.6}).
  This construction is inverse to that which associates, to any projective connection in the sense of \cref{I.5.6}, a projective connection in the sense of \cref{I.5.6bis}.
\end{env}

\begin{env}{5.11}
\label{I.5.11}
  Let $X$ be a smooth curve, $\sh{L}$ an invertible sheaf on $X$, and $E$ a second-order ordinary differential equation on $\sh{L}$.
  We have seen, in \cref{I.4.5}, that $E$ defines a connection on the bundle $P^1(\sh{L})$ of first-order jets of sections of $\sh{L}$, and we obtain from it a connection on
  \[
    \bigwedge^2 P^1(\sh{L}) \simeq \sh{L}\otimes\Omega^1(\sh{L}) = \Omega^1\otimes\sh{L}^{\otimes2}.
  \]

  If $X$ is a compact connected curve of genus~$g$, then the bundle $\Omega^1\otimes\sh{L}^{\otimes2}$ is thus necessarily of degree~$0$, and we have that
  \[
    \deg(\sh{L}) = 1-g.
  \]

  Let $V$ be the rank-$2$ local system of solutions of $E$;
  we have \cref{I.4.5} that $\sh{O}\otimes V\xrightarrow{\sim}P^1(\sh{L})$, and the linear form $\lambda\colon P^1(\sh{L})\to\sh{L}$ defines a section $\lambda_0$ of the projective bundle associated to the vector bundle $P^1(\sh{L})$.

  Locally on $X$, $V$ is isomorphic to the constant local system $\underline{\CC^2}$;
  the choice of an isomorphism $\sigma\colon V\to\underline{\CC^2}$ identifies $\lambda_0$ with a map $\lambda_{0,\sigma}$ from $X$ to $\PP^1(\CC)$;
  by \cref{I.4.8}, the differential of this map is everywhere non-zero, and so $\lambda_{0,\sigma}$ allows us to transport the canonical projective connection of $\PP^1(\CC)$ to $X$.
  This connection does not depend on the choice of $\sigma$, and so the differential equation $E$ defines a projective connection on $X$.
\end{env}

\begin{proposition}{5.12}
\label{I.5.12}
  Let $\sh{L}$ be an invertible sheaf on a smooth curve $X$.
  The construction in \cref{I.5.10} gives a bijection between
  \begin{enumerate}[a)]
    \item the set of second-order ordinary differential equations on $\sh{L}$ ; and
\oldpage{36}
    \item the set of pairs consisting of a projective connection on $X$ and a connection on $\Omega^1(\sh{L}^{\otimes2})$.
  \end{enumerate}
\end{proposition}

\begin{proof}
  The problem is local on $X$;
  we can thus suppose that $X$ is an open subset of $\CC$, and that $\sh{L}=\sh{O}$.
  An equation $E$ can then be written as
  \[
    E\colon y'' + a(x)y' + b(x)y = 0.
  \]
  If we identify $P^1(\sh{L})$ with $\sh{O}^2$, then the matrix of the connection \cref{I.5.10} defined by $E$ on $\wedge^2 P^1(\sh{L})\sim\sh{O}$ is then $-a(x)\tr M$, where $M$ is the matrix in \cref{I.4.8.1}.

  Let $\varphi$ be the identity map from $X$, an open subset of $\PP^1(\CC)$, to itself endowed with the projective connection defined by $E$.
  Identifying $\Omega$ with $\sh{O}$ by means of the given local coordinate, we then have that
  \[
  \label{I.5.12.1}
    \theta(\varphi) = \frac13b - \frac{1}{12}(a^2+2a').
  \tag{5.12.1}
  \]
  Indeed, if $f$ and $g$ are two linearly independent solutions of $E$, then the map with projective coordinates
  \[
    (f,g)\colon X\to\PP^1(\CC)
  \]
  \todo the projective connection.
  We have that
  \[
    \begin{aligned}
      f''
      &= -(af'+bf)
    \\f'''
      &= -a(-af'-bf) - bf' - a'f' - b'f
    \\&= (a^2-a'-b)f' + (ab-b')f.
    \end{aligned}
  \]

  Equation~\cref{I.5.9.3} gives (using the same notation for determinants as before)
  \[
    \begin{aligned}
      \theta(\varphi)
      &= \frac{
        \detrow{f}{f'}
        \left(
          \frac16(a^2-a'-b)\detrow{f}{f'}
          +\frac12 b\detrow{f}{f'}
        \right)
        - (\frac12a)^2\detrow{f}{f'}^2
      }{
        \detrow{f}{f'}^2
      }
    \\&= \frac16(a^2-a'-b) + \frac12 b - \frac14 a^2
    \\&= \frac13 b - \frac{1}{12}(a^2+2a').
    \end{aligned}
  \]

  We conclude by noting that $(a,b)$ is uniquely determined by $(-a,\frac13b-\frac{1}{12}(a^2+2a'))$, and that, for any holomorphic function $g$ on an open subset $U$ of $\CC$, there exists a unique projective connection on $U$ satisfying $\theta(\varphi)=g$, for $\varphi$ \todo the connection (same proof as for \cref{I.5.10}, or \cref{I.5.8}).
\end{proof}


\oldpage{37}
\section{Multiform functions of finite determination}
\label{I.6}

\begin{env}{6.1}
\label{I.6.1}
  Let $X$ be a non-empty connected topological space that is both locally path connected and locally simply path connected, and let $x_0$ be a point of $X$.
  We denote by $\pi\colon \widetilde{X}_{x_0}\to X$ the universal covering of $(X,x_0)$, and by $\widetilde{x}_0$ the base point of $\widetilde{X}_{x_0}$.
\end{env}

If $\sh{F}$ is a sheaf on $X$, then we pose:
\begin{definition}{6.2}
\label{I.6.2}
  A \emph{multiform section} of $\sh{F}$ on $X$ is a global section of the inverse image $\pi^*\sh{F}$ of $\sh{F}$ on $\widetilde{X}_{x_0}$.
\end{definition}



%% Bibliography %%

\nocite{*}

\begin{thebibliography}{26}

  \bibitem{1}
  {\sc Atiyah,~M. and Hodge,~W.L.}
  \newblock Integrals of the second kind on an algebraic variety.
  \newblock {\em Ann. of Math.} \textbf{62} (1955), 56--91.

  \bibitem{2}
  {\sc Baily,~W.L. and Borel,~A.}
  \newblock Compactification of arithmetic quotients of bounded domains.
  \newblock {\em Ann. of Math.} \textbf{84} (1966), 442--528.

  \bibitem{3}
  {\sc Baker,~A.}
  \newblock Linear forms in the logarithms of algebraic numbers II.
  \newblock {\em Mathematika} \textbf{14} (1967), 102--107.

  \bibitem{4}
  {\sc Berthelot,~P.}
  \newblock Cohomologie $p$-cristalline des sch\'{e}mas.
  \newblock {\em CR Acad. Sci. Paris} \textbf{269} (1969), 297--300, 357--360, and 397--400.

  \bibitem{5}
  {\sc Brieskorn,~E.}
  \newblock Die monodromie der isolierten singularit\"{a}ten von hyperfl\"{a}chen.
  \newblock {\em Manuscripta math.} \textbf{2} (1970), 103--161.

  \bibitem{6}
  {\sc Gelfond,~A.}
  \newblock Sur le septi\`{e}me probl\`{e}me de D.~Hilbert.
  \newblock {Doklady Akad. Nauk. URSS} \textbf{2} (1934), 4--6.

  \bibitem{7}
  {\sc Godement,~R.}
  \newblock {\em Topologie alg\'{e}brique et th\'{e}orie des faisceaux}.
  \newblock Hermann, 1958.

  \bibitem{8}
  {\sc Griffiths,~P.A.}
  \newblock ``Some results on Moduli and Periods of Integrals on Algebraic Manifolds III''.
  \newblock (Mimeographed notes from Princeton).

  \bibitem{9}
  {\sc Grothendieck,~A.}
  \newblock On the De Rham cohomology of algebraic varieties.
  \newblock {\em Publ. Math. IHES} \textbf{29} (1966), 95--103.

  \bibitem{10}
  {\sc Grothendieck,~A. (notes by Coates,~I. and Jussila,~O.)}
  \newblock ``Crystals and the De Rham cohomology of schemes'', in: {\em dix expos\'{e}s sur la cohomologie des sch\'{e}mas}.
  \newblock North Holl. Publ. Co., 1968.

  \bibitem{11}
  {\sc Gunning,~R.}
  \newblock {\em Lectures on Riemann surfaces}.
  \newblock Princeton Math. Notes, 1966.

  \bibitem{12}
  {\sc Hironaka,~H.}
  \newblock Resolution of singularities of an algebraic variety over a field of characteristic zero, I and II.
  \newblock {\em Ann. of Math.} \textbf{79} (1964).

  \bibitem{13}
  {\sc Ince,~E.L.}
  \newblock {\em Ordinary differential equations}.
  \newblock Dover, 1956.

  \bibitem{14}
  {\sc Katz,~N.}
  \newblock Nilpotent connections and the monodromy theorem. Applications of a result of Turrittin.
  \newblock {\em Publ. Math. IHES} \textbf{39} (1970), 175--232.

  \bibitem{15}
  {\sc Katz,~N. and Oda,~T.}
  \newblock On the differentiation of De Rham cohomology classes with respect to parameters.
  \newblock {\em J. Math. Kyoto Univ.} \textbf{8} (1968), 199--213.

  \bibitem{16}
  {\sc Leray,~J.}
  \newblock Un compl\'{e}ment au th\'{e}or\`{e}me de N.~Nilsson sur les int\'{e}grales de formes diff\'{e}rentielles \`{a} support singulier alg\'{e}brique.
  \newblock {\em Bull. Soc. Math. France} \textbf{95} (1967), 313--374.

  \bibitem{17}
  {\sc Lojasiewicz,~S.}
  \newblock Triangulation of semi-analytic sets.
  \newblock {\em Annali della Scuola Normale Sup. di Pisa Ser III} \textbf{18} (1964), 449--474.

  \bibitem{18}
  {\sc Lojasiewicz,~S.}
  \newblock (Mimeographed notes by IHES).

  \bibitem{19}
  {\sc Manin,~Y.}
  \newblock Moduli Fuchsiani.
  \newblock {\em Annali della Scuola Normale Sup. di Pisa Ser III} \textbf{19} (1965), 113--126.

  \bibitem{20}
  {\sc Nagata,~M.}
  \newblock Embedding of an abstract variety in a complete variety.
  \newblock {\em J. Math. Kyoto} \textbf{2} (1962), 1--10.

  \bibitem{21}
  {\sc Nagata,~M.}
  \newblock A generalization of the embedding problem.
  \newblock {\em J. Math. Kyoto} \textbf{3} (1963), 89--102.

  \bibitem{22}
  {\sc Nilsson,~N.}
  \newblock Some growth and ramification properties of certain integrals on algebraic manifolds.
  \newblock {\em Arkiv f\"{o}r Math.} \textbf{5} (1963--65), 527--540.

  \bibitem{23}
  {\sc Plemelj,~J.}
  \newblock {\em Monatsch. Math. Phys.} \textbf{19} (1908), 211.

  \bibitem[GAGA]{GAGA}
  {\sc Serre,~J.P.}
  \newblock Géométrie algébrique et géométrie analytique.
  \newblock {\em Ann. Inst. Fourier. Grenoble} \textbf{6} (1956).

  \bibitem{25}
  {\sc Turrittin,~H.L.}
  \newblock Convergent solutions of ordinary homogeneous differential equations in the neighbourhood of a singular point.
  \newblock {\em Acta Math.} \textbf{93} (1955), 27--66.

  \bibitem{26}
  {\sc Turrittin,~H.L.}
  \newblock ``Asymptotic expansions of solutions of systems of ordinary linear differential equations containing a parameter'', in: {\em Contributions to the theory of nonlinear oscillations.}
  \newblock Princeton, 1952.

\end{thebibliography}

\end{document}

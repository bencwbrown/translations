\documentclass{report}

\title{Regular singular differential equations}
\author{Pierre Deligne}
\date{}

\usepackage{amssymb,amsmath}

\usepackage{hyperref}
\usepackage[nameinlink]{cleveref}
\usepackage{enumerate}
\usepackage{graphicx}

\usepackage{mathrsfs}
%% Fancy fonts --- feel free to remove! %%
\usepackage{Baskervaldx}
\usepackage{mathpazo}


\usepackage{fancyhdr}
\usepackage{lastpage}
\usepackage{xstring}
\makeatletter
\ifx\pdfmdfivesum\undefined
  \let\pdfmdfivesum\mdfivesum
\fi
\edef\filesum{\pdfmdfivesum file {\jobname}}
\pagestyle{fancy}
\fancypagestyle{plain}{}
\fancyhf{}
\lhead{\footnotesize\nouppercase\leftmark}
\rhead{\footnotesize Version: \texttt{\StrMid{\filesum}{1}{8}}}
\cfoot{\small\thepage\ of \pageref*{LastPage}}


\crefname{section}{\S\!}{\S\S\!}
\crefname{equation}{}{}


%% Theorem environments %%

\usepackage{amsthm}

  \theoremstyle{plain}

  \newtheorem{innercustomproposition}{Proposition}
  \crefname{innercustomproposition}{Proposition}{Propositions}
  \newenvironment{proposition}[1]
    {\renewcommand\theinnercustomproposition{#1}\innercustomproposition}
    {\endinnercustomproposition}

  \newtheorem{innercustomlemma}{Lemma}
  \crefname{innercustomlemma}{Lemma}{Lemmas}
  \newenvironment{lemma}[1]
    {\renewcommand\theinnercustomlemma{#1}\innercustomlemma}
    {\endinnercustomlemma}

  \newtheorem{innercustomcorollary}{Corollary}
  \crefname{innercustomcorollary}{Corollary}{Corollaries}
  \newenvironment{corollary}[1]
    {\renewcommand\theinnercustomcorollary{#1}\innercustomcorollary}
    {\endinnercustomcorollary}


  \theoremstyle{definition}

  \newtheorem{innercustomdefinition}{Definition}
  \crefname{innercustomdefinition}{Definition}{Definitions}
  \newenvironment{definition}[1]
    {\renewcommand\theinnercustomdefinition{#1}\innercustomdefinition}
    {\endinnercustomdefinition}

  \newtheorem{innercustomenv}{}
  \crefname{innercustomenv}{}{}
  \newenvironment{env}[1]
    {\renewcommand\theinnercustomenv{#1}\innercustomenv}
    {\endinnercustomenv}


%% Shortcuts %%

\newcommand{\sh}{\mathscr}
\newcommand{\cat}{\mathcal}
\newcommand{\sbullet}{{\mathbin{\vcenter{\hbox{\scalebox{.5}{$\bullet$}}}}}}
\newcommand{\id}{\mathrm{Id}}

\renewcommand{\geq}{\geqslant}
\renewcommand{\leq}{\leqslant}

\DeclareMathOperator{\Spec}{Spec}
\DeclareMathOperator{\Ker}{Ker}
\DeclareMathOperator{\Hom}{Hom}
\DeclareMathOperator{\End}{End}
\DeclareMathOperator{\iHom}{\underline{Hom}}
\DeclareMathOperator{\iEnd}{\underline{End}}
\DeclareMathOperator{\DD}{D}
\DeclareMathOperator{\dd}{d\!}

\newcommand{\todo}{\textbf{ !TODO! }}
\newcommand{\oldpage}[1]{\marginpar{\footnotesize$\Big\vert$ \textit{p.~#1}}}


%% Document %%

\usepackage{embedall}
\begin{document}

\maketitle

\renewcommand{\abstractname}{Translator's note.}

\begin{abstract}
  \renewcommand*{\thefootnote}{\fnsymbol{footnote}}
  \emph{This text is one of a series\footnote{\url{https://thosgood.com/translations/}} of translations of various papers into English.}
  \emph{The translator takes full responsibility for any errors introduced in the passage from one language to another, and claims no rights to any of the mathematical content herein.}

  \medskip
  
  \emph{What follows is a translation of the French book:}

  \medskip\noindent
  \textsc{Deligne, P.}
  \emph{Equations Diff\'{e}rentielles \`{a} Points Singuliers R\'{e}guliers.}
  Springer--Verlag, Lecture Notes in Mathematics \textbf{163} (1970).
  {\footnotesize\url{https://publications.ias.edu/node/355}}

  \medskip
  \noindent\emph{We have also made changes following the errata, which was written in April 1971, by P. Deligne, at Warwick University.}
\end{abstract}

\setcounter{footnote}{0}

\tableofcontents


%% Content %%


\setcounter{chapter}{-1}

\chapter{Introduction}
\label{0}

\oldpage{1}
If $X$ is a (non-singular) complex-analytic manifold, then there is an equivalence between the notions of
\begin{enumerate}[a)]
  \item local systems of complex vectors on $X$; and
  \item vector bundles on $X$ endowed with an integrable connection.
\end{enumerate}

The latter of these two notions can be adapted in an evident way to the case where $X$ is a non-singular algebraic variety over a field $k$ (which we will take here to be of characteristic $0$).
However, general algebraic vector bundles with integrable connections are pathological (see \cref{II.6.19});
we only obtain a reasonable theory if we impose a ``regularity'' condition at infinity.
By a theorem of Griffiths \cite{8}, this condition is automatically satisfied for ``Gauss-Manin connections'' (see \cref{II.7}).
In dimension one, this is closely linked to the idea of regular singular points of a differential equation (see \cref{I.4} and \cref{II.1}).

In Chapter~I, we explain the different forms that the notion of an integrable connection can take.
In Chapter~II, we prove the fundamental facts concerning regular connections.
In Chapter~III, we translate certain results that we have obtained into the language of Nilsson class functions, and, as an application of the regularity theorem (\cref{II.7}), we explain the proof by Brieskorn \cite{5} of the monodromy theorem.

These notes came from the non-crystalline part of a seminar given at Harvard during the autumn of 1969, under the title: ``Regular singular differential equations and crystalline cohomology''.

I thank the assistants of this seminar, who had to be subjected to often unclear expos\'{e}s, and who allowed me to bring numerous simplifications.

I also thank N.~Katz, with whom I had numerous and useful conversations, and to whom are due the principal results of section~\cref{II.1}.


\section*{Notations and terminology}

\oldpage{2}
Within a single chapter, the references follow the decimal system.
A reference to a different chapter (resp. to the current introduction) is preceded by the Roman numeral of the chapter (resp. by 0).

We will use the following definitions:
\begin{enumerate}[({0.}1)]
  \item\label{0.1}
    \emph{analytic space}:
    the analytic spaces are complex and of locally-finite dimension.
    They are assumed to be $\sigma$-compact, but not necessarily separated.
  \item\label{0.2}
    \emph{multiform function}:
    a synonym for multivalued function --- for a precise definition, see \cref{I.6.2}.
  \item\label{0.3}
    \emph{immersion}:
    following the tradition of algebraic geometers, immersion is a synonym for ``embedding''.
  \item\label{0.4}
    \emph{smooth}:
    a morphism $f\colon X\to S$ of analytic spaces is smooth if, locally on $X$, it is isomorphic to the projection from $D^n\times S$ to $S$, where $D^n$ is an open polydisc.
  \item\label{0.5}
    \emph{locally paracompact}:
    a topological space is locally paracompact if every point has a paracompact neighbourhood (and thus a fundamental system of paracompact neighbourhoods).
  \item\label{0.6}
    non-singular (or smooth) \emph{complex algebraic variety}:
    a smooth scheme of finite type over $\Spec(\mathbb{C})$.
  \item\label{0.7}
    (complex) \emph{analytic manifold}:
    a non-singular (or smooth) analytic space.
  \item\label{0.8}
    \emph{covering}:
    following the tradition of topologists, a covering is a continuous map $f\colon X\to Y$ such that every point $y\in Y$ has a neighbourhood $V$ such that $f|V$ is isomorphic to the projection from $F\times V$ to $V$, where $F$ is discrete.
\end{enumerate}


\renewcommand{\thechapter}{\Roman{chapter}}

\chapter{Dictionary}
\label{I}

\oldpage{3}
In this chapter, we explain the relations between various aspects and various uses of the notion of ``local systems of complex vectors''.
The equivalence between the points of view considered has been well known for a long time.

The ``crystalline'' point of view has not been considered;
see \cite{4,10}.


\section{Local systems and the fundamental group}
\label{I.1}

\begin{definition}{1.1}
\label{I.1.1}
  Let $X$ be a topological space.
  A \emph{complex local system} on $X$ is a sheaf of complex vectors on $X$ that, locally on $X$, is isomorphic to a constant sheaf $\mathbb{C}^n$ (n$\in\mathbb{N}$).
\end{definition}

\begin{env}{1.2}
\label{I.1.2}
  Let $X$ be a locally path-connected and locally simply path-connected topological space, along with a basepoint $x_0\in X$.
  To avoid any ambiguity, we point out that:
  \begin{enumerate}[a)]
    \item The fundamental group $\pi_1(X,x_0)$ of $X$ at $x_0$ has elements given by homotopy classes of loops based at $x_0$;
    \item If $\alpha,\beta\in\pi_1(X,x_0)$ are represented by loops $a$ and $b$, then $\alpha\beta$ is represented by the loop $ab$ obtained by juxtaposing $b$ and $a$, in that order.
  \end{enumerate}

  Let $\sh{F}$ be a locally constant sheaf on $X$.
  For every path $a\colon[0,1]\to X$, the inverse image $a^*\sh{F}$ of $\sh{F}$ on $[0,1]$ is a locally constant, and thus constant, sheaf, and there exists exactly one isomorphism between $a^*\sh{F}$ and the constant sheaf defined by the set $(a^*\sh{F})_0 = \sh{F}_{a(0)}$.
  This isomorphism defines an isomorphism $a(\sh{F})$ between $(a^*\sh{F})_0$ and $(a^*\sh{F})_1$, i.e. an isomorphism
  \[
    a(\sh{F})\colon \sh{F}_{a(0)} \to \sh{F}_{a(1)}.
  \]
  This isomorphism depends only on the homotopy class of $a$, and satisfies $ab(\sh{F}) = a(\sh{F})\cdot b(\sh{F})$.
  In particular, $\pi_1(X,x_0)$ acts (on the left) on the fibre $\sh{F}_{x_0}$ of $\sh{F}$ at $x_0$.
  It is well known that:
\end{env}

\begin{proposition}{1.3}
\label{I.1.3}
  Under the hypotheses of \cref{I.1.2}, with $X$ connected, the functor $\sh{F}\mapsto\sh{F}_{x_0}$ is an equivalence between the category of locally constant sheaves on $X$ and the category of sets endowed with an action by the group $\pi_1(X,x_0)$.
\end{proposition}

\oldpage{4}
\begin{corollary}{1.4}
\label{I.1.4}
  Under the hypotheses of \cref{I.1.2}, with $X$ connected, the functor $\sh{F}\mapsto\sh{F}_{x_0}$ is an equivalence between the category of complex local systems on $X$ and the category of complex finite-dimensional representations of $\pi_1(X,x_0)$.
\end{corollary}

\begin{env}{1.5}
\label{I.1.5}
  Under the hypotheses of \cref{I.1.2}, if $a\colon[0,1]\to X$ is a path, and $b$ a loop based at $a(0)$, then $aba^{-1}=a(b)$ is a path based at $a(1)$.
  Its homotopy class depends only on the homotopy classes of $a$ and $b$.
  This construction defines an isomorphism between $\pi_1(X,a(0))$ and $\pi_1(X,a(1))$.
\end{env}

\begin{proposition}{1.6}
\label{I.1.6}
  Under the hypotheses of \cref{I.1.5}, there exists, up to unique isomorphism, exactly one locally constant sheaf of groups $\Pi_1(X)$ on $X$ (\emph{the fundamental groupoid}), endowed, for all $x_0\in X$, with an isomorphism
  \[
  \label{I.1.6.1}
    \Pi_1(X)_{x_0} \simeq \pi_1(X,x_0)
  \tag{1.6.1}
  \]
  and such that, for every path $a\colon[0,1]\to X$, the isomorphism in \cref{I.1.5} between $\pi_1(X,a(0))$ and $\pi_1(X,a(1))$ can be identified, via \cref{I.1.6.1}, with the isomorphism in \cref{I.1.2} between $\Pi_1(X)_{a(0)}$ and $\Pi_1(X)_{a(1)}$.
\end{proposition}

If $X$ is connected, with base point $x_0$, then the sheaf $\Pi_1(X)$ corresponds, via the equivalence in \cref{I.1.3}, to the group $\pi_1(X,x_0)$ endowed with its action over itself by inner automorphisms.

\begin{proposition}{1.7}
\label{I.1.7}
  If $\sh{F}$ is a locally constant sheaf on $X$, then there exists exactly one action (said to be \emph{canonical}) of $\Pi_1(X)$ on $\sh{F}$ that, at each $x_0\in X$, induces the action from \cref{I.1.2} of $\pi_1(X,x_0)$ on $\sh{F}$.
\end{proposition}


\section{Integrable connections and local systems}
\label{I.2}

\oldpage{5}

\begin{env}{2.1}
  Let $X$ be an analytic space \hyperref[0.1]{(0.1)}.
  We define a (holomorphic) \emph{vector bundle} on $X$ to be a locally free sheaf of modules that is of finite type over the structure sheaf $\sh{O}$ of $X$.
  If $\sh{V}$ is a vector bundle on $X$, and $x$ a point of $X$, then we denote by $\sh{V}_{(x)}$ the free $\sh{O}_{(x)}$-module of finite type of germs of sections of $\sh{V}$.
  If $\mathfrak{m}_x$ is the maximal ideal of $\sh{O}_{(x)}$, then we define the \emph{fibre at $x$ of the vector bundle $\sh{V}$} to be the \todo of finite rank
  \[
  \label{I.2.1.1}
    \sh{V}_x = \sh{V}_{(x)} \otimes_{\sh{O}_{(x)}} \sh{O}_{(x)}/\mathfrak{m}_x.
  \tag{2.1.1}
  \]

  If $f\colon X\to Y$ is a morphism of analytic spaces, then the \emph{inverse image} of a vector bundle $\sh{V}$ on $Y$ is the vector bundle $f^*\sh{V}$ on $X$ given by the inverse image of $\sh{V}$ as a coherent module:
  if $f^\sbullet\sh{V}$ is the sheaf-theoretic inverse image of $\sh{V}$, then
  \[
  \label{I.2.1.2}
    f^*\sh{V} \simeq \sh{O}_X \otimes_{f^*\sh{O}_Y} f^\sbullet\sh{V}
  \tag{2.1.2}
  \]

  In particular, if $x\colon P\to X$ is the morphism from the point space $P$ to $X$ defined by a point $x$ of $X$, then
  \[
    \label{I.2.1.3}
      \sh{V}_x \simeq x^*\sh{V}.
    \tag{2.1.3}
  \]
\end{env}

\begin{env}{2.2}
\label{I.2.2}
  Let $X$ be a complex-analytic manifold \hyperref[0.7]{(0.7)} and $\sh{V}$ a vector bundle on $X$.
  The elders would have defined a (holomorphic) connection on $\sh{V}$ as the data, for every pair of points $(x,y)$ that are first order infinitesimal neighbours in $X$, of an isomorphism $\gamma_{y,x}\colon\sh{V}_x\to\sh{V}_y$ that depends holomorphically on $(x,y)$ and is such that $\gamma_{x,x}=\id$.

  Suitably interpreted, this ``definition'' coincides with the currently fashionable definition (\cref{I.2.2.4}) given below (which we not be use in the rest of the section).

  It suffices to understand ``point'' to mean ``point with values in any analytic space'':

\oldpage{6}
  \begin{env}{2.2.1}
  \label{I.2.2.1}
    \emph{A point in an analytic space $X$ with values in an analytic space $S$} is a morphism from $S$ to $X$.
  \end{env}

  \begin{env}{2.2.2}
  \label{I.2.2.2}
    If $Y$ is a subspace of $X$, then the \emph{$n$th infinitesimal neighbourhood} of $Y$ in $X$ is the subspace of $X$ defined locally by the $(n+1)$th power of the ideal of $\sh{O}_X$ that defines $Y$.
  \end{env}

  \begin{env}{2.2.3}
  \label{I.2.2.3}
    Two points $x,y\in X$ with values in $S$ are said to be \emph{first order infinitesimal neighbours} if the map $(x,y)\colon S\to X\times X$ that they define factors through the first order infinitesimal neighbourhood of the diagonal of $X\times X$.
  \end{env}

  \begin{env}{2.2.4}
  \label{I.2.2.4}
    If $X$ is a complex-analytic manifold and $\sh{V}$ is a vector bundle on $X$, then a (\emph{holomorphic}) \emph{connection} $\gamma$ on $\sh{V}$ consists of the following data:

    for every pair $(x,y)$ of points of $X$ with values in an arbitrary analytic space $S$, with $x$ and $y$ first order infinitesimal neighbours, an isomorphism $\gamma_{x,y}\colon x^*\sh{V}\to y^*\sh{V}$;
    this data is subject to the conditions:
    \begin{enumerate}[(i)]
      \item (functoriality) For any $f\colon T\to S$ and any first order infinitesimal neighbours $x,y\colon S\rightrightarrows X$, we have $f^*(\gamma_{y,x})=\gamma_{yf,xf}$.
      \item We have $\gamma_{x,x}=\id$.
    \end{enumerate}
  \end{env}
\end{env}

\begin{env}{2.3}
\label{I.2.3}
  Let $X_1$ be the first-order infinitesimal neighbourhood of the diagonal $X_0$ of $X\times X$, and let $p_1$ and $p_2$ be the two projections of $X_1$ to $X$.
  By definition, the vector bundle $P^1(\sh{V})$ of first-order jets of sections of $\sh{V}$ is the bundle $(p_1)_*p_2^*\sh{V}$.
  We denote by $j^1$ the first-order differential operator that sends each section of $\sh{V}$ to its first-order jet:
  \[
    j^1\colon \sh{V} \to P^1(\sh{V}) \simeq \sh{O}_{X_1}\otimes_{\sh{O}_X}\sh{V}.
  \]

  A connection (\cref{I.2.2.4}) can be understood as a homomorphism (which is automatically an isomorphism)
  \[
  \label{I.2.3.1}
    \gamma = p_1^*\sh{V} \to p_2^*\sh{V}
  \tag{2.3.1}
  \]
  which induces the identity over $X_0$.
  Since
  \[
    \Hom_{X_1}(p_1^*\sh{V},p_2^*\sh{V}) \simeq \Hom(\sh{V},(p_1)_*p_2^*\sh{V}),
  \]
\oldpage{7}
  a connection can also be understood as a ($\sh{O}$-linear) homomorphism
  \[
  \label{I.2.3.2}
    \DD\colon \sh{V} \to P^1(\sh{V})
  \tag{2.3.2}
  \]
  such that the obvious composite arrow
  \[
    \sh{V}\xrightarrow{\DD} P^1(\sh{V}) \to \sh{V}
  \]
  is the identity.
  The sections $\DD s$ and $j^1(s)$ of $P^1(v)$ thus have the same image in $\sh{V}$, and $j^1(s)-\DD(s)$ can be identified with a section $\nabla s$ of $\Omega_X^1\otimes\sh{V} \simeq \Ker(P^1(\sh{V})\to\sh{V})$:
  \[
  \label{I.2.3.3}
    \nabla\colon \sh{V} \to \Omega^1(X)
  \tag{2.3.3}
  \]
  \[
  \label{I.2.3.4}
    j^1(s) = \DD(s)+\nabla s.
  \tag{2.3.4}
  \]

  In other words, a connection (\cref{I.2.2.4}), allowing us to compare two neighbouring fibres of $\sh{V}$, also allows us to define the differential $\nabla s$ of a section of $\sh{V}$.

  Conversely, equation~\cref{I.2.3.4} allows us to define $\DD$, and thus $\gamma$, from the covariant derivative $\nabla$.
  For $\DD$ to be linear, it is necessary and sufficient for $\nabla$ to satisfy the identity
  \[
  \label{I.2.3.5}
    \nabla(fs) = \dd f\cdot s + f\cdot\nabla s
  \tag{2.3.5}
  \]

  Definition~\cref{I.2.2.4} is thus equivalent to the following definition, due to J.L.~Koszul.
\end{env}

\begin{definition}{2.4}
\label{I.2.4}
  Let $\sh{V}$ be a (holomorphic) vector bundle on a complex-analytic manifold $X$.
  A \emph{holomorphic connection} (or simply, \emph{connection}) on $\sh{V}$ is a $\mathbb{C}$-linear homomorphism
  \[
    \nabla\colon \sh{V} \to \Omega_X^1(\sh{V}) = \Omega_X^1\otimes_{\sh{O}}\sh{V}
  \]
  that satisfies the Leibniz identity (\cref{I.2.3.5}) for local sections $f$ of $\sh{O}$ and $s$ of $\sh{V}$.
  We call $\nabla$ the \emph{covariant derivative} defined by the connection.
\end{definition}

\begin{env}{2.5}
\label{I.2.5}
  If the vector bundle $\sh{V}$ is endowed with a connection $\Gamma$ with covariant derivative $\nabla$, and if $w$ is a holomorphic vector field on $X$, then we set, for every local section $v$ of $\sh{V}$ over an open subset $U$ of $X$,
  \[
    \nabla_w(v) = \langle \nabla v,w \rangle \in \sh{V}(U).
  \]
  We call $\nabla_w\colon \sh{V} \to \sh{V}$ the \emph{covariant derivative along the vector field $w$}.
\end{env}

\oldpage{8}
\begin{env}{2.6}
\label{I.2.6}
  If ${}_1\!\Gamma$ and ${}_2\!\Gamma$ are connections on $X$, with covariant derivatives ${}_1\!\nabla$ and ${}_2\!\nabla$ (respectively), then ${}_2\!\nabla-{}_1\!\nabla$ is a $\sh{O}$-linear homomorphism from $\sh{V}$ to $\Omega_X^1(\sh{V})$.
  Conversely, the sum of ${}_1\!\nabla$ and such a homomorphism defines a connection on $\sh{V}$.
  Thus connections on $\sh{V}$ form a principal homogeneous space (or torsor) on $\iHom(\sh{V},\Omega_X^1(\sh{V})) \simeq \Omega_X^1(\iEnd(\sh{V}))$.
\end{env}

\begin{env}{2.7}
\label{I.2.7}
  If vector bundles are endowed with connections, then every vector bundle obtained by a ``tensor operation'' is again endowed with a connection.
  This is evident with \cref{I.2.2.4}.
  More precisely, let $\sh{V}_1$ and $\sh{V}_2$ be vector bundles endowed with connections with covariant derivatives $\nabla_1$ and $\nabla_2$.

  \begin{env}{2.7.1}
  \label{I.2.7.1}
    We define a connection on $\sh{V}_1\oplus\sh{V}_2$ by the formula
    \[
      \nabla_w(v_1+v_2) = {}_1\!\nabla_w(v_1) + {}_2\!\nabla_w(v_2)
    \]
  \end{env}

  \begin{env}{2.7.2}
  \label{I.2.7.2}
    We define a connection on $\sh{V}_1\otimes\sh{V}_2$ by the Leibniz formula
    \[
      \nabla_w(v_1\otimes v_2) = \nabla_w v_1\cdot v_2 + v_1\cdot\nabla_w v_2.
    \]
  \end{env}

  \begin{env}{2.7.3}
  \label{I.2.7.3}
    We define a connection on $\iHom(\sh{V}_1,\sh{V}_2)$ by the formula
    \[
      (\nabla_w f)(v_1) = {}_2\!\nabla_2(f(v_1)) - f({}_1\!\nabla v_1).
    \]
  \end{env}

  The canonical connection on $\sh{O}$ is the connection for which $\nabla f=\dd f$.
  
  Let $\sh{V}$ be a vector bundle endowed with a connection.
  \begin{env}{2.7.4}
  \label{I.2.7.4}
    We define a connection on the dual $\sh{V}^\vee$ of $\sh{V}$ via \cref{I.2.7.3} and the defining isomorphism $\sh{V}^\vee = \iHom(\sh{V},\sh{O})$.
    We have
    \[
      \langle \nabla_w v',v \rangle = \partial_w\langle v',v \rangle - \langle v',\nabla_w v \rangle.
    \]
  \end{env}

  We leave it to the reader to verify that these formulas do indeed define connections.
  For \cref{I.2.7.2}, for example, one must verify that, firstly, the given formula defines a $\mathbb{C}$-bilinear map from $(\sh{V}_1\otimes\sh{V}_2)$, which means that the right-hand side $II(v_1,v_2)$ is $\mathbb{C}$-bilinear and such that $II(fv_1,v_2)=II(v_1,fv_2)$;
  secondly, one must also verify identity~\cref{I.2.3.5}.
\end{env}

\begin{env}{2.8}
\label{I.2.8}
  An $\sh{O}$-homomorphism $f$ between vector bundles $\sh{V}_1$ and $\sh{V}_2$ endowed with connections
\oldpage{9}
  is said to be \emph{compatible with the connections} if
  \[
    {}_2\!\nabla\cdot f = f\cdot{}_1\!\nabla.
  \]
  By \cref{I.2.7.3}, this reduces to saying that $\nabla f=0$, if $f$ is thought of as a section of $\iHom(\sh{V}_1,\sh{V}_2)$.
  For example, by \cref{I.2.7.3}, the canonical map
  \[
    \Hom(\sh{V}_1,\sh{V}_2)\otimes\sh{V}_1 \to \sh{V}_2
  \]
  is compatible with the connections.
\end{env}


%% Bibliography %%

\nocite{*}

\end{document}

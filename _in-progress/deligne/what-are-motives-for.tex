\documentclass{article}

\title{What use are motives?}
\author{Pierre Deligne}
\date{}

\usepackage{amssymb,amsmath}

\usepackage{hyperref}
\usepackage[nameinlink]{cleveref}
\usepackage{enumerate}

\usepackage{mathrsfs}
%% Fancy fonts --- feel free to remove! %%
\usepackage{Baskervaldx}
\usepackage{mathpazo}


\usepackage{fancyhdr}
\usepackage{lastpage}
\usepackage{xstring}
\makeatletter
\ifx\pdfmdfivesum\undefined
  \let\pdfmdfivesum\mdfivesum
\fi
\edef\filesum{\pdfmdfivesum file {\jobname}}
\pagestyle{fancy}
\makeatletter
\let\runauthor\@author
\let\runtitle\@title
\makeatother
\fancyhf{}
\lhead{\footnotesize\runtitle}
\rhead{\footnotesize Version: \texttt{\StrMid{\filesum}{1}{8}}}
\cfoot{\small\thepage\ of \pageref*{LastPage}}


\crefname{section}{\S\!}{\S\S\!}
\crefname{equation}{}{}


%% Theorem environments %%

\usepackage{amsthm}


%% Shortcuts %%

\newcommand{\sh}{\mathscr}
\newcommand{\cat}{\mathcal}
\newcommand{\pr}{\mathrm{pr}}
\newcommand{\ZZ}{\mathbb{Z}}
\newcommand{\CC}{\mathbb{C}}

\renewcommand{\geq}{\geqslant}
\renewcommand{\leq}{\leqslant}

\DeclareMathOperator{\Pic}{Pic}

\newcommand{\todo}{\textbf{ !TODO! }}
\newcommand{\oldpage}[1]{\marginpar{\footnotesize$\Big\vert$ \textit{p.~#1}}}


%% Document %%

\usepackage{embedall}
\begin{document}

\maketitle
\thispagestyle{fancy}

\renewcommand{\abstractname}{Translator's note.}

\begin{abstract}
  \renewcommand*{\thefootnote}{\fnsymbol{footnote}}
  \emph{This text is one of a series\footnote{\url{https://thosgood.com/translations/}} of translations of various papers into English.}
  \emph{The translator takes full responsibility for any errors introduced in the passage from one language to another, and claims no rights to any of the mathematical content herein.}
  
  \emph{What follows is a translation of the French paper:}

  \medskip\noindent
  \textsc{Deligne, P.}
  ``A quoi servent les motifs?''.
  \emph{Proc. Symp. in Pure Math.} \textbf{55} (1994), 143--161.
  {\footnotesize\url{https://publications.ias.edu/deligne/paper/413}}.
\end{abstract}

\setcounter{footnote}{0}

\tableofcontents
\bigskip


%% Content %%

\oldpage{143}
The first of the ``standard conjectures'' (Grothendieck~\cite{19}, Kleiman~\cite{20}), the Lefschetz-style one, demands that certain cohomology classes be algebraic.
Anyway, if motives are the ``direct factors'' of algebraic varieties $X$ defined by projectors (algebraic cycles on $X\times X$), then their definition is only reasonable if we have enough algebraic cycles.
On this problem --- the construction of interesting algebraic cycles --- progress has been sparse.

Grothendieck tried to establish a catalogue of projective constructions of cycles (\todo), cf. \cite[p.~197]{19}.
For those for which we have calculated their cohomology class, the class can be expressed in terms of Chern classes of obvious vector bundles.
Even though I do not know of any counterexamples, it seems unlikely to me that the ring of cycle classes on $X\times X$ generated by the diagonal, the divisors, and the inverse images of the Chern classes of $X$ by the $\pr_i$ (for $i=1,2$) will always contain the cycles required by the first standard conjecture (for example, the K\"{u}nneth components of the diagonal).

Instead of trying to construct cycles, we can try to construct vector bundles, and then take their Chern classes.
It is, in fact, like this that K.~Kodaira and D.C.~Spencer (1953) prove the Hodge conjecture for divisors (a theorem of Lefschetz), with the group $\Pic(X)=H^1(X,\sh{O}^*)$ being accessible, over $\CC$, by the exponential exact sequence
\[
  0 \to 2\pi i\ZZ \to \sh{O} \to \sh{O}^* \to 0.
\]
Similarly, on an abelian variety, it is easier to write the functional equation of the $\Theta$ functions (a cocycle for a line bundle) than to define a $\Theta$ function, itself giving the divisor $\Theta=0$.

\oldpage{144}
Unfortunately, in higher rank, we do not know how to construct vector bundles whose Chern classes are interesting any more than we know how to construct interesting cycles.

On $\CC$, the Hodge conjecture would give the desired cycles.
Two difficulties in proving this conjecture have been found.
The first: Atiyah and Hirzebruch have shown that it could only be true rationally (cf. Atiyah-Hirzebruch~\cite{1}).

The second: from the point of view of Hodge theory, the class of a cycle $Z$ of codimension $d$ on a smooth projective $X$ lives naturally, not in $H_\ZZ^{d,d}:=H^{d,d}\cap H_\ZZ$, but in an extension $E_d$ of this group by the intermediate Jacobian:
\[
  J^d(X) := H^{2d-1}(X,\CC)/F^d + H^{2d-1}(X,\ZZ),
\]
where $F^d$ denotes the $d$-th term of the Hodge filtration of $H^{2d-1}(X,\CC)$.


%% Bibliography %%

\nocite{*}

\end{document}

\documentclass{article}

\title{The fundamental group of the projective line minus three points}
\author{P. Deligne}
\date{1989}

\usepackage{amssymb,amsmath}

\usepackage{hyperref}
\usepackage[nameinlink]{cleveref}
\usepackage{enumerate}
\usepackage{tikz-cd}
\usepackage{mathtools}

\usepackage{mathrsfs}
%% Fancy fonts --- feel free to remove! %%
\usepackage{Baskervaldx}
\usepackage{mathpazo}


\usepackage{fancyhdr}
\usepackage{lastpage}
\usepackage{xstring}
\makeatletter
\ifx\pdfmdfivesum\undefined
  \let\pdfmdfivesum\mdfivesum
\fi
\edef\filesum{\pdfmdfivesum file {\jobname}}
\pagestyle{fancy}
\makeatletter
\let\runauthor\@author
\let\runtitle\@title
\makeatother
\fancyhf{}
\lhead{\footnotesize\runtitle}
\rhead{\footnotesize Version: \texttt{\StrMid{\filesum}{1}{8}}}
\cfoot{\small\thepage\ of \pageref*{LastPage}}


\crefname{section}{Section}{Sections}
\crefname{equation}{}{}


%% Theorem environments %%

\usepackage{amsthm}


%% Shortcuts %%

\newcommand{\sh}{\mathscr}
\newcommand{\cat}{\mathcal}
\newcommand{\bb}{\mathbb}

\renewcommand{\geq}{\geqslant}
\renewcommand{\leq}{\leqslant}

\DeclareMathOperator{\Gal}{Gal}

\newcommand{\todo}{\textbf{ !TODO! }}
\newcommand{\oldpage}[1]{\marginpar{\footnotesize$\Big\vert$ \textit{p.~#1}}}


%% Document %%

\usepackage{embedall}
\begin{document}

\maketitle
\thispagestyle{fancy}

\renewcommand{\abstractname}{Translator's note.}

\begin{abstract}
  \renewcommand*{\thefootnote}{\fnsymbol{footnote}}
  \emph{This text is one of a series\footnote{\url{https://github.com/thosgood/translations}} of translations of various papers into English.}
  \emph{The translator takes full responsibility for any errors introduced in the passage from one language to another, and claims no rights to any of the mathematical content herein.}
  
  \emph{What follows is a translation of the French paper:}

  \medskip\noindent
  \textsc{Deligne, P.}
  "Le Groupe Fondamental de la Droite Projective Moins Trois Points."
  In \emph{Galois Groups over $\mathbb{Q}$}, Springer-Verlag, Mathematical Sciences Research Institute Publications, Volume~\textbf{16} (1989), pp.~79--297.
  \textsc{DOI:} \href{https://doi.org/10.1007/978-1-4613-9649-9\_3}{10.1007/978-1-4613-9649-9\_3}.
\end{abstract}

\setcounter{footnote}{0}

\tableofcontents
\bigskip


%% Content %%


\section*{Leitfaden}
\[
  \begin{tikzcd}
    1 \ar[dd, no head] & 4 \ar[d, no head] & & 9 \ar[d, no head] &
  \\& 5 \ar[d, no head] & & 10 \ar[dl, no head] \ar[dr, no head] &
  \\2 \ar[dd, no head] \ar[dr, no head] & 6 \ar[d, no head] & 11 \ar[ddr, no head] & & 12 \ar[ddl, no head]
  \\& 7 \ar[d, no head] \ar[drr, no head] & & &
  \\3 \ar[dddddrr, no head] & 8 & & 13 \ar[d, no head] &
  \\ & & & 14 \ar[d, no head] &
  \\ & & & 15 \ar[d, no head] &
  \\ & & & 16 \ar[d, no head] \ar[dr, no head] \ar[ddl, no head] &
  \\ & & & 17 \ar[dl, no head, dashed] & 18
  \\ & & 19 & &
  \end{tikzcd}
\]


\section*{}

The present article owes much to A.~Grothendieck.
He invented the philosophy of motives, which is our guiding thread.
Around five years ago, he also said to me, with force, that the profinite completion $\hat{\pi}_1$ of the fundamental group of $X\coloneqq\bb{P}^1(\bb{C})\setminus\{0,1,\infty\}$, with the action of $\Gal(\overline{\bb{Q}}/\bb{Q})$, is a remarkable object, and that it must be studied.


%% Bibliography %%

\nocite{*}
\bibliographystyle{acm}

\end{document}

\documentclass{article}

\title{What are motives for?}
\author{Pierre Deligne}
\date{}

\usepackage{amssymb,amsmath}

\usepackage{hyperref}
\usepackage[nameinlink]{cleveref}
\usepackage{enumerate}

\usepackage{mathrsfs}
%% Fancy fonts --- feel free to remove! %%
\usepackage{Baskervaldx}
\usepackage{mathpazo}


\usepackage{fancyhdr}
\usepackage{lastpage}
\usepackage{xstring}
\makeatletter
\ifx\pdfmdfivesum\undefined
  \let\pdfmdfivesum\mdfivesum
\fi
\edef\filesum{\pdfmdfivesum file {\jobname}}
\pagestyle{fancy}
\makeatletter
\let\runauthor\@author
\let\runtitle\@title
\makeatother
\fancyhf{}
\lhead{\footnotesize\runtitle}
\rhead{\footnotesize Version: \texttt{\StrMid{\filesum}{1}{8}}}
\cfoot{\small\thepage\ of \pageref*{LastPage}}


\crefname{section}{\S\!}{\S\S\!}
\crefname{equation}{}{}


%% Theorem environments %%

\usepackage{amsthm}

\theoremstyle{plain}

\newtheorem{innercustomnullities}{Vanishings}
\newenvironment{nullities}[1]
  {\renewcommand\theinnercustomnullities{#1}\innercustomnullities}
  {\endinnercustomnullities}

\theoremstyle{definition}

\newtheorem*{example}{Example}
\newtheorem*{remark}{Remark}


%% Shortcuts %%

\newcommand{\sh}{\mathscr}
\newcommand{\cat}{\mathcal}
\newcommand{\pr}{\mathrm{pr}}
\newcommand{\ZZ}{\mathbb{Z}}
\newcommand{\QQ}{\mathbb{Q}}
\newcommand{\CC}{\mathbb{C}}
\newcommand{\RR}{\mathbb{R}}
\renewcommand{\AA}{\mathbb{A}}
\newcommand{\GG}{\mathbb{G}}
\newcommand{\PP}{\mathbb{P}}
\newcommand{\FF}{\mathbb{F}}
\newcommand{\mot}{\mathrm{mot}}
\newcommand{\ab}{\mathrm{ab}}
\newcommand{\abs}{\mathrm{abs}}
\newcommand{\un}{\mathrm{un}}
\newcommand{\GL}{\mathrm{GL}}

\renewcommand{\geq}{\geqslant}
\renewcommand{\leq}{\leqslant}

\DeclareMathOperator{\Pic}{Pic}
\DeclareMathOperator{\Hom}{Hom}
\DeclareMathOperator{\End}{End}
\DeclareMathOperator{\Lie}{Lie}
\DeclareMathOperator{\Ext}{Ext}
\DeclareMathOperator{\real}{real}
\DeclareMathOperator{\Gr}{Gr}
\DeclareMathOperator{\Aut}{Aut}
\DeclareMathOperator{\Rep}{Rep}
\DeclareMathOperator{\ind}{ind}
\DeclareMathOperator{\Spec}{Spec}
\DeclareMathOperator{\Ker}{Ker}
\DeclareMathOperator{\Gal}{Gal}
\DeclareMathOperator{\cl}{cl}
\DeclareMathOperator{\ch}{ch}
\DeclareMathOperator{\Ch}{Ch}

\newcommand{\oldpage}[1]{\marginpar{\footnotesize$\Big\vert$ \textit{p.~#1}}}


%% Document %%

\usepackage{embedall}
\begin{document}

\maketitle
\thispagestyle{fancy}

\renewcommand{\abstractname}{Translator's note.}

\begin{abstract}
  \renewcommand*{\thefootnote}{\fnsymbol{footnote}}
  \emph{This text is one of a series\footnote{\url{https://thosgood.com/translations/}} of translations of various papers into English.}
  \emph{The translator takes full responsibility for any errors introduced in the passage from one language to another, and claims no rights to any of the mathematical content herein.}
  
  \emph{What follows is a translation of the French paper:}

  \medskip\noindent
  \textsc{Deligne, P.}
  ``A quoi servent les motifs?''.
  \emph{Proc. Symp. in Pure Math.} \textbf{55} (1994), 143--161.
  {\footnotesize\url{https://publications.ias.edu/deligne/paper/413}}.
\end{abstract}

\setcounter{footnote}{0}

\setcounter{tocdepth}{1}
\tableofcontents
\bigskip


%% Content %%

\emph{[Trans.] There remain one or two words of which I am uncertain, since I am very ignorant of the language of motivic geometry. I have underlined them and labelled them with a question mark.}
\bigskip

\oldpage{143}
The first of the ``standard conjectures'' (Grothendieck~\cite{19}, Kleiman~\cite{20}), the Lefschetz-style one, says that certain cohomology classes are algebraic.
Anyway, if motives are the ``direct factors'' of algebraic varieties $X$ defined by projectors (algebraic cycles on $X\times X$), then their definition is only reasonable if we have enough algebraic cycles.
On this problem --- the construction of interesting algebraic cycles --- progress has been sparse.

Grothendieck tried to establish a catalogue of projective constructions of cycles, cf. \cite[p.~197]{19}.
For those for which we have calculated their cohomology class, the class can be expressed in terms of Chern classes of obvious vector bundles.
Even though I do not know of any counterexamples, it seems unlikely to me that the ring of cycle classes on $X\times X$ generated by the diagonal, the divisors, and the inverse images of the Chern classes of $X$ by the $\pr_i$ (for $i=1,2$) will always contain the cycles required by the first standard conjecture (for example, the K\"{u}nneth components of the diagonal).

Instead of trying to construct cycles, we can try to construct vector bundles, and then take their Chern classes.
It is, in fact, like this that K.~Kodaira and D.C.~Spencer (1953) prove the Hodge conjecture for divisors (a theorem of Lefschetz), with the group $\Pic(X)=H^1(X,\sh{O}^*)$ being accessible, over $\CC$, by the exponential exact sequence
\[
  0 \to 2\pi i\ZZ \to \sh{O} \to \sh{O}^* \to 0.
\]
Similarly, on an abelian variety, it is easier to write the functional equation of the $\Theta$ functions (a cocycle for a line bundle) than to define a $\Theta$ function, itself giving the divisor $\Theta=0$.

\oldpage{144}
Unfortunately, in higher rank, we do not know how to construct vector bundles whose Chern classes are interesting any more than we know how to construct interesting cycles.

Over $\CC$, the Hodge conjecture would give the desired cycles.
Two difficulties in proving this conjecture have been found.
The first: Atiyah and Hirzebruch have shown that it could only be true rationally (cf. Atiyah-Hirzebruch~\cite{1}).

The second: from the point of view of Hodge theory, the class of a cycle $Z$ of codimension $d$ on a smooth projective $X$ lives naturally, not in $H_\ZZ^{d,d}:=H^{d,d}\cap H_\ZZ$, but in an extension $E_d$ of this group by the intermediate Jacobian:
\[
  J^d(X) := H^{2d-1}(X,\CC)/F^d + H^{2d-1}(X,\ZZ),
\]
where $F^d$ denotes the $d$-th term of the Hodge filtration of $H^{2d-1}(X,\CC)$.
In the category of mixed Hodge structures, this is, respectively, $\Hom(\ZZ(d),H^{2d}(X))$ and $\Ext^1(\ZZ(d),H^{2d-1}(X))$.
The Hodge conjecture says that the group of cycle classes is sent to a subgroup of finite index of the quotient $H_\ZZ^{d,d}$.
However, we have no idea what should be the image in $E_d$.
Furthermore, the aesthetic awkwardness that causes this ignorance renders the methods of P.A.~Griffiths, to prove the Hodge conjecture by induction on the dimension of the variety $X$, generally inapplicable.
This method is inspired by that which Lefschetz~\cite{24} uses for surfaces.
The idea is that, if $(H_t)_{t\in\PP^1}$ is a pencil of hyperplane sections, then constructing the cycle $Z$ on $X$ reduces to constructing the cycles $Z_i=Z\cap H_t$.
In good cases:
\begin{enumerate}[(a)]
  \item the cohomology class $c\in H_\ZZ^{d,d}(X)$ that we wish to be that of a cycle $Z$ restricts to zero on the $H_t$ ;
  \item if $Z$ exists, then its cohomology class $c$ determines the class of the $Z_t$ in the intermediate Jacobians $J^d(H_t)$ ; and
  \item the construction of $Z$ reduces to constructing the cycles $Z_t$ of classes defining a given section of the family of the $J^d(H_t)$.
\end{enumerate}
We can thus conclude that every element of $J^d(H_t)$ is the class of a cycle of codimension $d$ that is cohomologous to zero.
Using this method, Zucker~\cite{32} proves the Hodge conjecture for cubic hypersurfaces in $\PP^5$.
Note, however, that, if every element of $J^d(H_t)$ is the class of a cycle that is cohomologous to zero, then $H^{2d-1}(H_t)$ is of Hodge type $\{(d,d-1),(d-1,d)\}$ (see \cref{1.6}), and the converse is a consequence of the Hodge conjecture applied to the product of $H_t$ with a suitable abelian variety.
Furthermore, this surjectivity implies the existence of a curve $C_t$ and an algebraic cycle $W_t$ on $H_t\times C_t$ that sends $H^1(C_t)$ to $H^{2d-1}(H_t)$.
We can then expect that $H^{2d}(X)$ is controlled by the $H^2$ of the surface fibred over $\PP^1$ with fibres given by the $C_t$, and, for an $H^2$, we can use the Hodge conjecture anyway.

\oldpage{145}
The aim of these notes is to show that, despite this lack of progress on the problem of constructing cycles, the philosophy of motives is a powerful tool.

I thank S.~Bloch for his comments on a first version of these notes.


\section{Motives}
\label{1}

According to what we can and want to do, we have various definitions of motives available to us --- or even none.
We need to distinguish pure motives, typically given by the cohomology of non-singular projective varieties, and mixed motives, where open and singular varieties are allowed.
The notion of a motive over $S$ (a family of motives parametrised by $S$) poses yet more problems.

\subsection{}
\label{1.1}

For a field $k$, we always want the category of motives over $k$ to be a $\QQ$-linear abelian category $\sh{M}(k)$ with finite-dimensional $\Hom$ groups.
In the pure case, we want it to be semi-simple and graded (by weights).
In the mixed case, every motive $M$ should admit a finite increasing filtration $W$, with $\Gr_n^W(M)$ pure of weight~$n$.
Some other essential structures are the following:
\begin{enumerate}[(a)]
  \item Every algebraic variety $X$ on $k$ should have motivic cohomology groups $H_\mot^i(X)$, which are objects of $\sh{M}(k)$.
    In the pure case, we restrict to non-singular projective varieties $X$, and $H_\mot^i(X)$ should then be a pure motive of weight~$i$.
  \item For each of the usual cohomology theories $H$, we should have a ``realisation'' functor $\real$, and isomorphisms
    \[
      H^i(X) = \real H_\mot^i(X).
    \]
  \item There should be a tensor product $\otimes$, with which the realisation functors are compatible.
\end{enumerate}

\subsection{}
\label{1.2}

The tensor product structure described above allows us to apply the theory of Tannakian categories, invented by Grothendieck to study the formalism of motives.
References: Saavedra~\cite{28}, Deligne--Milne~\cite{12}, Deligne~\cite{15}.

For $k$ of characteristic~$0$, this theory says that the category of motives over $k$, with its tensor product, should be equivalent to the category of linear representations of a scheme $G$ of affine groups over $\QQ$.
Note that any two groups $G_1$ and $G_2$ that are inner forms of one another have the same category of representations.
The \emph{motivic Galois group} $G$ is thus not uniquely determined by $\sh{M}(k)$.
The theory tells us that a choice of $G$ and of an equivalence $\sh{M}(k)\simeq\Rep(G)$ is equivalent to the choice of an exact functor to $\QQ$-vector spaces that is compatible with the tensor product of $\sh{M}(k)$: a \emph{fibre functor} $\omega$.
To such an $\omega$ corresponds $G:=\underline{\Aut}^\otimes(\omega)$, the group scheme of $\otimes$-automorphisms of $\omega$.
We can think of $\omega$, or, rather, $\omega\circ H_\mot$, as a cohomology theory with values in $\QQ$-vector spaces.

\oldpage{146}
If we have an embedding of $k$ into $\CC$, then a possible cohomology theory is: ``singular cohomology of the topological space of complex points''.
For $k=\QQ$, another possible choice is ``de Rham cohomology''.

In characteristic~$p$, an example by Serre shows that there cannot be any reasonable cohomology theory with values in $\QQ$-vector spaces: if $E$ is a supersingular elliptic curve, then the algebra $\End(E)\otimes\QQ$ is a quaternion algebra, and thus has no linear representation of dimension~$2$ over $\QQ$.
From the existence of a cohomology theory with values in vector spaces over an extension of $\QQ$ it follows, however, that $\sh{M}(k)$ should be the category of representations of a suitable gerbe.

The Tannakian theory is a \emph{linear} analogue of a \emph{set-theoretic} theory, namely that of the profinite $\pi_1$, also due to Grothendieck (\cite{SGA1}).
The analogy is as follows: if $\sh{R}$ is the category of finite \'{e}tale covers of a connected scheme $S$ (resp. a Tannakian category over $K$), and $\omega$ is a fibre functor, with values in finite sets (resp. in finite-dimensional $K$-vector spaces), then $\sh{R}$ is naturally equivalent to the category of finite $G$-sets (resp. of representations of $G$), where $G$ is the profinite group $\Aut(\omega)$ (resp. the affine group scheme $\underline{\Aut}^\otimes(\omega)$).
From here we get the terminology ``motivic Galois group'', with the $\pi_1$ theory, in the case of spectra of fields, becoming the Galois version.

\subsection{}
\label{1.3}

In general, the motivic Galois group $G$ is enormous.
For example, except for when $k$ is algebraic over a finite field, we expect $G$ to admit $\mathrm{PGL}(2)^{\operatorname{card}(k)}$ as a quotient.
It ``doesn't know'' that motives $M_t$ can form a family that depends algebraically on $t$.

Even though $G$ is enormous, its existence facilitates the handling of ``motivic'' objects.

\begin{example}
  The category $\ind\sh{M}(k)$ of ind-objects of the category $\sh{M}(k)$ of motives over $k$ inherits the tensor product from $\sh{M}(k)$.
  This allows us to define a commutative Hopf algebra in $\ind\sh{M}(k)$ as being an object $H$ endowed with a product $H\otimes H\to H$ and coproduct $H\to H\otimes H$, as well as a unit $1\to H$, counit $H\to 1$, and antipode $H\to H$, satisfying suitable axioms.
  We define the category of \emph{motivic affine group schemes} as being the opposite of the category of commutative Hopf algebras in $\ind\sh{M}(k)$.
  If $\omega$ is a fibre functor, with values in $\QQ$-vector spaces, from the motivic Galois group $G:=\underline{\Aut}^\otimes(\omega)$, then the functor $\omega$ induces an equivalence between $\ind\sh{M}(k)$ and the category of linear representations --- not necessarily of finite dimension --- of $G$, and $H\mapsto\Spec\omega(H)$ is an equivalence between the category of motivic affine group schemes and the category of affine group schemes over $\QQ$ endowed with an action of $G$.
\end{example}

Here are three examples of such objects.

\subsubsection{}
\label{1.3.1}

The motivic version of $\pi_1(X,x)$ made unipotent, cf. Deligne~\cite{14}, especially §§5, 7, and 10 to 13.

\oldpage{147}
\subsubsection{}
\label{1.3.2}

The motivic Galois group: there exists a motivic version $G_\mot$ such that, for every fibre functor $\omega$, the corresponding motivic Galois group is given by the application of $\omega$ to $G_\mot$.
This gives a construction that works in any Tannakian category, cf. Deligne~\cite[\S8]{15}.
In the analogy with the profinite-$\pi_1$ theory, its analogue is the following: for varying $x$, the $\pi_1(X,x)$ form a local system on $X$, or, more precisely, an group object in the category of pro-objects of the category of \'{e}tale covers of $X$.
The $\pi_1$ at $x$ is obtained by taking its fibre at $x$.

\subsubsection{}
\label{1.3.3}

Let $\sh{M}$ be a full subcategory of the category of mixed motives over $k$, stable under $\otimes$, taking duals, and sub-quotients, and let $\sh{M}_\mathrm{pur}$ be the full subcategory of direct sums of pure motives in $\sh{M}$.
We have an exact functor that is compatible with the tensor product, given by $M\mapsto\Gr^W(M)$, from $\sh{M}$ to $\sh{M}_\mathrm{pur}$.
In $\sh{M}_\mathrm{pur}$ there exists a motivic pro-unipotent group scheme $U$, acting functorially on $\Gr^W(M)$ for $M$ in $\sh{M}$, and such that $M\mapsto\Gr^W(M)$ is an equivalence between $\sh{M}$ and the category of representations (in $\sh{M}_\mathrm{pur}$) of $U$.
This is a consequence of Deligne~\cite[8.17]{15}.

We can similarly define and work with motivic schemes, motivic torsors, \ldots{}.


\subsection{}
\label{1.4}

The abelianisation $G^\ab$ of the motivic Galois group $G$ is of a more reasonable size.
It is independent of the chosen realisation functor (and of its existence).
Conjecturally, if $k_1\subset k_2$ are algebraically closed fields, then $G^\ab$ is the same for $k_1$ and $k_2$, and the corresponding subcategory of motives is generated by the $H_\mot^1$ of CM-type abelian varieties.

In characteristic~$0$, if we take ``motives'' to mean absolute Hodge cycles, and we restrict to the subcategory generated by the $H^1$ of abelian varieties, then we know how to calculate $G^\ab$.
For $\sh{M}(\QQ)$, we even know how to calculate the quotient of $G$ by the derived group of $\Ker(G\to\Gal(\overline{\QQ}/\QQ))$.
See Deligne~\cite{13} or the talk by Schappacher at this conference.

For $\sh{M}(\FF_q)$, every motive should be a direct sum of pure motives, and the motivic Galois group should contain an element $F$, the Frobenius, whose powers are dense in $G$.
In particular, $G$ should be commutative.
The conjectural calculation of $\sh{M}(\FF_q)$ was made by Grothendieck (unpublished).
See Langlands--Rapoport~\cite{23} and the talk by Milne at this conference.


\subsection{}
\label{1.5}

One source of \cref{1.1}~(a) and (b) is the following example.
A smooth projective curve $X$ defines its Jacobian via $J(X)=\Pic^0(X)$.
Abelian varieties, up to isogeny, over $k$ form a semi-simple abelian category, with the $\Hom(A,B)$ being finite-dimensional over $\QQ$, and, for every usual cohomology theory $H$, the dual $H_1(X)$ of $H^1(X)$ is given by applying a ``realisation'' functor to $J(X)$.
Note that $H_1(X)$ is also the Tate twist $H^1(X)(1)$ of $H^1(X)$.
For $\ell$-adic cohomology, $\real_\ell(A)$ is the Tate module $V_\ell(A)=T_\ell(A)\otimes\QQ_\ell$.
For $k$ of characteristic~$0$, and
\oldpage{148}
de Rham cohomology, $\real_{\mathrm{DR}}(A)$ is the Lie algebra given by the universal additive extension of $A$.

We thus wish to be able to identify abelian varieties with certain motives of weight $-1$.
For some applications (for example, the study of Shimura varieties), we would be content with even having a category of motives that contains abelian varieties and is stable under $\otimes$.
In characteristic~$0$, the theory of absolute Hodge cycles gives us such a theory (Deligne--Brylinksi~\cite{11}, Deligne--Milne~\cite{12}, or the talk by Panschishkin at this conference).
If we do not restrict ourselves to pure motives, then another key example is that of smooth curves that are not necessarily complete.
More generally, we can consider a smooth projective curve $\overline{X}$, disjoint $S$ and $T$ in $\overline{X}$, and the cohomology $H^1(\overline{X}\setminus S,\operatorname{rel}T)$.
A \emph{$1$-motive} $K^\bullet$ is a complex of group schemes concentrated in degrees $-1$ and $0$, such that, over the algebraic closure, $K^{-1}$ is a free $\ZZ$-module of finite type, and $K^0$ is an extension of an abelian variety by a torus.
Suppose, for simplicity, that $k$ is algebraically closed.
For every usual cohomology theory, the $H^1$ in question, or even its Tate twist $H^1(\overline{X}\setminus S,\operatorname{rel}T)(1)$, is then given by applying a realisation functor to the following $1$-motive.

Let $J_T(\overline{X})$ be the generalised Jacobian classifying invertible sheaves of degree~$0$ over $\overline{X}$ that are trivialised over $T$.
This is an extension of the abelian variety $\Pic^0(\overline{X})$ by the character group torus $\Ker(\ZZ^T\xrightarrow{\Sigma}\ZZ)$.
Each $s\in S$ defines an invertible sheaf $\sh{O}(s)$, trivialised over $T$ and of degree~$1$, whence
\[
\label{1.5.1}
  \Ker(\ZZ^S\to\ZZ) \to J_T(\overline{X}).
\tag{1.5.1}
\]
This is the aforementioned $1$-motive.
See Deligne~\cite[§10]{9}.

\begin{remark}
  For $S,T\neq\varnothing$, the $1$-motive \cref{1.5.1} also determines the category of invertible sheaves on $\overline{X}\setminus S$, trivialised over $T$: a point $x\in J_T(\overline{X})$ defines such an invertible sheaf $\sh{L}_x$, and an isomorphism from $\sh{L}_x$ to $\sh{L}_y$ which is the identity on $T$ is identified with $k\in\Ker(\ZZ^S\to\ZZ)$, with $y-x=\delta k$.
\end{remark}

Again, we wish to be able to identify $1$-motives with certain (mixed) motives.
For some applications (the study at infinity of Shimura varieties), we would be content with even having a category of motives that contains the $1$-motives and is stable under $\otimes$.
In characteristic~$0$, we know how to do this (Brylinksi~\cite{7}).


\subsection{}
\label{1.6}

Abelian varieties have spaces of modules, and these allow us to view certain quotients of symmetric Hermitian spaces by arithmetic groups in an algebraic way.
The case of motives is more complicated.
If $S$ is a scheme over $\CC$, and $M$ is a family of pure motives parametrised by $S$, then $M$ gives a variation of Hodge structures $M_h$ over $S(\CC)$, which is polarised if $M$ is, via Hodge realisation.
From this, we get a map $\varphi$ from $S(\CC)$ to the space $\mathscr{C}$ that classifies polarised Hodge structures with the same Hodge numbers as $M$.

The variation $M_h$ gives, in particular, a local system of complex vector spaces $M_\CC$ endowed with a Hodge filtration $F$ that varies continuously.
\oldpage{149}
The holomorphicity of $F$ and the ``transversality'' condition discovered by Griffiths~\cite{17} (see also Griffiths~\cite{18}) can be expressed as follows.
Let $T$ be the tangent bundle of $S$, where $S$ is thought of as a $C^\infty$ manifold.
The complex structure of $S$ endows $T$ with a complex structure.
Let $\varphi\colon T\otimes_\RR\CC\to T$ be the $\CC$-linear extension of the identity on $T$, and $T''\subset T\otimes_\RR\CC$ be the kernel of $\varphi$.
We define a Hodge filtration on $T\otimes\CC$ by
\[
  F^{-1}=T\otimes\CC,
  \quad
  F^0=T'',
  \quad
  F^1=0.
\]
If $t$ and $m$ are $C^\infty$ sections of $T\otimes\CC$ and $M_\CC$ (respectively), then the flat structure of $M_\CC$ allows us to define $\nabla_t m$.
Then the condition is that $(t,m)\mapsto\nabla_t m$ be compatible with the Hodge filtrations:
\[
  \begin{array}{llll}
    \mbox{$t$ in $T''$,}
    & \mbox{$m$ in $F^i$,}
    & \mbox{$\nabla_t m$ in $F^i$:}
    & \mbox{holomorphicity;}
  \\\mbox{$t$ in $T$,}
    & \mbox{$m$ in $F^i$,}
    & \mbox{$\nabla_t m$ in $F^{i-1}$:}
    & \mbox{transversality.}
  \end{array}
\]

There exists a complex structure on the classifying space $\mathscr{C}$, and a distribution $\tau\subset T$ such that the holomorphicity and transversality conditions become holomorphicity of the classifying map $\varphi\colon S(\CC)\to\mathscr{C}$ and tangency of $\varphi(S)$ to $\tau$.

The distribution $\tau$ is, in general, not integrable.
The set of points of $\mathscr{C}$ that corresponds to a direct factor some $H^i(X)$, where $X$ is non-singular and projective, is the countable union of the subvarieties that are tangent to $\tau$.
When $\tau$ is not equal to the entire tangent bundle, then we have no description of it, even conjecturally.

An analogous argument, also due to Griffiths, explains why the points of the intermediate Jacobian $J^d(X)$ can not all be algebraic cycle classes, except for when $H^{2d-1}(X)$ is of type $\{(d-1,d)\},(d,d-1)$.
Let $f\colon X\to S$ be a family of non-singular projective varieties.
Set $X_s=f^{-1}(s)$.
The intermediate Jacobians
\[
  J^d(X_s) = H^{2d-1}(X_s,\CC)/H^{2d-1}(X_s,\ZZ) + F^d
\]
form a holomorphic bundle $J^d$ of complex toruses, and the transversality axiom ensures that the Gauss--Manin connection (expressing that the $H^{2d-1}(X_s,\CC)$ form a local system) passes to the quotient to define a differential operator $D$, defined on the sheaf of sections of $J^d$ and with values in the sheaf of $1$-forms with values in the holomorphic bundle of the $H^{2d-1}(X_s,\CC)/F^{d-1}$.
An algebraic family $(Z_s)_{s\in S}$ of algebraic cycles that are cohomologous to zero on the $X_s$ defines a holomorphic section $z$ of $J^d$.
The result of Griffiths is that $Dz=0$.
The section $z$ defines a holomorphic family of mixed Hodge structures that is an extension of $\ZZ$ (of type $(0,0)$) by $H^{2d-1}(X)(d)$, and $Dz=0$ is equivalent to saying that this family satisfies the transversality axiom.


\subsection{}
\label{1.7}

When the distribution $\tau$ is the entire tangent bundle, $\mathscr{C}$ is an arithmetic quotient of a symmetric Hermitian domain.
This description of complex points of Shimura varieties as modules of Hodge structures suggests that these varieties are spaces of modules of motives.

\oldpage{150}
Let $S$ be the algebraic $\RR$-group $\CC^*$, i.e. $S=R_{\CC/\RR}(\GG_m)$.
Giving an action of $S$ on the real vector space $V$ is equivalent to giving a decomposition of $V\otimes\CC$ as a direct sum of $V^{p,q}$ with $\overline{V^{p,q}}=V^{q,p}$, with $z\in S(\RR)=\CC^*$ acting on $V^{p,q}$ via multiplication by $z^{-p}\overline{z}^{-q}$.
We define $w\colon\GG_m\to S$ (resp. $\mu\colon\GG_m\to S$, defined over $\CC$) by the condition that, for all $V$, $w(\lambda)$ (resp. $\mu(\lambda)$) acts on $V^{p,q}$ via multiplication by $\lambda^{p+q}$ (resp. $\lambda^p$).

The data defining a Shimura variety $\mathrm{Sh}_K(G,X)$ consists of a reductive group $G$ over $\QQ$, a $G(\RR)$-conjugation class of morphisms $h\colon S\to G_\RR$, with $w_X:= h\circ w$ central (and thus independent of $h$), and a compact open subgroup $K$ of $G(\AA^f)$.
We consider the case where $w_X$ is defined over $\QQ$, and where $\operatorname{int}h(i)$ is a Cartan involution of the quotient $G/w_X(\GG_m)$.
The \emph{dual field} $E(G,X)$ is the field of definition of the conjugacy class of $h\mu$ (for arbitrary $h$ in $X$).
This is a subfield of $\CC$.
The Shimura variety is defined over $E(G,X)$, with complex points $K\setminus X\times G(\AA^f)/G(\QQ)$.
Its construction in the general case is due to Borovoi~\cite{6}.

A point of the Shimura variety over a field $\FF$ containing $E(G,X)$ should correspond to
\begin{enumerate}[(a)]
  \item an exact $\otimes$-functor $x$ from the category $\Rep(G)$ of representations of $G$ to the category of pure motives over $\FF$ ; and
  \item an integer structure.
    In terms of finite adelic realisations (the restriction of the product of $\ell$-adic realisations), we can describe this as an isomorphism of $\otimes$-functors
    \[
      x(V)_{\AA^f} \xrightarrow{\sim} V\otimes\AA^f,
    \]
    given up to composition with an element of $K$.
\end{enumerate}
The following condition should be satisfied.
For simplicity, we suppose that $\FF$ can be embedded in $\CC$.
\begin{enumerate}[(a)]
\setcounter{enumi}{2}
  \item Let $\iota$ be an embedding of $\FF$ into $\CC$, extending the identity embedding of $E(G,X)$ into $\CC$.
    There should exist $h\in X$ such that the following $\otimes$-functors from $\Rep(G)$ to Hodge structures are isomorphic:
    \begin{enumerate}[(1)]
      \item $(V,\rho)\mapsto V$, endowed with the Hodge structure defined by $\rho\circ h$;
      \item $(V,\rho)\mapsto$ the Hodge realisation of $x(V)$, after extending the base field to $\CC$ by $\iota$.
    \end{enumerate}
\end{enumerate}
The fact that condition~(c) is independent of the chosen complex embedding $\iota$ --- assumed to be an extension of the inclusion of $E(G,X)$ into $\CC$ --- is not obvious.
Sometimes, however, it follows from (b) and more algebraic conditions (d) and (e), consequences of (c), that follow.
\begin{enumerate}[(a)]
\setcounter{enumi}{3}
  \item The de Rham realisation defines a fibre functor $V\mapsto x(V)_{\mathrm{DR}}$ on $\Rep(G)$ that corresponds to a $G$-torsor $P$ over $F$.
    The Hodge filtration of the $X(V)_{\mathrm{DR}}$ is exact and compatible with the tensor product,
\oldpage{151}
    and thus comes from a parabolic $Q$ of $G^P$ and from $\mu_{\mathrm{DR}}\colon\GG_m\to Z(Q/\mathscr{R}_uQ)$, where $Z$ denotes the centre, that lifts to a conjugation class of morphisms from $\GG_m$ to $Q$, Saavedra~\cite[IV, 2.4, p.~229]{28}.
    Since $\FF$ contains $E(G,X)$, it makes sense to ask for the conjugation class corresponding to maps from $\GG_m$ to $G$ to coincide with that of $h\circ\mu$ (for $h\in X$).
    Condition~(c) implies this.
    This explains the appearance of the dual field.
  \item For a representation $(V,\rho)$ of weight $0$: $\rho\circ w$ trivial and $V\otimes V\to\QQ$ a symmetric invariant bilinear form such that, on $V_\RR$, $B(v,h(i)w)$ is positive-definite and symmetric, we ask for $x(V)\otimes x(V)\to 1$ to be positive, for the desired polarisation (loc. cit. V, 2.4, p.~276) of the category of motives.
\end{enumerate}


\subsection{}
\label{1.8}

This motivic interpretation of Shimura varieties has been a guide for how to elaborate the axioms that characterise them, as well as for the determination of their conjugates (Borovo\v{i}~\cite{6}).

Let $p$ be a prime number such that $G_{\QQ_p}$ extends to a reductive group over $\ZZ_p$, and let $K$ be the product of $G(\ZZ_p)$ with a subgroup of the restriction of the product of the $G(\QQ_\ell)$, for $\ell\neq p$.
The dual field $E(G,X)$ is then unramified at $p$.
We hope that the Shimura variety has a natural reduction $\mod p$, and that we can try to paraphrase the conjectural motivic description above in order to conjecture the structure of the set of its points over a finite field of characteristic~$p$.
There have been difficulties: how can we interpret (c) and the $p$-part of (b) to work together with (d)?
For progress in this direction, see Langlands--Rapoport~\cite{23}.


\section{Cohomology theories}
\label{2}

\subsection{}
\label{2.1}

A geometric construction, possible in one of the usual cohomology theories, should make sense ``motivically'', and thus have an analogue in the other usual theories.

This principal has been crucial in developing mixed Hodge theory.
Grothendieck saw that each $H^i(X)$ should split as subquotients of cohomology of non-singular projective varieties.
The mixed motive $H_\mot^i(X)$ should thus have a class in the Grothendieck group of pure motives.
In characteristic~$0$, the Hodge number $h^{pq}$ of the Hodge realisation of a pure motive defines a homomorphism from this Grothendieck group to $\ZZ$.
The Hodge numbers $h^{pq}(H_\mot^i(X))$ should thus have some meaning.

To go any further, we must convince ourselves that every motive has a weight filtration $W$, that is increasing, and with $\Gr_i^W(M)$ pure of weight~$i$ (= a direct factor of $H_\mot^i(X)$ for non-singular projective $X$).
Is is for the $H^1$ of curves, i.e. for the $1$-motives, that I am convinced of this fact, and the first test that the definition of mixed Hodge structures had to pass was that they recover $1$-motives over $\CC$ as a particular case (Deligne~\cite[§10]{9}).

\oldpage{152}
The same principle of transfer has allowed us to conjecture the asymptotic behaviour of a variation of Hodge structures on a punctured disc, or a product of punctured discs.

\bigskip

\textbf{Warning.}
Let $(S,\eta,s)$ be a line, with $S$ the spectrum of a complete discrete valuation ring.
By Raynaud, an abelian variety on the generic point $\eta$, with semi-stable reduction, admits a rigid-analytic description as the cokernel of an arrow defining a $1$-motive.
At the same time, in Hodge theory, if $\sh{H}$ is a variation of polarised Hodge structures of type $\{(1,0),(0,1)\}$ on a punctured disk $D^*$, then the monodromy endows $\sh{H}_0$ with a weight filtration $W$ that, combined with the original Hodge filtration, makes $\sh{H}$ into a variation of mixed Hodge structures on a punctured neighbourhood of $0$.

We should not hope for an analogous behaviour for motives, since the transversality condition is non-trivial in Hodge theory.
In general, an object that describes the asymptotic behaviour should exist only on the punctured Zariski tangent space (assuming a semi-stable reduction), and it should not be enough to allow us to reconstruct the object we started with.
For example, if $\sh{H}$ is a variation of polarised Hodge structures on $D^*=D\setminus\{0\}$, then the asymptotic nilpotent orbit due to W.~Schmid is a variation of mixed Hodge structures on the punctured (at $0$) tangent space, and, if $\sh{H}$ is extended to $D$, then it is the constant variation with value being the fibre at $0$ of the extension.

The theory of Morihiko Saito (\cite{29}), which gives the six operations (and the evanescent cycles) in mixed Hodge theory is in part inspired by the $\ell$-adic theory and by the motivic point of view.
Conversely, it suggests that, if we wish to consider motives over a base $S$, then instead of wanting to have motives $M$ with $\ell$-adic realisations of $\ell$-adic sheaves on $S$, it might be preferable to have motives with realisations in perverse sheaves.
In any case, it is only in such a framework that we might hope to have a weight filtration.

It is again the philosophy of motives that led Grothendieck to conjecture the existence of the ``mysterious'' functor linking the $p$-adic \'{e}tale cohomology of a variety defined over $\QQ_p$, assumed to be of good reduction, and its de Rham cohomology:
since it exists for $H^1$, i.e. for motives which are the abelian varieties, it should always exist, in a way that is compatible with the tensor product.


\subsection{}
\label{2.2}

Let $X_0$ be an algebraic variety over $k$, and let $X$ be given by extension of scalars of $X_0$ by an algebraic closure $\overline{k}$ of $k$.
For $x\in X_0(k)$, the profinite fundamental group $\pi_1(X,x)$ is endowed with an action of $\Gal(\overline{k}/k)$.
Suppose, for simplicity, that $X$ is normal.
The functor ``the fibre at $x$'' then gives an equivalence of categories:
\[
  \begin{gathered}
  \mbox{smooth $\QQ_\ell$-sheaves on $X_0$}
\\\downarrow
\\\mbox{$\ell$-adic representations of $\Gal(\overline{k}/k)$ endowed with an equivariant action of $\pi_1(X,x)$.}
  \end{gathered}
\]

\oldpage{153}
I do not know in which sense $\pi_1(X,x)$ is motivic.
The situation changes for its pro-$\ell$-completion $\pi_1(X,x)_\ell^\wedge$.
The abelianisation of this group is $H^1(X,\ZZ_\ell)$, and its $H_2$ is contained in that of $X$.
There should exist a motivic group scheme $\pi_1(X,x)_\un$ over $k$ such that
\begin{enumerate}[(a)]
  \item For every usual cohomology theory $H$, the corresponding realisation of $\pi_1(X,x)_\un$ over $\overline{k}$ has ``local systems'' (in the sense of $H$) on $X$ as its linear representations, which are unipotent.
  \item The functor ``the fibre at $x$'' gives an equivalence of categories:
    \[
      \begin{gathered}
        \mbox{smooth motives over $X_0$ that are iterated extensions of \underline{universe (?)} images of motives over $k$}
      \\\mbox{of motives over $k$}
      \\\downarrow
      \\\mbox{motives over $k$ endowed with an action of $\pi_1(X,x)_\un$.}
      \end{gathered}
    \]
\end{enumerate}

In (a), the local systems in question are not assumed to be motivic.
By (b), those that are motivic, however, form a faithful system of representations.

This philosophy has been inspired by the Hodge theory of $\pi_1$: see Steenbrink--Zucker~\cite{30}.

More generally, suppose that we have a set $A$ of pure motives over $X_0$.
The above corresponds to $A=\varnothing$, or, equivalently, $A$ consisting of just the unit motive.
There should exist a motivic group scheme $\pi_1(X,x)_A$ over $k$, where the $\pi_1$ is motivic relative to $A$, such that
\begin{enumerate}[(a')]
  \item Over $\overline{k}$, its realisation relative to the theory $H$ has ``local systems'' on $X$ that are iterated extensions of subquotients of tensor products of realisations of objects of $A$ and their duals as its representations.
  \item The functor ``the fibre at $x$'' gives an equivalence of categories:
    \[
      \begin{gathered}
        \mbox{smooth motives over $X_0$ that are tensor products of objects of $A$,}
      \\\mbox{of their duals, and of inverse images of motives over $k$}
      \\\downarrow
      \\\mbox{motives over $k$ endowed with an action of $\pi_1(X,x)_A$.}
      \end{gathered}
    \]
\end{enumerate}

In (a'), the subquotients are taken over $X$ and are not assumed to be motivic.

In Hodge theory, this suggests the following questions.
Let $X$ be smooth and connected over $\CC$, and let $x\in X$.
Let $A$ be a set of variations of polarisable Hodge structures on $X$.
Let $\sh{A}$ be the Tannakian category of local $\QQ$-systems on $X$ generated (by $\otimes$, dual, subquotients, and extensions) by the underlying local systems of the objects of $A$.
Let $\sh{A}_H$ be the Tannakian category of variations of mixed Hodge structures on $X$ that are admissible (in the sense of Steenbrink--Zucker~\cite{30}) and generated (by $\otimes$, dual, subquotients, and extensions) by $A$ and the constant variations of the associated graded algebra by polarisable weights \underline{(?)}.

On $\sh{A}$, we have the fibre functor ``the fibre at $x$'', denoted by $\omega_x$.
Let $G=\Aut^\otimes(\omega_x)$.
This is the Zariski closure of $\pi_1(X,x)$ in the product of the $\GL(H_x)$, where $H$ is an object of $\sh{A}$.
Let $\sh{H}(G)$ be the Hopf algebra of which $G$ is the spectrum.

\oldpage{154}
For $M$ in $\sh{A}_H$, the underlying local system $M_Q$ is in $\sh{A}$, so that $G$ acts on its fibre $(M_Q)_x$ at $x$: this fibre is an $\sh{H}(G)$-comodule:\footnote{\emph{[Trans.] In the original, this equation is labelled (2.2.21), but it seems like it should instead be (2.2.1).}}
\[
\label{2.2.1}
  (M_Q)_x \to (M_Q)_x\otimes\sh{H}(G).
\tag{2.2.1}
\]

The Hopf algebra $\sh{H}(G)$ should admit the structure of a mixed Hodge ind-structure, i.e. $G$ should be an affine group scheme in the Tannakian category of mixed Hodge structures, in such a way that
\begin{enumerate}[(a)]
  \item the arrows in \cref{2.2.1} are mixed Hodge morphisms ; and
  \item the functor $M\mapsto M_x$ gives an equivalence between $\sh{A}$ and the category of mixed Hodge structures $H$ with polarisable associated graded algebras that are endowed with a comodule structure $H\to H\otimes\sh{H}(G)$ that is a mixed Hodge morphism.
\end{enumerate}


\section{Absolute cohomology}
\label{3}

\subsection{}
\label{3.1}

In each of the usual cohomology theories, the functors $H^i(X)$ come from a functor $R\Gamma(X)$ with values in a triangulated category $\sh{D}$.
This is endowed with a $t$-structure (the data of subcategories $\sh{D}^{\leq0}$ and $\sh{D}^{\geq0}$ satisfying suitable axioms: Beilinson--Bernstein--Deligne~\cite[1.3]{3}), with its heart (the abelian subcategory given by the intersection of $\sh{D}^{\leq0}$ and $\sh{D}^{\geq0}$) being the category in which $H^\bullet$ takes its values, and
\[
  H^i(X) = H^iR\Gamma(X)
\]
in the sense of this $t$-structure.

We might hope that these functors $R\Gamma$ are the ``realisations'' of a motivic functor $R\Gamma$:
we wish for the existence of a triangulated category $\sh{D}(k)$, with a $t$-structure whose heart is the category $\sh{M}(k)$ of mixed motives over $k$, as well as a functor $R\Gamma_\mot$ with values in $\sh{D}(k)$, which recover the $R\Gamma$ in the various theories by applying realisation functors.

A triangulated category $\sh{D}$ with a $t$-structure is not always the derived category of its heart $\sh{C}$.
For example: let $X$ be a finite connected CW-complex, and let $\sh{D}\subset D^b(X,\QQ)$ be the derived subcategory of the derived category consisting of complexes of sheaves with locally constant cohomology.
Its heart is the abelian category of locally constant sheaves of $\QQ$-vector spaces over $X$.
We have a natural functor $D^b(\sh{C})\to\sh{D}$, but this functor is only an equivalence if $X$ is a $K(\pi,1)$.

In this section, we assume the existence of $\sh{D}(k)$ and of the aforementioned ``realisation'' functors.
We also assume the existence of a ``Tate twist'' in $\sh{D}(k)$, which is an auto-equivalence of the category, and corresponds to the usual Tate twist via the realisation functors.
Finally, we assume that the category $\sh{M}(k)$ is Tannakian, and that the $\ell$-adic cohomology if a fibre functor.
This implies that, if $\varphi\colon M_1\to M_2$ is a morphism of motives whose $\ell$-adic realisation is an isomorphism, then $\varphi$ is an isomorphism.

For $M_1$ and $M_2$ in $\sh{M}(k)$, set
\[
  \Ext^i(M_1,M_2) = \Hom_{\sh{D}(k)}(M_1,M_2[i]).
\]
\oldpage{155}
These are the usual $\Hom$ and $\Ext$ for $i=0,1$, but, for $i>1$, they can differ from the Yoneda $\Ext$, which are the $\Hom(M_1,M_2[i])$ in $D^b(\sh{M}(k))$.

For each of the usual cohomology theories, these motivic $\Ext^i$ should be sent to the $\Ext^i$ of the corresponding realisations of $M_1$ and $M_2$.


\subsection{}
\label{3.2}

If $X$ is an algebraic variety over $k$< then we define the \emph{absolute cohomology groups} of $X$ to be the
\[
  H^i_\abs(X) := \Hom_{\sh{D}(k)}(1,R\Gamma_\mot(X)[i]),
\]
where $1$ is the unit motive.
We will also consider the
\[
  H^i_\abs(X,\QQ(j)) := \Hom_{\sh{D}(k)}(1,R\Gamma_\mot(X)(j)[i]).
\]

These groups are often called the \emph{motivic cohomology groups}.
They are vector spaces over $\QQ$.
They are not to be confused with the $H^i_\mot(X)$, which are motives over $k$.
The two are linked by a spectral sequence:
\[
\label{3.2.1}
  E_2^{pq} = \Ext^p(1,H_\mot^q(X)) \Rightarrow H_\abs^{p+q}(X).
\tag{3.2.1}
\]
If $M$ is a motive, then we write $H^p_\abs(M)$ to mean $\Ext^p(1,M)=\Hom_{\sh{D}(k)}(1,M[p])$.
With this notation, the spectral sequence \cref{3.2.1} can be written as
\[
\label{3.2.2}
  E_2^{pq} = H_\abs^p(H_\mot^q(X)) \Rightarrow H_\abs^{p+q}(X).
\tag{3.2.2}
\]

The $\ell$-adic analogue is the following:
a variety $X$ over $k$ (with algebraic closure $\overline{k}$) defines $\ell$-adic cohomology groups $H^i(X\otimes_k\overline{k},\QQ_\ell)$, which we wish to be realisations of motives $H_\mot^i(X)$.
The variety $X$ also has an absolute \'{e}tale cohomology $H^i(X,\QQ_\ell)$, which is the abutment of a spectral sequence:
\[
  H^p(\Gal(\overline{k}/k), H^q(X\otimes_k\overline{k},\QQ_\ell)) \Rightarrow H^i(X,\QQ_\ell).
\]

If $X$ is projective and smooth over $k$, then we wish for $R\Gamma_\mot(X)$ to be (non-canonically) isomorphic to the direct sum in $\sh{D}(X)$ of the $H_\mot^i(X)[-i]$.
This would follow from the following hypothesis:
\begin{quote}
  \textbf{($*$)}
\itshape
  If $\sh{L}$ is an invertible sheaf on $X$, then we have
  \[c_1(\sh{L})\colon R\Gamma_\mot(X)\to R\Gamma_\mot(X)(1)[2]\]
  which, after taking the $\ell$-adic realisation, gives the cup product with the Chern class $c_1(\sh{L})$ of
  \[\sh{L}\colon H^i(X\otimes_k\overline{k},\QQ_\ell) \to H^{i+2}(X\otimes_k\overline{k},\QQ_\ell(1)).\]
\end{quote}

In fact, take $\sh{L}$ to be ample.
Since the $\ell$-adic realisation functor is a fibre functor, it follows from ($*$) and the difficult Lefschetz theorem that, for $X$ of pure dimension $N$, the
\[
  c_1(\sh{L})^i\colon H_\mot^{N-i}(X) \to H_\mot^{N+i}(X)(i)
\]
are isomorphisms.
We can then apply Deligne~\cite{8}.

\oldpage{156}
Analysing the proof, we can thus obtain, for every ample invertible sheaf $\sh{L}$, multiple canonical decompositions:
\[
  \varphi_\sh{L}\colon R\Gamma_\mot(X) \leftrightarrow \bigoplus H_\mot^i(X)[-i].
\]
See Deligne, ``D\'{e}ocmpositions dans la cat\'{e}gorie d\'{e}riv\'{e}e'', in this volume.

\bigskip

\textbf{Warning.}
These arguments are essentially rational, since they depend on the difficult Lefschetz theorem.
If we had an integer variety of $\sh{D}(X)$, then we should expect for $R\Gamma(X)$ in this setting to not always be a sum of the $H^i(X)[-i]$.


\subsection{}
\label{3.3}

Let $X$ be smooth over $k$.
The various ``classes'' of an algebraic cycle $Z$ of codimension~$d$ should come from a motivic class
\[
\label{3.3.1}
  \cl(Z) \in \Hom(1,R\Gamma_\mot(X)[2d](d)) =: H_\abs^{2d}(X,\QQ(d)).
\tag{3.3.1}
\]
The class in \cref{3.3.1} should depend only on the linear equivalence class of $Z$.

If $X$ is smooth and projective, and if we have chosen a decomposition
\[
  R\Gamma_\mot(X) \simeq \bigoplus H_\mot^i(X)[-i]
\]
(cf. \cref{3.2}), then the class in \cref{3.3.1} is equivalent to a series of classes
\[
\label{3.3.2}
  \cl_n(Z) \in \Ext^n(1,H_\mot^{2d-n}(X)(d)) = H_\abs^n(H_\mot^{2d-n}(X)(d)).
\tag{3.3.2}
\]
The first of these classes that is non-zero does not depend on the chosen decomposition.


\subsection{}
\label{3.4}

\textbf{Example.}
In Hodge theory, $Z$ defines first of all an integer cohomology class of type $(d,d)$, i.e. a morphism of Hodge structures
\[
  \ZZ \to H^{2d}(X)(d).
\]
If this class is zero, then $Z$ has a class in the intermediate Jacobian:
\[
  J_d(X) = H^{2d-1}(X,\CC)/F^d\oplus H^{2d-1}(X,\ZZ),
\]
which is exactly the group
\[
  \Ext^1(\ZZ,H^{2d-1}(X)(d))
\]
in the category of mixed Hodge structures.
The story ends here, since the abelian category of mixed Hodge $\QQ$-structures is of cohomological dimension~$1$.


\subsection{}
\label{3.5}

\textbf{Example.}
Let $X$ be smooth and projective, and of pure dimension~$N$ over an algebraically closed field $k$.
Let $\sh{O}(1)$ be an ample invertible sheaf.
We will show how we can associate a class $\cl(D)$ in $\Pic^0(X)\otimes\QQ$ to a divisor $D$, where the theory of motives interprets this Picard group as $H_\abs^1(X)(1)$.

\oldpage{157}
By a theorem of A.~Weil~\cite[Theorem~7]{31}, the map
\[
  E\mapsto E\cdot c_1(\sh{O}(1))^{N-1}
\]
from linear equivalence classes of divisors to linear equivalence classes of $0$-cycles induces an isogeny
\[
  \varphi\colon \Pic^0(X) \to J(X),
\]
and thus a bijection from $\Pic^0(X)\otimes\QQ$ to $J(X)\otimes\QQ$.
We set
\[
  \cl_1(D) = \varphi^{-1}\cl
  \left(
    D\cdot c_1(\sh{O}(1))^{N-1}
    - \frac{\deg(D\cdot c_1(\sh{O}(1))^{N-1})}{\deg(c_1(\sh{O}(1))^N)}
    \cdot c_1(\sh{O}(1))^N
  \right).
\]


\subsection{}
\label{3.6}

\textbf{Example.}
We now work, not over the spectrum of a field, but instead over a base $S$ which is smooth over an algebraically closed field $k$.
Let $f\colon X\to S$ be smooth and projective over $S$, and let $Z$ be a cycle of codimension~$d$.
We work in $\ell$-adic cohomology, with $\ell$ invertible over $S$.
The cycle $Z$ has a class in
\[
  H^{2d}(X,\ZZ_\ell(d)) = \Hom_{D(S)}(\ZZ_\ell, Rf_*\ZZ_\ell[2d](d)).
\]
If we choose a decomposition
\[
\label{3.6.1}
  Rf_*\QQ_\ell \simeq \sum R^if_*\QQ_\ell[-i],
\tag{3.6.1}
\]
then this class gives rise to a series of classes:
\[
\label{3.6.2}
  \cl_n(Z) \in \Ext^n(\ZZ,R^{2d-n}f_*\QQ_\ell(d)) = H^n(S,R^{2d-n}f_*\QQ_\ell(d)).
\tag{3.6.2}
\]

If $X$ is of the form $X_0\times_{\Spec(k)}S$, and if the decomposition \cref{3.6.1} comes from the unique analogous decomposition of the cohomology of $X_0$, then these classes can be identified with the K\"{u}nneth components of the cohomology class of $Z$.


\subsection{}
\label{3.7}

Suppose that $X$ is smooth over a field.
Decompose the K-theory group $K_n(X)\otimes\QQ$ by the eigenvalues of the Adams operations:
\[
  K_n(X)\otimes\QQ = \bigoplus K_n(X)^{(j)}
\]
where, on $K_n(X)^{(j)}$, the Adams operation $\Psi_k$ acts by multiplication by $k^j$.
Recall that $K_0(X)^{(j)}$ is the Chow group of cycles of codimension~$j$, modulo rational equivalence, tensored with $\QQ$.

The example of the usual cohomology theories leads to us hope that the Chern classes
\[
  \ch^j\colon K_n(X)\otimes\QQ \to H_\abs^{2j-n}(X,\QQ(j))
\]
will factor through $K_n(X)^{(j)}$.

An optimistic conjecture is that the ``Chern class'' morphisms are isomorphisms:
\[
\label{3.7.1}
  \ch^j\colon K_n(x)^{(j)} \xrightarrow{\sim} H_\abs^{2j-n}(X,\QQ(j)) := \Hom(1,R\Gamma(X)(j)[2j-n]).
\tag{3.7.1}
\]

\oldpage{158}
This conjecture underlies the terminology ``absolute motivic cohomology group'' for the $K_n(X)^{(j)}$, as well as the notation
\[
  H_\sh{M}^{2j-n}(X,\QQ(j)) := K_n(X)^{(j)}.
\]

Block~\cite{4} proposed an interpretation of the $K_n(X)^{(j)}$ as ``higher Chow groups'' $\Ch^j(X;n)$.
His argument has a hole (moving lemma insufficient) that he promises to fix.
This interpretation makes the definition of $\ch^j$ more natural, but the interpretation in terms of K-theory remains necessary for explicit calculation in the case of number fields.
The higher Chow groups have the advantage of having an natural integer structure.


\subsection{}
\label{3.8}

Thanks to Beilinson, the conjecture \cref{3.7.1} can be deduced from certain hypotheses, of which the most important are
\begin{enumerate}[(a)]
  \item the functor $X\mapsto K_\bullet(X)$ factors through $R\Gamma_\mot$ ; and
  \item the category $\sh{D}(k)$ is the derived category of the category of motives.
\end{enumerate}
The conjecture has striking consequences for the two sides of \cref{3.7.1}.
They disappear, for trivial reasons, in different regions of the $(n,j)$-plane:
\begin{enumerate}[(a)]
  \item If $M$ is a motive, then $\Ext^i(1,M)=0$ for $i<0$, and $\Ext^0(1,M)=\Hom(1,M)$.
    From this, it follows that $\Hom(1,R\Gamma(X)(j)[N])=0$ for $N<0$, or for $N=0$ and $j\neq0$.
    In K-theory, this gives the conjecture by Beilinson and Soul\'{e} that $K_n(X)^{(j)}=0$ for $2j<n$, as well as for $2j\leq n$ if $j\neq0$.
  \item The Chow group $\Ch^j(X)$ of algebraic cycles of codimension~$j$ is zero for $j>\dim X$, and, similarly, $\Ch^j(X;n)$ is zero for $j>\dim(X)+n$.
    Furthermore, $\Ch^j(X;n)=0$ for $n<0$.
\end{enumerate}
Applying \cref{3.7.1}, we thus deduce:

\begin{nullities}{3.8.1}
\label{3.8.1}
  \begin{enumerate}[(i)]
    \item If $M$ is a pure motive of weight $w$, then $\Hom^i(1,M)=0$ for $i>-w$.
    \item If $M$ is effective of weight $w$, and $b>0$, then $\Hom^i(1,M(w+b))=0$ for $i>w+b$.
  \end{enumerate}
\end{nullities}

\begin{proof}
  For (i), suppose that $M$ is a direct factor of $H_\mot^a(X)(c)$, with $c\geq0$, and $X$ a smooth projective variety.
  The $\Ext^i(1,M)$ is then a direct factor of $\Ch^c(X;2c-i-a)$, and, if it is non-zero, then $2c-i-a\geq0$, i.e. $i\leq 2c-a=-w$.

  For (ii), We again use the weak Lefschetz theorem (the $\ell$-adic, and thus motivic if the $\ell$-adic cohomology is a fibre functor, version) to suppose that $\dim(X)\leq a$.
  We suppose that $c\geq a$, and we must prove the vanishing of $\Ext^i(1,H_\mot^a(X)(c))$ for $i>(c-a)+a=c$.
  This inequality is equivalent to $c>a+(2c-i-a)\geq\dim(X)+(2c-i-a)$.
\end{proof}

If \cref{3.7.1} is an isomorphism then this would also imply the weak Lefschetz theorems for the Chow groups, as well as the conjectures on the Chow groups presented at this conference by J.P. Murre.


\oldpage{159}
\subsection{}
\label{3.9}

If $M$ is the unit motive, then (\ref{3.8.1})(ii) demands that
\[
\label{3.9.1}
  H_\abs^n(\Spec(k),\QQ(j)) = 0 \quad\mbox{for $n>j$}.
\tag{3.9.1}
\]

It is not clear to me if we should hope for an analogue of \cref{3.9.1} over $\ZZ$.
If we define the $H_\abs^n(\Spec(k),\ZZ(j))$ as being the higher Chow groups, then \cref{3.9.1} is trivially true.

A more concrete question is the following: suppose that we could define Galois cohomology classes in the $H^n(\Gal(\overline{k}/k),\ZZ_\ell(j))$ in a way that was ``independent of $\ell$''; if $n>j$, then \cref{3.9.1} demands that these classes be torsion; should we hope for them to actually be zero?


\subsection{}
\label{3.10}

\textbf{Example.}
Let $G$ be an simply connected and absolutely simple algebraic group over a field $k$.
Over the algebraic closure $\overline{k}$ of $k$, the relations between the cohomology of $G$, $BG$, and $BT$ (where $T$ is a maximal torus) show that, in each of the usual cohomology theories, $H^i(G_{\overline{k}})=0$ for $i=1,2$, and that $H^3(G_{\overline{k}})(2)$ has a natural generator $a$.
Now work in $\ell$-adic cohomology.
Over $k$, the absolute $\ell$-adic cohomology of $G$ is the direct sum of that of a point and of a reduced cohomology $\widetilde{H}$, related to $G_{\overline{k}}$ by a Hochschild--Serre exact sequence.
We thus have that $\widetilde{H}^i(G,\ZZ_\ell(2))=0$ for $i\leq2$, and that $\widetilde{H}^3(G,\ZZ_\ell(2))$ is isomorphic to $\ZZ_\ell$, with a natural generator $a$.
This generator is primitive, and thus defines a homomorphism
\[
  G(k) \to H^3(\Spec(k),\ZZ_\ell(2)).
\]
We can show that, as predicted by \cref{3.9}, this arrow is zero:
by Mercurev--Suslin~\cite{27}, the $H^3$ is torsion free, and a trace argument shows that the image is torsion:
$G$ splits over a finite extension $k'$ of $k$, and $G(k')$ is equal to its derived group, at least if $|k'|\geq4$.

The same circle of ideas allows us to construct a canonical central extension of $G(k)$ by $H^2(\Spec(k),\ZZ/n(2))$, equal to $K_2(k)/nK_2(k)$ by Mercurev--Suslin~\cite{27}.
Recall that, for split $G$, a canonical central extension of $G(k)$ by $K_2(k)$ was constructed by Matsumoto~\cite[5.11]{25}.


\subsection{}
\label{3.11}

If $M$ is a motive over $\QQ$, then $H_\abs^1(M)$ appears in the Beilinson conjecture concerning the behaviour at $s=0$ of the $L$-function of $M$ (Beilinson~\cite{2} and the talks by Soul\'{e}, Nekovar, and Scholl at this conference).
I refer to these talks for a description of the conjecture and what is known about it.


\subsection{}
\label{3.12}

If $F$ is a number field, then \cref{3.7.1} predicts that, in the category of motives over $F$, we have that
\[
\label{3.12.1}
  \Ext^1(\QQ(0),\QQ(j)) = K_{2j-1}(F)\otimes\QQ,
\tag{3.12.1}
\]
with vanishing of the $\Ext^n$ for $n\geq2$.
The claim in \cref{3.12.1} imposes severe restrictions on the category of motives that are iterated extensions of Tate motives, and on the realisation systems attached to an object of this category.
See Deligne~\cite{14}, where an example considered in detail is that of the Lie algebra of the unipotent $\pi_1$ of the projective line minus three points.

\oldpage{160}
Here is a programme to prove the consequences of \cref{3.12.1} for such realisation systems.
\begin{enumerate}[(A)]
  \item Define a Tannakian category $\sh{T}$, with simple objects being the tensor powers $\QQ(n)$, for $n\in\ZZ$, of an object $\QQ(1)$ of rank~$1$ for which \cref{3.12.1} is true ``by definition''.
\end{enumerate}

The idea is that Bloch's higher Chow groups $\Ch^n(\Spec(F),i)$ are given by a simplicial complex that is graded by the codimension~$n$ of cycles, and that the product is almost given by a graded simplicial commutative algebra structure.
We say ``almost'', since the intersection of two cycles is not always defined, but only ``in general''.
Rationally, it should be possible to replace this almost algebra by a true commutative bigraded differential algebra, with $d$ of degree $(1,0)$, and
\begin{itemize}
  \item $A^{\bullet,m}=0$ for $m<0$ ;
  \item $A^{\bullet,0}$ consisting only of $\QQ$ in degree~$0$ ; and
  \item $H^\bullet(A^{\bullet,n})$ consisting only of $K_{2n-1}(F)\otimes\QQ$ in degree~$1$.
\end{itemize}
We can then define $\sh{T}$ as a suitable category of bigraded differential modules, and with morphisms up to homotopy.

This part of the programme has essentially been carried out by May~\cite{26}.
For a cubic variety, see the talk of Bloch in this volume.

\begin{enumerate}[(A)]
\setcounter{enumi}{1}
  \item Define, for each of the usual cohomology theories, a ``realisation'' fibre functor on the Tannakian category $\sh{T}$ of (A).
  \item Show that the mixed Tate motives in which we are interested (for example, $\Lie\pi_1(\PP^1\setminus\{0,1,\infty\})$) --- or, rather, the systems of their realisations --- come from an object of $\sh{T}$.
\end{enumerate}


%% Bibliography %%

\nocite{*}

\begin{thebibliography}{34}

  \bibitem{1}
  {\sc Atiyah,~M. and Hirzebruch,~F.}
  \newblock Analytic cycles on complex manifolds
  \newblock {\em Topology} \textbf{1} (1962), 25--85.

  \bibitem{2}
  {\sc Beilinson,~A.A.}
  \newblock Higher regulators and values of $L$-functions.
  \newblock {\em Current Problems in Mathematics} \textbf{24} (1984), 191--238.

  \bibitem{3}
  {\sc Beilinson,~A.A., Bernstein,~J., and Deligne,~P.}
  \newblock Faisceaux pervers.
  \newblock {\em Ast\'{e}rique} \textbf{100} (1982).

  \bibitem{4}
  {\sc Bloch,~S.}
  \newblock Algebraic cycles and higher K-theory.
  \newblock {\em Adv. in Math.} \textbf{61} (1986), 267--304.

  \bibitem{5}
  {\sc Bloch,~S.}
  \newblock Algebraic cycles and higher K-theory, A correction.
  \newblock Preprint (1989).

  \bibitem{6}
  {\sc Borovol,~M.}
  \newblock Langlands conjecture concerning conjugation of connected Shimura varieties.
  \newblock {\em Selecta Math. Soviet.} \textbf{31} (1983--84), 3--59.

  \bibitem{7}
  {\sc Brylinski,~J.L.}
  \newblock $1$-motifs et formes automorphes, in: {\em Journ\'{e}es SMF de Dijon (1981)}.
  \newblock {Publ. Math. de l'Univ. de Paris~VII} \textbf{15} (1981), 43--106.

  \bibitem{8}
  {\sc Deligne,~P.}
  \newblock Th\'{e}or\`{e}mes de Lefschetz et crit\`{e}res de d\'{e}g\'{e}n\'{e}rescence de suites spectrales.
  \newblock {\em Inst. Hautes \'{E}tudes Sci. Publ. Math.} \textbf{35} (1968), 107--126.

  \bibitem{9}
  {\sc Deligne,~P.}
  \newblock Th\'{e}orie de Hodge III.
  \newblock {\em Inst. Hautes \'{E}tudes Sci. Publ. Math.} \textbf{44} (1974), 5-77.

  \bibitem{10}
  {\sc Deligne,~P.}
  \newblock Vari\'{e}t\'{e}s de Shimura: Interpr\'{e}tation modulaire et techniques de construction de mod\`{e}les canoniques.
  \newblock {\em Proc. Sympos. Pure Math.} \textbf{33} (1979), 247--290.

  \bibitem{11}
  {\sc Deligne,~P. (notes by Brylinski,~J.L.).}
  \newblock Cycles de Hodge absolus et p\'{e}riodes des int\'{e}grales des vari\'{e}t\'{e}s ab\'{e}liennes.
  \newblock {\em M\'{e}moires de la Soc. Math. France} \textbf{2} (1980), 23--33.

  \bibitem{12}
  {\sc Deligne,~P. (notes by Milne,~J.).}
  \newblock Hodge cycles on abelian varieties.
  \newblock {\em SLN} \textbf{900} (1982), 9--100.

  \bibitem{13}
  {\sc Deligne,~P.}
  \newblock Motifs et groupe de Taniyama
  \newblock {\em SLN} \textbf{900} (19), 261--279.

  \bibitem{14}
  {\sc Deligne,~P.}
  \newblock Le groupe fondamental de la droite projective moins trois points, in: {\em Galois groups over $\mathbb{Q}$}.
  \newblock MSRI publications no.~\textbf{16}, 1989, 79--297.

  \bibitem{15}
  {\sc Deligne,~P.}
  \newblock Cat\'{e}gories tannakiennes, in: {\em Grothendieck Festschrift}, vol.~\textbf{2}
  \newblock Birkha\"{u}ser, 1990, 111--195.

  \bibitem{16}
  {\sc Deligne,~P. and Milne,~J.}
  \newblock Tannakian categories.
  \newblock {\em SLN} \textbf{900} (1982), 101--228.

  \bibitem{17}
  {\sc Griffiths,~P.A.}
  \newblock Periods of integrals on algebraic manifolds, II.
  \newblock {\em Amer. J. Math.} \textbf{90} (1968), 805--865.

  \bibitem{18}
  {\sc Griffiths,~P.A.}
  \newblock Periods of integrals on algebraic manifolds: Summary of main results and discussion of open problems.
  \newblock {\em Bull. Amer. Math. Soc.} \textbf{76} (1970), 228--296.

  \bibitem{19}
  {\sc Grothendieck,~A.}
  \newblock Standard conjectures on algebraic cycles, in: {\em Bombay Coll. on Algebraic Geometry (1968)}.
  \newblock Oxford, 1969, 193--199.

  \bibitem{20}
  {\sc Kleiman,~S.L.}
  \newblock Algebraic cycles and the Weil conjectures, in: {\em Dix Expos\'{e}s sur la Th\'{e}orie des Sch\'{e}mas}.
  \newblock Adv. Stud. Pure Math. \textbf{3}, 1968, 359--386.

  \bibitem{21}
  {\sc Kodaira,~K. and Spencer,~D.C.}
  \newblock Divisor classes on algebraic varieties.
  \newblock {\em Proc. Nat. Acad. Sci.} \textbf{39} (1953), 872--877.

  \bibitem{22}
  {\sc Kriz,~I. and May,~P.}
  \newblock An addendum to ``Notes towards Block--Deligne motives''.
  \newblock Preprint (1991).

  \bibitem{23}
  {\sc Langlands,~R.P. and Rapoport,~M.}
  \newblock Shimura Variet\"{a}ten und Gerben.
  \newblock {\em Crelle} \textbf{378} (1987), 113--220.

  \bibitem{24}
  {\sc Lefschetz,~S.}
  \newblock L'Analysis Situs et la G\'{e}om\'{e}trie Alg\'{e}brique, in: {\em Selected Papers}.
  \newblock Chelsea, 1924.

  \bibitem{25}
  {\sc Matsumoto,~H.}
  \newblock Sur les sous-groupes arithm\'{e}tiques des groupes semi-simples d\'{e}ploy\'{e}s.
  \newblock {\em Ann. Sci. \'{E}cole Norm. Sup.} \textbf{2} (1969), 1--62.

  \bibitem{26}
  {\sc May,~P.}
  \newblock Notes toward Bloch--Deligne motives.
  \newblock Preprint (1991).

  \bibitem{27}
  {\sc Merkurjev,~A.S. and Suslin,~A.A.}
  \newblock K-cohomology of Severi--Brauer varieties and the norm residue homomorphisms.
  \newblock {\em Izv. Akad. Sci. URSS} \textbf{46} (1968), 1011-1046.
  \newblock (English translation: {\em Math. USSR Izv.} \textbf{21} (1983), 307--340).

  \bibitem{28}
  {\sc Saavedra,~N.}
  \newblock Cat\'{e}gories Tannakiennes.
  \newblock Springer--Verlag, Lecture Notes in Math. \textbf{265}, 1972.

  \bibitem{29}
  {\sc Saito, Morihiko}
  \newblock Mixed Hodge Modules.
  \newblock {\em Res. Inst. Math. Sci.} (1987).

  \bibitem{30}
  {\sc Steenbrink,~J. and Zucker,~S.}
  \newblock Variations of Mixed Hodge Structures, I and II.
  \newblock {\em Invent. Math.} \textbf{80} (1985), 489--542 and 543--565.

  \bibitem{31}
  {\sc Weil,~A.}
  \newblock Crit\`{e}res d'\'{e}quivalence en g\'{e}om\'{e}trie alg\'{e}brique.
  \newblock {\em Math. Ann.} \textbf{128} (1954), 95--127.

  \bibitem{32}
  {\sc Zucker,~S.}
  \newblock The Hodge conjecture for cubic fourfolds.
  \newblock {\em Comp. Math.} \textbf{34} (1977), 199--209.

  \bibitem[SGA]{SGA}
  {\em S\'{e}minaire de G\'{e}om\'{e}trie Alg\'{e}brique du Bois-Marie}, directed by {\sc Grothendieck,~A.}

  \bibitem[SGA1]{SGA1}
  {\em Rev\^{e}tements \'{e}tales et groupe fondamental.}
  \newblock Springer--Verlag, Lecture Notes in Math. \textbf{224}, 1971.

\end{thebibliography}

\end{document}
